% This file is a demonstration on how a (HA-)prosper file should be
% changed to make it work with beamer.



% Copyright notice:

% Except for the following block, this file is the original
% Doc/Introduction.tex created by Hendri Adriaens
% http://center.uvt.nl/phd_stud/adriaens as part of the HA-prosper
% package.


% Changed:

\documentclass{beamer}

% You might wish to try this instead of the above line:
%\documentclass[class=article]{beamer}
%\usepackage{beamerbasearticle}
%\usepackage{hyperref}

\usepackage[framesassubsections]{beamerprosper}

\mode<presentation>
{
  \definecolor{beamerstructure}{RGB}{43,79,112}
  \definecolor{sidebackground}{RGB}{230,242,250}
  \color{beamerstructure}
  \usepackage[tab,width=3.25cm]{beamerthemesidebar}
  \usepackage{times}
  \userightsidebarcolortemplate{\color{sidebackground}}
  \beamertemplateballitem
  \beamertemplateboldframetitle
  \beamertemplateboldtitlepage
}

%\documentclass[pdf]{prosper}
%\usepackage[toc,highlight,HA]{HA-prosper}



% Original file:


% Introduction to the `HA-Prosper' package.
% Created by: Hendri Adriaens
%             http://center.uvt.nl/phd_stud/adriaens
%             Center for Economic Research
%             Tilburg University, the Netherlands

%================================================
% Please also read the manual of HA-prosper and
% of the specific style you are using since some
% features of this example might not be supported
% by the style you use.
%================================================

\title{Introduction to the HA-prosper package}
\subtitle{A package for use with prosper}
\author{Hendri Adriaens\\
\institution{CentER}\\
\institution{\href{http://center.uvt.nl/phd_stud/adriaens}{http://center.uvt.nl/phd\string_stud/adriaens}}}

\DefaultTransition{Wipe}
\TitleSlideNav{FullScreen}
\NormalSlideNav{ShowBookmarks}
\LeftFoot{\href{http://center.uvt.nl/phd_stud/adriaens}{Hendri Adriaens}, \today}
\RightFoot{Introduction to the HA-prosper package}

\HAPsetup{iacolor=lightgray,stype=1}

\begin{document}

% ==================================================================================
% Slide 1
\maketitle
% ==================================================================================


% ==================================================================================
% Slide 2
\tsectionandpart{Introduction}
% ==================================================================================


% ==================================================================================
% Slide 3
\overlays{2}{
\begin{slide}{Welcome}
\begin{itemstep}
\item Welcome to the introduction of the HA-prosper package.
\item The main features of HA-prosper are:
\begin{itemize}
\item table of contents;
\item portrait slides support;
\item notes;
\item prosper bug solutions.
\end{itemize}
\end{itemstep}
\end{slide}
}
% ==================================================================================


% ==================================================================================
% Slide 4
\overlays{2}{
\begin{slide}{Styles}
\begin{itemstep}
\item The HA-prosper packages adds some functionality to prosper, but\dots
\item The available styles show how to extend these possibilities
even further and for instance:
\begin{itemize}
\item embed additional navigational elements on slides;
\item use multiple slide layouts in the same presentation;
\item put figures on slides together with text;
\item etcetera.
\end{itemize}
\end{itemstep}
\end{slide}
}
% ==================================================================================


% ==================================================================================
% Slide 5
\tsectionandpart{Features}
% ==================================================================================


% ==================================================================================
% Slide 6
\overlays{4}{
\begin{slide}{Table of contents}
\begin{itemstep}
\item A table of contents can be put on every slide;
\item A table of contents entry is created from the slide title
or from text that is given as an optional argument;
\item The table of contents has the following features:
\begin{itemize}
\item Highlighting of the current slide or section is possible;
\item Items can be omitted;
\item Parts of the table of contents can be hidden when these are
unnecessary.
\end{itemize}
\item The style that you use should of course support the inclusion
of the table of contents.
\end{itemstep}
\end{slide}
}
% ==================================================================================


% ==================================================================================
% Slide 7
\overlays{6}{
\begin{slide}{More features}
HA-prosper contains more features which are fully described in the manual.
\begin{itemstep}[sstart=2]
\item Portrait slides
\item Notes
\item Dualslide
\item Blackslide
\item and a lot more\dots
\end{itemstep}
\end{slide}
}
% ==================================================================================


% ==================================================================================
% Slide 8
\overlays{3}{
\begin{slide}{Prosper bug solutions}
\begin{itemstep}
\item Numbering of equations, tables and figures on overlays is supported;
\item The `\textbackslash and' command for authors is supported;
\item Solved placement of left and right footers.
\end{itemstep}
\end{slide}
}
% ==================================================================================


% ==================================================================================
% Slide 9
\tsectionandpart{Questions and comments}
% ==================================================================================


% ==================================================================================
% Slide 10
\overlays{3}{
\begin{slide}{Comments and contributions}
\begin{itemstep}
\item Comment and contributions are always welcome.
\item In case you want to contribute a style or template, please
insert both
\begin{itemize}
\item documentation;
\item an example that demonstrates all the features that your template provides.
\end{itemize}
\item You can contact
\href{http://stuwww.uvt.nl/~hendri/Personal/contact.html}{\underline{Hendri Adriaens}}
(click the link) in case you have comments or questions about
developing a new template or if want to submit your style.
\end{itemstep}
\end{slide}
}
% ==================================================================================


% ==================================================================================
% Slide 11
\overlays{2}{
\begin{slide}{Questions}
\begin{itemstep}
\item If you have questions, please first consult the documentation
of the following packages (depending on your problem):
\begin{itemize}
\item HA-prosper;
\item Prosper;
\item Hyperref;
\item PSTricks.
\end{itemize}
\item In case you cannot find the answer and your question is related
to HA-prosper, you can post a message to the HA-prosper mailinglist:\par
\href{http://listserv.surfnet.nl/archives/ha-prosper.html}{http://listserv.surfnet.nl/archives/ha-prosper.html}.
\end{itemstep}
\end{slide}
}
% ==================================================================================


\end{document}
