\documentclass{ltxdoc}

% Copyright 2003 by Till Tantau <tantau@users.sourceforge.net>.
%
% This program can be redistributed and/or modified under the terms
% of the LaTeX Project Public License Distributed from CTAN
% archives in directory macros/latex/base/lppl.txt.

\usepackage{pgf,xcolor}
\usepackage[left=2.25cm,right=2.25cm,top=2.5cm,bottom=2.5cm,nohead]{geometry}
\usepackage{amsmath,amssymb}
\usepackage[pdfborder={0 0 0}]{hyperref}

\pgfdeclareimage[width=6.66666cm,height=5cm]{themebars}{themebars}
\pgfdeclareimage[width=6.66666cm,height=5cm]{themebars2}{themebars2}
\pgfdeclareimage[width=6.66666cm,height=5cm]{themeboxes}{themeboxes}
\pgfdeclareimage[width=6.66666cm,height=5cm]{themeboxes2}{themeboxes2}
\pgfdeclareimage[width=6.66666cm,height=5cm]{themeclassic}{themeclassic}
\pgfdeclareimage[width=6.66666cm,height=5cm]{themeclassic2}{themeclassic2}
\pgfdeclareimage[width=6.66666cm,height=5cm]{themelined}{themelined}
\pgfdeclareimage[width=6.66666cm,height=5cm]{themelined2}{themelined2}
\pgfdeclareimage[width=6.66666cm,height=5cm]{themeplain}{themeplain}
\pgfdeclareimage[width=6.66666cm,height=5cm]{themeplain2}{themeplain2}
\pgfdeclareimage[width=6.66666cm,height=5cm]{themesidebar}{themesidebar}
\pgfdeclareimage[width=6.66666cm,height=5cm]{themesidebar2}{themesidebar2}
\pgfdeclareimage[width=6.66666cm,height=5cm]{themesidebardark}{themesidebardark}
\pgfdeclareimage[width=6.66666cm,height=5cm]{themesidebardark2}{themesidebardark2}
\pgfdeclareimage[width=6.66666cm,height=5cm]{themesidebartab}{themesidebartab}
\pgfdeclareimage[width=6.66666cm,height=5cm]{themesidebartab2}{themesidebartab2}
\pgfdeclareimage[width=6.66666cm,height=5cm]{themesidebardarktab}{themesidebardarktab}
\pgfdeclareimage[width=6.66666cm,height=5cm]{themesidebardarktab2}{themesidebardarktab2}
\pgfdeclareimage[width=6.66666cm,height=5cm]{themesplit}{themesplit}
\pgfdeclareimage[width=6.66666cm,height=5cm]{themesplit2}{themesplit2}
\pgfdeclareimage[width=6.66666cm,height=5cm]{themesplitcondensed}{themesplitcondensed}
\pgfdeclareimage[width=6.66666cm,height=5cm]{themesplitcondensed2}{themesplitcondensed2}
\pgfdeclareimage[width=6.66666cm,height=5cm]{themetree}{themetree}
\pgfdeclareimage[width=6.66666cm,height=5cm]{themetree2}{themetree2}
\pgfdeclareimage[width=6.66666cm,height=5cm]{themetreebars}{themetreebars}
\pgfdeclareimage[width=6.66666cm,height=5cm]{themetreebars2}{themetreebars2}

\def\beamer{\textsc{beamer}}
\def\pdf{\textsc{pdf}}
\def\pgf{\textsc{pgf}}
\def\pstricks{\textsc{pstricks}}

\def\declare#1{{\color{red!75!black}#1}}

\def\command#1{\list{}{\leftmargin=2em\itemindent-\leftmargin\def\makelabel##1{\hss##1}}%
\item\extractcommand#1@\par\topsep=0pt}
\def\endcommand{\endlist}
\def\extractcommand#1#2@{\strut\declare{\texttt{\string#1}}#2}

\def\example{\par\smallskip\noindent\textit{Example: }}

\def\environment#1{\list{}{\leftmargin=2em\itemindent-\leftmargin\def\makelabel##1{\hss##1}}%
\extractenvironement#1@\par\topsep=0pt}
\def\endenvironment{\endlist}
\def\extractenvironement#1#2@{%
\item{{\ttfamily\char`\\begin\char`\{\declare{#1}\char`\}}#2}%
  {\itemsep=0pt\parskip=0pt\item{\meta{environment contents}}%
  \item{\ttfamily\char`\\end\char`\{\declare{#1}\char`\}}}}

\def\classoption#1{\list{}{\leftmargin=2em\itemindent-\leftmargin\def\makelabel##1{\hss##1}}%
\item{{\ttfamily\char`\\documentclass[\declare{#1}]\char`\{beamer\char`\}}}\par\topsep=0pt}
\def\endclassoption{\endlist}


\def\smallpackage{\vbox\bgroup\package}
\def\endsmallpackage{\egroup\endpackage}
\def\package#1{\list{}{\leftmargin=2em\itemindent-\leftmargin\def\makelabel##1{\hss##1}}%
\extracttheme#1@\par\topsep=0pt}
\def\endpackage{\endlist}
\def\extracttheme#1#2@{%
\item{{{\ttfamily\char`\\usepackage}#2{\ttfamily\char`\{\declare{#1}\char`\}}}}}


\newcommand\opt[1]{{\color{black!50!green}#1}}
\renewcommand\oarg[1]{\opt{{\ttfamily[}\meta{#1}{\ttfamily]}}}
\newcommand\sarg[1]{\opt{{\ttfamily\char`\<}\meta{#1}{\ttfamily\char`\>}}}
\newcommand\ssarg[1]{{\ttfamily\char`\<}\meta{#1}{\ttfamily\char`\>}}




\begin{document}

\title{User's Guide to the Beamer Class, Version 1.00\\
\Large\href{http://latex-beamer.sourceforge.net}{\texttt{http://latex-beamer.sourceforge.net}}}
\author{Till Tantau\\
  \href{mailto:tantau@users.sourceforge.net}{\texttt{tantau@users.sourceforge.net}}}

\maketitle

\tableofcontents

\section{Introduction}

\subsection{Overview}

This user's guide explains the functionality of the \beamer\ class.
It is a \LaTeX\ class that allows you to create a presentation with a
projector. It can also be used to create slides. It behaves 
similarly to other packages like \textsc{prosper}, but has the
advantage that it works together directly with |pdflatex|, but
also with |dvips|.

To use the \beamer\ class, proceed as follows:
\begin{enumerate}
\item
  Specify |beamer| as document class instead of
  |article|.
\item
  Structure your \LaTeX\ text using |section| and
  |subsection| commands.
\item
  Place the text of the individual slides inside |frame|
  commands.
\item
  Run |pdflatex| on the text (or |latex|,
  |dvips|, and |ps2pdf|).
\end{enumerate}

The \beamer\ class has several useful features: You don't need any
external programs to use it other than |pdflatex|, but it works
also with |dvips|. You can easily and intuitively create
sophisticated overlays. Finally, you can easily change the whole slide
theme or only parts of it. The following code shows a typical usage of
the class.

\begin{verbatim}
\documentclass{beamer}

\usepackage{beamerthemesplit}

\title{Example Presentation Created with the Beamer Package}
\author{Till Tantau}
\date{\today}

\begin{document}

\frame{\titlepage}

\section[Outline]{}
\frame{\tableofcontents}

\section{Introduction}
\subsection{Overview of the Beamer Class}
\frame
{
  \frametitle{Features of the Beamer Class}

  \begin{itemize}
  \item<1-> Normal LaTeX class.
  \item<2-> Easy overlays.
  \item<3-> No external programs needed.      
  \end{itemize}
}
\end{document}
\end{verbatim}

Run |pdflatex| on this code (twice) and then use, for example, the
Acrobat Reader to present the resulting |.pdf| file in a
presentation. You can also, alternatively, use |dvips|; see
Section~\ref{section-postscript} for details.

As can be seen, the text looks almost like a normal \LaTeX\ text. The
main difference is the usage of the |\frame| command. This
command takes one parameter, which is the text that should be shown on
the frame. Typically, the contents of a frame is shown on a single
slide. However, in case you use overlay commands inside a frame, a
single frame command may produce several slides. An example is the
last frame in the above example. There, the |\item| commands
are followed by \emph{overlay specifications} like |<1->|,
which means ``from slide 1 on.'' Such a specification causes the item
to be shown only on the specified slides of the frame (see
Section~\ref{section-overlay} for details). In the above example, a
total of five slides are produced: a title page slide, an outline
slide, a slide showing only the first of the three items, a slide
showing the first two of them, and a slide showing all three items.
 
To structure your text, you can use the commands |\section| and
|\subsection|. These commands will not only create entries in the
table of contents, but will also in the navigation bars.


\subsection{How to Read this User's Guide}

This user guide is both intended as a tutorial and as a reference
guide. If you have not yet installed the package, please read
Section~\ref{section-installation} first. If you do not have much
experience with preparing presentations, 
Section~\ref{section-workflow} might be especially helpful. The later
sections explain the basic usage of the |beamer| class as well as
advanced features. If you wish to adjust the way your presentations
look (for example, if you wish to add a default logo of your
institution to every presentation in the future), please read the last
section. 

In this guide you will find the descriptions of all ``public''
commands provided by the |beamer| class. In each such
description, the described command, environment, or option is printed 
in red. Text shown in green is optional and can be left out.




\section{Installation and Compatibility}

\label{section-installation}


To use the beamer class, you just need to put the files of the
\beamer\ package in a directory that is read by \TeX. To uninstall the
class, simply remove these files once more. The same is true of the
\textsc{pgf} package, which you will also need.

Unfortunately, there are different ways of making \TeX\ ``aware'' of
the files in the \beamer\ package. Which way you should choose depends
on how permanently you intend to use the class.


\subsection{Installing Debian and Red Hat Packages}

Currently, there are no out-of-the-box Debian or Red Hat packages of
the beamer class available.



\subsection{Temporary Installation}

If you only wish to install the beamer class for a quick appraisal, do
the following: Obtain the latest source version (ending
|.tar.gz|) of the \beamer\ package from 
\href{http://sourceforge.net/projects/latex-beamer/}{|http://sourceforge.net/projects/latex-beamer/|}
(most likely, you have already done this). Next, you also need at
least version 0.40 of the \textsc{pgf} package, which can be found at
the same place. Finally, you need at least version 1.03 of the
\textsc{xcolor} package, which can also be found at that place
(although the version on CTAN might be newer).

In all cases, the packages contain a bunch of files (for the \beamer\
class, |beamer.cls| is one of these files and happens to be the
most important one, for the \textsc{pgf} package |pgf.sty| is
the most important file). 
Place all files in three directories. For example,
|/home/tantau/beamer/|, |/home/tantau/pgf/|, and
|/home/tantau/xcolor/| would work fine for me. Then setup the
environment variable called |TEXINPUTS| to be the following
string (how exactly this is done depends on your operating system and
shell): 

\begin{verbatim}
.:/home/tantau/beamer:/home/tantau/pgf:/home/tantau/xcolor:
\end{verbatim}

Naturally, if the |TEXINPUTS| variable is already defined
differently, you should \emph{add} the three directories to the list. Do
not forget to place a colon at the end (corresponding to an empty
path), which will include all standard directories.



\subsection{Installation in a texmf Tree}

For a more permanent installation, you can place the files of the
\beamer\ package and of the \textsc{pgf} package (see the previous
subsection on how to obtain them) in an appropriate |texmf|
tree. 

When you ask \TeX\ to use a certain class or package, it usually looks
for the necessary files in so-called |texmf| trees. These trees
are simply huge directories that contain these files. By default,
\TeX\ looks for files in three different |texmf| trees:
\begin{itemize}
\item
  The root |texmf| tree, which is usually located at
  |/usr/share/texmf/|, |c:\texmf\|, or\\
  |c:\Program Files\TeXLive\texmf\|.
\item
  The local  |texmf| tree, which is usually located at
  |/usr/local/share/texmf/|, |c:\localtexmf\|, or\\
  |c:\Program Files\TeXLive\texmf-local\|.
\item
  Your personal  |texmf| tree, which is located in your home
  directory.   
\end{itemize}

You should install the packages either in the local tree or in
your personal tree, depending on whether you have write access to the
local tree. Installation in the root tree can cause problems, since an
update of the whole \TeX\ installation will replace this whole tree.

Inside whatever |texmf| directory you have chosen, create
the sub-sub-sub-directories
\begin{itemize}
\item
  |texmf/tex/latex/beamer| and
\item
  |texmf/tex/latex/pgf|
\item
  |texmf/tex/latex/xcolor|
\end{itemize}
and place all files in these three directories.

Finally, you need to rebuild \TeX's filename database. This done by
running the command  |texhash| or |mktexlsr| (they are
the same). In MikTeX, there is a menu option to do this.

\vskip1em
For a more detailed explanation of the standard installation process
of packages, you might wish to consult
\href{http://www.ctan.org/installationadvice/}{|http://www.ctan.org/installationadvice/|}.
However, note that the \beamer\ package does not come with a
|.ins| file (simply skip that part).



\subsection{Testing the Installation}

To test your installation, copy the file |beamerexample1.tex|
from the examples subdirectory to some place where you usually
create presentations. Then run the command |pdflatex| several times on
the file and check whether the resulting |beamerexample1.pdf|
looks correct. If so, you are all set.

If you have updated from a previous version and you have trouble
\TeX ing some old file, it sometimes helps to delete all the extra
files \TeX\ creates automatically (like the |.aux| and
|.head| files).


\subsection{Compatibility}

When using certain packages together with the |beamer| class, extra
options or precautions may be necessary.

\begin{package}{{CJK}}
  When using the |CJK| package, you must use the class option
  \declare{|CJK|}. See |beamerexample4.tex| for an example.
\end{package}

\begin{package}{{listings}}
  Note that you must treat |lstlisting| environments exactly the same
  way as you would treat |verbatim| environments.
\end{package}






\section{Workflow}

\label{section-workflow}

This section presents a possible workflow for creating a beamer
presentation and possibly a handout to go along with it. Technical
questions are addressed, like which programs to call with 
which parameters, and hints are given on how to create a
presentation. If you have already created numerous presentations, you
may wish to skip the first of the following steps 
and only have a look at how to convert the |.tex| file into a
|.pdf| of |.ps| file.


\subsection{Step Zero: Know the Time Constraints}

When you start to create a presentation, the very first thing you
should worry about is the amount of time you have for your
presentation. Depending on the occasion, this can
be anything between 2 minutes and two hours. A simple rule for the
number of frames is that you should have at most one frame per
minute.

In most situations, you will have less time for your presentation that
you would like. \emph{Do not try to squeeze more into a
  presentation than time allows for.} No matter how important some
detail seems to you, it is better to leave it out, but get the main
message across, than getting neither the main message nor the detail
across. 

In many situations, a quick appraisal of how much time you have will
show that you won't be able to mention certain details. Knowing this can
save you hours of work on preparing slides that you would have to remove
later anyway.




\subsection{Step One: Setup the Files}

It is advisable that you create a folder for each
presentation. Even though your presentation will usually reside in a
single file, \TeX\ produces so many extra files that things can easily
get very confusing otherwise. The folder's name should ideally start
with the date of your talk in ISO format (like 2003-12-25 for a
Christmas talk), followed by some reminder text of what the talk is
all about. Putting the date at the front in this format causes your
presentation folders to be listed nicely when you have several of them
residing in one directory. If you use an extra directory for each
presentation, you can call your main file |main.tex|. 

To create an initial |main.tex| file for your talk, copy an
existing file (like the file |beamerexample1.tex| that comes along
with the contribution) and delete everything that is not going to be
part of your talk. Adjust the |\author{}| and other fields as 
appropriate. 

If you wish your talk to reside in the same file as some different,
non-presentation article version of your text, it is advisable to
setup a more elaborate file scheme. See
Section~\ref{section-article-version-workflow} for details.




\subsection{Step Two: Structure You Presentation}

With the time constraints in mind, make a mental inventory of the
things you can reasonably talk about within the time available. Then
categorize the inventory into sections and subsections. For very long
talks (like a 90 minute lecture), you might also divide your talk into
independent parts (like a ``review of the previous lecture part'' and
a ``main part''). Put |\section{}| and |\subsection{}| commands into
the (more or less empty) main file. Do not create any frames until you
have a first working version of a possible table of contents. Do not
feel afraid to change it later on as you work on the talk.

You should not use more than four sections and not less than two per
part. Even four sections are usually too much, unless they follow 
a very easy pattern. Five and more sections are simply too hard to
remember for the audience. After all, when you present the table of
contents, the audience will not yet really be able to grasp the
importance and relevance of the different sections and will most
likely have forgotten them by the time you reach them.

Ideally, a table of contents should be understandable by itself. In
particular, it should be comprehensible \emph{before} someone has
heard your talk. Keep section and subsection titles
self-explaining. Note each part has its own table of contents. 

Both the sections and the subsections should follow a logical
pattern. Begin with an explanation of what your talk is all about. (Do
not assume that everyone knows this. The Ignorant Audience Law states:
The audience always knows less than you think it should know, even if
you take the Ignorant Audience Law into account.) Then explain what
you or someone else has found out concerning the subject
matter. Always conclude your talk with a summary that repeats the main
message of the talk in a short and simple way. People pay most
attention at the beginning and at the end of talks. The summary is
your ``second chance'' to get across a message.

You can also add an appendix part using the |\appendix| command. Put
everything into this part which you do not actually intend to talk
about, but which might come in handy in questions are asked.



\subsection{Step Three: Creating a PDF or PostScript File}

Once a first version of the structure is finished, you should create a
first PDF or PostScript file of your (still empty) talk. This file
will only contain the title page and the table of contents. The file
might  look like this:

\begin{verbatim}
\documentclass{beamer}
% This is the file main.tex

\usepackage{beamerthemesplit}

\title{Example Presentation Created with the Beamer Package}
\author{Till Tantau}
\date{\today}

\begin{document}

\frame{\titlepage}

\section[Outline]{}
\frame{\tableofcontents}

\section{Introduction}
\subsection{Overview of the Beamer Class}
\subsection{Overview of Similar Classes}

\section{Usage}
\subsection{...}
\subsection{...}

\section{Examples}
\subsection{...}
\subsection{...}

\end{document}
\end{verbatim}



\subsubsection{Creating PDF}

To create a |PDF| version of this file, run the program
|pdflatex| on |main.tex| at least twice. Your need to run it twice, so
that \TeX\ can create the table of contents. (It may even be necessary
to run it more often since there are all sorts of auxilliary files
created.) In the following example, the greater-than sign is the prompt. 

\begin{verbatim}
> pdflatex main.tex
    ... lots of output ...
> pdflatex main.tex
    ... lots of output ...
\end{verbatim}

You can next use a program like the Acrobat Reader or |xpdf|
to view the resulting presentation.

\begin{verbatim}
> acroread main.pdf
\end{verbatim}

When printing a presentation using Acrobat, make sure that the option
``expand small pages to paper size'' is enabled. This is necessary,
because slides are only 128mm times 96mm.

To put several slides onto one page (useful for the handout version)
or to enlarge the slides, you can use the program |pdfnup|. Also, many
commercial programs can perform this task.



\subsubsection{Creating PostScript}
\label{section-postscript}

To create a PostScript version, you should first ascertain that the
\textsc{hyperref} package (which is automatically loaded by the
\beamer\ class) uses the option |dvips| or some compatible
option, see the documentation of the \textsc{hyperref} package for
details. Whether this is the case depends on the contents of your
local |hyperref.cfg| file. You can enforce the usage of this
option by passing |dvips| or a compatible option to the
\beamer\ class (write |\documentclass[dvips]{beamer}|), which
will pass this option on to the \textsc{hyperref} package.

You can then run |latex| twice, followed by |dvips|.

\begin{verbatim}
> latex main.tex
    ... lots of output ...
> latex main.tex
    ... lots of output ...
> dvips -P pdf main.dvi
\end{verbatim}

The option (|-P pdf|) tells |dvips| to use
Type~1 outline fonts instead of the usual Type~3 bitmap fonts. You may
wish to omit this option if there is a problem with it. 

If you wish each slide to completely fill a letter-sized page, use the
following commands instead:

\begin{verbatim}
> dvips -P pdf -tletter main.dvi -o main.temp.ps
> psnup -1 -W128mm -H96mm -pletter main.temp.ps main.ps
\end{verbatim}

For A4-sized paper, use:

\begin{verbatim}
> dvips -P pdf -ta4 main.dvi -o main.temp.ps
> psnup -1 -W128mm -H96mm -pa4 main.temp.ps main.ps
\end{verbatim}

In order to create a white margin around the whole page (which is sometimes
useful for printing), add the option |-m 1cm| to the options of
|psnup|. 

To put two or four slides on one page, use |-2|, respectively
|-4| instead of |-1| as the first parameter for
|psnup|. In this case, you may wish to add the option
|-b 1cm| to add a bit of space around the individual slides.

You can convert a PostScript file to a pdf file using

\begin{verbatim}
> ps2pdf main.ps main.pdf
\end{verbatim}



\subsection{Step Four: Create Frames}

Once the table of contents looks satisfactory, start creating frames
for your presentation. In the following, some guidelines that I stick
to are given on what to put on slides and what not to put. You can
certainly ignore any of these guideline, but you should be aware of
it when you ignore a rule and you should be able to justify it to
yourself. 


\subsubsection{Guidelines on What to Put on a Frame}

\begin{itemize}
\item
  A frame with too little on it is better than a
  frame with too much on it.
\item
  Do not assume that everyone in the audience is an expert on the
  subject matter. (Remember the Ignorant Audience Law.) Even if the
  people listening to you should be experts, they may last have heard
  about things you consider obvious several years ago. You should
  always have the time for a quick reminder of what exactly a
  ``semantical complexity class'' or an ``$\omega$-complete partial
  ordering'' is.
\item
  Never put anything on a slide that you are not going to explain
  during the talk, not even to impress anyone with how
    complicated your subject matter really is. However, you may
  explain things that are not on a slide.
\item
  Keep it simple. Typically, your audience will see a slide for less
  than 50 seconds. They will not have the time to puzzle through long
  sentences or complicated formulas.
\end{itemize}



\subsubsection{Guidelines on Text}

\begin{itemize}
\item
  Put a title on each frame. The title explains the contents of the
  frame to people who did not follow all details on the slide.
\item
  Ideally, titles on consecutive frames should ``tell a story'' all by
  themselves.
\item
  \emph{Never} use a smaller font size to ``squeeze more on a frame.''
\item
  Prefer enumerations and itemize environments over plain text. Do not
  use long sentences.
\item
  Text and numbers in figures should have the \emph{same} size as
  normal text. Illegible numbers on axes usually ruin a chart and its
  message. 
\end{itemize}


\subsubsection{Guidelines on Graphics}

\begin{itemize}
\item
  Put (at least) one graphic on each slide, whenever
  possible. Visualizations help an audience enormously.
\item
  Usually, place graphics to the left of the text. (Use the
  |columns| environment.) 
\item
  Graphics should have the same typographic parameters as the
  text: Use the same fonts (at the same size) in graphics as in the
  main text. A small dot in a graphic should have exactly the same 
  size as a small dot in a text. The line width should be the same as
  the stroke width used in creating the glyphs of the font. For
  example, an 11pt non-bold Computer Modern font has a stroke width of
  0.4pt.
\item
  While bitmap graphics, like photos, can be much more colorful than the
  rest of the text, vector graphics should follow the same ``color
  logic'' as the main text (like black~= normal lines, red~= hilighted
  parts, green~= examples, blue~= structure).
\item
  Like text, you should explain everything that is shown on a
  graphic. Unexplained details make the audience puzzle whether this
  was something important that they have missed. Be careful when
  importing graphics from a paper or some other source. They usually
  have much more detail than you will be able to explain.
\end{itemize}

For technical hints on how to create graphics, see
Section~\ref{section-graphics}.


\subsubsection{Guidelines on Colors}

\begin{itemize}
\item
  Use colors sparsely. The prepared themes are already quite
  colorful (blue~= structure, red~= alert, green~= example). If you
  add more colors, you should have a \emph{very} good reason.
\item
  Be careful when using bright colors on white background,
  \emph{especially} when using green. What looks good on your monitor
  may look bad during a presentation due to the different ways
  monitors, beamers, and printers reproduce colors. Add lots of black
  to pure colors when you use them on bright backgrounds.
\item
  Maximize contrast. Normal text should be black on white or at least
  something very dark on something very bright. \emph{Never} do things
  like ``light green text on not-so-light green background.''
\item
  Background shadings decrease the legibility without increasing the
  information content. Do not add a background shading just because it
  ``somehow looks nicer.''
\item
  Inverse video (bright text on dark background) can be a problem
  during presentations in bright environments since only a small
  percentage of the presentation area is light up by the
  beamer. Inverse video is harder to reproduce on printouts and on
  transparencies. 
\end{itemize}


\subsubsection{Guidelines on Animations and Special Effects}

\begin{itemize}
\item
  Use animations to explain the dynamics of systems, algorithms, etc.
\item
  Do \emph{not} use animations just to attract the attention of your
  audience. This often distracts attention away from the main topic of the
  slide.
\item
  Do \emph{not} use distracting special effects like ``dissolving''
  slides unless you have a very good reason for using them. If you use
  them, use them sparsely. 
\end{itemize}


\subsubsection{Using the Draft Option}

While working on your presentation, it may sometimes be useful to
\TeX\ your |.tex| file quickly and have the presentation contain only
the most important information. This is especially true if you have a
slow machine. In this case, the |draft| class option is useful.

\begin{classoption}{{draft}}
  Causes the head lines, foot lines, and sidebars to be replaced by
  gray rectangles (their sizes are still computed, though). Many
  other packages, including |pgf| and |hyperref|, also ``speedup''
  when this option is given.
\end{classoption}



\subsection{Step Five: Test Your Presentation}

\emph{Always} test your presentation. For this, you should
vocalize or subvocalize your talk in a quite environment. Typically,
this will show that your talk is too long. You should then remove
parts of the presentation, such that it fits into the allotted time
slot. Do \emph{not} attempt to talk faster in order to squeeze the
talk into the given amount of time. You are almost sure to loose your
audience this way.

Do not try to create the ``perfect'' presentation immediately. Rather,
test and retest the talk and modify it as needed. 




\subsection{Step Six: Optionally Create a Handout or an Article Version}

Once your talk is fixed, you can create a handout, if this seems
appropriate. For this, use the class option |handout| as
explained in Section~\ref{handout}. Typically, you might wish
to put several handout slides on one page. See
Section~\ref{section-postscript} on how to do this.

You may also wish to create an article version of your talk. An
``article version'' of your presentation is a normal \TeX\ text
typeset using, for example, the document class |article| or perhaps
|lncs| or a similar document class. The \beamer\ class offers
facilities to have this version coexist with your presentation version
in one file and to share code. Also, you can include slides of your
presentation as figures in your article version. Details on how to
setup the article version can be found in
Section~\ref{section-article}.  






\section{Frames and Overlays}

\label{section-overlay}

\subsection{Frames}

\subsubsection{Frame Creation}

A presentation consists of a series of frames. Each frame consists of
a series of slides. You create a frame using the command
|\frame|. This command takes one parameter, namely the
contents of the frame. All of this text that is not tagged by overlay
specifications (see Section~\ref{subsection-overlay}) is shown on all
slides of the frame.  

\begin{command}{\frame\oarg{overlay specification}\marg{frame text}}
  The \meta{overlay specification} dictates which slides of a frame are
  to be shown, see Section~\ref{subsection-restriction} for details. 
  The \meta{frame text} can be normal \LaTeX\ text, but may not contain
  |\verb| commands or |verbatim| environments, unless special
  precautions are taken, see Section~\ref{section-verbatim}.
  \example
  \begin{verbatim}
\frame
{
  Some text...

  Some more...
}
  \end{verbatim}
\end{command}

\begin{command}{\plainframe\oarg{overlay specification}\marg{frame text}}
  This command creates a frame in which the head lines, foot lines,
  and side bars are suppressed. This is useful for creating single
  frames with different head and foot lines or for creating frames
  showing big pictures that completely fill the frame.

  \example A frame with a picture completely filling the frame:  
\begin{verbatim}
\plainframe{\hfill\pgfimage[height=9.6cm]{bigimagefilename}\hfill}
\end{verbatim}
  
  \example A title page, in which the head and foot lines are replaced
  by two graphics.
\begin{verbatim}
\usetitlepagetemplate{
  \beamerline{\pgfuseimage{toptitle}}
  \vskip0pt plus 1filll

  \begin{centering}
    \Large{\textbf{\inserttitle}}
    
    \insertdate
  \end{centering}

  \vskip0pt plus 1filll
  \beamerline{\pgfuseimage{bottomtitle}}
}

\begin{document}
\plainframe{\titlepage}
\end{verbatim}
\end{command}




\subsubsection{Components of a Frame}

Each frame consists of several components:
\begin{enumerate}\itemsep=0pt\parskip=0pt
\item a head line,
\item a foot line,
\item a left side bar,
\item a right side bar,
\item navigation symbols,
\item a logo,
\item a frame title, and
\item some frame contents.
\end{enumerate}

A frame need not have all of these components. Usually, the first six
components are automatically setup by the theme you are
using. To change them, you must install an appropriate template, see
Section~\ref{section-head-templates} for the head and foot lines and
Section~\ref{section-sidebar-templates} for the side bars. To install
a logo, invoke the following command in the preamble, \emph{after}
having loaded the theme:

\begin{command}{\logo\marg{logo text}}
  The \meta{logo text} is usually a command for including a
  graphic.
  \example
\begin{verbatim}
\pgfdeclareimage[height=0.5cm]{logo}{tu-logo}
\logo{\pgfuseimage{logo}}
\end{verbatim}  
\end{command}

The frame title is shown prominently at the top of the frame and can
be specified with the following command:

\begin{command}{\frametitle\marg{frame title text}}
  You should end the \meta{frame title text} with a period, if the title is a
  proper sentence. Otherwise, there should not be a period.
\example
\begin{verbatim}
\frame{
  \frametitle{A Frame Title is Important.}

  Frame contents.
}
\end{verbatim}
\end{command}

Be default, all material for a slide is vertically centered. You can
change this using the following class options:

\begin{classoption}{slidestop}
  Place text of slides at the (vertical) top of the slides. This
  corresponds to a vertical ``flush.''
\end{classoption}

\begin{classoption}{slidescentered}
  Place text of slides at the (vertical) center of the slides. This is
  the default.
\end{classoption}



\subsubsection{Restricting the Slides of a Frame}
\label{subsection-restriction}

The number of slides in a frame is automatically
calculated. If the largest number mentioned in any
overlay specification inside the frame is 4, four slides are
introduced (despite the fact that a specification like |<4->|
might suggest that more than four slides would be possible).

You can also specify the number of slides in the frame ``by hand.'' To
do so, you pass an optional argument to the |\frame| command,
given in \emph{square} brackets. This argument is also a 
slide specification. The frame will contain only the slides
specified in this argument. Consider the following example.

\begin{verbatim}
\frame[1-2,4-]
{
  This is slide number \only<1>{1}\only<2>{2}\only<3>{3}%
  \only<4>{4}\only<5>{5}.
}
\end{verbatim}
This command will create a frame containing four slides. The first
will contain the text ``This is slide number~1,'' the second ``This is
slide number~2,'' the third ``This is slide number~4,'' and the fourth
``This is slide number~5.''

A useful specification is just |[0]|, which causes the frame to
have to no slides at all. For example, |\frame[handout:0]| causes
the frame to be suppressed in the handout version, but to be shown
normally in all other versions.


\subsubsection{Verbatim Commands and Listings inside Frames}
\label{section-verbatim}

The |\verb| command, the |verbatim| environment, the
|lstlisting| environment, and related environments that allow
you to typeset arbitrary text work only in
frames that contain a single slide or that are suppressed
altogether. Furthermore, you must explicitly specify that the frame
contains only one slide; like this:
\begin{verbatim}
\frame[all:1]
{
  \frametitle{Our Search Procedure}

\begin{verbatim}
  int find(int* a, int n, int x)
  {
    for (int i = 0; i<n; i++)
      if (a[i] == x)
        return i;
  }
\end{verbatim}
\unskip{\MacroFont|\end{verbatim}|}
\begin{verbatim}
}
\end{verbatim}

Instead of |\frame[all:1]| you could also have specified
|\frame[1]|, but this works only for the presentation version of
the talk, not for the handout version. To make verbatim accessible
also in the handout version, you would have to specify
\verb!\frame[1| handout: 1]! and even more if you also have a
transparencies version. The specification |\frame[all:1]| states
that the frame has just one slide in all versions.

If you need to use verbatim commands in frames that contain several
slides, you must \emph{declare} your verbatim texts before the frame
starts. This is done using two special commands:


\begin{command}{\defverb\marg{command name}\opt{|*|}%
    \meta{delimiter symbol}\meta{verbatim text}\meta{delimiter symbol}}
  Declares a verbatim text for later use. The declaration should be
  done outside the frame. Once declared, the text can be used
  in overlays like normal text. The one-line \meta{verbatim text} must
  be delimited by a special \meta{delimiter symbol} (works like the
  |\verb| command). Adding a star makes spaces visible.

\example
\begin{verbatim}
\defverb\mytext!int main (void) { ...!
\defverb\mytextspaces*!int  main  (void ){  ...!

\frame
{
  \begin{itemize}
  \item<1-> In C you need a main function.
  \item<2-> It is declare like this: \mytext
  \item<3-> Spaces are not important: \mytextspaces
  \end{itemize}
}
\end{verbatim}
\end{command}


\begin{command}{\defverbatim\marg{command name}\marg{text}}
  The \meta{text} may contain a |verbatim|,  |verbatim*|,
  |lstlisting|, or a related environment. The command \marg{command
    name} can be used later inside frames. The declaration
  should be done outside the frame. Once declared, the text can be
  used in overlays like normal text.
  
  \example
\begin{verbatim}
\defverbatim\algorithm{
\begin{verbatim}
int main (void)
{
  cout << "Hello world." << endl;
  return 0;
}
\end{verbatim}
\unskip{\MacroFont|\end{verbatim}|}
\begin{verbatim}
}

\frame
{
  Our algorithm:
  \alert<1>{\algorithm}
  \uncover<2>{Note the return value.}
}
\end{verbatim}
\end{command}


\subsection{Overlays}

\subsubsection{The Pauses Environment}

The |pauses| environment offers an easy, but not very flexible
way of creating frames that are uncovered piecewise. The environment
itself does not have an immediate effect. But if you use the command
|\pause| inside the environment, only the text of the environment
up to the |\pause| command is shown on the first slide. On the
second slide, everything is shown up to the second |\pause|, and
so forth. Note that the |\pause| command can only be used on the 
same level of nesting as the |pauses| environment.

A much more fine-grained control over what is shown on each slide can
be attained using overlay specifications, see the next
subsections. However, for many simple cases the |\pause|
command is sufficient.

If you use multiple  |pauses| environments on one frame, the
slide counting for the second environment starts where the first one
left off, see the following example. You can nest |pauses|
environments, but this will not always have the effect you might
expect. 

\begin{verbatim}
\frame{
  \begin{pauses}
    Shown from first slide on.
    \pause
    Shown from second slide on.
    \pause
    Shown from third slide on.
  \end{pauses}

  Shown from first slide on (not affected by the environment).

  \begin{pauses}
    Shown from third slide on. (continued from above)
    \pause
    Shown from fourth slide on.
  \end{pauses}
}
\end{verbatim}

As a convenience, a |pauses| environment is automatically setup
inside each frame, each |itemize|, each |description|,
and each |enumerate|. Thus, by simply using the |\pause|
command on the outermost level of any frame or after items in lists or
descriptions, you uncover the rest of the frame or list only on the
next slide.

\begin{environment}{{pauses}\oarg{start slide number}}
  The content of the environment is shown piecewise. Each
  |\pause| command used inside uncovers a bit more of the
  environment's text. The main use of \meta{start slide
    number} is to set it to~0. The effect of this is that the first
  |\pause| has no effect, which can be useful if the |pauses|
  environment immediately starts with a |\pause| command. This happens 
  sometimes when the environment's content is created automatically.

\example
\begin{verbatim}
\frame
{
  \begin{pauses}
    Shown from slide 1 onward.
    \pause

    Shown from slide 2 onward.
  \end{pauses}
}
\end{verbatim}
\end{environment}
As mentioned above, in the above example the |pauses|
environment could also have been omitted, as the |\frame| command
inserts it automatically.

\begin{command}\pause
  When used inside a \texttt{pauses} environment, this command causes
  the text following it to be shown only from the next slide on.

  \example
\begin{verbatim}
\frame
{
  \begin{itemize}
  \item
    A    
    \pause
  \item
    B
    \pause
  \item
    C
  \end{itemize}
}
\end{verbatim}
\end{command}


\subsubsection{Commands with Overlay Specifications}
\label{subsection-overlay}

An overlay specification is a comma-separated list of slides and
ranges. Ranges are specified like this: |2-5|, which
means slide two through to five. The start or the beginning of a range
can be omitted. For example, |3-| means ``slides three, four,
five, and so on'' and |-5| means the same as |1-5|. A
complicated example is |-3,6-8,10,12-15|, which selected the
slides 1, 2, 3, 6, 7, 8, 10, 12, 13, 14, and 15.

Overlay specifications can be written behind certain commands. If such
an overlay specification is present, the command will only ``take
effect'' on the specified slides. What exactly ``take effect'' means
depends on the command. Consider the following example.

\begin{verbatim}
\frame
{
  \textbf{This line is bold on all three slides.}
  \textbf<2>{This line is bold only on the second slide.}
  \textbf<3>{This line is bold only on the third slide.}
}
\end{verbatim}

For the command |\textbf|, the overlay specification causes the
text to be set in boldface only on the specified slides. On all other
slides, the text is set in a normal font.

You cannot add an overlay specification to every command, but only to
those listed in the following. However, it is quite easy to redefine a
command such that it becomes ``overlay specification aware.''

For the following commands, adding an overlay specification causes the
command to be simply ignored on slides that are not included in the
specification: |\textbf|, |\textit|, |\textsl|,
|\textrm|, |\textsf|, |\color|, |\alert|,
|\structure|. If a command takes several arguments, like
|\color|, the specification directly follows the command as in
the following example.

\begin{verbatim}
\frame
{
  \color<2-3>[rgb]{1,0,0} This text is red on slides 2 and 3, otherwise black.
}
\end{verbatim}

For the following commands, the effect of an overlay specification is
special:

\begin{command}{\only\sarg{overlay specification}\marg{text}}
  If the \meta{overlay specification} is present, the \meta{text} is
  inserted only into the specified slides. For other slides, the text
  is simply thrown away. In particular, it occupies no space.
  \example |\only<3->{Text inserted from slide 3 on.}|
\end{command}

There exists a variant of |\only|, namely |\pgfonly|, that
should be used inside \pgf\ pictures instead of |\only|. The
command |\pgfonly| inserts appropriate |\ignorespaces|
commands that are needed by \pgf.

\begin{command}{\uncover\sarg{overlay specification}\marg{text}}
  If the \meta{overlay specification} is present, the \meta{text} is
  shown (``uncovered'') only on the specified slides. On other slides, the
  text still occupies space and it is still typeset, but it is not
  shown or only shown as if transparent. For details on how to specify
  whether the text is invisible or just transparent, see
  Section~\ref{section-transparent}. 
  \example |\uncover<3->{Text shown from slide 3 on.}|
\end{command}

\begin{command}{\invisible\sarg{overlay specification}\marg{text}}
  The \meta{text} occupies space and it is typeset, but it is not
  shown. If the \meta{overlay specification} is given, this command takes
  effect only on the specified slides. This command is a conter-part to
  |\uncover|, but not quite: unlike |\uncover|, invisible 
  text is never shown in a transparent way, but is guaranteed to
  really be invisible.
  \example |\invisible<-2>{Text shown from slide 3 on.}|
\end{command}

\begin{command}{\alt\ssarg{overlay specification}%
    \marg{default text}\marg{alternative text}}
  The default text is shown on the specified slides, otherwise the
  alternative text. The specification must always be present.
  \example |\alt<2>{On Slide 2}{Not on slide 2.}|
\end{command}


\begin{command}{\temporal\ssarg{overlay specification}%
    \marg{before slide text}\marg{default text}\marg{after slide text}}
  This command alternates between three different texts, depending on
  whether the current slide is temporally before the specified
  slides, is one of the specified slides, or comes after them. If the
  \meta{overlay specification} is not an interval (that is, if it has
  a ``hole''), the ``hole'' is considered to be part of the before slides.
  \example
\begin{verbatim}
  \temporal<3-4>{Shown on 1, 2}{Shown on 3, 4}{Shown 5, 6, 7, ...}
  \temporal<3,5>{Shown on 1, 2, 4}{Shown on 3, 5}{Shown 6, 7, 8, ...}
\end{verbatim}

  As a possible application of the |\temporal| command consider the
  following example: 
  \example
\begin{verbatim}
\def\colorize<#1>{%
  \temporal<#1>{\color{structure!50}}{\color{black}}{\color{black!50}}}

\frame{
  \begin{itemize}
    \colorize<1> \item First item.
    \colorize<2> \item Second item.
    \colorize<3> \item Third item.
    \colorize<4> \item Fourth item.
  \end{itemize}
}
\end{verbatim}
\end{command}


\begin{command}{\item\sarg{overlay specification}\oarg{item label}}
  Adding an \meta{overlay specification} to an item in a list causes
  this item to be uncovered only on the specified slides. This is
  useful for creating lists that are uncovered piecewise. Note that
  you are not required to stick to an order in which items are
  uncovered. If present, the optional \meta{item label} comes after
  the overlay specification. 
  
  \example
\begin{verbatim}
\frame
{
  \begin{itemize}
  \item<1-> First point, shown on all slides.
  \item<2-> Second point, shown on slide 2 and later.
  \item<2-> Third point, also shown on slide 2 and later.
  \item<3-> Fourth point, shown on slide 3.
  \end{itemize}
}

\frame
{
  \begin{enumerate}
  \item<3->[0.] A zeroth point, shown at the very end.
  \item<1-> The first an main point.
  \item<2-> The second point.
  \end{enumerate}
}
\end{verbatim}

  \example In the following example a list is uncovered item-wise. The
  last uncovered item is furthermore hilighted.  
\begin{verbatim}
\frame
{
  The advantages of the beamer class are
  \begin{enumerate}
  \item<1-> \alert<1>{It is easy to use.}
  \item<2-> \alert<2>{It is easy to extend.}
  \item<3-> \alert<3>{It works together with \texttt{pdflatex}.}
  \item<4-> \alert<4>{It has nice overlays.}
  \end{enumerate}
}
\end{verbatim}
\end{command}

The related command |\bibitem| is also overlay-specification-aware
in the same way as |\item|.

\begin{command}{\label\sarg{overlay specification}\marg{label name}}
  If the \meta{overlay specification} is present, the label is only
  inserted on the specified slide. Inserting a label on more than one
  slide will cause a `multiple labels' warning. \emph{However}, if no
  overlay specification is present, the specification is automatically
  set to just `1' and the label is thus inserted only on the first
  slide. This is typically the desired behaviour since it does not
  really matter on which slide the label is inserted, \emph{except} if
  you use an |\only| command. Then you need to specifiy a slide.
  \example
\begin{verbatim}
\frame
{
  \begin{align}
    a &= b + c   \label{first}\\ % no specification needed
    c &= d + e   \label{second}\\% no specification needed
  \end{align}

  Blah blah, \uncover<2>{more blah blah.}

  \only<3>{Specification is needed now.\label<3>{mylabel}}
}
\end{verbatim}
\end{command}

\subsubsection{Environments with Overlay Specifications}

Environments can also be equipped with overlay specifications. For
most of the predefined environments, see subsection~\ref{predefined},
adding an overlay specifications causes the whole environment to be
uncovered only on the specified slides. This is useful for showing
things incrementally as in the following example.

\begin{verbatim}
\frame
{
  \frametitle{A Theorem on Infinite Sets}

  \begin{theorem}<1->
    There exists an infinite set.
  \end{theorem}

  \begin{proof}<3->
    This follows from the axiom of infinity.
  \end{proof}

  \begin{example}<2->
    The set of natural numbers is infinite.
  \end{example}
}
\end{verbatim}
In the example, the first slide only contains the theorem, on the
second slide an example is added, and on the third slide the proof is
also shown.

The two special environments |onlyenv| and |uncoverenv| are
``environment versions'' of the commands |\only| and |\uncover|.


\begin{environment}{{onlyenv}\sarg{overlay specification}}
  If the \meta{overlay specification} is given, the contents of the
  environment is inserted into the text only on the specified slides. 
  \example
\begin{verbatim}
\frame
{
  This line is always shown.
  \begin{onlyenv}<2>
    This line is inserted on slide 2.
  \end{onlyenv}
}
\end{verbatim}
\end{environment}


\begin{environment}{{uncoverenv}\sarg{overlay specification}}
  If the \meta{overlay specification} is given, the contents of the
  environment is shown only on the specified slides. It still occupies
  space on the other slides.
  \example
\begin{verbatim}
\frame
{
  This word is 
  \begin{uncoverenv}<2>
    visible
  \end{uncoverenv}
  only on slide 2.
}
\end{verbatim}
\end{environment}


\subsubsection{Dynamically Changing Text}

You may sometimes wish to have some part of a frame change dynamically
from slide to slide. On each slide of the frame, something different
should be shown inside this area. You could achieve the effect of
dynamically changing text by giving a list of |\only| commands like this:
\begin{verbatim}
  \only<1>{Initial text.}
  \only<2>{Replaced by this on second slide.}
  \only<3>{Replaced again by this on third slide.}
\end{verbatim}
The trouble with this approach is that it may lead to slight, but
annoying differences in the heights of the lines, which may cause the
whole frame to ``whobble'' from slide to slide. This problem becomes
much more severe if the replacement text is several lines long.

To solve this problem, you can use two environments:
|overlayarea| and |overprint|. The first is more flexible,
but less user-friendly.

\begin{environment}{{overlayarea}\marg{area width}\marg{area height}}
  Everything within the environment will be placed in a rectangular
  area of the specified size. The area will have the same size on all
  slides of a frame, regardless of its actual contents. 
  \example
\begin{verbatim}
\begin{overlayarea}{\textwidth}{3cm}
  \only<1>{Some text for the first slide.\\Possibly several lines long.}
  \only<2>{Replacement on the second slide.}
\end{overlayarea}
\end{verbatim}
\end{environment}

\begin{environment}{{overprint}\oarg{area width}}
  The \meta{area width} defaults to the text width.
  Inside the environment, use |\onslide| commands to specify
  different things that should be shown for this environment on
  different slides. The |\onslide| commands are used like
  |\item| commands. Everything within the environment will be
  placed in a rectangular area of the specified width. The height and
  depth of the area are chosen large enough to accommodate the largest
  contents of the area. The overlay specifications of the
  |\onslide| commands must be disjoint. This may be a problem for
  handouts, since, there, all overlay specifications defaul to |1|. If
  you use the option |handout|, you can disable all but one
  |\onslide| by setting the others to |0|.
  \example
\begin{verbatim}
\begin{overprint}
  \onslide<1| handout:1>
    Some text for the first slide.\\
    Possibly several lines long.
  \onslide<2| handout:0>
    Replacement on the second slide. Supressed for handout.
\end{overprint}
\end{verbatim}
\end{environment}



\subsection{Making Commands and Environments Overlay-Specification-Aware}

This subsection explains how you can make your own commands
overlay-specification-aware. Also, it explains how to setup counters
correctly that should be increased from frame to frame (like equation
numbering), but not from slide to slide. You may wish to skip this
section, unless you  want to write your own extensions to the \beamer\
class. 
 
You can define a new command that is overlay-specification-aware using
the following command.

\begin{command}{\newoverlaycommand\marg{command name}%
    \marg{default text}\marg{alternative text}}
  Declares the new command named \meta{command name}. If this command is
  encountered by \TeX, it is checked whether an overlay specification
  follows. If not, the \meta{default text} is inserted. If there is a
  specification, the \meta{default text} is also inserted if the current slide
  is specified, otherwise the  \meta{alternative text} is inserted.
  \example
\begin{verbatim}
\newoverlaycommand{\SelectRedAsColor}{\color[rgb]{1,0,0}}{}

\frame
{
  \SelectRedAsColor<2>
  The second slide of this frame is all in red. 
}
\end{verbatim}
\end{command}

\begin{command}{\renewoverlaycommand\marg{existing command name}%
    \marg{default text}\marg{alternative text}}
  Redeclares a command that already exists in the same way as
  |\newoverlaycommand|. Inside the parameters, you can 
  still access to original definitions using the command
  |\original|, see the example.
  \example
\begin{verbatim}
\renewoverlaycommand{\tiny}{\original{\tiny}}{}

\frame
{
  \tiny<2>This text is tiny on slide 2.
}
\end{verbatim}
\end{command}


\begin{command}{\newoverlayenvironment\marg{environment name}%
    \oarg{parameter number}%
    \marg{default begin}\marg{default end}\\
    \marg{alternative begin}\marg{alternative end}}
  Declares a new environment that is overlay specification aware. If
  this environment encountered, it is 
  checked whether an overlay specification follows. If not or if it is
  found and the current slide is specified, the default begin and end
  are used. Otherwise, the alternative begin and end are used.

  If the \meta{parameter number} is specified, it must currently
  be~1. In this case, the begin commands must take one parameter. This
  parameter will \emph{preceed} the overlay specification, see the
  examples. 
  \example
\begin{verbatim}
\newoverlayenvironment{mytheorem}{\alert{Theorem}:}{}{Theorem:}{}

\frame
{
  \begin{mytheorem}<2>
    This theorem is hilighted on the second slide.
  \end{mytheorem}
}
\end{verbatim}

\begin{verbatim}
\newoverlayenvironment{mytheorem}[1]{\alert{Theorem #1}:}{}{Theorem #1:}{}

\frame
{
  \begin{mytheorem}{of Tantau}<2>
    This theorem is hilighted on the second slide.
  \end{mytheorem}
}
\end{verbatim}
\end{command}

The following two commands can be used to ensure that a certain
counter is automatically reset on subsequent slides of a frame. This
is necessary for example for the equation count. You might want this
count to be increased from frame to frame, but certainly not from
overlay slide to overlay slide. For equation counters and footnote
counters (you should not use footnotes), these commands have already
been invoked.

\begin{command}{\resetcounteronoverlays\marg{counter name}}
  After you have invoked this command, the value of the specified
  counter will be the same on all slides of every frame. 
  \example |\resetcounteronoverlays{equation}|
\end{command}
 
\begin{command}{\resetcountonoverlays\marg{count register name}}
  The same as |\resetcounteronoverlays|, except that this
  command should be used with counts that have been created using the
  \TeX\ primitive |\newcount| instead of \LaTeX's  |\definecounter|. 
  \example
\begin{verbatim}
\newcount\mycount
\resetcountonoverlays{mycount}
\end{verbatim}
\end{command}






\section{Structuring a Presentation}

\subsection{Global Structure of Presentations}

Ideally, during most presentations you would like to present your
slides in a perfectly linear fashion, presumably by pressing the
page-down-key once for each slide. However, there are different
reasons why you might have to deviate from this linear order:
\begin{itemize}
\item
  Your presentation may contain ``different levels of detail'' that
  may or may not be skipped or expanded, depending on the audience's
  reaction.
\item
  You are asked questions and wish to show supplementary slides.
\item
  You are asked questions about an earlier slide, which forces you to 
  find and then jump to that slide.
\end{itemize}
You cannot really prepare against the last kind of questions. In this
case, you can use the navigation bars and symbols to find the slide
you are interested in, see \ref{section-navigation-bars}.

Concerning the first two kinds of deviations, the \beamer\ class
offers several ways of preparing such ``planned detours'' or ``planned
short cuts''.
\begin{itemize}
\item
  You can easily add predefined ``skip buttons.'' When such a button
  is pressed, you jump over a well-defined part of your talk. Skip
  button have two advantages over just pressing the forward key
  is rapid succession: first, you immediately end up at the correct
  position and, second, the button's label can give the audience a
  visual feedback of what exactly will be skipped. For example, when
  you press a skip button labeled ``Skip proof'' nobody will start
  puzzling over what he or she has missed.
\item
  You can add an appendix to your talk. The appendix is kept
  ``perfectly separated'' from the main talk. Only once you ``enter''
  the appendix part (presumably by hyperjumping into it), does the
  appendix structure become visible. You can put all frames that you
  do not intend to show during the normal course of your talk, but
  which you would like to have handy in case someone asks, into this
  appendix.
\item
  You can add ``goto buttons'' and ``return buttons'' to create
  detours. Pressing a goto button will jump to a certain part of the
  presentation where extra details can be shown. In this part, there
  is a return button present on each slide that will jump back to the
  place where the goto button was pressed.
\end{itemize}


\subsection{Commands for Creating the Global Structure}

\subsubsection{Adding a Title Page}

You can use the |\titlepage| command to insert a title page into
a frame. 

The |\titlepage| command will arrange the following elements on
the title page: the document title, the author(s)'s names, their
affiliation, a title graphic, and a date.

\begin{command}{\titlepage}
  Inserts the text of a title page into the current frame.
  \example |\frame{\titlepage}|
\end{command}

Before you invoke the title page command, you must specify all
elements you wish to be shown. This is done using the following
commands: 

\begin{command}{\title\oarg{short title}\marg{title}}
  The \meta{short tile} is used in head lines and foot lines. Inside
  the \meta{title} line breaks can be inserted using the
  double-backslash command.
  \example
\begin{verbatim}
\title{The Beamer Class}
\title[Short Version]{A Very Long Title\\Over Several Lines}
\end{verbatim}
\end{command}

\begin{command}{\author\oarg{short author names}\marg{author names}}
  The names should be separated using the
  command |\and|. In case authors have different affiliations,
  they should be suffixed by the command |\inst| with different
  parameters.
  \example|\author[Hemaspaandra et al.]{L. Hemaspaandra\inst{1} \and T. Tantau\inst{2}}|
\end{command}

\begin{command}{\institute\oarg{short institute}\marg{institute}}
  If more than one institute is given, they should be separated using
  the command |\and| and they should be prefixed by the command
  |\inst| with different parameters.
  \example
\begin{verbatim}
\institute[Universities of Rochester and Berlin]{
  \inst{1}Department of Computer Science\\
  University of Rochester
  \and
  \inst{2}Fakult\"at f\"ur Elektrotechnik und Informatik\\
  Technical University of Berlin}
\end{verbatim}
\end{command}

\begin{command}{\date\oarg{short date}\marg{date}}
  \example|\date{\today}| or |\date[STACS 2003]{STACS Conference, 2003}|.
\end{command}


\begin{command}{\titlegraphic\marg{text}}
  The \meta{text} is shown as title graphic. Typically, a picture
  environment is used as \meta{text}.
  \example|\titlegraphic{\pgfuseimage{titlegraphic}}|
\end{command}




\subsubsection{Adding Sections and Subsections}

You can structure your text using the commands |\section| and
|\subsection|. Unlike standard \LaTeX, these commands will not
create a heading at the position where you use them. Rather, they will
add an entry to the table of contents and also to the navigation
bars.

In order to create a line break in the table of contents (usually not
a good idea), you can use the command |\breakhere|. Note that the
standard command |\\| does not work.

\begin{command}{\section\oarg{short section name}\marg{section name}}
  Starts a section. No heading is created, but the \meta{section name}
  is only shown in the table of contents and the \meta{short section name}
  is only shown in the navigation bar. If the \meta{section name} is
  empty, a navigation entry is created, but no entry in the table of
  contents. This is useful for sections like a ``table of contents
  section.''
  \example|\section[Summary]{Summary of Main Results}| or
  |\section[Outline]{}| 
\end{command}


\begin{command}{\subsection\oarg{short subsection name}\marg{subsection name}}
  This command works the same way as the |\section| command.
  \example|\subsection{Some Subsection}|
\end{command}




\subsubsection{Adding Parts}

If you give a long talk (like a lecture), you may wish to break up
your talk into several parts. Each such part acts like a little ``talk
of its own'' with its own table of contents, its own navigation bars,
and so on. Inside one part, the sections and subsections of the other
parts are not shown at all.

To create a new part, use the |\part| command. All sections and
subsections following this command will be ``local'' to that part.
Like the |\section| and |\subsection| command, the |\part| command
does not cause any frame or special text to be produced. However,
it is often advisable for the start of a new part to use the command
|\partpage| to insert some text into a frame that ``advertises'' the
beginning of a new part. See |beamerexample3.tex| for an example.

\begin{command}{\part\oarg{short part name}\marg{part name}}
  Starts a part. The \meta{part name} will be shown when the
  |\partpage| command is used. The \meta{shown part name} is not shown
  anywhere by default, but it is accessible via the command
  |\insertshortpart|.
  \example
\begin{verbatim}
\begin{document}
  \frame{\titlepage}

  \section[Outlines]{}
  \subsection{Part I: Review of Previous Lecture}
  \frame{
    \frametitle{Outline of Part I}
    \tableofcontents[part=1]}
  \subsection{Part II: Today's Lecture}
  \frame{
    \frametitle{Outline of Part II}
    \tableofcontents[part=2]}

  \part{Review of Previous Lecture}
  \frame{\partpage}
  \section[Previous Lecture]{Summary of the Previous Lecture}
  \subsection{Topics}
  \frame{...}
  \subsection{Learning Objectives}
  \frame{...}
  
  \part{Today's Lecture}
  \frame{\partpage}
  \section{Topic A}
  \frame{\tableofcontents[current]}
  \subsection{Foo}
  \frame{...}
  \section{Topic B}
  \frame{\tableofcontents[current]}
  \subsection{bar}
  \frame{...}
\end{document}
\end{verbatim}
\end{command}

\begin{command}{\partpage}
  Works like |\titlepage|, only that the current part, not the current
  presentation is ``advertised.'' The appearance can be changed by
  adjusting the part page template, see
  Section~\ref{section-part-page-template}. 
  \example |\frame{\partpage}|
\end{command}



\subsubsection{Adding a Table of Contents}

You can create a table of contents using the command
|\tableofcontents|. Unlike the normal \LaTeX\ table of contents
command, this command takes an optional parameter in square brackets
that can be used to create certain special effects.

\begin{command}{\tableofcontents\oarg{comma-separated option list}}
  Inserts a table of contents into the current frame. To change how
  the table of contents is typeset, you need to modify the appropriate
  templates, see Section~\ref{section-toc-templates}. 
  \example
\begin{verbatim}
\section[Outline]{}
\frame{\tableofcontents}

\section{Introduction}
\frame{\tableofcontents[current]}
\subsection{Why?}
\frame{...}
\frame{...}
\subsection{Where?}
\frame{...}

\section{Results}
\frame{\tableofcontents[current]}
\subsection{Because}
\frame{...}
\subsection{Here}
\frame{...}
\end{verbatim}

  The following options can be given:
  \begin{itemize}
  \item
    \declare{|part=|\meta{part number}} causes the table of contents
    of part \meta{part number} to be shown, instead of the table of
    contents of the current part (which is the default). This option
    can be combined with the other options, although combining it with
    the |current| option obviously makes no sense.
  \item
    \declare{|current|} causes all but the current section to be
    shown in a semi-transparent way.
  \item
    \declare{|pausesections|} causes a |\pause| command to
    be issued before each section. This is useful if you wish to show
    the table of contents in an incremental way.
  \item
    \declare{|pausesubsections|} causes a |\pause| command to
    be issued before each subsection.
  \item
    \declare{|hidesubsections|} causes the subsections to be
    omitted. However, if used together with the |current| option,
    the subsections of the current section are not omitted.
  \item
    \declare{|shadesubsections|} causes the subsections to
    be shown in a semi-transparent way.
  \end{itemize}
  The last two commands are useful if you do not wish to show too many
  details when presenting the talk outline.
\end{command}




\subsubsection{Adding a Bibliography}

You can use the bibliography environment and the |\cite| commands
of \LaTeX\ in a \beamer\ presentation. However, there are a few things
to keep in mind:

\begin{itemize}
\item
  It is a bad idea to present a long bibliography in a 
  beamer presentation. Present only very few references.
\item
  Present references only if they are intended as ``further reading,''
  for example at the end of a lecture.
\item
  Using the |\cite| commands can be confusing since the audience
  has little chance of remembering the citations. If you cite the
  references, always cite them with full author name and year like
  ``[Tantau, 2003]'' instead of something like ``[2,4]'' or
  ``[Tan01,NT02]''.
\end{itemize}

Keeping the above warnings in mind, proceed as follows to create the
bibliography: 

For a beamer presentation, you will typically have to typeset your
bibliography items partly ``by hand.'' Nevertheless, you \emph{can}
use |bibtex| to create a ``first approximation'' of the
bibliography. Copy the content of the file |main.bbl| into your
presentation. If you are not familiar with |bibtex|, you may wish
to consult its documentation. It is a  powerful tool for
creating high-quality citations.

Using |bibtex| or your editor, place your bibliographic
references in the environment |thebibliography|. This
(standard \LaTeX) environment takes one parameter, which should be the
longest |\bibitem| label in the following list of bibliographic
entries.

\begin{environment}{{thebibliography}\marg{longest label text}}
  Inserts a bibliography into the current frame. The \meta{longest
    label text} is used to determine the indent of the list. However,
  several templates for the typesetting of the bibliography (see
  Section~\ref{section-bib-templates}) ignore this 
  parameter since they replace the references by a symbol.

  Inside the environment, use a (standard \LaTeX) |\bibitem| command
  for each reference item. Inside each item, use a (standard \LaTeX)
  |\newblock| command to separate the authors's names, the title, the
  book/journal reference, and any notes. Each of these commands may
  introduce a new line or color or other formatting, as specified by
  the template for bibliographies.

  The environment must be placed inside a frame. If the bibliography
  does not fit on one frame, you should 
  split it (create a new frame and a second |thebibliography|
  environment). Even better, you should reconsider whether it is a good
  idea to present so many references.
  \example
\begin{verbatim}
\frame{
  \frametitle{For Further Reading}

  \begin{thebibliography}{Dijkstra, 1982}
  \bibitem[Solomaa, 1973]{Solomaa1973}
    A.~Salomaa.
    \newblock {\em Formal Languages}.
    \newblock Academic Press, 1973.

  \bibitem[Dijkstra, 1982]{Dijkstra1982}
    E.~Dijkstra.
    \newblock Smoothsort, an alternative for sorting in situ.
    \newblock {\em Science of Computer Programming}, 1(3):223--233, 1982.
  \end{thebibliography}
 }
\end{verbatim}
\end{environment}

\begin{command}{\bibitem\sarg{overlay specification}%
    \oarg{citation text}\marg{label name}}
  The \meta{citation text} is inserted into the text when the item is
  cited using |\cite{|\meta{label name}|}| in the main presentation
  text. For a \beamer\ presentation, this should usually be as long as
  possible.  

  Use |\newblock| commands to separate the authors's names, the title, the
  book/journal reference, and any notes. If the \meta{overlay specification}
  is present, the entry will only be shown on the
  specified slides.
  \example
\begin{verbatim}
\bibitem[Dijkstra, 1982]{Dijkstra1982}
  E.~Dijkstra.
  \newblock Smoothsort, an alternative for sorting in situ.
  \newblock {\em Science of Computer Programming}, 1(3):223--233, 1982.
\end{verbatim}
\end{command}

Unlike normal \LaTeX, the default template for the
bibliography does not repeat the citation text (like ``[Dijkstra,
1982]'') before each item in the bibliography. Instead, a cute, small
article symbol is drawn. The rationale is that the audience will not be
able to remember any abbreviated citation texts till the end of the
talk. If you really insist on using abbreviations, you can use the
command |\beamertemplatetextbibitems| to restore the default
bevahior, see also Section~\ref{section-bib-templates}.




\subsubsection{Adding an Appendix}

You can add an appendix to your talk by using the |\appendix|
command. You should put frames and perhaps whole subsections into the
appendix that you do not intend to show during your presentation, but
which might be useful to answer a question. The |\appendix| command
essentially just start a new part named |\appendixname|. However, it
also sets up certain hyperlinks. 
Like other parts, the appendix is kept separate of your actual
talk.

\begin{command}{\appendix}
  Starts the appendix. All frames, all |\subsection| commands, and all
   |\section| commands used after this command will not be shown as
   part of the normal navigation bars.
  \example
\begin{verbatim}
\begin{document}
\frame{\titlepage}
\section[Outline]{}
\frame{\tableofcontents}
\section{Main Text}
\frame{Some text}
\section[Summary]{}
\frame{Summary text}

\appendix
\section{\appendixname}
\frame{\tableofcontents}
\subsection{Additional material}
\frame{Details}
\frame{Text omitted in main talk.}
\subsection{Even more additional material}
\frame{More details}
\end{document}
\end{verbatim}
\end{command}





\subsubsection{Adding Hyperlinks and Buttons}

To create an anticipated nonlinear jumps in your talk structure, you
can add hyperlinks to your presentation. A hyperlink is a text
(usually rendered as a button) that, when you click on it, jumps the
presentation to some other slide. Creating such a button is a
three-step process: 
\begin{enumerate}
\item
  You specify a target using the command |\hypertarget|. In some
  cases, see below, this step may be skipped.
\item
  You render the button using |\beamerbutton| or a similar
  command. This will \emph{render} the button, but clicking it will
  not yet have any effect. 
\item
  You put the button inside a |\hyperlink| command. Now clicking it
  will jump to the target of the link.  
\end{enumerate}

\begin{command}{\hypertarget\sarg{overlay specification}%
    \marg{target name}\marg{text}}
  If the \meta{overlay specification} is present, the \meta{text} is
  the target for hyper jumps to \meta{target name} only on the
  specified slide. On all other slides, the text is shown
  normally. Note that you \emph{must} add an overlay specification to
  the |\hypertarget| command whenever you use it on frames that have
  multiple slides (otherwise |pdflatex| rightfully complains
  that you have defined the same target on different slides).
  \example
\begin{verbatim}
\frame{
  \begin{itemize}
  \item<1-> First item.
  \item<2-> Second item.
  \item<3-> Third item.
  \end{itemize}

  \hyperlink{jumptosecond}{\beamergotobutton{Jump to second slide}}
  \hypertarget<2>{jumptosecond}{}
}
\end{verbatim}
\end{command}

The following commands can be used to specify in an abstract way what
a button will be used for. How exactly these buttons are rendered is
governed by a template, see Section~\ref{section-navigation-buttons}.

\begin{command}{\beamerbutton\marg{button text}}
  Draws a button with the given \meta{button text}.
  \example |\hyperlink{somewhere}{\beamerbutton{Go somewhere}}|
\end{command}

\begin{command}{\beamergotobutton\marg{button text}}
  Draws a button with the given \meta{button text}. Before the text, a
  small symbol (usually a right-pointing arrow) is inserted that
  indicates that pressing this button will jump to another ``area'' of
  the presentation.

  \example |\hyperlink{detour}{\beamergotobutton{Go to detour}}|
\end{command}

\begin{command}{\beamerskipbutton\marg{button text}}
  The symbol drawn for this button is usually a double right
  arrow. Use this button if pressing it will skip over a
  well-defined part of your talk.

  \example
\begin{verbatim}
\frame{
  \begin{theorem}
    ...
  \end{theorem}

  \begin{overprint}
  \onslide<1>
    \hfill\hyperlinkframestartnext{\beamerskipbutton{Skip proof}}
  \onslide<2>
    \begin{proof}
      ...
    \end{proof}
  \end{overprint}
}
\end{verbatim}
\end{command}

\begin{command}{\beamerreturnbutton\marg{button text}}
  The symbol drawn for this button is usually a left pointing
  arrow. Use this button if pressing it will return from a detour. 

  \example
\begin{verbatim}
\frame{
  \begin{theorem}
    ...
  \end{theorem}

  \hyperlink{detour}{\beamergotobutton{Go to proof details}}
  \hypertarget{returnhere}{}
}
\appendix
\frame{
  \hypertarget{detour}{}
  \begin{proof}
    ...
  \end{proof}
  \hfill\hyperlink{returnhere}{\beamerreturnbutton{Return}}
}  
\end{verbatim}
\end{command}

To make a button ``clickable'' you must place it in a command like
|\hyperlink|. The command |\hyperlink| is a standard command of the
|hyperref| package. The \beamer\ class defines a whole bunch of other
hyperlink commands that you can also use.

\begin{command}{\hyperlink\marg{target name}\marg{link text}}
  The \meta{link text} is typeset in the usual way. If you click
  anywhere on this text, you will jump to the slide on which the
  |\hypertarget| command was used with the parameter \meta{target
    name}. 
\end{command}

The following commands have a predefined target; otherwise they behave
exactly like |\hyperlink|.

\begin{command}{\hyperlinkslideprev\marg{link text}}
  Clicking the text jumps one slide back.
\end{command}

\begin{command}{\hyperlinkslidenext\marg{link text}}
  Clicking the text jumps one slide forward.
\end{command}
  
\begin{command}{\hyperlinkframestart\marg{link text}}
  Clicking the text jumps to the first slide of the current frame.
\end{command}

\begin{command}{\hyperlinkframeend\marg{link text}}
  Clicking the text jumps to the last slide of the current frame.
\end{command}

\begin{command}{\hyperlinkframestartnext\marg{link text}}
  Clicking the text jumps to the first slide of the next frame.
\end{command}

\begin{command}{\hyperlinkframeendprev\marg{link text}}
  Clicking the text jumps to the last slide of the previous frame.
\end{command}

The previous four command exist also with ``|frame|'' replaced by
``|subsection|'' everywhere, and also again with  ``|frame|'' replaced
by ``|section|''.

\begin{command}{\hyperlinkpresentationstart\marg{link text}}
  Clicking the text jumps to the first slide of the presentation.
\end{command}

\begin{command}{\hyperlinkpresentationend\marg{link text}}
  Clicking the text jumps to the last slide of the presentation. This
  \emph{excludes} the appendix.
\end{command}

\begin{command}{\hyperlinkappendixstart\marg{link text}}
  Clicking the text jumps to the first slide of the appendix. If there
  is no appendix, this will jump to the last slide of the document.
\end{command}

\begin{command}{\hyperlinkappendixend\marg{link text}}
  Clicking the text jumps to the last slide of the appendix.
\end{command}

\begin{command}{\hyperlinkdocumentstart\marg{link text}}
  Clicking the text jumps to the first slide of the presentation.
\end{command}

\begin{command}{\hyperlinkdocumentend\marg{link text}}
  Clicking the text jumps to the last slide of the presentation or, if
  an appendix is present, to the last slide of the appendix.
\end{command}





\subsection{Navigation Bars and Symbols}
\label{section-navigation-bars}

Navigation bars and symbols are two independent concepts that can be
used to navigate through a presentation. They are created
automatically.


\subsubsection{Using the Navigation Bars}

Most themes that come along with the \beamer\ class show some kind of
navigation bar during your talk. Although these navigation bars take
up quite a bit of space, they are often useful for two reasons:

\begin{itemize}
\item
  They provide the audience with a visual feedback of how much of your
  talk you have covered and what is yet to come. Without such
  feedback, an audience will often puzzle whether something you are
  currently introducing will be explained in more detail later on or
  not.
\item
  You can click on all parts of the navigation bar. This will directly
  ``jump'' you to the part you have clicked on. This is particularly
  useful to skip certain parts of your talk and during a ``question
  session,'' when you wish to jump back to a particular frame someone
  has asked about.
\end{itemize}

Some navigation bars can be ``compressed'' using the following option:

\begin{classoption}{compress}
  Tries to make all navigation bars as small as possible. For example,
  all small frame representations in the navigation bars for a single
  section are shown alongside each other. Normally, the representations
  for different subsections are shown in different lines. Furthermore,
  section and subsection navigations are compressed into one line.
\end{classoption}

When you click on one of the icons representing a frame in a
navigation bar (by default this is icon is a small circle), the
following happens: 
\begin{itemize}
\item
  If you click on (the icon of) any frame other than the current frame, the
  presentation will jump to the first slide of the frame you clicked
  on.
\item
  If you click on the current frame and you are not on the last slide
  of this frame, you will jump to the last slide of the frame.
\item
  If you click on the current frame and you are on the last slide, you
  will jump to the first slide of the frame.
\end{itemize}
By the above rules you can:
\begin{itemize}
\item
  Jump to the beginning of a frame from somewhere else by clicking on
  it once.  
\item
  Jump to the end of a frame from somewhere else by clicking on it
  twice.
\item
  Skip the rest of the current frame by clicking on it once.
\end{itemize}

I also tried making a jump to an already-visited frame jump
automatically to the last slide of this frame. However, this turned
out to be more confusing than helpful. With the current implementation
a double-click always brings you to the end of a slide, regardless
from where you ``come.''

By clicking on a section or subsection in the navigation bar, you will
jump to that section. Clicking on a section is particularly useful if
the section starts with a |\tableofcontents[current]|, since you
can use it to jump to the different subsections.

By clicking on the document title in a navigation bar (not all themes
show it), you will jump to the first slide of your presentation
(usually the title page) \emph{except} if you are already at the first
slide. On the first slide, clicking on the document title will jump to
the end of the presentation, if there is one. Thus by \emph{double}
clicking the document title in a navigation bar, you can jump to the end.



\subsubsection{Using the Navigation Symbols}
\label{section-navigation-symbols}

Navigation symbols are small icons that are shown on every slide
by default. The following symbols are shown: 
\begin{enumerate}
\item
  A slide icon, which is depicted as  a single rectangle. To the left and
  right of this symbol, a left and right arrow are shown.
\item
  A frame icon, which is depicted as three slide icons ``stacked on top of
  each other''. This symbols is framed by arrows.
\item
  A subsection icon, which is depicted as a highlighted subsection
  entry in a table of contents. This  symbols is framed by arrows.
\item
  A section icon, which is depicted as a highlighted section entry
  (together with all subsections) in a table of contents. This symbols
  is framed by arrows.
\item
  A presentation icon, which is depicted as a completely highlighted
  table of contents.
\item
  An appendix icon, which is depicted as a completely highlighted
  table of contents consisting of only one section. (This icon is only
  shown if there is an appendix.
\item
  Back and forward icons, depicted as circular arrows.
\item
  A ``search'' or ``find'' icon, depicted as a detective's
  magnifying glass.
\end{enumerate}

Clicking on the left arrow next to an icon always jumps to (the
last slide of) the previous slide, frame, subsection, or
section. Clicking on the right arrow next to an icon always jump to
(the first slide of) the next slide, frame, subsection, or section. 

Clicking \emph{on} any of these icons has different effects:
\begin{enumerate}
\item
  If supported by the viewer application, clicking on a slide icon
  pops up a window that allows you to enter a slide number to which
  you wish to jump.
\item
  Clicking on the left side of a frame icon will jump to the first
  slide of the frame, clicking on the right side will jump to the last
  slide of the frame (this can be useful for skipping overlays).
\item
  Clicking on the left side of a subsection icon will jump to the
  first slide of the subsection, clicking on the right side will jump
  to the last slide of the subsection.
\item
  Clicking on the left side of a section icon will jump to the
  first slide of the section, clicking on the right side will jump
  to the last slide of the section.
\item
  Clicking on the left side of the presentation icon will jump to the
  first slide, clicking on the right side will jump to the last slide
  of the presentation. However, this does \emph{not} include the
  appendix. 
\item
  Clicking on the left side of the appendix icon will jump to the
  first slide of the appendix, clicking on the right side will jump to
  the last slide of the appendix.
\item
  If supported by the viewer application, clicking on the back and
  forward symbols jumps to the previously visited slides.
\item
  If supported by the viewer application, clicking on the search icon
  pops up a window that allows you to enter a search string. If found,
  the viewer application will jump to this string.
\end{enumerate}

You can reduce the number of icons that are shown or their layout by
adjusting the navigation symbols template, see
Section~\ref{section-navigation-symbols-template}. 







\subsection{Command for Creating the Local Structure}

Just like your whole presentation, each frame should also be
structured. A frame that is solely filled with some long text is very
hard to follow. It is your job to structure the contents of each frame
such that, ideally, the audience immediately seems which information
is important, which information is just a detail, how the presented
information is related, and so on.

\LaTeX\ provides different commands for structuring text ``locally,''
for example, via the |itemize| environment. These environments
are also available in the beamer class, although their appearance has
been slightly changed. Furthermore, the \beamer\ class also defines
some new commands and environments, see below, that may help you to
structure your text.


\subsubsection{Itemizations, Enumerations, and Descriptions}

There are three predefined environments for creating lists, namely
|enumerate|, |itemize|, and |description|. The first
two can be nested to depth two, but not further (this would
create totally unreadable slides).

The |\item| command is overlay-specification-aware. If an overlay
specification is provided, the item will only be shown on the
specified slides, see the following example. If the |\item|
command is to take an optional argument and an overlay specification,
the overlay specification comes first as in |\item<1>[Cat]|.

\begin{verbatim}
\frame
{
  There are three important points:
  \begin{enumerate}
  \item<1-> A first one,
  \item<2-> a second one with a bunch of subpoints,
    \begin{itemize}
    \item first subpoint. (Only shown from second slide on!).
    \item<3-> second subpoint added on third slide.
    \item<4-> third subpoint added on fourth slide.
    \end{itemize}
  \item<5-> and a third one.
  \end{enumerate}
}
\end{verbatim}


\begin{environment}{{itemize}}
  Used to display a list of items that do not have a special
  ordering. Inside the environment, use an |\item| command for
  each topic. The appearence of the items can be changed using
  templates, see Section~\ref{section-templates}.
  \example
\begin{verbatim}
\begin{itemize}
\item This is important.
\item This is also important.
\end{itemize}
\end{verbatim}
\end{environment}


\begin{environment}{{enumerate}}
  Like |itemize|, except that the list is ordered.
  \example
\begin{verbatim}
\begin{enumerate}
\item This is important.
\item This is also important.
\end{enumerate}
\end{verbatim}
\end{environment}


\begin{environment}{{description}\oarg{long text}}
  Like |itemize|, but used to display an list that explains or defines
  labels. The width of \meta{long text} is used to set the indent. Normally,
  you choose the widest label in the description and copy it here.
  \example
\begin{verbatim}
\begin{description}
\item[Lion] King of the savanna.
\item[Tiger] King of the jungle.
\end{description}

\begin{description}[longest label]
\item<1->[short] Some text.
\item<2->[longest label] Some text.
\item<3->[long label] Some text.
\end{description}
\end{verbatim}
\end{environment}



\subsubsection{Block Environments and Simple Structure Commands}
\label{predefined}

The \beamer\ class predefines a number of useful environments and
commands. Using these commands makes is easy to change the appearance
of a document by changing the theme.


\begin{command}{\alert\sarg{overlay specification}\marg{hilighted text}}
  The given text is hilighted, typically be coloring the text red. If
  the \meta{overlay specification} is present, the command only has an
  effect on the specified slides.
  \example |This is \alert{important}.|
\end{command}

\begin{command}{\structure\sarg{overlay specification}\marg{text}}
  The given text is marked as part of the structure, typically by
  coloring it in the |structure| color. If the \meta{overlay
    specification} is present, the command only has an effect on the
  specified slides.
  \example|\structure{Paragraph Heading.}|
\end{command}

\begin{environment}{{block}\marg{block title}\sarg{overlay specification}}
  Inserts a block, like a definition or a theorem, with the title
  \meta{block title}. If the \meta{overlay specification} is present,
  the block is shown only on the specified slides. In the example, the
  definition is shown only from slide 3 onward.
  \example
\begin{verbatim}
  \begin{block}{Definition}<3->
    A \alert{set} consists of elements.
  \end{block}
\end{verbatim}
\end{environment}


\begin{environment}{{alertblock}\marg{block title}\sarg{overlay specification}}
  Inserts a block whose title is hilighted.  If the \meta{overlay specification} is present,
  the block is shown only on the specified slides.
  \example
\begin{verbatim}
  \begin{alertblock}{Wrong Theorem}
    $1=2$.
  \end{alertblock}
\end{verbatim}
\end{environment}

\begin{environment}{{exampleblock}\marg{block title}\sarg{overlay specification}}
  Inserts a block that is supposed to be an example. If the \meta{overlay specification} is present,
  the block is shown only on the specified slides.
  \example
\begin{verbatim}
  \begin{exampleblock}{Example}
    The set $\{1,2,3,5\}$ has four elements.
  \end{exampleblock}
\end{verbatim}
\end{environment}
  
Predefined English block environments, that is, block environments
with fixed title, are: |Theorem|, |Proof|, |Corollary|,
|Fact|, |Example|, and |Examples|. You can also use these
environments with a lowercase first letter, the result  is the
same. The following German block environments are also predefined:
|Problem|, |Loesung|, |Definition|, |Satz|,
|Beweis|, |Folgerung|, |Lemma|, |Fakt|,
|Beispiel|, and |Beispiele|. See the following example for
their usage.

\begin{verbatim}
\frame
{
  \frametitle{A Theorem on Infinite Sets}

  \begin{theorem}<1->
    There exists an infinite set.
  \end{theorem}

  \begin{proof}<2->
    This follows from the axiom of infinity.
  \end{proof}

  \begin{example}<3->
    The set of natural numbers is infinite.
  \end{example}
}
\end{verbatim}



\subsubsection{Figures and Tables}

You can use the standard \LaTeX\ environments |figure| and
|table| much the same way you would normally use them. However,
any placement specification will be ignored. Figures and tables are
immediately inserted where the environments start. If there are too
many of them to fit on the frame, you must manually split them among
additional frames.

\example
\begin{verbatim}
\frame{
  \begin{figure}
    \pgfuseimage{myfigure}
    \caption{This caption is placed below the figure.}
  \end{figure}

  \begin{figure}
    \caption{This caption is placed above the figure.}
    \pgfuseimage{myotherfigure}
  \end{figure}
}
\end{verbatim}

You can adjust how the figure and table captions are typeset by
changing the corresponding template, see
Section~\ref{section-template-caption}.





\subsubsection{Splitting a Frame into Multiple Columns}

Three environments are used to create columns on a slide. Columns are
especially useful for placing a graphic next to a description/explanation.
The main environment for creating columns is called
|columns|. Inside this environment, you can place several
|column| environments. Each will create a new column.

\begin{environment}{{columns}}
  A multi-column area. Inside the environment you should place only
  |column| environments.
  \example
\begin{verbatim}
\begin{columns}
  \begin{column}{5cm}
    First column.
  \end{column}
  \begin{column}{5cm}
    Second column.
  \end{column}
\end{columns}
\end{verbatim}
\end{environment}

\begin{environment}{{columnsonlytextwidth}}
  This command has the same effect as |columns|, except that the
  columns will not occupy the whole page width, but only the text
  width. 
\end{environment}


\begin{environment}{{column}\marg{column width}}
  Creates a single column of width \meta{column width}. The column is
  centered vertically relative to the other columns.
\end{environment}







\section{Graphics, Colors, Animations, and Special Effects}

\subsection{Graphics}
\label{section-graphics}

Graphics often convey concepts or ideas much more efficiently than
text: A picture can say more than a thousand words. (Although,
sometimes a word can say more than a thousand pictures.) In the
following, the advantages and disadvantages of different possible ways
of creating graphics for beamer presentations are discussed.



\subsubsection{Including External Graphic Files}

One way of creating graphics for a presentation is to  use an 
external program, like |xfig| or the Gimp. These programs
have an option to \emph{export} graphic files in a format that can
then be inserted into the presentation.

The main advantage is:
\begin{itemize}
\item
  You can use a powerful program to create a high-quality graphic.
\end{itemize}

The main disadvantages are:
\begin{itemize}
\item
  You  have to worry about many files. Typically there are at least
  two for each presentation, namely the program's graphic data file and the
  exported graphic file in a format that can be read by \TeX.
\item
  Changing the graphic using the program does not automatically change
  the graphic in the presentation. Rather, you must reexport the
  graphic and rerun \LaTeX.
\item
  It may be difficult to get the line width, fonts, and font sizes
  right.
\item
  Creating formulas as part of graphics is often difficult or
  impossible.
\end{itemize}

You can use all the standard \LaTeX\ commands for inserting graphics,
like |\includegraphics| (be sure to use the package
|graphics|). Also, the |pgf| package offers commands for including
graphics. Either will work fine in most situations, so choose
whichever you like. Note that, like |\pgfdeclareimage|,
|\includegraphics| also includes an image only once in a |.pdf| file,
even if it used several times (as a matter of fact, the |graphics|
package is even a bit smarter about this than |pgf|). Internally, the
|beamer| class uses |pgf| since this makes it much easier to produce
both |.pdf| and |.ps| files from the same code and to use the color
mix-ins.



\subsubsection{Inlining Graphic Commands}

A different way of creating graphics is to insert
graphic drawing commands directly into your \LaTeX\ file. There are numerous
packages that help you do this. They have various degrees of
sophistication. Inlining graphics suffers from none of the
disadvantages mentioned above for including external graphic files,
but the main disadvantage is that it is often hard to use
these packages. In some sense, you ``program'' your graphics, which 
requires a bit of practice.

When choosing a graphic package, there are a few things to keep in
mind:
\begin{itemize}
\item
  Many packages produce poor quality graphics. This is especially true
  of the standard |picture| environment of \LaTeX.
\item
  Powerful packages that produce high-quality graphics often do not
  work together with |pdflatex|.
\item
  The most powerful and easiest-to-use package around, namely
  |pstricks|, does not work together with |pdflatex| and
  this is a fundamental problem. Due to the fundamental differences
  between \textsc{pdf} and PostScript, it is not possible to write a
  ``|pdflatex| backend for |pstricks|.''
\end{itemize}

A solution to the above problem (though not necessarily the best) is
to use the \textsc{pgf} package. It produces high-quality graphics and
works together with |pdflatex|, but also with normal
|latex|. It is not as powerful as |pstricks| (as pointed
out above, this is because of rather fundamental reasons) and not as
easy to use, but it should be sufficient in most cases.





\subsection{Color Management}

The color management of the \beamer\ class relies on the packages
|xcolor|, which is an extension of the |color| package,
and on |xxcolor|, which in turn is an extension of
|xcolor|. Hopefully, in the future |xxcolor| and
|xcolor| will merge into one package and perhaps they will
someday also merge together with |color|.


\subsubsection{Colors of Main Text Elements}

By default, the following colors are used in a presentation:
\begin{itemize}
\item
  Normal text is typeset in |black|.
\item
  All ``structural'' elements, like titles, navigation bars, block
  titles, and so on, are typeset using the color
  |structure|. By default, this color is bluish. Using one of
  the class options |red|, |gray|, or |brown|
  changes this. You can also change this color simply be redefining
  the color |structure|.
\item
  All ``alert'' text is typeset by mixing in 85\% of red. To change
  this, you can either redefine the color |alert|, or you can
  change the whole alert template.
\item
  All examples are typeset using 50\% of green. To change this, you
  must change the example templates.
\end{itemize}

\begin{classoption}{brown}
  Changes the main color of the navigation and title bars
  to a brownish color.
\end{classoption}

\begin{classoption}{red}
  Changes the main color of the navigation and title bars
  to a reddish color.
\end{classoption}

\begin{classoption}{gray}
  Changes the main color of the navigation and title bars
  to monochrome.
\end{classoption}

\subsubsection{Average Background Color}

\label{section-average}

In some situations, for example when creating a transparency effect,
it is useful to have access to the current background
color. One can then, for example, mix a color with the background
color to create a ``transparent'' color.

Unfortunately, it is not always clear what exactly the background
color is. If the background is a shading or a picture, different parts
of a slide have different background colors. In these cases, one can
at least try to mix-in an \emph{average} background color, called
|averagebackgroundcolor|. If a shading or picture is not too
colorful, this works fairly well.

To specify the average background color, use the following command:

\begin{command}{\beamersetaveragebackground\marg{color expression}}
  Installs the given color as the average background color. See the
  |xcolor| package for the syntax of color expressions.
  \example |\beamersetaveragebackground{red!10}|
\end{command}

If you use the commands from Section~\ref{section-backgrounds} for
installing a background coloring, the average background color is
computed automatically for you. When you directly use the command
|\usebackgroundtemplate|, you should must set the average
background color afterward.




\subsubsection{Transparency Effects}
\label{section-transparent}

By default, \emph{covered} items are not shown during a
presentation. Thus if you write |\uncover<2>{Text.}|, the text
is not shown on any but the second slide. On the other slide, the text
is not simply printed using the background color -- it is not shown at
all. This effect is most useful if your background does not have a
uniform color.

Sometimes however, you might prefer that covered items are not
completely covered. Rather, you would like them to be shown already in
a very dim or shaded way. This allows your audience to get a feeling
for what is yet to come, without getting distracted by it. Also, you
might wish text that is covered ``once more'' still to be visible to
some degree.

Ideally, there would be an option to make covered text
``transparent.'' This would mean that when covered text is shown, it
would instead be mixed with the background behind it. Unfortunately,
this is more or less impossible to implement since neither PostScript
nor \pdf\ currently support transparency.

Nevertheless, one can come ``quite close'' to transparent text using
the special command
\begin{verbatim}
\beamersetuncovermixins{#1}{#2}
\end{verbatim}
This commands allows you to specify in a quite general way how a
covered item should be rendered. You can even specify different ways
of rendering the item depending on how long it will take before this
item is shown or for how long it has already been covered once
more. The transparency effect will automatically apply to all colors,
\emph{except} for the colors in images and shadings. For images and
shadings there is an awkward workaround, see the documentation of the
\pgf\ package. 

As a convenience, two commands are defined in |beamertemplates|
that install a predefined uncovering behavior.

\begin{command}{\beamertemplatetransparentcovered}
  Makes all covered text nearly transparent. 
\end{command}

\begin{command}{\beamertemplatetransparentcovereddynamic}
  Makes all covered text nearly transparent, but is a dynamic way. The
  longer it will take till the text is uncovered, the stronger the
  transparency. 
\end{command}

\begin{command}{\beamersetuncovermixins\marg{not yet list}%
    \marg{once more list}}
  The \meta{not yet list} specifies  how to render covered items that
  have not  yet been uncovered. The \meta{once more list} specifies
  how to render covered items that have once more been covered. 
  If you leave one of the specifications empty, the corresponding
  covered items are completely covered, that is, they are invisible.
  \example
\begin{verbatim}
\beamersetuncovermixins
  {\mixinon<1>{15!averagebackgroundcolor}
    \mixinon<2>{10!averagebackgroundcolor}
    \mixinon<3>{5!averagebackgroundcolor}
    \mixinon<4->{2!averagebackgroundcolor}}
  {\mixinon<1->{15!averagebackgroundcolor}}
\end{verbatim}
  The \meta{not yet list} and the  \meta{once more list} can
  contain any number of the following two commands:
\end{command}

\begin{command}{\mixinon\ssarg{overlay specification}\marg{mix-in specification}}
  The \meta{overlay specification} specifies on which slides the
  \meta{mix-in specification} should be applied to all colors. Unlike
  other overlay specifications, this \meta{overlay specification} is a
  ``relative'' overlay specification. For example, the specification
  ``3'' here means ``things that will be uncovered three slides
  ahead,'' respectively ``things that have once more been covered for
  three slides.'' More precisely, if an item is uncovered for more
  than one slide and then covered once more, only the ``first moment
  of uncovering'' is used for the calculation of how long the item has
  been covered once more.

  \emph{Mix-in} specifications are a concept introduced by the
  |xcolor| package. The \meta{mix-in specification} specifies how colors
  should be altered by adding another color to them. The specification
  consists of two parts, separated by an exclamation mark. The first
  part is a number between 0 and 100, where 0 means ``do not mix in the
  text color at all'' and 100 means ``use only the text color''. The
  second part is the color that should be mixed in. This second part may
  be omitted (along with the exclamation mark), in which case ``white''
  is used as mix-in color. Any color that has been defined using the
  |\definecolor| command is permissible as a mix-in color.
  \example
\begin{verbatim}
\mixinon<1>{15!averagebackgroundcolor}
\end{verbatim}
  For all items that become uncovered on the next slide or that have
  just been covered on the previous slide (depending on whether this
 command is used as part of the first or second parameter of the command
  |\beamersetuncovermixins|), use only 15\% of the actual color and
  85\% of the average background color.
\end{command}

\begin{command}{\invisibleon\ssarg{overlay specification}}
  Text that is covered on the specified slides (once more,
  relative to the current slide), is not shown at all.
  \example
\begin{verbatim}
\invisibleon<2->
\end{verbatim}
  Makes everything totally covered that is not shown next or has just
  been shown.
\end{command}




\subsection{Animations}

A word of warning first: Animations can be very distracting. No matter
how cute a rotating, flying theorem seems to look and no matter
how badly you feel your audience needs some action to keep it happy,
most people in the audience will typically feel you are making fun of
them. 

\subsubsection{Using an External Viewer}

If you have created an animation using some external
program (like a renderer), you can use the capabilities of the
presentation program (like the Acrobat Reader) to show the
animation. Unfortunately, currently there is no portable way of doing
this and even the Acrobat Reader does not support this feature on all
platforms.


\subsubsection{Animations Created by Showing Slides in Rapid Succession}

You can create an animation in a portable way by using the
overlay commands of the \beamer\ package to create a series of slides
that, when shown in rapid succession, present an animation. This is a
flexible approach, but such animations will typically be rather static
since it will take some time to advance from one slide to the
next. This approach is mostly useful for animations where you want
to explain each ``picture'' of the animation.
When you advance slides ``by hand,'' that is, by pressing a forward
button, it typically takes at least a second for the next slide to
show.

More ``lively'' animations can be created by relying on a capability
of the viewer program. Some programs support
showing slides only for a certain number of seconds during a
presentation (for the Acrobat Reader this works only in full-screen
mode). By setting the number of seconds to zero, you can create a
rapid succession of slides.

To facilitate the creation of animations using this feature, commands
can be used: |\animate| and |\animatevalue|.

\begin{command}{\animate\ssarg{overlay specification}}
  The slides specified by \meta{overlay specification} will be shown
  only as shortly as  possible.
\example
\begin{verbatim}
\frame{
  \frametitle{A Five Slide Animation}
  \animate<2-4>

  The first slide is shown normally. When the second slide is shown
  (presumably after pressing a forward key), the second, third, and
  fourth slides ``flash by.'' At the end, the content of the fifth
  slide is shown.

  ... code for creating an animation with five slides ...
}
\end{verbatim}
\end{command}

\begin{command}{\animatevalue|<|\meta{start slide}|-|\meta{end slide}|>|%
    \marg{name}\marg{start value}\marg{end value}}
  The \meta{name} must be the name of a counter or a dimension.
  It will be varied between two values. For the slides in the
  specified range, the counter or dimension is set to an interpolated
  value that depends on the current slide number. On slides before the
  \meta{start slide}, the counter or dimension is set to \meta{start
    value}; on the slides after the \meta{end slide} it is set to
  \meta{end value}.
  \example
\begin{verbatim}
\newcount\opaqueness
\frame{
  \animate<2-10>
  \animatevalue<1-10>{\opaqueness}{100}{0}
  \begin{colormixin}{\the\opaqueness!averagebackgroundcolor}
    \frametitle{Fadeout Frame}

    This text (and all other frame content) will fade out when the
    second slide is shown. This even works with
    {\color{green!90!black}colored} \alert{text}.
  \end{colormixin}
}

\newcount\opaqueness
\newdimen\offset
\frame{
  \frametitle{Flying Theorems (You Really Shouldn't!)}

  \animate<2-14>

  \animatevalue<1-15>{\opaqueness}{100}{0}
  \animatevalue<1-15>{\offset}{0cm}{-5cm}
  \begin{colormixin}{\the\opaqueness!averagebackgroundcolor}
  \hskip\offset
    \begin{minipage}{\textwidth}
      \begin{theorem}
        This theorem flies out.
      \end{theorem}
    \end{minipage}
  \end{colormixin}

  \animatevalue<1-15>{\opaqueness}{0}{100}
  \animatevalue<1-15>{\offset}{-5cm}{0cm}
  \begin{colormixin}{\the\opaqueness!averagebackgroundcolor}
  \hskip\offset
    \begin{minipage}{\textwidth}
      \begin{theorem}
        This theorem flies in.
      \end{theorem}
    \end{minipage}
  \end{colormixin}
}
\end{verbatim}
\end{command}



\subsection{Slide Transitions}

\textsc{pdf} in general, and the Acrobat Reader in particular, offer a
standardized way of defining \emph{slide transitions}. Such a
transition is a visual effect that is used to show the slide. For
example, instead of just showing the slide immediately, whatever was
shown before might slowly ``dissolve'' and be replaced by the slide's
content.

Slide transitions should be used with great care. Most of the time,
they only distract. However, they can be useful in some situations:
For example, you might show a young boy on a slide and might wish to
dissolve this slide into slide showing a grown man instead. In this
case, the dissolving gives the audience visual feedback that the young
boy ``slowly becomes'' the man.

There are a number of commands that can be used to specify what effect
should be used when the current slide is presented. Consider the
following example:

\begin{verbatim}
\frame{
  \pgfuseimage{youngboy}
}
\frame{
  \transdissolve
  \pgfuseimage{man}
}
\end{verbatim}
The command |\transdissolve| causes the slide of the
second frame to be shown in a ``dissolved way.'' Note that the
dissolving is a property of the second frame, not of the first one. We
could have placed the command anywhere on the frame.

The transition commands are overlay-specification-aware. We could
collapse the two frames into one frame like this:
\begin{verbatim}
\frame{
  \only<1>{\pgfuseimage{youngboy}}
  \only<2>{\pgfuseimage{man}}
  \transdissolve<2>
}
\end{verbatim}
This states that on the first slide the young boy should be shown, on
the second slide the old man should be shown, and when the second
slide is shown, it should be  shown in a ``dissolved way.''

In the following, the different commands for creating transitional
effects are listed.

\begin{command}{\transblindshorizontal}
  Show the slide as if horizontal blinds where pulled away.
  \example|\transblindshorizontal|
\end{command}
  
\begin{command}{\transblindsvertical}
  Show the slide as if vertical blinds where pulled away.
  \example|\transblindsvertical<2,3>|
\end{command}
  
\begin{command}{\transboxin}
  Show the slide by moving to the center from all four sides.
  \example|\transboxin<1>|
\end{command}
  
\begin{command}{\transboxout}
  Show the slide by showing more and more of a rectangular area that
  is centered on the slide center.
  \example|\transboxout|
\end{command}
 
\begin{command}{\transdissolve}
  Show the slide by slowly dissolving what was shown before.
  \example|\transdissolve|
\end{command}
  
\begin{command}{\transglitter\marg{degree}}
  Show the slide with a glitter effect that sweeps in the specified
  direction. The \meta{degree} must be a multiple of 90.
  \example|\transglitter<2-3>{90}|
\end{command}
  
\begin{command}{\transsplitverticalin}
  Show the slide by sweeping two vertical lines from the sides inward.
  \example|\transsplitverticalin|
\end{command}
  
\begin{command}{\transsplitverticalout}
  Show the slide by sweeping two vertical lines from the center outward.
  \example|\transsplitverticalout|
\end{command}
  
\begin{command}{\transsplithorizontalin}
  Show the slide by sweeping two horizontal lines from the sides inward.
  \example|\transsplithorizontalin|
\end{command}
  
\begin{command}{\transsplithorizontalout}
  Show the slide by sweeping two horizontal lines from the center outward.
  \example|\transsplithorizontalout|
\end{command}
 
\begin{command}{\transwipe\marg{degree}}
  Show the slide by sweeping a single line in the specified direction,
  thereby ``wiping out'' the previous contents. The \meta{degree} must be a multiple of 90.
  \example|\transwipe{90}|
\end{command}



You can also specify how \emph{long} a given slide should be shown,
using the following overlay-specification-aware command:

\begin{command}{\transduration\marg{number of seconds}}
  In full screen mode, show the slide for \meta{number of seconds}.
  In zero is specified, the slide is shown as short as possible. This
  can be used to create interesting pseudo-animations.
  \example|\transduration<2>{1}|
\end{command}



\section{Managing Non-Presentation Versions and Material}

The \beamer\ package offers different ways of creating special
versions of your talk and adding material that are not shown during
the presentation. You can create a \emph{handout} version of the
presentation that can be distributed to the audience. You can also
create a version that is more suitable for a presentation using an
overhead projector. You can add notes for yourself that help
you remember what to say for specific slides. Finally, you can have a
completely independent ``article'' version of your presentation 
coexist in your main file. All special versions are created by
specifying different class options and rerunning \TeX\ on the main
file. 


\subsection{Creating Handouts}

\label{handout}

A \emph{handout} is a version of a presentation that is printed on
paper and handed out to the audience before or after the talk. (See
Section~\ref{section-postscript} for how to place numerous frames on one
pages, which is very useful for handouts.)  For the handout you
typically want to produce as few slides as possible per frame. In
particular, you do not want to print a new slide for each slide of a
frame. Rather, only the ``last'' slide should be printed. 

In order to create a handout, specify the class option
|handout|. If you do not specify anything else, this will cause
all overlay specifications to be suppressed. For most cases this will
create exactly the desired result.

\begin{classoption}{handout}
  Create a version that uses the |handout| overlay specifications.
\end{classoption}

In some cases, you may want a more complex behaviour. For example, if
you use many |\only| commands to draw an animation. In this case,
suppressing all overlay specifications is not such a good idea, since
this will cause all steps of the animation to be shown at the same
time. In some cases this is not desirable. Also, it might be desirable
to suppress some |\alert| commands that apply only to specific
slides in the handout.

For a fine-grained control of what is shown on a handout, you can use
\emph{alternate overlay specifications}. They specify which slides
 of a frame should be shown for a special version, for example for the
handout version. An alternate overlay specification is written
alongside the normal overlay specification inside the pointed
brackets. It is separated from the normal specification by a vertical
bar and a space. The version to which the alternate specification
applies is written first, followed by a colon. Here is an example:
\begin{verbatim}
  \only<1-3,5-9| handout:2-3,5>{Text}
\end{verbatim}
This specification says: ``Normally, insert the text on slides 1--3
and 5--9. For the handout version, insert the text only on slides
2,~3, and~5.'' If no alternate overlay specification is given for
handouts, the default is ``always.'' This causes the desirable effect
that if you do not specify anything, the overlay specification is
effectively suppressed for the handout.

An especially useful specification is the following:
\begin{verbatim}
  \only<3| handout:0>{Not shown on handout.}
\end{verbatim}
Since there is no zeroth slide, the text is not shown. Likewise,
\verb!\alert<3| handout:0>{Text}! will not alert the text on a
handout.

You can also use an alternate overlay specification for the optional
argument of the frame command as in the following example.
\begin{verbatim}
\frame[1-| handout:0]{Text...}
\end{verbatim}
This causes the frame to be suppressed in the handout version. Also,
you can restrict the presentation such that only specific slides of
the frame are shown on the handout:
\begin{verbatim}
\frame[1-| handout:4-5]{Text...}
\end{verbatim}

It is also possible to give only an alternate overlay
specification. For example, |\alert<handout:0>{...}| causes the
text to be always hilighted during the presentation, but never on the
handout version. Likewise, |\frame[handout:0]{...}| causes the
frame to be suppressed for the handout.

Finally, note that it is possible to give more than one alternate
overlay specification and in any order. For example, the following
specification states that the text should be inserted on the first
three slides in the presentation, in the first two slides of the
transparency version, and not at all in the handout.
\begin{verbatim}
  \only<trans:1-2| 1-3| handout:0>{Text}
\end{verbatim}

If you wish to give the same specification in all versions, you can do
so by specifying |all:| as the version. For example,
\begin{verbatim}
\frame[all:1-2]
{
  blah...
}
\end{verbatim}
ensures that the frame has two slides in all versions. 




\subsection{Creating Transparencies}

\label{trans}

The main aim of the \beamer\ class is to create presentations for
beamers. However, it is often useful to print transparencies as
backup, in case the hardware fails. A transparencies version of a talk
often has less slides than the main version, since it takes more time
to switch slides, but it may have more slides than the handout
version. For example, while in a handout an animation might be
condensed to a single slide, you might wish to print several slides
for the transparency version.

You can use the same mechanism as for creating handouts: Specify
|trans| as a class option and add alternate transparency
specifications for the |trans| version as needed. An elaborated
example of different overlay specifications for the presentation, the
handout, and the transparencies can be found in the file
|beamerexample1.tex|.

\begin{classoption}{trans}
  Create a version that uses the |trans| overlay
  specifications. 
\end{classoption}



\subsection{Adding Notes}

You can add notes to your slides using the command |\note|. A
note is a reminder to yourself of what you should say or should keep in
mind when presenting a frame. The |\note| command should be given
after the frame to which the note applies. Here is a typical example.
\begin{verbatim}
\frame{
  \begin{itemize}
  \item<1-> Eggs
  \item<2-> Plants
  \item<3-> Animals
  \end{itemize}
}
\note{Tell joke about eggs.}
\end{verbatim}
The note command will create a new page that contains your text plus
some information that should make it easier to match the note to the
frame while talking. 

Since you normally do not wish the notes to be part of your
presentation, you must explicitly specify the class option
|notes| to include notes. If this option is not specified, notes
are suppressed. If you specify |notesonly| instead of
|notes|, only notes will be included and all normal frames are
parsed, but not displayed. This is useful for printing the notes.

By default, you can fit only little on each note (they are only
intended to be reminders after all). Using the class option
|compressnotes| will allow you to squeeze much more on each note
card. 

\begin{classoption}{notes}
  Include notes in the output file. Normally, notes are not included.
\end{classoption}

\begin{classoption}{notesonly}
  Include only the notes in the output file. Useful for printing them.
\end{classoption}

\begin{classoption}{compressnotes}
  Squeezes as much text as possible on each note card.
\end{classoption}

\begin{command}{\note\marg{note text}}
  Creates a note page. Should be given right after a frame.
  \example|\note{Talk no more than 1 minute.}|
\end{command}

\begin{command}{\noteitems\marg{list of item commands}}
  Just like the |\note| command, except that an |itemize|
  environment is setup inside the note.
  \example
\begin{verbatim}
\frame{Bla bla...}
\noteitems{
\item Stress the importance.
\item Use no more than 2 minutes.
}
\end{verbatim}
\end{command}



\subsection{Creating an Article Version}

\label{section-article}

In the following, the ``article version'' of your presentation refers
to a normal \TeX\ text typeset using, for example, the document class
|article| or perhaps |lncs| or a similar document
class. This version of the presentation will typically follow
different typesetting rules and may even have a different
structure. Nevertheless, you may wish to have this version coexist
with your presentation in one file and you may wish to share some part
of it (like a figure or a formula) with your presentation.



\subsubsection{Article, Common, and Presentation Mode}

The class option |class=|\meta{class name}, where
\meta{class name} is the name of another document class like
|article| or |report|, causes the |beamer| class to transfer control
almost immediately to the class named \meta{class name}. None of the
normal commands defined by the beamer class will be 
defined, except for the three commands listed in the following. All
class options passed to the beamer class will be passed on to the
class \meta{class name}, \emph{except}, naturally, for the option
|class=|\meta{class name} itself.

\begin{classoption}{class={\normalfont\meta{another class
        name}}{\opt{,{\normalfont\meta{options for another class}}}}}
  Transfer control to document class \meta{another class name} with
  the options \meta{options for another class}. See
  Section~\ref{section-article} for details.
  \example
\begin{verbatim}
\documentclass[class=article,a4paper]{beamer}
\end{verbatim}
  This will cause the rest of the text to be typeset using the
  |article| class with the only class option being
  |a4paper|.
\end{classoption}

You can use three commands to specify which part of your text belong
to the article version, which belongs to the actual presentation, and
which belongs to both. These command switch between three different
modes: article mode, presentation mode, and common mode. While \TeX\
scans text in the article mode, this text is read normally when an
article is requested, but thrown away if a presentation is
requested. In presentation mode, the behavior is the other way
round. In common mode, the text is always inserted.

Right after the |documentclass| command and right after the
|\begin{document}| (provided it is the sole entry on a line with no
comments following and no leading spaces), \TeX\ is always in common
mode. If you do not wish this to be the case, simply append a comment
to the line.

If you use |\input| or |\include| or a related command to include
another file, make sure that when \TeX\ reaches the end of this file,
it is in common mode.
  
\begin{command}{\article}
  All text following this command will only be present in the article
  version. For the presentation version, this text will be completely
  ignored. This command must be the only command in a line and it must
  start the line.
\end{command}

\begin{command}{\presentation}
  All text following this command will only be present in the
  presentation version. For the article version, the text will be
  completely ignored. This command must be the only command in a line
  and it must start the line.
\end{command}

\begin{command}{\common}
  All text following this command will be present in both the article
  and the presentation version. This command must be the only command in
  a line and it must start the line.
\end{command}

\begin{verbatim}
\documentclass[class=article,a4paper]{beamer}
%\documentclass[red]{beamer}

\article
\usepackage{fullpage}

\common
\usepackage[english]{babel}
\usepackage{pgf}

\presentation
\usepackage{beamerthemesplit}

\begin{document}

\pgfdeclareimage[height=1cm]{myimage}{filename}

\presentation
  \frame{

\common
    \begin{figure}
      \pgfuseimage{myimage}
    \end{figure}
  
\presentation
    }
\end{document}
\end{verbatim}

The above commands cannot be used inside macros (they are implemented
similarly to |verbatim| environments, only that the contents is
sometimes thrown away instead of rendered). However, there is one
exception: Inside a |\frame|, these commands can be used,
provided they ``balance'' inside the frame and provided you switch
back to presentation mode by the end of the frame (as in the above
example). If you have problems with these commands inside 
a frame, try using a |\def| command outside the frame as in the 
following example:
\begin{verbatim}
\begin{document}
...
\common
  \def\myfigure{
    \begin{figure}
      \pgfimage{filename}
    \end{figure}}

\article
  \myfigure

\presentation
  \frame{\myfigure}
\end{document}
\end{verbatim}



\subsubsection{Workflow}
\label{section-article-version-workflow}
The following workflow steps are optional, but they can simplify the
creation of the article version.

\begin{itemize}
\item 
  In the main file |main.tex|, delete the first line, which sets the
  document class.
\item
  Create a file named, say, |main.beamer.tex| with the
  following content:
\begin{verbatim}
\documentclass{beamer}
\input{main.tex}
\end{verbatim} 
\item
  Create an extra file named, say, |main.article.tex| with the
  following content:
\begin{verbatim}
\documentclass[class=article]{beamer}
\setjobnamebeamerversion{main.beamer}
\input{main.tex}
\end{verbatim}
\item
  You can now run |pdflatex| or |latex| on the two files
  |main.beamer.tex| and |main.article.tex|. 
\end{itemize}

The command |\setjobnamebeamerversion| tells the article version where
to find the presentation version. This is necessary if you wish to include
slides from the presentattion version in an article as figures.

\begin{command}{\setjobnamebeamerversion\marg{filename without extension}}
  Tells the beamer class where to find the presentation version of the
  current file.  
\end{command}

An example of this workflow approach can be found in the |examples|
subdirectory for files starting with |beamerexample2|.



\subsubsection{Including Slides from the Presentation Version in the
  Article Version}

In order to include a slide from your presentation in your article
version, you must do two things: first, you must \emph{name} the slide
using the following command:

\begin{command}{\nameslide\sarg{overlay specification}\marg{slide
      name}}
  The specified slide gets assigned the name \meta{slide name}. This
  causes an entry with the slide's page number to be inserted into an
  auxiliary file with the extension |.snm|. Furthermore, a hyper
  target of the name \meta{slide name} is also setup (this is just a
  convenience for jumping to a named slide inside the presentation
  version). 
  \example
\begin{verbatim}
\frame{
  \nameslide<1>{slide1}
  \nameslide<2>{slide2}
  
  \frametitle{This is a frame with two overlays.}

  \begin{itemize}
  \item The first item$\dots$
    \pause
  \item $\dots$ and the second one.
  \end{itemize}
 }
\end{verbatim}
\end{command}

Once you have declared a slide, you can use the following command in
your article version to insert the slide into it:

\begin{command}{\includeslide\oarg{options}\marg{slide name}}
  This command calls |\pgfimage| with the given \meta{options} for
  the file specified by
  \begin{quote}
    |\setjobnamebeamerversion|\meta{filename}
  \end{quote}
  Furthermore, the option |page=|\meta{page of slide name} is passed
  to |\pgfimage|, where the \meta{page of slide name} is read
  internally from the file \meta{filename}|.snm|.
  \example

\begin{verbatim}
\article
  \begin{figure}
    \begin{center}
      \includeslide[height=5cm]{slide1}
    \end{center}
    \caption{The first slide (height 5cm). Note the partly covered second item.}
  \end{figure}
  \begin{figure}
    \begin{center}
      \includeslide{slide2}
    \end{center}
    \caption{The second slide (original size). Now the second item is also shown.}
  \end{figure}
\end{verbatim}  
\end{command}

The exact effect of passing the option |page=|\meta{page of slide
  name} to the command |\pgfimage| is explained in the documentation
of |pgf|. In essence, the following happens:
\begin{itemize}
\item
  For old version of |pdflatex| and for any version of |latex|
  together with |dvips|, the |pgf| package will look for a file named
  \begin{quote}
    \meta{filename}|.page|\meta{page of slide name}|.|\meta{extension}
  \end{quote}
  For each page of your |.pdf| or |.ps| file that is to be included in
  this way, you must create such a file by hand. For example, if the
  PostScript file of your presentation version is named
  |main.beamer.ps| and you wish to include the slides with page
  numbers 2 and~3, you must create (single page) files
  |main.beamer.page2.ps| and |main.beamer.page3.ps| ``by hand'' (or
  using some script). If these files cannot be found, |pgf| will
  complain.
\item
  For new versions of |pdflatex|, |pdflatex| also looks for the files
  according to the above naming scheme. However, if it fails to find
  them (because you have not produced them), it uses a special
  mechanism to directly extract the desired page from the presentation
  file |main.beamer.pdf|.
\end{itemize}








\section{Customization}

\subsection{Fonts}

\subsubsection{Serif Fonts and Sans Serif Fonts}

By default, the beamer class uses the Computer Modern sans-serif fonts
for typesetting a presentation. The Computer Modern font family is the
original font family designed by Donald Knuth himself for the \TeX\
program. A sans-serif font is a font in which the letters do not have
serifs (from Frensh \emph{sans}, which means ``without''). Serifs are
the little hooks at the ending of the strokes that make up a
letter. The font you are currently reading is a serif font. \textsf{By
  comparison, this text is in a sans-serif font.}

The choice Computer Modern sans-serif had the following reasons:

\begin{itemize}
\item
  The Computer Modern family has a very large number of symbols
  available that go well together.
\item
  Sans-serif fonts are (generally considered to be) easier to read
  when used in a presentation. In low resolution rendering, serifs
  decrease the legibility of a font.
\end{itemize}

While these reasons are pretty good, you still might wish to change the font:

\begin{itemize}
\item
  The Computer Modern fonts are a bit boring if you have seen them too
  often. Using another font (but not Times!) can give a fresh look.
\item
  Other fonts, especially Times, are sometime rendered better since
  they seem to have better internal hinting.
\item
  A presentation typeset in a serif font  creates a conservative
  impression, which might be exactly what you wish to create.
\end{itemize}

You must decide whether the text should be typeset in sans serif or in
serif. To choose this, use either the class option |sans| or
|serif|. By default, |sans| is selected, so you do not
need to specify this. Furthermore, you can specify one of the two
options |mathsans| or |mathserif|. These options
override the overall sans-serif/serif choice for math text.

\begin{classoption}{sans}
  Use a sans-serif font during the presentation. (Default.)
\end{classoption}

\begin{classoption}{serif}
  Use a serif font during the presentation.
\end{classoption}

\begin{classoption}{mathsans}
  Override the math font to be a sans-serif font.
\end{classoption}

\begin{classoption}{mathserif}
  Override the math font to be a serif font.
\end{classoption}

\subsubsection{Font Families}

Independently of the serif/sans-serif choice, you can switch the
document font. To do so, you should use one of the prepared packages
of \LaTeX's font mechanism. For example, to change to Times/Helvetica,
simply add 
\begin{verbatim}
\usepackage{times}
\end{verbatim}
in your preamble. Note that if you do not specify |serif| as a
class option, Helvetica (not Times) will be selected as the text
font.

There may be many other fonts available on your
installation. Typically, at least some of the following packages
should be available: |avant|, |bookman|, |chancery|, |charter|,
|euler|, |helvet|, |mathtime|, |mathptm|, |newcent|, |palatino|,
|pifont|, |times|, |utopia|.

If you use |times| together with the |serif| option, you
may wish to include also the package |mathptm|. If you use the
|mathtime| package (you have to buy some of the fonts), you
also need to specify the |serif| option.


\subsubsection{Font Sizes}

The default sizes of the fonts are chosen in a way that makes it
difficult to fit ``too much'' onto a slide. Also, it will ensure that 
your slides are readable even under bad conditions like a large
room and a small only a small projection area. However, you may wish
to enlarge or shrink the fonts a bit if you know this to be more
appropriate in your presentation environment. You can use the
following two options to achieve this:

\begin{classoption}{bigger}
  Makes all fonts a little bigger, which makes the text more
  readable. The downside is that less fits onto each frame.
\end{classoption}

\begin{classoption}{smaller}
  Makes all fonts a little smaller, which allows you to fit more onto
  frames. Normally, this is not a good idea.
\end{classoption}



\subsection{Margin Sizes}

The ``paper size'' of a beamer presentation is fixed to 128mm times
96mm. The aspect ratio of this size is 4:3, which is exactly what most
beamers offer these days. It is the job of the
presentation program (like |acroread|) to display the slides at
full screen size. The main advantage of using a small ``paper size''
is that you can use all your normal fonts at their natural sizes. In
particular, inserting a graphic with 11pt labels will result in
reasonably sized labels during the presentation.

You should refrain from changing the ``paper size.'' However, you
\emph{can} change the size of the left and right margins, which
default to 1cm. To change them, you should use the following two
commands:

\begin{command}{\beamersetleftmargin\marg{left margin dimension}}
  Sets a new left margin. This excludes the left side bar. Thus, it is
  the distance between the right edge of the left side bar and the left
  edge of the text. This command can only be used in the preamble
  (before the |document| environment is used).
  \example |\beamersetleftmargin{1cm}|
\end{command}

\begin{command}{\beamersetrightmargin\marg{left margin dimension}}
  Like |\beamersetleftmargin|, only for the right margin.
\end{command}

For more information on side bars, see
Section~\ref{section-sidebar-templates}. 



\subsection{Themes}

Just like \LaTeX\ in general, the \beamer\ class tries to separate the
contents of a text from the way it is typeset (displayed). There are two ways in
which you can change how a presentation is typeset: you can specify a
different theme and you can specify different templates. A theme is
a predefined collection of templates.

There exist a number of different predefined themes that can be used
together with the \beamer\ class. Feel free to add further themes.
Themes are used by including an appropriate \LaTeX\ style file, using
the standard |\usepackage| command.


\begin{smallpackage}{{beamerthemebars}}
  \example

  \pgfuseimage{themebars}\quad\pgfuseimage{themebars2}
\end{smallpackage}


\begin{package}{{beamerthemeboxes}\opt{|[headheight=|\meta{head height}|,footheight=|\meta{foot height}|]|}}
  \example

  \pgfuseimage{themeboxes}\quad\pgfuseimage{themeboxes2}

  \example
\begin{verbatim}
\usepackage[headheight=12pt,footheight=12pt]{beamerthemeboxes}
\end{verbatim}

  For this theme, you can specify an arbitrary number of templates for
  the boxes in the head line and in the foot line. You can add a
  template for another box by using the following commands.
\end{package}

\begin{command}{\addheadboxtemplate%
    \marg{background color command}\marg{box template}}
  Each time this command is invoked, a new box is added to the head
  line, with the first added box being shown on the left. All boxes
  will have the same size.
  \example
\begin{verbatim}
\addheadboxtemplate{\color{black}}{\color{white}\tiny\quad 1. Box}
\addheadboxtemplate{\color{structure}}{\color{white}\tiny\quad 2. Box}
\addheadboxtemplate{\color{structure!50}}{\color{white}\tiny\quad 3. Box}
\end{verbatim}
\end{command}

\begin{command}{\addfootboxtemplate%
    \marg{background color command}\marg{box template}}
  \example
\begin{verbatim}
\addfootboxtemplate{\color{black}}{\color{white}\tiny\quad 1. Box}
\addfootboxtemplate{\color{structure}}{\color{white}\tiny\quad 2. Box}
\end{verbatim}
\end{command}


\begin{smallpackage}{{beamerthemeclassic}}
  \example

  \pgfuseimage{themeclassic}\quad\pgfuseimage{themeclassic2}
\end{smallpackage}


\begin{smallpackage}{{beamerthemelined}}
  \example

  \pgfuseimage{themelined}\quad\pgfuseimage{themelined2}
\end{smallpackage}


\begin{smallpackage}{{beamerthemeplain}}
  \example

  \pgfuseimage{themeplain}\quad\pgfuseimage{themeplain2}
\end{smallpackage}


\begin{smallpackage}{{beamerthemesidebar}\opt{|[width=|\meta{sidebar width}|]|}}
  \example

  \pgfuseimage{themesidebar}\quad\pgfuseimage{themesidebar2}

  \example
\begin{verbatim}
\usepackage[width=3cm]{beamerthemesidebar}
\end{verbatim}
\end{smallpackage}


\begin{smallpackage}{{beamerthemesidebartab}\opt{|[width=|\meta{sidebar width}|]|}}
  \example

  \pgfuseimage{themesidebartab}\quad\pgfuseimage{themesidebartab2}
\end{smallpackage}


\begin{smallpackage}{{beamerthemesidebardark}\opt{|[width=|\meta{sidebar width}|]|}}
  \example

  \pgfuseimage{themesidebardark}\quad\pgfuseimage{themesidebardark2}
\end{smallpackage}


\begin{smallpackage}{{beamerthemesidebartabdark}\opt{|[width=|\meta{sidebar width}|]|}}
  \example

  \pgfuseimage{themesidebardarktab}\quad\pgfuseimage{themesidebardarktab2}
\end{smallpackage}


\begin{smallpackage}{{beamerthemesplit}}
  \example

  \pgfuseimage{themesplit}\quad\pgfuseimage{themesplit2}
\end{smallpackage}
\medskip

The theme |beamerthemesplitcondensed| is no longer
supported. Use |beamerthemesplit| with the |compress| class option
instead. 

\begin{smallpackage}{{beamerthemetree}}
  \example

  \pgfuseimage{themetree}\quad\pgfuseimage{themetree2}
\end{smallpackage}


\begin{smallpackage}{{beamerthemetreebars}}
  \example

  \pgfuseimage{themetreebars}\quad\pgfuseimage{themetreebars2}
\end{smallpackage}



\subsection{Templates}
\label{section-templates}

\subsubsection{Introduction to Templates}

If you only wish to modify a small part of how your presentation is
rendered, you do not need to create a whole new theme. Instead, you
can modify an appropriate template.

A template specifies how a part of a presentation is typeset. For
example, the frame title template dictates where the frame title is
put, which font is used, and so on.

As the name suggests, you specify a template by writing the exact
\LaTeX\ code you would also use when typesetting a single frame title
by hand. Only, instead of the actual title, you use the command
|\insertframetitle|.

For example, suppose we would like to have the frame title typeset in
red, centered, and boldface. If we were to typeset a single frame
title by hand, it might be done like this:
\begin{verbatim}
\frame
{
  \begin{centering}
    \color{red}
    \textbf{The Title of This Frame.}
    \par
  \end{centering}

  Blah, blah.
}
\end{verbatim}

In order to typeset the frame title in this way on all slides, we can
change the frame title template as follows:
\begin{verbatim}
\useframetitletemplate{
  \begin{centering}
    \color{red}
    \textbf{\insertframetitle}
    \par
  \end{centering}
}
\end{verbatim}

We can then use the following code to get the desired effect:
\begin{verbatim}
\frame
{
  \frametitle{The Title of This Frame.}

  Blah, blah.
}
\end{verbatim}

When rendering the frame, the \beamer\ class will use the code of the
frame title template to typeset the frame title and it will replace
every occurrence of |\insertframetitle| by the current frame
title.

In the following subsections all commands for changing templates are
listed, like the above-mentioned command
|\useframetitletemplate|. Inside these commands, you should use
the |\insertxxxx| commands listed in 
the next subsection. 

Some of the below subsections start with commands for using
\emph{predefined} templates. These commands are defined in the package
|beamertemplates|. Calling one of them will change a template in
a predefined way. Using them, you can use, for example, your favorite
theme together with a predefined background.

Here are a few hints that might be helpful when you wish to redefine a
template: 
\begin{itemize}
\item
  Usually, you might wish to copy code from an existing template. This
  code often takes care of some things that you may not yet have
  thought about.
\item
  When copying code from another template and when inserting this code
  in the preamble of your document (not in another style file), you may
  have to ``switch on'' the at-character (|@|). To do so, add the
  command |\makeatletter| before the |\usexxxtemplate| command and the
  command |\makeatother| afterward.
\item
  Most templates having to do with the frame components (head lines,
  side bars, etc.)\ can only be changed in the preamble. Other
  templates can be changed during the document.
\item
  The height of the head line and foot line templates is calculated
  automatically. This is done by typesetting the templates and then
  ``having a look'' at their heights. This recalculation of the
  heights takes place several times, but at least twice: once directly
  after a call to |\useheadtemplate|, respectively |\usefoottemplate|,
  and once before the |\begin{document}|. Because of this, your
  templates must be ``typesetable'' inside the preamble. In
  particular, any images you use must already be declared.
\item
  The left and right margins of the head and foot line templates are
  the same as of the normal text. In order to start the head line and
  foot line at the right page margin, you must insert a negative
  horizontal skip using |\hskip-\Gm@lmargin|. You may wish to add a
  |\hskip-\Gm@rmargin| at the end to avoid having \TeX\ complain about
  overfull boxes.
\item
  Getting the boxes right inside any template is often a bit of a
  hassle. You may wish to consult the \TeX\ book for the glorious
  details on ``Making Boxes.'' If your headline is simple, you might
  also try putting everything into a |pgfpicture| environment, which
  makes the placement easier.
\end{itemize}



\subsubsection{Title Page}


\paragraph{Template Installation Commands}\ 

\begin{command}{\usetitlepagetemplate\marg{title page template}}
  \example
\begin{verbatim}
\usetitlepagetemplate{
  \vbox{}
  \vfill
  \begin{centering}
    \Large\structure{\inserttitle}
    \vskip1em\par
    \normalsize\insertauthor\vskip1em\par
    {\scriptsize\insertinstitute\par}\par\vskip1em
    \insertdate\par\vskip1.5em
    \inserttitlegraphic
  \end{centering}
  \vfill
}
\end{verbatim}
\end{command}

If you wish to suppress the head and foot line in the title page, use
|\plainframe{\titlepage}|.



\paragraph{Inserts for this Template}\ 

\begin{command}{\insertauthor}
  Inserts the author names into a template.
\end{command}

\begin{command}{\insertdate}
  Inserts the date into a template.
\end{command}

\begin{command}{\insertinstitute}
  Inserts the institute into a template.
\end{command}

\begin{command}{\inserttitle}
  Inserts a version of the document title into a template that is
  useful for the title page. 
\end{command}

\begin{command}{\inserttitlegraphic}
  Inserts the title graphic into a template.
\end{command}



\subsubsection{Part Page}

\label{section-part-page-template}

\paragraph{Template Installation Commands}\ 

\begin{command}{\usepartpagetemplate\marg{part page template}}
  \example
\begin{verbatim}
\usepartpagetemplate{
  \begin{centering}
    \Large\structure{\partname~\insertromanpartnumber}
    \vskip1em\par
    \insertpart\par
  \end{centering}
  }
\end{verbatim}
\end{command}


\paragraph{Inserts for this Template}\ 

\begin{command}{\insertpart}
  Inserts the current part name.
\end{command}

\begin{command}{\insertpartnumber}
  Inserts the current part number as an Arabic number into a template.
\end{command}

\begin{command}{\insertpartromannumber}
  Inserts the current part number as a Roman number into a template.
\end{command}




\subsubsection{Background}

\label{section-backgrounds}

\paragraph{Predefined Templates}\ 

\begin{command}{\beamertemplateshadingbackground%
    \marg{color expression page bottom}\marg{color expression page top}}
  Installs a vertically shaded background such that the
  specified bottom color changes smoothly to the specified top
  color. \textbf{Use with care: Background shadings are often
    distracting!} However, a very light shading with warm colors can
  make a presentation more lively.
  \example
\begin{verbatim}
\beamertemplateshadingbackground{red!10}{blue!10}
% Bottom is light red, top is light blue
\end{verbatim}
\end{command}


\begin{command}{\beamertemplategridbackground}
  Installs a light grid as background.
\end{command}


\paragraph{Template Installation Commands}\ 

\begin{command}{\usebackgroundtemplate\marg{background template}}
  Installs a new background template. Call
  |\beamersetaveragebackground| after you have called this macro, see
  Section~\ref{section-average} for details.
  \example
\begin{verbatim}
\usebackgroundtemplate{%
  \color{red}%
  \vrule  height\paperheight width\paperwidth%
}
\end{verbatim}
\end{command}







\subsubsection{Table of Contents}

\label{section-toc-templates}

\paragraph{Template Installation Commands}\ 

\begin{command}{\usetemplatetocsection\oarg{mix-in specification}%
    \marg{template}\opt{\marg{grayed template}}}
  Installs a \meta{template} for rendering sections in the table of
  contents. If the \meta{mix-in specification} is present, the
  \meta{grayed template} may not be present and the grayed sections
  names are obtained by mixing in the  \meta{mix-in specification}. 
  If \meta{mix-in specification} is not present,  \meta{grayed
    template} must be present and is used to render grayed section
  names. 
  \example
\begin{verbatim}
\usetemplatetocsection
{\color{structure}\inserttocsection}
{\color{structure!50}\inserttocsection}

\usetemplatetocsection[50!averagebackgroundcolor]
{\color{structure}\inserttocsection}
\end{verbatim}
\end{command}

\begin{command}{\usetemplatetocsubsection\oarg{mix-in specification}%
    \marg{template}\opt{\marg{grayed template}}}
  See |\usetemplatetocsection|.
  \example
\begin{verbatim}
\usetemplatetocsubsection
{\leavevmode\leftskip=1.5em\color{black}\inserttocsubsection\par}
{\leavevmode\leftskip=1.5em\color{black!50}\inserttocsubsection\par}

\usetemplatetocsection[50!averagebackgroundcolor]
{\leavevmode\leftskip=1.5em\color{black}\inserttocsubsection\par}
\end{verbatim}
\end{command}

\paragraph{Inserts for this Template}\ 

\begin{command}{\inserttocsection}
  Inserts the table of contents version of the current section name
  into a template.
\end{command}

\begin{command}{\inserttocsubsection}
  Inserts the table of contents version of the current subsection name
  into a template. 
\end{command}





\subsubsection{Bibliography}

\label{section-bib-templates}

\paragraph{Predefined Templates}\

\begin{command}{\beamertemplatetextbibitems}
  Shows the citation text in front of references in a
  bibliography instead of a small symbol.
\end{command} 

\begin{command}{\beamertemplatearrowbibitems}
  Changes the symbol before references in a bibliography to
  a small arrow.
\end{command}

\begin{command}{\beamertemplatebookbibitems}
  Changes the symbol before references in a bibliography to
  a small book icon.
\end{command}

\begin{command}{\beamertemplatearticlebibitems}
  Changes the symbol before references in a bibliography to
  a small article icon. (Default)
\end{command}



\paragraph{Template Installation Commands}\ 

\begin{command}{\usebibitemtemplate\marg{citation template}}
  Installs a template for the citation text before the entry. (The 
  ``label'' of the item.)
  \example |\usebibitemtemplate{\color{structure}\insertbiblabel}|
\end{command}


\begin{command}{\usebibliographyblocktemplate%
    \marg{template 1}\marg{template 2}%
    \marg{template 3}\marg{template 4}}
  The text \meta{template~1} is inserted before the first block of the
  entry (the first block is all text before the first occurrence of a 
  |\newblock| command). The text \meta{template~2} is inserted before
  the second block (the text between the first and second occurrence
  of |\newblock|). Likewise for \meta{template~3} and \meta{template~4}. 

  The templates are inserted \emph{before} the blocks and you do not
  have access to the blocks themselves via insert commands. In the
  following example, the first |\par| commands ensure that the
  author, the title, and the journal are put on different lines. The
  color commands cause the author (first block) to be typeset using
  the theme color, the second block (title of the paper) to be typeset
  in black, and all other lines to be typeset in a washed-out version
  of the theme color. 
  \example
\begin{verbatim}
  \usebibliographyblocktemplate
  {\color{structure}}
  {\par\color{black}}
  {\par\color{structure!75}}
  {\par\color{structure!75}}
\end{verbatim}
\end{command}


\paragraph{Inserts for these Templates}\ 

\begin{command}{\insertbiblabel}
  Inserts the current citation label into a template.
\end{command}



\subsubsection{Frame Titles}

\paragraph{Template Installation Commands}\ 

\begin{command}{\useframetitletemplate\marg{frame title template}}
  \example
\begin{verbatim}
\useframetitletemplate{%
  \begin{centering}
    \structure{\textbf{\insertframetitle}}
    \par
  \end{centering}
}
\end{verbatim}
\end{command}

\paragraph{Inserts for this Template}\ 

\begin{command}{\insertframetitle}
  Inserts the current frame title into a template.
\end{command}




\subsubsection{Head Lines and Foot Lines}

\label{section-head-templates}

\paragraph{Predefined Templates}\ 

\begin{command}{\beamertemplateheadempty}
  Makes the head line empty.
\end{command}

\begin{command}{\beamertemplatefootempty}
  Makes the foot line empty.
\end{command}

\begin{command}{\beamertemplatefootpagenumber}
  Shows only the page number in the foot line.
\end{command}



\paragraph{Template Installation Commands}\ 

\begin{command}{\usefoottemplate\marg{foot line template}}
  The final height of the foot line is calculated by invoking this
  template just before the beginning of the document and by setting
  the foot line height to the height of the template.
  \example
\begin{verbatim}
\usefoottemplate{\hfil\tiny{\color{black!50}\insertpagenumber}}
\end{verbatim}
or
\begin{verbatim}
\usefoottemplate{%
  \vbox{%
    \tinycolouredline{structure!75}%
      {\color{white}\textbf{\insertshortauthor\hfill\insertshortinstitute}}%
    \tinycolouredline{structure}%
      {\color{white}\textbf{\insertshorttitle}\hfill}%
    }}
\end{verbatim}
\end{command}


\begin{command}{\useheadtemplate\marg{head line template}}
  See |\usefoottemplate|.
  \example
\begin{verbatim}
\useheadtemplate{%
  \vbox{%
  \vskip3pt%
  \beamerline{\insertnavigation{\paperwidth}}%
  \vskip1.5pt%
  \insertvrule{0.4pt}{structure!50}}%
}
\end{verbatim}
\end{command}



\paragraph{Inserts for these Templates}\ 

\begin{command}{\insertframenumber}
  Inserts the number of the current frame (not slide) into a template.
\end{command}

\begin{command}{\insertlogo}
  Inserts the logo(s) into a template.
\end{command}

\begin{command}{\insertnavigation\marg{width}}
  Inserts a horizontal navigation bar of the given \meta{width} into a
  template. The bar lists the sections and below them mini frames for
  each frame in that section.
\end{command}

\begin{command}{\insertpagenumber}
  Inserts the current page number into a template.
\end{command}

\begin{command}{\insertsection}
  Inserts the current section into a template.
\end{command}

\begin{command}{\insertsectionnavigation\marg{width}}
  Inserts a vertical navigation bar containing all sections, with the
  current section hilighted.
\end{command}

\begin{command}{\insertsectionnavigationhorizontal\marg{width}%
    \marg{left insert}\marg{right insert}}
  Inserts a horizontal navigation bar containing all sections, with
  the current section hilighted. The \meta{left insert} will be
  inserted to the left of the sections, the \marg{right insert} to the
  right. By inserting a triple fill (a
  |filll|) you can flush to bar to the left or right.
  \example
\begin{verbatim}
\insertsectionnavigationhorizontal{.5\textwidth}{\hskip0pt plus1filll}{}
\end{verbatim}
\end{command}

\begin{command}{\insertshortauthor}
  Inserts the short version of the author into a template.
\end{command}

\begin{command}{\insertshortdate}
  Inserts the short version of the date into a template.
\end{command}

\begin{command}{\insertshortinstitute}
  Inserts the short version of the institute into a template.
\end{command}

\begin{command}{\insertshortpart}
  Inserts the short version of the part name into a template.
\end{command}

\begin{command}{\insertshorttitle}
  Inserts the short version of the document title into a template.
\end{command}


\begin{command}{\insertsubsection}
  Inserts the current subsection into a template.
\end{command}

\begin{command}{\insertsubsectionnavigation\marg{width}}
  Inserts a vertical navigation bar containing all subsections of the
  current section, with the current subsection hilighted.
\end{command}

\begin{command}{\insertsubsectionnavigationhorizontal\marg{width}%
    \marg{left insert}\marg{right insert}}
  See |\insertsectionnavigationhorizontal|.
\end{command}


\begin{command}{\insertverticalnavigation\marg{width}}
  Inserts a vertical navigation bar of the given \meta{width} into a
  template. The bar shows a little table of contents. The individual
  lines are typeset using the templates
  |\usesectionsidetemplate| and |\usesubsectionsidetemplate|.
\end{command}

\begin{command}{\insertvrule\marg{color expression}\marg{thickness}}
  Inserts a rule of the given color and \meta{thickness} into a
  template. 
\end{command}





\subsubsection{Side Bars}

\label{section-sidebar-templates}

Side bars are vertical areas that stretch from the lower end of the
head line to the top of the foot line. There can be a side bar at the
left and one at the right (or even both). Side bars can show a table
of contents, but they could also be added for purely aesthetic
reasons.

When you install a side bar template, you must explicitly specify the
horizontal size of the side bar. The vertical size is determined
automatically. Each side bar can have its own background, which can be
setup using special side background templates.

Adding a sidebar of a certain size, say 1cm, will make the main text
1cm narrower. The distance between the inner side of a side
bar and the outer side of the text, as specified by
the command |\beamersetleftmargin| and its counterpart for the
right margin, is not changed when a side bar is installed.

Internally, the sidebars are typeset by showing them as part of the
headline. The \beamer\ class keeps track of six dimensions, three 
for each side: the variables |\beamer@leftsidebar| and
|\beamer@rightsidebar| store the (horizontal) sizes of the side
bars, the variables |\beamer@leftmargin| and
|\beamer@rightmargin| store the distance between sidebar and
text, and the macros |\Gm@lmargin| and  |\Gm@rmargin| store
the distance from the edge of the paper to the edge of the text. Thus
the sum |\beamer@leftsidebar| and |\beamer@leftmargin| is
exactly  |\Gm@lmargin|. Thus, if you wish to put some text right
next to the left side bar, you might write
|\hskip-\beamer@leftmargin| to get there.

In the following, only the commands for the left side bars are
listed. Each of these commands also exists for the right side bar,
with ``left'' replaced by ``right'' everywhere.


\begin{command}{\useleftsidebartemplate\marg{horizontal size}\marg{template}}
  When the side bar is typeset, the \meta{template} is invoked inside a
  |\vbox| of the height of the side bar. Thus, the below example
  will produce a side bar of half a centimeter width, in which the word
  ``top'' is printed just below the head line and ``bottom'' is printed
  just above the foot line.
  \example
\begin{verbatim}
\useleftsidebartemplate{1cm}{
  top
  \vfill
  bottom
}
\end{verbatim}
\end{command}

\begin{command}{\useleftsidebarbackgroundtemplate\marg{template}}
  The template is shown behind whatever is shown in the left side
  bar. 
  \example
\begin{verbatim}
\useleftsidebarbackgroundtemplate
  {\color{red}\vrule height\paperheight width\beamer@leftsidebar}
\end{verbatim}
\end{command}


\begin{command}{\useleftsidebarcolortemplate\marg{color expression}}
  Uses the given color as background for the side bar.
  \example
\begin{verbatim}
\useleftsidebarcolortemplate{\color{red}}
\useleftsidebarcolortemplate{\color[rgb]{1,0,0.5}}
\end{verbatim}
\end{command}

\begin{command}{\useleftsidebarverticalshadingtemplate\marg{bottom
      color expression}\marg{top color expression}}
  Installs a smooth vertical transition between the given colors as
  background for the side bar.
  \example
\begin{verbatim}
\useleftsidebarverticalshadingtemplate{white}{red}
\end{verbatim}
\end{command}


\begin{command}{\useleftsidebarhorizontalshadingtemplate\marg{left end
      color expression}\marg{right end color expression}}
  Installs a smooth horizontal transition between the given colors as
  background for the side bar.
  \example
\begin{verbatim}
\useleftsidebarhorizontalshadingtemplate{white}{red}
\end{verbatim}
\end{command}


\begin{command}{\usesectionsidetemplate\marg{current section
      template}\marg{other section template}}
  Both parameters should be |\hbox|es. The templates are used to
  typeset a section name inside a side navigation bar.
  \example
\begin{verbatim}
\usesectionsidetemplate
{\setbox\tempbox=\hbox{\color{black}\tiny{\kern3pt\insertsectionhead}}%
  \ht\tempbox=8pt%
  \dp\tempbox=2pt%
  \wd\tempbox=\beamer@sidebarwidth%
  \box\tempbox}
{\setbox\tempbox=\hbox{\color{structure!75}\tiny{\kern3pt\insertsectionhead}}%
  \ht\tempbox=8pt%
  \dp\tempbox=2pt%
  \wd\tempbox=\beamer@sidebarwidth%
  \box\tempbox}
\end{verbatim}
\end{command}



\begin{command}{\usesubsectionsidetemplate\marg{current subsection
      template}\marg{other subsection template}}
  See |\usesectionsidetemplate|.
  \example
\begin{verbatim}
\usesectionsidetemplate
{\setbox\tempbox=\hbox{\color{black}\tiny{\kern3pt\insertsectionhead}}%
  \ht\tempbox=8pt%
  \dp\tempbox=2pt%
  \wd\tempbox=\beamer@sidebarwidth%
  \box\tempbox}
{\setbox\tempbox=\hbox{\color{structure!75}\tiny{\kern3pt\insertsectionhead}}%
  \ht\tempbox=8pt%
  \dp\tempbox=2pt%
  \wd\tempbox=\beamer@sidebarwidth%
  \box\tempbox}
\end{verbatim}
\end{command}










\subsubsection{Buttons}
\label{section-navigation-buttons}

\paragraph{Predefined Templates}\ 

\begin{command}{\beamertemplateoutlinebuttons}
  Renders buttons as rectangles with rounded left and right
  border. Only the border (outline) is painted.
\end{command}

\begin{command}{\beamertemplatesolidbuttons}
  Renders buttons as filled rectangles with rounded left and right
  border.
\end{command}


\paragraph{Template Installation Commands}\ 

\begin{command}{\usebuttontemplate\marg{button template}}
  Installs a new button template. This template is invoked whenever a
  button should be rendered.
  \example
\begin{verbatim}
\usebuttontemplate{\color{structure}\insertbuttontext}
\end{verbatim}
\end{command}


\paragraph{Inserts}\ 

Inside the button template, the button text can be accessed via the
following command:

\begin{command}{\insertbuttontext}
  Inserts the text of the current button into a template. When called
  by  button creation commands, like |\beamerskipbutton|, the symbol
  will be part of this text.
\end{command}

The button creation commands automatically add the following three
inserts to the text to be rendered by |\insertbuttontext|:

\begin{command}{\insertgotosymbol}
  Inserts a small right-pointing arrow.
\end{command}

\begin{command}{\insertskipsymbol}
  Inserts a double right-pointing arrow.
\end{command}

\begin{command}{\insertreturnsymbol}
  Inserts a small left-pointing arrow.
\end{command}

You can redefine these commands to change these symbols.




\subsubsection{Navigation Bars}

\paragraph{Predefined Templates}\ 

\begin{command}{\beamertemplateboxminiframe}
  Changes the symbols in a navigation bar used to represent
  a frame to a small box.
\end{command}

\begin{command}{\beamertemplateticksminiframe}
  Changes the symbols in a navigation bar used to represent
  a frame to a small vertical bar of varying length.
\end{command}


\paragraph{Template Installation Commands}\ 

\begin{command}{\usesectionheadtemplate\marg{current section
      template}\marg{other section template}}
  The templates are used to render the section names in a navigation
  bar. 
  \example
\begin{verbatim}
\usesectionheadtemplate
{\color{structure}\tiny\insertsectionhead}
{\color{structure!50}\tiny\insertsectionhead}
\end{verbatim}
\end{command}
  

\begin{command}{\usesubsectionheadtemplate\marg{current subsection
      template}\marg{other subsection template}}
  See |\usesectionheadtemplate|.
  \example
\begin{verbatim}
\usesubsectionheadtemplate
{\color{structure}\tiny\insertsubsectionhead}
{\color{structure!50}\tiny\insertsubsectionhead}
\end{verbatim}
\end{command}

\begin{command}{\useminislidetemplate%
    \marg{template current frame icon}%
    \marg{template current subsection frame icon}\\%
    \marg{template other frame icon}%
    \marg{horizontal offset}%
    \marg{vertical offset}}
  The templates are used to draw frame icons in navigation bars. The
  offsets describe the offset between icons.
  \example
\begin{verbatim}
\useminislidetemplate
  {
    \color{structure}%
    \hskip-0.4pt\vrule height\boxsize width1.2pt%
  }  
  {%
    \color{structure}%
    \vrule height\boxsize width0.4pt%
  }
  {%
    \color{structure!50}%
    \vrule height\boxsize width0.4pt%
  }
  {.1cm}
  {.05cm}
\end{verbatim}
\end{command}




\subsubsection{Navigation Symbols}
\label{section-navigation-symbols-template}

\paragraph{Predefined Templates}\ 

\begin{command}{\beamertemplatenavigationsymbolsempty}
  Suppresses all navigation symbols.
\end{command}

\begin{command}{\beamertemplatenavigationsymbolsframe}
  Shows only the frame symbol as navigation symbol.
\end{command}

\begin{command}{\beamertemplatenavigationsymbolsvertical}
  Organizes the navigation symbols vertically.
\end{command}

\begin{command}{\beamertemplatenavigationsymbolshorizontal}
  Organizes the navigation symbols horizontally.
\end{command}



\paragraph{Template Installation Commands}\ 

\begin{command}{\usenavigationsymbolstemplate\marg{symbols template}}
  Installs a new symbols template. This template is invoked by themes
  at the place where the navigation symbols should be shown.
  \example
\begin{verbatim}
\usenavigationsymbolstemplate{\vbox{%
  \hbox{\insertslidenavigationsymbols}
  \hbox{\insertframenavigationsymbols}
  \hbox{\insertsubsectionnavigationsymbols}
  \hbox{\insertsectionnavigationsymbols}
  \hbox{\insertdocnavigationsymbols}
  \hbox{\insertbackfindforwardnavigationsymbols}}}
\end{verbatim}
\end{command}


\paragraph{Inserts for this Template}\ 

The following inserts are useful for the navigation symbols template:

\begin{command}{\insertslidenavigationsymbols}
  Inserts the slide navigation symbol, see
  Section~\ref{section-navigation-symbols}.
\end{command}

\begin{command}{\insertframenavigationsymbols}
  Inserts the frame navigation symbol, see
  Section~\ref{section-navigation-symbols}.
\end{command}

\begin{command}{\insertsubsectionnavigationsymbols}
  Inserts the subsection navigation symbol, see
  Section~\ref{section-navigation-symbols}.
\end{command}

\begin{command}{\insertsectionnavigationsymbols}
  Inserts the section navigation symbol, see
  Section~\ref{section-navigation-symbols}.
\end{command}

\begin{command}{\insertdocnavigationsymbols}
  Inserts the presentation navigation symbol and (if necessary) the
  appendix navigation symbol, see
  Section~\ref{section-navigation-symbols}.
\end{command}

\begin{command}{\insertbackfindforwardnavigationsymbols}
  Inserts a back, a find, and a forward navigation symbol, see
  Section~\ref{section-navigation-symbols}.
\end{command}





\subsubsection{Footnotes}

\paragraph{Template Installation Commands}\

\begin{command}{\usefootnotetemplate\marg{footnote template}}
  \example
\begin{verbatim}
\usefootnotetemplate{
  \parindent 1em
  \noindent
  \hbox to 1.8em{\hfil\insertfootnotemark}\insertfootnotetext}
\end{verbatim}
\end{command}


\paragraph{Inserts for these Templates}\

\begin{command}{\insertfootnotemark}
  Inserts the current footnote mark (like a raised number) into a
  template. 
\end{command}

\begin{command}{\insertfootnotetext}
  Inserts the current footnote text into a template. 
\end{command}





\subsubsection{Captions}
\label{section-template-caption}

\paragraph{Predefined Templates}\

\begin{command}{\beamertemplatecaptionwithnumber}
  Changes the caption template such that the number of the
  table or figure is also shown.
\end{command}

\begin{command}{\beamertemplatecaptionownline}
  Changes the caption template such that the word ``Table''
  or ``Figure'' has its own line.
\end{command}



\paragraph{Template Installation Commands}\

\begin{command}{\usecaptiontemplate\marg{caption template}}
  \example
\begin{verbatim}
\usecaptiontemplate{
  \small
  \structure{\insertcaptionname~\insertcaptionnumber:}
  \insertcaption
}
\end{verbatim}
\end{command}



\paragraph{Inserts for these Templates}\

\begin{command}{\insertcaption}
  Inserts the text of the current caption into a template.
\end{command}

\begin{command}{\insertcaptionname}
  Inserts the name of the current caption into a template. This word
  is either ``Table'' or ``Figure'' or, if the |babel| package is
  used, some translation thereof.
\end{command}

\begin{command}{\insertcaptionnumber}
  Inserts the number of the current figure or table into a template.
\end{command}






\subsubsection{Lists (Itemizations, Enumerations, Descriptions)}

\paragraph{Predefined Templates}\

\begin{command}{\beamertemplatedotitem}
  Changes the symbols shown in an |itemize|
  environment to dots.
\end{command}

\begin{command}{\beamertemplateballitem}
  Changes the symbols shown in an |itemize|
  environment to small plastic balls.
\end{command}


\paragraph{Template Installation Commands}\

\begin{command}{\useenumerateitemtemplate\marg{template}}
  The \meta{template} is used to render the default item in the top
  level of an enumeration. 
  \example |\useenumerateitemtemplate{\insertenumlabel}|
\end{command}


\begin{command}{\useitemizeitemtemplate\marg{template}}
  The \meta{template} is used to render the default item in the top
  level of an itemize list.
  \example |\useitemizeitemtemplate{\pgfuseimage{mybullet}}|
\end{command}


\begin{command}{\usesubitemizeitemtemplate\marg{template}}
  The \meta{template} is used to render the default item in the
  second level of an itemize list.
  \example |\usesubitemizeitemtemplate{\pgfuseimage{mysubbullet}}|
\end{command}

\begin{command}{\useitemizetemplate\marg{begin text}\marg{end text}}
  The \meta{begin text} is inserted at the beginning of a top-level
  itemize list, the \meta{end text} at its end.
  \example |\useitemizetemplate{}{}|
\end{command}

\begin{command}{\usesubitemizetemplate\marg{begin text}\marg{end text}}
  The \meta{begin text} is inserted at the beginning of a second-level
  itemize list, the \meta{end text} at its end.
  \example |\usesubitemizetemplate{\begin{small}}{\end{small}}|
\end{command}


\begin{command}{\useenumerateitemtemplate\marg{template}}
  The \meta{template} is used to render the default item in the
  top-level of an enumeration.  
  \example
  |\useenumerateitemtemplate{\insertenumlabel}|
\end{command}
\begin{command}{\usesubenumerateitemtemplate\marg{template}}
  The \meta{template} is used to render the default item in the second
  level of an enumeration. 
  \example
  |\usesubenumerateitemtemplate{\insertenumlabel-\insertsubenumlabel}|
\end{command}

\begin{command}{\useenumeratetemplate\marg{begin text}\marg{end text}}
  The \meta{begin text} is inserted at the beginning of a top-level
  enumeration, the \meta{end text} at its end.
  \example |\useenumeratetemplate{}{}|
\end{command}


\begin{command}{\usesubenumeratetemplate\marg{begin text}\marg{end text}}
  The \meta{begin text} is inserted at the beginning of a second-level
  enumeration, the \meta{end text} at its end.
  \example |\usesubenumeratetemplate{\begin{small}}{\end{small}}|
\end{command}


\begin{command}{\usedescriptiontemplate\marg{description
      template}\marg{default width}}
  The \meta{default width} is used as width of the default item, if no
  other width is specified; the width |\labelsep| is
  automatically added to this parameter.
  \example
  |\usedescriptionitemtemplate{\color{structure}\insertdescriptionitem}{2cm}|
\end{command}


\paragraph{Inserts for these Templates}\

\begin{command}{\insertdescriptionitem}
  Inserts the current item of a description environment into a
  template.
\end{command}

\begin{command}{\insertenumlabel}
  Inserts the current number of the top-level enumeration (as an
  Arabic number) into a template.
\end{command}


\begin{command}{\insertsubenumlabel}
  Inserts the current number of the second-level enumeration (as an
  Arabic number) into a template.
\end{command}





\subsubsection{Hilighting Commands}

\paragraph{Template Installation Commands}\

\begin{command}{\usealerttemplate\marg{alert template}}
  \example |\usealerttemplate{{\color{red}\insertalert}}|
\end{command}


\begin{command}{\usestructuretemplate\marg{structure template}}
  \example |\usestructuretemplate{{\color{structure}\insertstructure}}|
\end{command}


\paragraph{Inserts for these Templates}\

\begin{command}{\insertalert}
  Inserts the current alerted text into a template.
\end{command}

\begin{command}{\insertstructure}
  Inserts the current structure text into a template.
\end{command}






\subsubsection{Block Environments}

\paragraph{Predefined Templates}\

\begin{command}{\beamertemplateroundedblocks}
  Changes the block templates such that they are printed on a
  rectangular area with rounded corners. Quite cute.
\end{command}



\paragraph{Template Installation Commands}\

\begin{command}{\useblocktemplate\marg{block beginning
      template}\marg{block end template}}
  \example
\begin{verbatim}
\useblocktemplate
  {%
   \medskip%
    {\color{blockstructure}\textbf{\insertblockname}}%
    \par%
  }
  {\medskip}
\end{verbatim}
\end{command}


\begin{command}{\usealertblocktemplate\marg{block beginning
      template}\marg{block end template}}
  \example
\begin{verbatim}
\usealertblocktemplate
  {%
    \medskip
    {\alert{\textbf{\insertblockname}}}%
  \par}
  {\medskip}
\end{verbatim}
\end{command}


\begin{command}{\useexampleblocktemplate\marg{block beginning
      template}\marg{block end template}}
  \example
\begin{verbatim}
\useexampleblocktemplate
  {%
    \medskip
    \begingroup\color{darkgreen}{\textbf{\insertblockname}}
    \par}
  {%
     \endgroup
     \medskip
  }
\end{verbatim}
\end{command}


\paragraph{Inserts for these Templates}\

\begin{command}{\insertblockname}
  Inserts the name of the current block into a template.
\end{command}


\end{document}
