\documentclass{article}

% Copyright 2003 by Till Tantau <tantau@cs.tu-berlin.de>.
%
% This program can be redistributed and/or modified under the terms
% of the LaTeX Project Public License Distributed from CTAN
% archives in directory macros/latex/base/lppl.txt.

\usepackage{pgf}
\usepackage[left=2.5cm,right=2.5cm,top=2.5cm,bottom=2.5cm,nohead]{geometry}
\usepackage{amsmath,amssymb}
\usepackage[pdfborder={0 0 0}]{hyperref}

\pgfdeclareimage{themebars}{6.66666cm}{5cm}{themebars}
\pgfdeclareimage{themebars2}{6.66666cm}{5cm}{themebars2}
\pgfdeclareimage{themeboxes}{6.66666cm}{5cm}{themeboxes}
\pgfdeclareimage{themeboxes2}{6.66666cm}{5cm}{themeboxes2}
\pgfdeclareimage{themeclassic}{6.66666cm}{5cm}{themeclassic}
\pgfdeclareimage{themeclassic2}{6.66666cm}{5cm}{themeclassic2}
\pgfdeclareimage{themelined}{6.66666cm}{5cm}{themelined}
\pgfdeclareimage{themelined2}{6.66666cm}{5cm}{themelined2}
\pgfdeclareimage{themeplain}{6.66666cm}{5cm}{themeplain}
\pgfdeclareimage{themeplain2}{6.66666cm}{5cm}{themeplain2}
\pgfdeclareimage{themesidebar}{6.66666cm}{5cm}{themesidebar}
\pgfdeclareimage{themesidebar2}{6.66666cm}{5cm}{themesidebar2}
\pgfdeclareimage{themesidebardark}{6.66666cm}{5cm}{themesidebardark}
\pgfdeclareimage{themesidebardark2}{6.66666cm}{5cm}{themesidebardark2}
\pgfdeclareimage{themesidebartab}{6.66666cm}{5cm}{themesidebartab}
\pgfdeclareimage{themesidebartab2}{6.66666cm}{5cm}{themesidebartab2}
\pgfdeclareimage{themesidebardarktab}{6.66666cm}{5cm}{themesidebardarktab}
\pgfdeclareimage{themesidebardarktab2}{6.66666cm}{5cm}{themesidebardarktab2}
\pgfdeclareimage{themesplit}{6.66666cm}{5cm}{themesplit}
\pgfdeclareimage{themesplit2}{6.66666cm}{5cm}{themesplit2}
\pgfdeclareimage{themesplitcondensed}{6.66666cm}{5cm}{themesplitcondensed}
\pgfdeclareimage{themesplitcondensed2}{6.66666cm}{5cm}{themesplitcondensed2}
\pgfdeclareimage{themetree}{6.66666cm}{5cm}{themetree}
\pgfdeclareimage{themetree2}{6.66666cm}{5cm}{themetree2}
\pgfdeclareimage{themetreebars}{6.66666cm}{5cm}{themetreebars}
\pgfdeclareimage{themetreebars2}{6.66666cm}{5cm}{themetreebars2}

\def\beamer{\textsc{beamer}}
\def\pdf{\textsc{pdf}}
\def\pgf{\textsc{pgf}}
\def\pstricks{\textsc{pstricks}}
\def\bs{$\backslash$}

\def\Theme#1{\par\bigskip\noindent\textbf{Theme \texttt{#1}}\par}
\def\ClassOption#1{\par\bigskip\noindent\textbf{Class Option \texttt{#1}}\par}
\def\Environment#1{\par\bigskip\noindent\textbf{Environment \texttt{#1}}\par}
\def\Command#1{\par\bigskip\noindent\textbf{Command \texttt{#1}}\par}
\long\def\Parameters#1{\medskip\noindent Parameters:
  \begin{enumerate}\itemsep=0pt\parskip=0pt
    #1
  \end{enumerate}}
\long\def\Description#1{\unskip\medskip\noindent Description: #1}
\def\Example{\par\medskip\noindent Example: }

\begin{document}

\title{User's Guide to the Beamer Class, Version 0.92\\
\Large\href{http://latex-beamer.sourceforge.net}{\texttt{http://latex-beamer.sourceforge.net}}}
\author{Till Tantau\\
  \href{mailto:tantau@users.sourceforge.net}{\texttt{tantau@users.sourceforge.net}}}

\maketitle

\tableofcontents



\section{Introduction}


\subsection{Overview}

This user's guide explains the functionality of the \beamer\ class.
It is a \LaTeX\ class that allows you to create a beamer
presentation. It can also be used to create slides. It behaves
similarly to other packages like \textsc{prosper}, but has the
advantage that it works together directly with \texttt{pdflatex}, but
also with \texttt{dvips}.

To use the \beamer\ class, proceed as follows:
\begin{enumerate}
\item
  Specify \texttt{beamer} as document class instead of
  \texttt{article}.
\item
  Structure your \LaTeX\ text using \texttt{section} and
  \texttt{subsection} commands.
\item
  Place the text of the individual slides inside \texttt{frame}
  commands.
\item
  Run \texttt{pdflatex} on the text (or \texttt{latex},
  \texttt{dvips}, and \texttt{ps2pdf}).
\end{enumerate}

The \beamer\ class has several useful features: You don't need any
external programs to use it other than \texttt{pdflatex}, but it works
also with \texttt{dvips}. You can easily and intuitively create
sophisticated overlays. Finally, you can easily change the whole slide
theme or only parts of it. The following code shows a typical usage of
the class.

\begin{verbatim}
\documentclass{beamer}

\usepackage{beamerthemesplit}

\title{Example Presentation Created with the Beamer Package}
\author{Till Tantau}
\date{\today}

\begin{document}

\frame{\titlepage}

\section[Outline]{}
\frame{\tableofcontents}

\section{Introduction}
\subsection{Overview of the Beamer Class}
\frame
{
  \frametitle{Features of the Beamer Class}

  \begin{itemize}
  \item<1-> Normal LaTeX class.
  \item<2-> Easy overlays.
  \item<3-> No external programs needed.      
  \end{itemize}
}
\end{document}
\end{verbatim}

Run \texttt{pdflatex} on this code (twice) and then use, for example, the
Acrobat Reader to present the resulting \texttt{.pdf} file in a
presentation. You can also, alternatively, use \texttt{dvips}; see
Section~\ref{section-postscript} for details.

As can be seen, the text looks almost like a normal \LaTeX\ text. The
main difference is the usage of the \verb!\frame! command. This
command takes one parameter, which is the text that should be shown on
the frame. Typically, the contents of a frame is shown on a single
slide. However, in case you use overlay commands inside a frame, a
single frame command may produce several slides. An example is the
last frame in the above example. There, the \verb!\item! commands
are followed by \emph{overlay specifications} like \verb!<1->!,
which means ``from slide 1 on.'' Such a specification causes the item
to be shown only on the specified slides of the frame (see
Section~\ref{section-overlay} for details). In the above example, a
total of five slides are produced: a title page slide, an outline
slide, a slide showing only the first of the three items, a slide
showing the first two of them, and a slide showing all three items.
 
To structure your text, you can use the commands \verb!\section! and
\verb!\subsection!. These commands will not only create a table of
contents, but will also create navigation bars.



\subsection{Getting Started: Installation}

To use the beamer class, you just need to put the files of the
\beamer\ package in a directory that is read by \TeX. To uninstall the
class, simply remove these files once more. The same is true of the
\textsc{pgf} package, which you will also need.


Unfortunately, there are different ways of making \TeX\ ``aware'' of
the files in the \beamer\ package. Which way you should choose depends
on how permanently you intend to use the class.


\subsubsection{Installing Debian and Red Hat Packages}

Currently, there are no out-of-the-box Debian or Red Hat packages of
the beamer class available.



\subsubsection{Temporary Installation}

If you only wish to install the beamer class for a quick appraisal, do
the following: Obtain the latest source version (ending
\texttt{.tar.gz}) of the \beamer\ package from 
\href{http://sourceforge.net/projects/latex-beamer/}{\texttt{http://sourceforge.net/projects/latex-beamer/}}
(most likely, you have already done this). Next, you also need at
least version 0.34 of the \textsc{pgf} package, which can be found at
the same place. Finally, you need at least version 1.03 of the
\textsc{xcolor} package, which can also be found at that place
(although the version on CTAN might be newer).

In all cases, the packages contain a bunch of files (for the \beamer\
class, \texttt{beamer.cls} is one of these files and happens to be the
most important one, for the \textsc{pgf} package \texttt{pgf.sty} is
the most important file). 
Place all files in three directories. For example,
\texttt{/home/tantau/beamer/}, \texttt{/home/tantau/pgf/}, and
\texttt{/home/tantau/xcolor/} would work fine for me. Then setup the
environment variable called \texttt{TEXINPUTS} to be the following
string (how exactly this is done depends on your operating system and
shell): 

\begin{verbatim}
.:/home/tantau/beamer:/home/tantau/pgf:/home/tantau/xcolor:
\end{verbatim}

Naturally, if the \texttt{TEXINPUTS} variable is already defined
differently, you should \emph{add} the two directories to the list. Do
not forget to place a colon at the end (corresponding to an empty
path), which will include all standard directories.



\subsubsection{Installation in a texmf Tree}

For a more permanent installation, you can place the files of the
\beamer\ package and of the \textsc{pgf} package (see the previous
subsection on how to obtain them) in an appropriate \texttt{texmf}
tree. 

When you ask \TeX\ to use a certain class or package, it usually looks
for the necessary files in so-called \texttt{texmf} trees. These trees
are simply huge directories that contain these files. By default,
\TeX\ looks for files in three different \texttt{texmf} trees:
\begin{itemize}
\item
  The root \texttt{texmf} tree, which is usually located at
  \texttt{/usr/share/}, \verb!c:\texmf\!, or\\
  \verb!c:\Program Files\TeXLive\texmf\!.
\item
  The local  \texttt{texmf} tree, which is usually located at
  \texttt{/usr/local/share/}, \verb!c:\localtexmf\!, or\\
  \verb!c:\Program Files\TeXLive\texmf-local\!.
\item
  Your personal  \texttt{texmf} tree, which is located in your home
  directory.   
\end{itemize}

You should install the packages either in the local tree or in
your personal tree, depending on whether you have write access to the
local tree. Installation in the root tree can cause problems, since an
update of the whole \TeX\ installation will replace this whole tree.

Inside whatever \texttt{texmf} directory you have chosen, create
the sub-sub-sub-directories
\begin{itemize}
\item
  \texttt{texmf/tex/latex/beamer} and
\item
  \texttt{texmf/tex/latex/pgf}
\item
  \texttt{texmf/tex/latex/xcolor}
\end{itemize}
and place all files in these three directories.

Finally, you need to rebuild \TeX's filename database. This done by
running the command  \texttt{texhash} or \texttt{mktexlsr} (they are
the same). In MikTeX, there is a menu option to do this.

\vskip1em
For a more detailed explanation of the standard installation process
of packages, you might wish to consult
\href{http://www.ctan.org/installationadvice/}{\texttt{http://www.ctan.org/installationadvice/}}.
However, note that the \beamer\ package does not come with a
\texttt{.ins} file (simply skip that part).



\subsection{Testing the Installation}

To test your installation, copy the file \texttt{beamerexample.tex}
from the documentation subdirectory to some place where you usually
create presentations. Then run the command \texttt{pdflatex} twice on
the file and check whether the resulting \texttt{beamerexample.pdf}
looks correct. If so, you are all set.

If you have updated from a previous version and you have trouble
\TeX ing some old file, it sometimes help to delete all the extra
files \TeX\ creates automatically (like the \texttt{.aux} and
\texttt{.head} files).



\section{Workflow}

This section presents a possible workflow for creating a beamer
presentation and possibly a handout to go along with it. Technical
questions are addressed, like which programs to call with 
which parameters, and hints are given on how to create a
presentation. If you have already created numerous presentations, you
may wish to skip the first of the following steps 
and only have a look at how to convert the \texttt{.tex} file into a
\texttt{.pdf} of \texttt{.ps} file.


\subsection{Step Zero: Know the Time Constraints}

When you start to create a presentation, the very first thing you
should worry about is the amount of time you have for your
presentation. Depending on the occasion, this can
be anything between 2 minutes and two hours. A simple rule for the
number of frames is that you should have at most one frame per
minute.

In most situations, you will have less time for your presentation that
you would like. \emph{Do not try to squeeze more into a
  presentation than time allows for.} No matter how important some
detail seems to you, it is better to leave it out, but get the main
message across, than getting neither the main message nor the detail
across. 

In many situations, a quick appraisal of how much time you have will
show that you won't be able to mention certain details. Knowing this can
save you hours of work on preparing slides that you would have to remove
later anyway.




\subsection{Step One: Setup the Files}

It is advisable that you create a folder for each
presentation. Even though your presentation will usually reside in a
single file, \TeX\ produces so many extra files that things can easily
get very confusing otherwise. The folder's name should ideally start
with the date of your talk in ISO format (like 2003-12-25 for a
Christmas talk), followed by some reminder text of what the talk is
all about. Putting the date at the front in this format causes your
presentation folders to be listed nicely when you have several of them
residing in one directory. If you use an extra directory for each
presentation, you can call your main file
\verb!main.tex!. 

To create an initial \verb!main.tex! file for your talk, copy an
existing file (like the file \verb!beamerexample.tex! that comes along
with the contribution) and delete everything that is not going to be
part of your talk. Adjust the \verb!\author{}! and other fields as 
appropriate. 





\subsection{Step Two: Structure You Presentation}

With the time constraints in mind, make a mental inventory of the
things you can reasonably talk about within the time available. Then
categorize the inventory into sections and subsections. Put
\verb!\section{}! and \verb!\subsection{}! commands into the (more or
less empty) main file. Do not create any frames until you
have a first working version of a possible table of contents. Do not
feel afraid to change it later on as you work on the talk.

You should not use more than four sections and not less than two. Even
four sections are usually too much, unless they follow a very easy
pattern. Five and more sections are simply too hard to remember for the
audience. After all, when you present the table of contents, the
audience will not yet really be able to grasp the importance and
relevance of the different sections and will most likely have
forgotten them by the time you reach them.

Ideally, a table of contents should be understandable by itself. In
particular, it should be comprehensible \emph{before} someone has
heard your talk. Keep section and subsection titles self-explaining.

Both the sections and the subsections should follow a logical
pattern. Begin with an explanation of what your talk is all about. (Do
not assume that everyone knows this. The Ignorant Audience Law states:
The audience always knows less than you think it should know, even if
you take the Ignorant Audience Law into account.) Then explain what
you or someone else has found out concerning the subject
matter. Always conclude your talk with a summary that repeats the main
message of the talk in a short and simple way. People pay most
attention at the beginning and at the end of talks. The summary is
your ``second chance'' to get across a message.




\subsection{Step Three: Creating a PDF or PostScript File}

Once a first version of the structure is finished, you should create a
first PDF or PostScript file of your (still empty) talk. This file
will only contain the title page and the table of contents. The file
might  look like this:

\begin{verbatim}
\documentclass{beamer}
% This is the file main.tex

\usepackage{beamerthemesplit}

\title{Example Presentation Created with the Beamer Package}
\author{Till Tantau}
\date{\today}

\begin{document}

\frame{\titlepage}

\section[Outline]{}
\frame{\tableofcontents}

\section{Introduction}
\subsection{Overview of the Beamer Class}
\subsection{Overview of Similar Classes}

\section{Usage}
\subsection{...}
\subsection{...}

\section{Examples}
\subsection{...}
\subsection{...}

\end{document}
\end{verbatim}



\subsubsection{Creating PDF}

To create a \texttt{PDF} version of this file, run the program
\verb!pdflatex! on \verb!main.tex! at least twice. Your need to run it
twice, so that \TeX\ can create the table of contents. In the
following example, the greater-than sign is the prompt.

\begin{verbatim}
> pdflatex main.tex
    ... lots of output ...
> pdflatex main.tex
    ... lots of output ...
\end{verbatim}


You can next use a program like the Acrobat Reader or \texttt{xpdf}
to view the resulting presentation.

\begin{verbatim}
> acroread main.pdf
\end{verbatim}

When printing a presentation, make sure that the option ``expand small
pages to paper size'' is enabled. This is necessary, because slides are
only 128mm times 96mm.



\subsubsection{Creating PostScript}
\label{section-postscript}

To create a PostScript version, you should first ascertain that the
\textsc{hyperref} package (which is automatically loaded by the
\beamer\ class) uses the option \texttt{dvips} or some compatible
option, see the documentation of the \textsc{hyperref} package for
details. Whether this is the case depends on the contents of your
local \texttt{hyperref.cfg} file. You can enforce the usage of this
option by passing \texttt{dvips} or a compatible option to the
\beamer\ class (write \verb!\documentclass[dvips]{beamer}!), which
will pass this option on to the \textsc{hyperref} package.

You can then run \verb!latex! twice, followed by \verb!dvips!.

\begin{verbatim}
> latex main.tex
    ... lots of output ...
> latex main.tex
    ... lots of output ...
> dvips -P pdf main.dvi
\end{verbatim}

The option (\verb!-P pdf!) tells \verb!dvips! to use
Type~1 outline fonts instead of the usual Type~3 bitmap fonts. You may
wish to omit this option if there is a problem with it. 

If you wish each slide to completely fill a letter-sized page, use the
following commands instead:

\begin{verbatim}
> dvips -P pdf -tletter main.dvi -o main.temp.ps
> psnup -1 -W128mm -H96mm -pletter main.temp.ps main.ps
\end{verbatim}

For A4-sized paper, use:

\begin{verbatim}
> dvips -P pdf -ta4 main.dvi -o main.temp.ps
> psnup -1 -W128mm -H96mm -pa4 main.temp.ps main.ps
\end{verbatim}

In order to create a white margin around the whole page (which is sometimes
useful for printing), add the option \verb!-m 1cm! to the options of
\verb!psnup!. 

To put two or four slides on one page, use \verb!-2!, respectively
\verb!-4! instead of \verb!-1! as the first parameter for
\verb!psnup!. In this case, you may wish to add the option
\verb!-b 1cm! to add a bit of space around the individual slides.

You can convert a PostScript file to a pdf file using

\begin{verbatim}
> ps2pdf main.ps main.pdf
\end{verbatim}



\subsection{Step Four: Create Frames}

Once the table of contents looks satisfactory, start creating frames
for your presentation. In the following, some guidelines that I stick
to are given on what to put on slides and what not to put. You can
certainly ignore any of these guideline, but you should be aware of
it when you ignore a rule and you should be able to justify it to
yourself. 


\subsubsection{Guidelines on What to Put on a Frame}

\begin{itemize}
\item
  A frame with too little on it is better than a
  frame with too much on it.
\item
  Do not assume that everyone in the audience is an expert on the
  subject matter. (Remember the Ignorant Audience Law.) Even if the
  people listening to you should be experts, they may last have heard
  about things you consider obvious several years ago. You should
  always have the time for a quick reminder of what exactly a
  ``semantical complexity class'' or an ``$\omega$-complete partial
  ordering'' is.
\item
  Never put anything on a slide that you are not going to explain
  during the talk, not even to impress anyone with how
    complicated your subject matter really is. However, you may
  explain things that are not on a slide.
\item
  Keep it simple. Typically, your audience will see a slide for less
  than 50 seconds. They will not have the time to puzzle through long
  sentences or complicated formulas.
\end{itemize}



\subsubsection{Guidelines on Text}

\begin{itemize}
\item
  Put a title on each frame. The title explains the contents of the
  frame to people who did not follow all details on the slide.
\item
  Ideally, titles on consecutive frames should ``tell a story'' all by
  themselves.
\item
  \emph{Never} use a smaller font size to ``squeeze more on a frame.''
\item
  Prefer enumerations and itemize environments over plain text. Do not
  use long sentences.
\item
  Text and numbers in figures should have the \emph{same} size as
  normal text. Illegible numbers on axes usually ruin a chart and its
  message. 
\end{itemize}


\subsubsection{Guidelines on Graphics}

\begin{itemize}
\item
  Put (at least) one graphic on each slide, whenever
  possible. Visualizations help an audience enormously.
\item
  Usually, place graphics to the left of the text. (Use the
  \texttt{columns} environment.) 
\item
  Graphics should have the same typographic parameters as the
  text: Use the same fonts (at the same size) in graphics as in the
  main text. A small dot in a graphic should have exactly the same 
  size as a small dot in a text. The line width should be the same as
  the stroke width used in creating the glyphs of the font. For
  example, an 11pt non-bold Computer Modern font has a stroke width of
  0.4pt.
\item
  While bitmap graphics, like photos, can be much more colorful than the
  rest of the text, vector graphics should follow the same ``color
  logic'' as the main text (like black~= normal lines, red~= hilighted
  parts, green~= examples, blue~= structure).
\item
  Like text, you should explain everything that is shown on a
  graphic. Unexplained details make the audience puzzle whether this
  was something important that they have missed. Be careful when
  importing graphics from a paper or some other source. They usually
  have much more detail than you will be able to explain.
\end{itemize}

For technical hints on how to create graphics, see
Section~\ref{section-graphics}.


\subsubsection{Guidelines on Colors}

\begin{itemize}
\item
  Use colors sparsely. The prepared themes are already quite
  colorful (blue~= structure, red~= alert, green~= example). If you
  add more colors, you should have a \emph{very} good reason.
\item
  Be careful when using bright colors on white background,
  \emph{especially} when using green. What looks good on your monitor
  may look bad during a presentation due to the different ways
  monitors, beamers, and printers reproduce colors. Add lots of black
  to pure colors when you use them on bright backgrounds.
\item
  Maximize contrast. Normal text should be black on white or at least
  something very dark on something very bright. \emph{Never} do things
  like ``light green text on not-so-light green background.''
\item
  Background shadings decrease the legibility without increasing the
  information content. Do not add a background shading just because it
  ``somehow looks nicer.''
\item
  Inverse video (bright text on dark background) can be a problem
  during presentations in bright environments since only a small
  percentage of the presentation area is light up by the
  beamer. Inverse video is harder to reproduce on printouts and on
  transparencies. 
\end{itemize}


\subsubsection{Guidelines on Animations and Special Effects}

\begin{itemize}
\item
  Use animations to explain the dynamics of systems, algorithms, etc.
\item
  Do \emph{not} use animations just to attract the attention of your
  audience. This often distracts attention away from the main topic of the
  slide.
\item
  Do \emph{not} use distracting special effects like ``dissolving''
  slides unless you have a very good reason for using them. If you use
  them, use them sparsely. 
\end{itemize}



\subsection{Step Five: Test Your Presentation}

\emph{Always} test your presentation. For this, you should
vocalize or subvocalize your talk in a quite environment. Typically,
this will show that your talk is too long. You should then remove
parts of the presentation, such that it fits into the allotted time
slot. Do \emph{not} attempt to talk faster in order to squeeze the
talk into the given amount of time. You are almost sure to loose your
audience this way.

Do not try to create the ``perfect'' presentation immediately. Rather,
test and retest the talk and modify it as needed. 




\subsection{Step Six: Optionally Create a Handout}

Once your talk is fixed, you can create a handout, if this seems
appropriate. For this, use the class option \verb!handout! as
explained in Section~\ref{handout}. Typically, you might wish
to put several handout slides on one page. See
Section~\ref{section-postscript} on how to do this.







\section{Frames and Overlays}

\label{section-overlay}

\subsection{Frames}

\subsubsection{Frame Creation}

A presentation consists of a series of frames. Each frame consists of
a series of slides. You create a frame using the command
\verb!\frame!. This command takes one parameter, namely the
contents of the frame. All of this text that is not tagged by overlay
specifications (see Section~\ref{subsection-overlay}) is shown on all
slides of the frame.  

\Command{frame}
\Parameters{
\item
  optional parameter in square brackets: a specification of slides to
  be shown, see subsection \ref{subsection-restriction} for details. 
\item
  the frame's contents.
}
\Example
\begin{verbatim}
\frame
{
  Some text...

  Some more...
}
\end{verbatim}

\Command{plainframe}
\Parameters{
\item
  optional parameter in square brackets: a specification of slides to
  be shown, see subsection \ref{subsection-restriction} for details. 
\item
  the frame's contents.
}
\Description{
  This command creates a frame in which the head lines, foot lines,
  and side bars are suppressed. This is useful for creating single
  frames with different head and foot lines or for creating frames
  showing big pictures that completely fill the frame.
  }
\Example A frame with a picture completely filling the frame:  
\begin{verbatim}
\pgfdeclareimage{bigimage}{}{9.6cm}{bigimagefilename}
\plainframe{\hfill\pgfuseimage{bigimage}\hfill}
\end{verbatim}

\Example A title page, in which the head and foot lines are replaced
by two graphics.
\begin{verbatim}
\usetitlepagetemplate{
  \beamerline{\pgfuseimage{toptitle}}
  \vskip0pt plus 1filll

  \begin{centering}
    \Large{\textbf{\inserttitle}}
    
    \insertdate
  \end{centering}

  \vskip0pt plus 1filll
  \beamerline{\pgfuseimage{bottomtitle}}
}

\begin{document}
\plainframe{\titlepage}
\end{verbatim}


\subsubsection{Components of a Frame}

Each frame consists of up to six components:
\begin{enumerate}\itemsep=0pt\parskip=0pt
\item a head line,
\item a foot line,
\item a left side bar,
\item a right side bar,
\item a frame title, and
\item some frame contents.
\end{enumerate}

A frame need not have all of these components. Usually, the first four
components are automatically setup by the theme you are
using. To change them, you must install an appropriate template, see
Section~\ref{section-head-templates} for the head and foot lines and
Section~\ref{section-sidebar-templates} for the side bars.

The frame title is shown prominently at the top of the frame. To
specify the title, use the command \verb!\frametitle!. You should end
the frame title with a period, if the title is a proper
sentence. Otherwise, there should not be a period.

\Command{frametitle}
\Parameters{
\item a title for the frame.
}
\Example
\begin{verbatim}
\frame{
  \frametitle{A Frame Title is Important.}

  Frame contents.
}
\end{verbatim}



\subsubsection{Restricting the Slides of a Frame}
\label{subsection-restriction}

As mentioned above, the number of slides in a frame is automatically
calculated. If the largest number mentioned in any
specification is 4, four slides are introduced (despite the fact
that a specification like \verb!<4->! might suggest that more than
four slides would be possible).

You can also specify the number of slides in the frame ``by hand.'' To
do so, you pass an optional argument to the \verb!\frame! command,
given in \emph{square} brackets. This argument is also a 
slide specification. The frame will contain only the slides
specified in this argument. Consider the following example.

\begin{verbatim}
\frame[1-2,4-]
{
  This is slide number \only<1>{1}\only<2>{2}\only<3>{3}%
  \only<4>{4}\only<5>{5}.
}
\end{verbatim}
This command will create a frame containing four slides. The first
will contain the text ``This is slide number~1,'' the second ``This is
slide number~2,'' the third ``This is slide number~4,'' and the fourth
``This is slide number~5.''



\subsubsection{Verbatim Commands and Listings inside Frames}

The \verb!\verb! command, the verbatim environment, the lstlisting
environment, and related environments work only in frames that contain
a single slide. Furthermore, you must explicitly specify that the
frame contains only one slide; like this: 
\begin{verbatim}
\frame[all:1]
{
  \frametitle{Our Search Procedure}

\begin{verbatim}
  int find(int* a, int n, int x)
  {
    for (int i = 0; i<n; i++)
      if (a[i] == x)
        return i;
  }
\end{verbatim}
\unskip\verb!  \end{verbatim}!
\begin{verbatim}
}
\end{verbatim}

Instead of \verb!\frame[all:1]! you could also have specified
\verb!\frame[1]!, but this works only for the presentation version of
the talk, not for the handout version. To make verbatim accessible
also in the handout version, you would have to specify
\verb!\frame[1| handout: 1]! and even more if you also have a
transparencies version. The specification \verb!\frame[all:1]! states
that the frame has just one slide in all versions.

If you need to use verbatim commands in frames that contain several
slides, you must \emph{declare} your verbatim texts before the frame
starts. This is done using two special commands:


\Command{defverb}
\Parameters{
\item command name (including a backslash)
\item a one-line verbatim text, delimited by a special symbol (works
  like the \texttt{verb} command). Adding a star before the second
  parameter make spaces visible.
}
\Description{
  Declares a verbatim text for later use. The declaration should be
  done outside the frame. Once declared, the text can be used
  in overlays like normal text.
  }
\Example
\begin{verbatim}
\defverb\mytext!int main (void) { ...!
\defverb\mytextspaces*!int  main  (void ){  ...!

\frame
{
  \begin{itemize}
  \item<1-> In C you need a main function.
  \item<2-> It is declare like this: \mytext
  \item<3-> Spaces are not important: \mytextspaces
  \end{itemize}
}
\end{verbatim}


\Command{defverbatim}
\Parameters{
\item command name (including a backslash)
\item a normal parameter that contains a \texttt{verbatim},
  \texttt{verbatim*}, \texttt{lstlisting}, or a related environment. 
}
\Description{
  Declares a verbatim environment for later use. The declaration
  should be done outside the frame. Once declared, the text can be
  used in overlays like normal text.
  }
\Example
\begin{verbatim}
\defverbatim\algorithm{
\begin{verbatim}
int main (void)
{
  cout << "Hello world." << endl;
  return 0;
}
\end{verbatim}
\unskip\verb!\end{verbatim}}!
\begin{verbatim}
\frame
{
  Our algorithm:

  \alert<1>{\algorithm}

  \uncover<2>{Note the return value.}
}
\end{verbatim}



\subsection{Overlays}

\subsubsection{The Pauses Environment}

The \texttt{pauses} environment offers an easy, but not very flexible
way of creating frames that are uncovered piecewise. The environment
itself does not have an immediate effect. But if you use the command
\verb!\pause! inside the environment, only the text of the environment
up to the \verb!\pause! command is shown on the first slide. On the
second slide, everything is shown up to the second \verb!\pause!, and
so forth. Note that the \verb!\pause! command can only be used on the 
same level of nesting as the \texttt{pauses} environment.

A much more fine-grained control over what is shown on each slide can
be attained using overlay specifications, see the next
subsections. However, for many simple cases the \verb!\pause!
command is sufficient.

If you use multiple  \texttt{pauses} environments on one frame, the
slide counting for the second environment starts where the first one
left off, see the following example. You can nest \texttt{pauses}
environments, but this will not always have the effect you might
expect. 

\begin{verbatim}
\frame{
  \begin{pauses}
    Shown from first slide on.
    \pause
    Shown from second slide on.
    \pause
    Shown from third slide on.
  \end{pauses}

  Shown from first slide on (not affected by the environment).

  \begin{pauses}
    Shown from third slide on. (continued from above)
    \pause
    Shown from fourth slide on.
  \end{pauses}
}
\end{verbatim}

As a convenience, a \texttt{pauses} environment is automatically setup
inside each frame, each \texttt{itemize}, each \texttt{description},
and each \texttt{enumerate}. Thus, by simply using the \verb!\pause!
command on the outermost level of any frame or after items in lists or
descriptions, you uncover the rest of the frame or list only on the
next slide.

\Environment{pauses}
\Parameters{
\item Put the text before first \texttt{pause} from this slide on,
then continue increasing the slide number. Optional parameter, given
in square brackets.
}
\Description{
  The content of the environment is shown piecewise. Each
  \texttt{pause} command used inside uncovers a bit more of the
  environment's text. The optional parameter's main use is to set is
  to~0. The effect of this is that the first \texttt{pause} has no
  effect, which can be useful if the \texttt{pauses} environment
  immediately starts with a \texttt{pause} command. This happens
  sometimes, when the environment's content is created automatically.
}
\Example
\begin{verbatim}
\frame
{
  \begin{pauses}
    Shown from slide 1 onward.
    \pause

    Shown from slide 2 onward.
  \end{pauses}
}
\end{verbatim}
As mentioned above, in the above example the \texttt{pause}
environment could also have been omitted, as the \verb!\frame! command
inserts it automatically.


\Command{pause}
\Description{
  When used inside a \texttt{pauses} environment, this command causes
  the text following it to be shown only from the next slide on.
}
\begin{verbatim}
\frame
{
  \begin{itemize}
  \item
    A    
    \pause
  \item
    B
    \pause
  \item
    C
  \end{itemize}
}
\end{verbatim}



\subsubsection{Commands with Overlay Specifications}
\label{subsection-overlay}

An overlay specification is a comma-separated list of slides and
ranges. Ranges are specified like this: \verb!2-5!, which
means slide two through to five. The start or the beginning of a range
can be omitted. For example, \verb!3-! means ``slides three, four,
five, and so on'' and \verb!-5! means the same as \verb!1-5!. A
complicated example is \verb!-3,6-8,10,12-15!, which selected the
slides 1, 2, 3, 6, 7, 8, 10, 12, 13, 14, and 15.

Overlay specifications can be written behind certain commands. If such
an overlay specification is present, the command will only ``take
effect'' on the specified slides. What exactly ``take effect'' means
depends on the command. Consider the following example.

\begin{verbatim}
\frame
{
  \textbf{This line is bold on all three slides.}
  \textbf<2>{This line is bold only on the second slide.}
  \textbf<3>{This line is bold only on the third slide.}
}
\end{verbatim}

For the command \verb!\textbf!, the overlay specification causes the
text to be set in boldface only on the specified slides. On all other
slides, the text is set in a normal font.

You cannot add an overlay specification to every command, but only to
those listed in the following. However, it is quite easy to redefine a
command such that it becomes ``overlay specification aware.''

For the following commands, adding an overlay specification causes the
command to be simply ignored on slides that are not included in the
specification: \verb!\textbf!, \verb!\textit!, \verb!\textsl!,
\verb!\textrm!, \verb!\textsf!, \verb!\color!, \verb!\alert!,
\verb!\structure!. If a command takes several arguments, like
\verb!\color!, the specification directly follows the command as in
the following example.

\begin{verbatim}
\frame
{
  \color<2-3>[rgb]{1,0,0} This text is red on slides 2 and 3, otherwise black.
}
\end{verbatim}

For the following commands, the effect of an overlay specification is
special:

\Command{only}
\Parameters{
\item a text
}
\Description{
  If an overlay specification is present, the text is inserted only
  into the specified slides. For other slides, the text is simply
  thrown away. In particular, it occupies no space.}
\Example \verb!\only<3->{Text inserted from slide 3 on.}!

There exists a variant of \verb!\only!, namely \verb!\pgfonly!, that
should be used inside \pgf\ pictures instead of \verb!\only!. The
command \verb!\pgfonly! inserts appropriate \verb!\ignorespaces!
commands that are needed by \pgf.

\Command{uncover}
\Parameters{
\item a text
}
\Description{
  If an overlay specification is present, the text is shown
  (``uncovered'') only on the specified slides. On other slides, the
  text still occupies space and it is still typeset, but it is not
  shown or only shown as if transparent. For details on how to specify
  whether the text is invisible or just transparent, see
  Section~\ref{section-transparent}. 
}
\Example \verb!\uncover<3->{Text shown from slide 3 on.}!

\Command{invisible}
\Parameters{
\item a text
}
\Description{
  The text is occupies space and it is still typeset, but it is not
  shown. If an overlay specification is given, this command takes
  effect only on the specified slides. This command is a conter-part to
  \texttt{uncover}, but not quite: unlike \texttt{uncover}, invisible 
  text is never shown in a transparent way, but is guaranteed to
  really be invisible.
}
\Example \verb!\invisible<-2>{Text shown from slide 3 on.}!

\Command{alt}
\Parameters{
\item a slide specification in pointed brackets.
\item a main text
\item an alternative text
}
\Description{
  The main text is shown on the specified slides, otherwise the
  alternative text. The specification must always be present.}
\Example \verb!\alt<2>{On Slide 2}{Not on slide 2.}!

\Command{temporal}
\Parameters{
\item a slide specification in pointed brackets.
\item a text to be put on all slides before the specified slides
\item a text to be put on the specified slides
\item a text tot be put on all slides after the specified slides
}
\Description{
  This command alternates between three different texts, depending on
  whether the current slide is temporally before the specified
  slides, is one of the specified slides, or comes after them. If the
  specification is not an interval (that is, if it has a ``hole''),
  the ``hole'' is considered to be part of the before slides.}
\Example
\begin{verbatim}
  \temporal<3-4>{Shown on 1, 2}{Shown on 3, 4}{Shown 5, 6, 7, ...}
  \temporal<3,5>{Shown on 1, 2, 4}{Shown on 3, 5}{Shown 6, 7, 8, ...}
\end{verbatim}

As a possible application of the \verb!\temporal! command consider the
following example: 

\begin{verbatim}
\def\colorize<#1>{%
  \temporal<#1>{\color{structure!50}}{\color{black}}{\color{black!50}}}

\frame{
  \begin{itemize}
    \colorize<1> \item First item.
    \colorize<2> \item Second item.
    \colorize<3> \item Third item.
    \colorize<4> \item Fourth item.
  \end{itemize}
}
\end{verbatim}


\Command{item}
\Description{
  Adding an overlay specification to an item in a list causes this
  item to be uncovered only on the specified slides. This is useful
  for creating lists that are uncovered piecewise. Note that you are
  not required to stick to an order in which items are uncovered.
  }
\Example
\begin{verbatim}
\frame
{
  \begin{itemize}
  \item<1-> First point, shown on all slides.
  \item<2-> Second point, shown on slide 2 and later.
  \item<2-> Third point, also shown on slide 2 and later.
  \item<3-> Fourth point, shown on slide 3.
  \end{itemize}
}

\frame
{
  \begin{enumerate}
  \item<3->[0.] A zeroth point, shown at the very end.
  \item<1-> The first an main point.
  \item<2-> The second point.
  \end{enumerate}
}
\end{verbatim}

In the following concluding example, a list is uncovered
item-wise. The last uncovered item is furthermore hilighted. 

\begin{verbatim}
\frame
{
  The advantages of the beamer class are
  \begin{enumerate}
  \item<1-> \alert<1>{It is easy to use.}
  \item<2-> \alert<2>{It is easy to extend.}
  \item<3-> \alert<3>{It works together with \texttt{pdflatex}.}
  \item<4-> \alert<4>{It has nice overlays.}
  \end{enumerate}
}
\end{verbatim}

The related command \verb!\bibitem! is also overlay-specification-aware
in the same way as \verb!\item!.

\Command{hypertarget}
\Parameters{
\item a target name
\item some text
}
\Description{
  If an overlay specification is present, the text is the specified
  target for hyperjumps only on the specified slide. On all other
  slides, the text is shown normally. Note that you \emph{must} add an
  overlay specification to the \texttt{hypertarget} command whenever
  you use it on frames that have multiple slides (otherwise
  \texttt{pdflatex} rightfully complains that you have defined the
  same target on different slides).}
\Example
\begin{verbatim}
\frame{
  \begin{itemize}
  \item<1-> First item.
  \item<2-> Second item.
  \item<3-> Third item.
  \end{itemize}

  \hyperlink{jumptoend}{Jump to last slide of the frame.}
  \hypertarget<3>{jumptoend}{}
}
\end{verbatim}


\Command{label}
\Parameters{
\item a target
}
\Description{
  If an overlay specification is present, the label is only inserted
  on the specified slide. Inserting a label on more than one slide
  will cause a `multiple labels' warning. \emph{However}, if no
  overlay specification is present, the specification is automatically
  set to just `1' and the label is thus inserted only on the first
  slide. This is typically the desired behaviour since it does not
  really matter on which slide the label is inserted, \emph{except} if
  you use an \texttt{only} command. Then you need to specifiy a slide.
}
\Example
\begin{verbatim}
\frame
{
  \begin{align}
    a &= b + c   \label{first}\\ % no specification needed
    c &= d + e   \label{second}\\% no specification needed
  \end{align}

  Blah blah, \uncover<2>{more blah blah.}

  \only<3>{Specification is needed now.\label<3>{mylabel}}
}
\end{verbatim}


\subsubsection{Environments with Overlay Specifications}

Environments can also be equipped with overlay specifications. For
most of the predefined environments, see subsection~\ref{predefined},
adding an overlay specifications causes the whole environment to be
uncovered only on the specified slides. This is useful for showing
things incrementally as in the following example.

\begin{verbatim}
\frame
{
  \frametitle{A Theorem on Infinite Sets}

  \begin{theorem}<1->
    There exists an infinite set.
  \end{theorem}

  \begin{proof}<3->
    This follows from the axiom of infinity.
  \end{proof}

  \begin{example}<2->
    The set of natural numbers is infinite.
  \end{example}
}
\end{verbatim}
In the example, the first slide only contains the theorem, on the
second slide an example is added, and on the third slide the proof is
also shown.

The two special environments \verb!onlyenv! and \verb!uncoverenv! are
``environment versions'' of the commands \verb!\only! and \verb!\uncover!.


\Environment{onlyenv}
\Description{
  If an overlay specification is given, the contents of the
  environment is inserted into the text only on the specified slides. }
\Example
\begin{verbatim}
\frame
{
  This line is always shown.
  \begin{onlyenv}<2>
    This line is inserted on slide 2.
  \end{onlyenv}
}
\end{verbatim}

\Environment{uncoverenv}
\Description{
  If an overlay specification is given, the contents of the
  environment is shown only on the specified slides. It still occupies
  space on the other slides.}
\Example
\begin{verbatim}
\frame
{
  This word is 
  \begin{uncoverenv}<2>
    visible
  \end{uncoverenv}
  only on slide 2.
}
\end{verbatim}


\subsubsection{Dynamically Changing Text}

You may sometimes wish to have some part of a frame change dynamically
from slide to slide. On each slide of the frame, something different
should be shown inside this area. You could achieve the effect of
dynamically changing text by giving a list of \verb!\only! commands like this:
\begin{verbatim}
  \only<1>{Initial text.}
  \only<2>{Replaced by this on second slide.}
  \only<3>{Replaced again by this on third slide.}
\end{verbatim}
The trouble with this approach is that it may lead to slight, but
annoying differences in the heights of the lines, which may cause the
whole frame to ``whobble'' from slide to slide. This problem becomes
much more severe if the replacement text is several lines long.

To solve this problem, you can use two environments:
\verb!overlayarea! and \verb!overprint!. The first is more flexible,
but less user-friendly.

\Environment{overlayarea}
\Parameters{
\item
  The width of the area.
\item
  The height of the area.
  }
\Description{
  Everything within the environment will be placed in a rectangular
  area of the specified size. The area will have the same size on all
  slides of a frame, regardless of its actual contents. }
\Example
\begin{verbatim}
\begin{overlayarea}{\textwidth}{3cm}
  \only<1>{Some text for the first slide.\\Possibly several lines long.}
  \only<2>{Replacement on the second slide.}
\end{overlayarea}
\end{verbatim}


\Environment{overprint}
\Parameters{
\item
  Optional parameter in square brackets: width of the overprint
  area. Default: text width.
  }
\Description{
  Inside the environment, use \texttt{onslide} commands to specify
  different things that should be shown for this environment on
  different slides. The \texttt{onslide} commands are used like
  \texttt{item} commands. Everything within the environment will be
  placed in a rectangular area of the specified width. The height and
  depth of the area are chosen large enough to accommodate the largest
  contents of the area. The overlay specifications of the
  \texttt{onslide} commands must be disjoint.}
\Example
\begin{verbatim}
\begin{overprint}
  \onslide<1>
    Some text for the first slide.\\
    Possibly several lines long.
  \onslide<2>
    Replacement on the second slide.
\end{overprint}
\end{verbatim}




\subsection{Making Commands and Environments Overlay-Specification-Aware}

This subsection explains how you can make your own commands
overlay-specification-aware. Also, it explains how to setup counters
correctly that should be increased from frame to frame (like equation
numbering), but not from slide to slide. You may wish to skip this
section, unless you  want to write your own extensions to the \beamer\
class. 
 
You can define a new command that is overlay-specification-aware using
the following command.

\Command{newoverlaycommand}
\Parameters{
\item name of the command
\item commands to be executed on the specified slides
\item commands to be executed otherwise
}
\Description{
  Declares a new command. If this command is encountered, it is
  checked whether an overlay specification follows. If not, the
  commands given in the second parameter are executed. If there is a
  specification, the second parameter is executed if the current slide
  is specified, otherwise the third parameter is executed.
  }
\Example
\begin{verbatim}
\newoverlaycommand{\SelectRedAsColor}{\color[rgb]{1,0,0}}{}
...
\frame
{
  \SelectRedAsColor<2>
  The second slide of this frame is all in red. 
}
\end{verbatim}


\Command{renewoverlaycommand}
\Parameters{
\item name of a command to be redefined
\item commands to be executed on the specified slides
\item commands to be executed otherwise
}
\Description{
  Redeclares a command that already exists in the same way as
  \texttt{newoverlaycommand}. Inside the parameters, you can 
  still access to original definitions using the command
  \texttt{original}, see the example.
  }
\Example
\begin{verbatim}
\renewoverlaycommand{\tiny}{\original{\tiny}}{}
...
\frame
{
  \tiny<2>This text is tiny on slide 2.
}
\end{verbatim}



\Command{newoverlayenvironment}
\Parameters{
\item name of the environment
\item begin commands to be executed on the specified slides
\item end commands to be executed on the specified slides
\item begin commands to be executed otherwise
\item end commands to be executed otherwise
}
\Description{
  Declares a new environment that is overlay specification aware. If
  this environment encountered, it is 
  checked whether an overlay specification follows. If not or if it is
  found and the current slide is specified, the second and third
  parameters form the beginning and end of the environment. Otherwise,
  the fourth and fifth parameters are used.

  This command can take one optional parameter, given in square
  brackets after the first parameter. If this parameter is specified,
  it must currently be~1. In this case, the begin commands must take
  one parameter. This parameter will \emph{preceed} the overlay
  specification, see the examples.
  }
\Example
\begin{verbatim}
\newoverlayenvironment{mytheorem}{\alert{Theorem}:}{}{Theorem:}{}

\frame
{
  \begin{mytheorem}<2>
    This theorem is hilighted on the second slide.
  \end{mytheorem}
}
\end{verbatim}

\begin{verbatim}
\newoverlayenvironment{mytheorem}[1]{\alert{Theorem #1}:}{}{Theorem #1:}{}

\frame
{
  \begin{mytheorem}{of Tantau}<2>
    This theorem is hilighted on the second slide.
  \end{mytheorem}
}
\end{verbatim}


The following two commands can be used to ensure that a certain
counter is automatically reset on subsequent slides of a frame. This
is necessary for example for the equation count. You might want this
count to be increased from frame to frame, but certainly not from
overlay slide to overlay slide. For equation counters and footnote
counters (you should not use footnotes), these commands have already
been invoked.

\Command{resetcounteronoverlays}
\Parameters{
\item name of a \LaTeX\ counter
}
\Description{
  After you have invoked this command, the value of the specified
  counter will be the same on all slides of every frame. 
}
\Example \verb!\resetcounteronoverlays{equation}!
 
\Command{resetcountonoverlays}
\Parameters{
\item name of a \TeX\ count register
}
\Description{
  The same as \texttt{resetcounteronoverlays}, except that this
  command should be used with counts that have been created using the
  \TeX\ primitive \texttt{newcount} instead of \LaTeX 's
  \texttt{definecounter}. 
}
\Example
\begin{verbatim}
\newcount\mycount
\resetcountonoverlays{mycount}
\end{verbatim}







\section{Structuring a Presentation}

\subsection{Kinds of Global Structures of Presentations}

Still needs to be written.

\subsubsection{Linear Global Structure}

%\subsubsection{Pyramidal Global Structures}

\subsubsection{Nonlinear Global Structure}



\subsection{Commands and Environments for Creating Global Structure}


\subsubsection{Adding a Title Page}

You can use the \verb!\titlepage! command to insert a title page into
a frame. 

The \verb!\titlepage! command will arrange the following elements on
the title page: the document title, the author(s)'s names, their
affiliation, a title graphic, and a date.

\Command{titlepage}
\Description{Inserts the text of a title page into the current frame.}
\Example \verb!\frame{\titlepage}!
\vskip1em

Before you invoke the title page command, you must specify all
elements you wish to be shown. This is done using the following
commands: 

\Command{title}
\Parameters{
\item A shorter version of the title for inclusion in head lines and
  foot lines. This parameter is optional and given in square brackets.
\item A title for the document. Line breaks can be inserted using the
  double-backslash command.
}
\Example
\begin{verbatim}
\title{The Beamer Class}

\title[Short Version]{A Very Long Title\\Over Several Lines}
\end{verbatim}

\Command{author}
\Parameters{
\item A shorter version of the authors for inclusion in head lines and
  foot lines. This parameter is optional and given in square brackets.
\item Names of the authors.
}
\Description{
  The names should be separated using the
  command \texttt{and}. In case authors have different affiliations,
  they should be suffixed by the command \texttt{inst} with different
  parameters.}
\Example\verb!\author[Hemaspaandra et al.]{Lane Hemaspaandra\inst{1} \and Till Tantau\inst{2}}!

\Command{institute}
\Parameters{
\item A shorter version of the institute's name for inclusion in head
  lines and foot lines. This parameter is optional and given in square
  brackets.
\item Institute(s) where the authors work.
}
\Description{
  If more than one institute is given, they should be separated using
  the command \texttt{and} and they should be prefixed by the command
  \texttt{inst} with different parameters.}
\Example
\begin{verbatim}
\institute[Universities of Rochester and Berlin]{
  \inst{1}Department of Computer Science\\
  University of Rochester
  \and
  \inst{2}Fakult\"at f\"ur Elektrotechnik und Informatik\\
  Technical University of Berlin}
\end{verbatim}


\Command{date}
\Parameters{
\item A shorter version of the date for inclusion in head
  lines and foot lines. This parameter is optional and given in square
  brackets.
\item A text to be shown as date or occasion at which the talk was held.
}
\Example\verb!\date{\today}! or \verb!\date[STACS 2003]{STACS Conference, 2003}!.


\Command{titlegraphic}
\Parameters{
\item A text to be shown as title graphic. Typically, a picture
  environment is used as text.
}
\Example\verb!\titlegraphic{\pgfuseimage{titlegraphic}}!



\subsubsection{Adding Table of Contents}

You can create a table of contents using the command
\verb!\tableofcontents!. Unlike the normal \LaTeX\ table of contents
command, this command takes an optional parameter in square brackets
that can be used to create certain special effects.

\Command{tableofcontents}
\Parameters{
\item A list of options, separated by commas. The valid options and
their effects are explained below.
}
\Description{
  Inserts a table of contents into the current frame. To change how
the table of contents is typeset, you need to modify the appropriate
templates, see Section~\ref{section-toc-templates}.
}
\Example
\begin{verbatim}
\section[Outline]{}
\frame{\tableofcontents}

\section{Introduction}
\frame{\tableofcontents[current]}
\subsection{Why?}
\frame{...}
\frame{...}
\subsection{Where?}
\frame{...}

\section{Results}
\frame{\tableofcontents[current]}
\subsection{Because}
\frame{...}
\subsection{Here}
\frame{...}
\end{verbatim}

The options of the command \verb!\tableofcontents! have the following
effects: 
\begin{itemize}
\item
  The option \texttt{current} causes all but the current section to be
  shown in a semi-transparent way.
\item
  The option \texttt{pausesections} causes a \verb!\pause! command to
  be issued before each section. This is useful if you wish to show
  the table of contents in an incremental way.
\item
  The option \texttt{pausesubsections} causes a \verb!\pause! command to
  be issued before each subsection.
\item
  The option \texttt{hidesubsections} causes the subsections to be
  omitted. However, if used together with the \texttt{current} option,
  the subsections of the current section are not omitted.
\item
  The option \texttt{shadesubsections} causes the subsections to
  be shown in a semi-transparent way.
\end{itemize}

The last two commands are useful if you do not wish to show too many
details when presenting the talk outline.





\subsubsection{Adding Sections and Subsections}

You can structure your text using the commands \verb!\section! and
\verb!\subsection!. Unlike standard \LaTeX, these commands will not
create a heading at the position where you use them. Rather, they will
add an entry to the table of contents and also to the navigation
bars.

In order to create a line break in the table of contents (usually not
a good idea), you can use the command \verb!\breakhere!. Note that the
standard command \verb!\\! does not work.

\Command{section}
\Parameters{
\item (optional, in square brackets) text to be shown in horizontal
  navigation bars 
\item text to be shown in the table of contents; if empty, no entry is
  created.
}
\Description{
  Starts a section. No heading is created, the section name is only
  shown in the table of contents and in the navigation bar. If the
  main parameter is empty, but the parameter in square brackets is
  not, a navigation entry is created, but no entry in the table of
  contents. This is useful for sections like a ``table of contents
  section.''} 
\Example\verb!\section[Summary]{Summary of Main Results}! or
\verb!\section[Outline]{}! 


\Command{subsection}
\Parameters{
\item (optional, in square brackets) text to be shown in horizontal
  navigation bars 
\item text to be shown in the table of contents; if empty, no entry is
  created.
}
\Description{
  Starts a subsection. No heading is created, the subsection name is only
  shown in the table of contents and in the navigation bar. If the
  main parameter is empty, but the parameter in square brackets is
  not, a navigation entry is created, but no entry in the table of
  contents.}
\Example\verb!\subsection{Some Subsection}!




\subsubsection{Adding a Bibliography}

You can use the bibliography environment and the \verb!\cite! commands
of \LaTeX\ in a \beamer\ presentation. However, there are a few things
to keep in mind:

\begin{itemize}
\item
  It is a bad idea to present a long bibliography in a 
  beamer presentation. Present only very few references.
\item
  Present references only if they are intended as ``further reading,''
  for example at the end of a lecture.
\item
  Using the \verb!\cite! commands can be confusing since the audience
  has little chance of remembering the citations. If you cite the
  references, always cite them with full author name and year like
  ``[Tantau, 2003]'' instead of something like ``[2,4]'' or
  ``[Tan01,NT02]''.
\end{itemize}

Keeping the above warnings in mind, proceed as follows to create the
bibliography: 

For a beamer presentation, you will typically have to typeset your
bibliography items partly ``by hand.'' Nevertheless, you \emph{can}
use \verb!bibtex! to create a ``first approximation'' of the
bibliography. Copy the content of the file \verb!main.bbl! into your
presentation. If you are not familiar with \verb!bibtex!, you may wish
to consult its documentation. It is a  powerful tool for
creating high-quality citations.

Using \verb!bibtex! or just your editor, you place your bibliographic
references into an environment called \verb!thebibliography!. This
(standard \LaTeX) environment takes one parameter, which should be the
longest \verb!bibitem! label in the following list of bibliographic
entries.

\Environment{thebibliography}
\Parameters{
\item Text of the longest label. Inside the environment, use one
  \texttt{bibitem} command for each reference.
}
\Description{
  Inserts a bibliography into the current frame. Must be placed inside
a frame. If the bibliography does not fit on one frame, you should
split it (create a new frame and a second \texttt{thebibliography}
environment). Even better, you should reconsider whether it is a good
idea to present so many references.}
\Example

\begin{verbatim}
\frame{
  \frametitle{For Further Reading}

  \begin{thebibliography}{Dijkstra, 1982}
  \bibitem[Solomaa, 1973]{Solomaa1973}
    A.~Salomaa.
    \newblock {\em Formal Languages}.
    \newblock Academic Press, 1973.

  \bibitem[Dijkstra, 1982]{Dijkstra1982}
    E.~Dijkstra.
    \newblock Smoothsort, an alternative for sorting in situ.
    \newblock {\em Science of Computer Programming}, 1(3):223--233, 1982.
  \end{thebibliography}
 }
\end{verbatim}

The parameter of the \verb!thebibliography! environment is used to
determine the indent of the list. However, several templates for the
typesetting of the bibliography (see
Section~\ref{section-bib-templates}) ignore this parameter since they
replace the references by a symbol.

Inside the \verb!thebibliography! environment, use a (standard \LaTeX)
\verb!\bibitem! command for each reference item. Inside each item, use a
(standard \LaTeX) \verb!\newblock! command to separate the authors's
names, the title, the book/journal reference, and any notes. Each of
these commands may introduce a new line or color or other formatting,
as specified by the template for bibliographies.


\Command{bibitem}
\Parameters{
\item
  The text to be inserted into the text when the item is cited in the
  presentation (optional in square brackets). For a beamer
  presentation, this should usually be as long as possible.
\item
  A label to be used with the \texttt{cite} commands.
  }
\Description{
  Adds a reference item to the bibliography. Use \texttt{newblock}
  commands to  separate the authors's names, the title, the
  book/journal reference, and any notes. If an overlay specification
  is present, it must come directly after the word
  \texttt{bibitem}. If present, the entry will only be shown on the
  specified slides.
}
\Example

\begin{verbatim}
\bibitem[Dijkstra, 1982]{Dijkstra1982}
  E.~Dijkstra.
  \newblock Smoothsort, an alternative for sorting in situ.
  \newblock {\em Science of Computer Programming}, 1(3):223--233, 1982.
\end{verbatim}

Note that, unlike normal \LaTeX, the default template for the
bibliography does not repeat the citation text (like ``[Dijkstra,
1982]'') before each item in the bibliography. Instead, a cute, small
article symbol is drawn. The rationale is that the audience will not be
able to remember any abbreviated citation texts till the end of the
talk. If you really insist on using abbreviations, you can use the
command \verb!beamertemplatetextbibitems! to restore the default
bevahior, see also Section~\ref{section-bib-templates}.



\subsubsection{Adding an Appendix}

You can add an appendix to your talk by using the \verb!\appendix!
command. You should put frames and perhaps whole subsections into the
appendix that you do not intend to show during your presentation, but
which might be useful to answer a question. 

The appendix acts like an additional section with subsections
all of its own, but it is not shown in the normal table of contents or
in the navigation bars. Thus, it is kept perfectly separate of your
actual talk. However, once you ``enter'' the slides that make up the
appendix (either by continuing past the last slide of the actual talk
or by using a hyper-jump), the navigation bars show (only) the
contents of the appendix and the table of contents commands (only)
show the subsections of the appendix.

Inside the appendix, you should not use the \verb!\section! command,
but you can use the \verb!\subsection! command to structure the
appendix. Use the \verb!\tableofcontents! command to insert a table of
contents of the appendix into the current frame.

\Command{appendix}
\Description{
  Starts the appendix. All frames and all \texttt{subsection} commands
  used after this command will not be shown as part of the normal
  navigation bars.
}
\Example

\begin{verbatim}
\begin{document}
\frame{\titlepage}
\section[Outline]{}
\frame{\tableofcontents}
\section{Main Text}
\frame{Some text}
\section[Summary]{}
\frame{Summary text}

\appendix
\frame{\tableofcontents}
\subsection{Additional material}
\frame{Details}
\frame{Text omitted in main talk.}
\subsection{Even more additional material}
\frame{More details}
\end{document}
\end{verbatim}








\subsection{The Navigation Bars}

Most themes that come along with the \beamer\ class show some kind of
navigation bar during your talk. Although these navigation bars take
up quite a bit of space, they are often useful for two reasons:

\begin{itemize}
\item
  They provide the audience with a visual feedback of how much of your
  talk you have covered and what is yet to come. Without such
  feedback, an audience will often puzzle whether something you are
  currently introducing will be explained in more detail later on or
  not.
\item
  You can click on all parts of the navigation bar. This will directly
  ``jump'' you to the part you have clicked on. This is particularly
  useful to skip certain parts of your talk and during a ``question
  session,'' when you wish to jump back to a particular frame someone
  has asked about.
\end{itemize}


\subsubsection{Using the Navigation Bars to Navigate Between Frames}

When you click on one of the icons representing a frame (by default
this is icon is a small circle), the following happens:
\begin{itemize}
\item
  If you click on (the icon of) any frame other than the current frame, the
  presentation will jump to the first slide of the frame you clicked
  on.
\item
  If you click on the current frame and you are not on the last slide
  of this frame, you will jump to the last slide of the frame.
\item
  If you click on the current frame and you are on the last slide, you
  will jump to the first slide of the frame.
\end{itemize}
By the above rules you can:
\begin{itemize}
\item
  Jump to the beginning of a frame from somewhere else by clicking on
  it once.  
\item
  Jump to the end of a frame from somewhere else by clicking on it
  twice.
\item
  Skip the rest of the current frame by clicking on it once.
\end{itemize}

I also tried making a jump to an already-visited frame jump
automatically to the last slide of this frame. However, this turned
out to be more confusing than helpful. With the current implementation
a double-click always brings you to the end of a slide, regardless
from where you ``come.''


\subsubsection{Using the Navigation Bars to Navigate Between Sections}

By clicking on a section or subsection in the navigation bar, you will
jump to that section. Clicking on a section is particularly useful if
the section starts with a \verb!\tableofcontents[current]!, since you
can use it to jump to the different subsections.

By clicking on the document title in a navigation bar (not all themes
show it), you will jump to the first slide of your presentation
(usually the title page) \emph{except} if you are already at the first
slide. On the first slide, clicking on the document title will jump to
the appendix, if there is one. Thus by \emph{double} clicking the
document title in a navigation bar, you can jump to the appendix.





\subsection{The Local Structure of Frames}

Just like your whole presentation, each frame should also be
structured. A frame that is solely filled with some long text is very
hard to follow. It is your job to structure the contents of each frame
such that, ideally, the audience immediately seems which information
is important, which information is just a detail, how the presented
information is related, and so on.

\LaTeX\ provides different commands for structuring text ``locally,''
for example, via the \texttt{itemize} environment. These environments
are also available in the beamer class, although their appearance has
been slightly changed. Furthermore, the \beamer\ class also defines
some new commands and environments, see below, that may help you to
structure your text.


\subsubsection{Itemizations, Enumerations, and Descriptions}

There are three predefined environments for creating lists, namely
\verb!enumerate!, \verb!itemize!, and \verb!description!. The first
two of there can be nested to depth two, but not further (this would
create totally unreadable slides).

The \verb!\item! command is overlay-specification-aware. If an overlay
specification is provided, the item will only be shown on the
specified slides, see the following example. If the \verb!\item!
command is to take an optional argument and an overlay specification,
the overlay specification comes first as in \verb!\item<1>[Cat]!.

\begin{verbatim}
\frame
{
  There are three important points:
  \begin{enumerate}
  \item<1-> A first one,
  \item<2-> a second one with a bunch of subpoints,
    \begin{itemize}
    \item first subpoint. (Only shown from second slide on!).
    \item<3-> second subpoint added on third slide.
    \item<4-> third subpoint added on fourth slide.
    \end{itemize}
  \item<5-> and a third one.
  \end{enumerate}
}
\end{verbatim}


\Environment{itemize}
\Description{
  Used to display a list of items that do not have a special
  ordering. Inside the environment, use an \texttt{item} command for
  each topic. The appearence of the items can be changed using
  templates, see Section~\ref{section-templates}.}
\Example
\begin{verbatim}
\begin{itemize}
\item This is important.
\item This is also important.
\end{itemize}
\end{verbatim}


\Environment{enumerate}
\Description{
  Used to display an ordered list of items. Inside the environment,
  use an \texttt{item} command for each topic. The appearence of the
  items can be changed using templates, see
  Section~\ref{section-templates}.}
\Example
\begin{verbatim}
\begin{enumerate}
\item This is important.
\item This is also important.
\end{enumerate}
\end{verbatim}


\Environment{description}
\Parameters{
\item
  Some text, given as an optional parameter in square brackets. The
  width of the labels will be set to the width of this text. Normally,
  you choose the widest label in the description and copy it here.
  }
\Description{
  Used to display an list that explains or defines labels. Inside the
  environment, use an \texttt{item} with an argument in square brackets
  for each topic. The appearence of the items can be changed using
  templates, see Section~\ref{section-templates}.}
\Example
\begin{verbatim}
\begin{description}
\item[Lion] King of the savanna.
\item[Tiger] King of the jungle.
\end{description}

\begin{description}[longest label]
\item<1->[short] Some text.
\item<2->[longest label] Some text.
\item<3->[long label] Some text.
\end{description}
\end{verbatim}




\subsubsection{Block Environments and Simple Structure Commands}
\label{predefined}

The \beamer\ class predefines a number of useful environments and
commands. Using these commands makes is easy to change the appearance
of a document by changing the theme.


\Command{alert}
\Parameters{
\item a text to be hilighted.
  }
\Description{
  The given text is hilighted, typically be coloring the text red. If
  an overlay specification is given, the command only has an effect on
  the specified slides.
  }
\Example\verb!This is \alert{important}.!


\Command{structure}
\Parameters{
\item a text to be marked as part of the structure of the text.
  }
\Description{
  The given text is marked as part of the structure, typically be
  coloring the text in the structure color. If
  an overlay specification is given, the command only has an effect on
  the specified slides.
  }
\Example\verb!\structure{Paragraph Heading.}!

\Environment{block}
\Parameters{
\item a block title
  }
\Description{
  Inserts a block, like a definition or a theorem, with a title. If
  an overlay specification is given, the block is shown only on the
  specified slides. In the example, the definition is shown only from
  slide 3 onwards.
  }
\Example
\begin{verbatim}
  \begin{block}{Definition}<3->
    A \alert{set} consists of elements.
  \end{block}
\end{verbatim}

\Environment{alertblock}
\Parameters{
\item a block title
  }
\Description{
  Inserts a block whose title is hilighted. If
  an overlay specification is given, the block is shown only on the
  specified slides.
  }
\Example
\begin{verbatim}
  \begin{alertblock}{Wrong Theorem}
    $1=2$.
  \end{alertblock}
\end{verbatim}

\Environment{exampleblock}
\Parameters{
\item a block title
  }
\Description{
  Inserts a block that is supposed to be an example. If
  an overlay specification is given, the block is shown only on the
  specified slides.
  }
\Example
\begin{verbatim}
  \begin{exampleblock}{Example}
    The set $\{1,2,3,5\}$ has four elements.
  \end{exampleblock}
\end{verbatim}

Predefined English block environments, that is, block environments
with fixed title, are: \verb!Theorem!, \verb!Proof!, \verb!Corollary!,
\verb!Fact!, \verb!Example!, and \verb!Examples!. You can also use these
environments with a lowercase first letter, the result  is the
same. The following German block environments are also predefined:
\verb!Problem!, \verb!Loesung!, \verb!Definition!, \verb!Satz!,
\verb!Beweis!, \verb!Folgerung!, \verb!Lemma!, \verb!Fakt!,
\verb!Beispiel!, and \verb!Beispiele!. See the following example for
their usage

\begin{verbatim}
\frame
{
  \frametitle{A Theorem on Infinite Sets}

  \begin{theorem}<1->
    There exists an infinite set.
  \end{theorem}

  \begin{proof}<2->
    This follows from the axiom of infinity.
  \end{proof}

  \begin{example}<3->
    The set of natural numbers is infinite.
  \end{example}
}
\end{verbatim}


\subsubsection{Figures and Tables}

You can use the standard \LaTeX\ environments \texttt{figure} and
\texttt{table} much the same way you would normally use them. However,
any placement specification will be ignored. Figures and tables are
immediately inserted where the environments start. If there are too
many of them to fit on the frame, you must manually split them among
additional frames.

\Example
\begin{verbatim}
\frame{
  \begin{figure}
    \pgfuseimage{myfigure}
    \caption{This caption is placed below the figure.}
  \end{figure}

  \begin{figure}
    \caption{This caption is placed above the figure.}
    \pgfuseimage{myotherfigure}
  \end{figure}
}
\end{verbatim}

You can adjust how the figure and table captions are typeset by
changing the corresponding template, see
Section~\ref{section-template-caption}.





\subsubsection{Splitting a Frame into Multiple Columns}

Three environments are used to create columns on a slide. Columns are
especially useful for placing a graphic next to a description/explanation.
The main environment for creating columns is called
\verb!columns!. Inside this environment, you can place several
\verb!column! environments. Each will create a new column.

\Environment{columns}
\Description{
  A multi-column area. Inside the environment you should place only
  \texttt{column} environments.}
\Example
\begin{verbatim}
\begin{columns}
  \begin{column}{5cm}
    First column.
  \end{column}
  \begin{column}{5cm}
    Second column.
  \end{column}
\end{columns}
\end{verbatim}

\Environment{columnsonlytextwidth}
\Description{
  This command has the same effect as \texttt{columns}, except that the
  columns will not occupy the whole page width, but only the text
  width. 
  }

\Environment{column}
\Parameters{
\item The width of the column.
}
\Description{
  Creates a single column of the specified width. The column is
  centered vertically relative to the other columns.}






\section{Graphics, Colors, Animations, and Special Effects}

\subsection{Graphics}
\label{section-graphics}

Graphics often convey concepts or ideas much more efficiently than
text: A picture can say more than a thousand words. (Although,
sometimes a word can say more than a thousand pictures.) In the
following, the advantages and disadvantages of different possible ways
of creating graphics for beamer presentations are discussed.



\subsubsection{Including External Graphic Files}

One way of creating graphics for a presentation is to  use an 
external program, like \texttt{xfig} or the Gimp. These programs
have an option to \emph{export} graphic files in a format that can
then be inserted into the presentation.

The main advantage is:
\begin{itemize}
\item
  You can use a powerful program to create a high-quality graphic.
\end{itemize}

The main disadvantages are:
\begin{itemize}
\item
  You  have to worry about many files. Typically there are at least
  two for each presentation, namely the program's graphic data file and the
  exported graphic file in a format that can be read by \TeX.
\item
  Changing the graphic using the program does not automatically change
  the graphic in the presentation. Rather, you must reexport the
  graphic and rerun \LaTeX.
\item
  It may be difficult to get the line width, fonts, and font sizes
  right.
\item
  Creating formulas as part of graphics is often difficult or
  impossible.
\end{itemize}


In principle, you can use all the standard \LaTeX\ commands for
inserting graphics, like the command \verb!\includegraphic!. However,
it may be advisable to use the special commands from the \textsc{pgf}
package instead for this particular purpose. The reason is that
\verb!\includegraphic! will put a copy of the graphic into the file
upon each invocation. If a frame includes a graphic and
shows this graphic on ten slides, then ten copies of the possibly
large graphic file will be inserted into the presentation file. This
can result in huge files.

The \textsc{pgf} package offers a solution to this (but, currently,
only if you use \verb!pdflatex!): There, you must first \emph{declare}
every graphic. Once you have done this, you can \emph{use} the graphic
as often as you want and the graphic data will be put only once into
the \verb!.pdf! file. Furthermore, if you use \verb!latex! instead of
\verb!pdflatex!, the \textsc{pgf} package will automatically search
for a graphic file with the extension \texttt{.eps} instead of the
extensions appropriate for \verb!pdflatex!.

The \textsc{pgf} commands are used as follows: To declare an image,
you write, somewhere early in your file,
\begin{verbatim}
\pgfdeclareimage{icon}{9pt}{10pt}{iconfile}
\end{verbatim}
The first parameter is a string by which you can refer to the image
later on. The second two parameters are the width and height of the
image, although one (but not both) can be omitted in which case the
missing value is computed automatically such that the image's aspect
ratio remains correct. The last parameter is the graphic file name
\emph{without} the extension. Depending on whether a \textsc{pdf} or a
PostScript file is created, the package will try appropriate
extensions automatically (\texttt{.eps} for normal \LaTeX,
\texttt{.png}, \texttt{.jpg}, and \texttt{.pdf} for
\texttt{pdflatex}). 

To use a previously declared image, just write 
\begin{verbatim}
\pgfuseimage{icon}
\end{verbatim}

For more details, consult the \textsc{pgf} User Manual.


\subsubsection{Inlining Graphic Commands}

A different way of creating graphics is to insert
graphic drawing commands directly into your \LaTeX\ file. There are numerous
packages that help you do this. They have various degrees of
sophistication. Inlining graphics suffers from none of the
disadvantages mentioned above for including external graphic files,
but the main disadvantage is that it is often hard to use
these packages. In some sense, you ``program'' your graphics, which 
requires a bit of practice.

When choosing a graphic package, there are a few things to keep in
mind:
\begin{itemize}
\item
  Many packages produce poor quality graphics. This is especially true
  of the standard \texttt{picture} environment of \LaTeX.
\item
  Powerful packages that produce high-quality graphics often do not
  work together with \texttt{pdflatex}.
\item
  The most powerful and easiest-to-use package around, namely
  \texttt{pstricks}, does not work together with \texttt{pdflatex} and
  this is a fundamental problem. Due to the fundamental differences
  between \textsc{pdf} and PostScript, it is not possible to write a
  ``\texttt{pdflatex} backend for \texttt{pstricks}.''
\end{itemize}

A solution to the above problem (though not necessarily the best) is
to use the \textsc{pgf} package. It produces high-quality graphics and
works together with \texttt{pdflatex}, but also with normal
\texttt{latex}. It is not as powerful as \texttt{pstricks} (as pointed
out above, this is because of rather fundamental reasons) and not as
easy to use, but it should be sufficient in most cases.





\subsection{Color Management}

The color management of the \beamer\ class relies on the packages
\texttt{xcolor}, which is an extension of the \texttt{color} package,
and on \texttt{xxcolor}, which in turn is an extension of
\texttt{xcolor}. Hopefully, in the future \texttt{xxcolor} and
\texttt{xcolor} will merge into one package and perhaps they will
someday also merge together with \texttt{color}.


\subsubsection{Colors of Main Text Elements}

By default, the following colors are used in a presentation:
\begin{itemize}
\item
  Normal text is typeset in \texttt{black}.
\item
  All ``structural'' elements, like titles, navigation bars, block
  titles, and so on, are typeset using the color
  \texttt{structure}. By default, this color is bluish. Using one of
  the class options \texttt{red}, \texttt{gray}, or \texttt{brown}
  changes this. You can also simply redefine this color to a different
  color using the \verb!\definecolor! command.
\item
  All ``alert'' text is typeset by mixing in 85\% of red. To change
  this, you can either redefine the color \texttt{alert}, or you can
  change the whole alert template.
\item
  All examples are typeset using 50\% of green. To change this, you
  must change the example templates.
\end{itemize}



\subsubsection{Average Background Color}

In some situations, for example when creating a transparency effect,
it is useful to have access to the current background
color. One can then, for example, mix a color with the background
color to create a ``transparent'' color.

Unfortunately, it is not always clear what exactly the background
color is. If the background is a shading or a picture, the color
changes all the time. In these cases, one can at least try to mix-in
an \emph{average} background color, called
\texttt{averagebackgroundcolor}. If a shading or picture is not too
colorful, this works fairly well.

To specify the average background color, use the following command:

\Command{beamersetaveragebackground}
\Parameters{
\item name of a color or a color expression (see the \texttt{xcolor}
package).
}
\Description{
  Installs the given color as the average background color.
}
\Example \verb/\beamersetaveragebackground{red!10}/

If you use the commands from Section~\ref{section-backgrounds} for
installing a background coloring, the average background color is
computed automatically for you. When you directly use the command
\verb!\usebackgroundtemplate!, you should must set the average
background color afterward.




\subsubsection{Transparency Effects}
\label{section-transparent}

By default, \emph{covered} items are not shown during a
presentation. Thus if you write \verb!\uncover<2>{Text.}!, the text
is not shown on any but the second slide. On the other slide, the text
is not simply printed using the background color -- it is not shown at
all. This effect is most useful if your background does not have a
uniform color.

Sometimes however, you might prefer that covered items are not
completely covered. Rather, you would like them to be shown already in
a very dim or shaded way. This allows your audience to get a feeling
for what is yet to come, without getting distracted by it. Also, you
might wish text that is covered ``once more'' still to be visible to
some degree.

Ideally, there would be an option to make covered text
``transparent.'' This would mean that when covered text is shown, it
would instead be mixed with the background behind it. Unfortunately,
this is more or less impossible to implement since neither PostScript
nor \pdf\ currently support transparency.

Nevertheless, one can come ``quite close'' to transparent text using
the special command
\begin{verbatim}
\beamersetuncovermixins{#1}{#2}
\end{verbatim}
This commands allows you to specify in a quite general way how a
covered item should be rendered. You can even specify different ways
of rendering the item depending on how long it will take before this
item is shown or for how long it has already been covered once
more. The transparency effect will automatically apply to all colors,
\emph{except} for the colors in images and shadings. For images and
shadings there is an awkward workaround, see the documentation of the
\pgf\ package. 

As a convenience, two commands are defined in \texttt{beamertemplates}
that install a predefined uncovering behavior.

\Command{beamertemplatetransparentcovered}
\Description{
  Makes all covered text nearly transparent. 
}

\Command{beamertemplatetransparentcovereddynamic}
\Description{
  Makes all covered text nearly transparent, but is a dynamic way. The
  longer it will take till the text is uncovered, the stronger the
  transparency. 
}

\Command{beamersetuncovermixins}
\Parameters{
\item
  A specification of how to render covered items that have \emph{not
  yet} been uncovered.
\item
  A specification of how to render covered items that have \emph{once
  more} been covered.
}
\Description{
  The format of the specifications is explained below. If you leave
one of the specifications empty, the corresponding covered items are
completely covered, that is, they are invisible.
}
\Example
\begin{verbatim}
\beamersetuncovermixins
  {\mixinon<1>{15!averagebackgroundcolor}
    \mixinon<2>{10!averagebackgroundcolor}
    \mixinon<3>{5!averagebackgroundcolor}
    \mixinon<4->{2!averagebackgroundcolor}}
  {\mixinon<1->{15!averagebackgroundcolor}}
\end{verbatim}

The specifications passed to \verb!\beamersetuncovermixins! can
contain any number of the following two commands:

\Command{mixinon}
\Parameters{
\item
  An overlay specification given in brackets.
\item
  A mix-in specification.
}
\Description{
  The overlay specification specifies on which slides the second
parameter should be applied to all colors. Unlike other overlay
specifications, the first parameter is a ``relative'' overlay
specification. For example, the specification ``3'' here means
``things that will be uncovered three slides ahead,'' respectively
``things that have once more been covered for three slides.'' More
precisely, if an item is uncovered for more than one slide and then
covered once more, only the ``first moment of uncovering'' is used for
the calculation of how long the item has been covered once more.

  A \emph{mix-in} specification is a concept introduced by the
\texttt{xcolor} package. A mix-in specification specifies how colors
should be altered by adding another color to them. The specification
consists of two parts, separated by an exclamation mark. The first
part is a number between 0 and 100, where 0 means ``do not mix in the
text color at all'' and 100 means ``use only the text color''. The
second part is the color that should be mixed in. This second part may
be omitted (along with the exclamation mark), in which case ``white''
is used as mix-in color. Any color that has been defined using the
\texttt{definecolor} command is permissible as a mix-in color.
}
\Example
\begin{verbatim}
\mixinon<1>{15!averagebackgroundcolor}
\end{verbatim}
For all items that become uncovered on the next slide or that have
just been covered on the previous slide (depending on whether this
command is used as part of the first or second parameter of the command
\verb!\beamersetuncovermixins!), use only 15\% of the actual color and
85\% of the average background color.

\Command{invisibleon}
\Parameters{
\item
  An overlay specification given in brackets.
}
\Description{
  Text that is covered on the specified slides (once more,
  relative to the current slide), is not shown at all.}
\Example
\begin{verbatim}
\invisibleon<2->
\end{verbatim}
Makes everything totally covered that is not shown next or has just
been shown.



\subsection{Animations}

A word of warning first: Animations can be very distracting. No matter
how cute a rotating, flying theorem looks like to you and no matter
how badly you feel your audience needs some action to keep happy,
most people in the audience will typically feel you are making fun of
them. 

\subsubsection{Using an External Viewer}

If you have created an animation using some external
program (like a renderer), you can use the capabilities of the
presentation program (like the Acrobat Reader) to show the
animation. Unfortunately, currently there is no portable way of doing
this and even the Acrobat Reader does not support this feature on all
platforms.


\subsubsection{Animations Created by Showing Slides in Rapid Succession}

You can create an animation in a portable way by using the
overlay commands of the \beamer\ package to create a series of slides
that, when shown in rapid succession, present an animation. This is a
flexible approach, but such animations will typically be rather static
since it will take some time to advance from one slide to the
next. This approach is mostly useful for animations where you want
to explain each ``picture'' of the animation.
When you advance slides ``by hand,'' that is, by pressing a forward
button, it typically takes at least a second for the next slide to
show.

More ``lively'' animations can be created by relying on a capability
of the viewer program. Some programs support
showing slides only for a certain number of seconds during a
presentation (for the Acrobat Reader this works only in full-screen
mode). By setting the number of seconds to zero, you can create a
rapid succession of slides.

To facilitate the creating of animations in using the feature, you can
use two commands: \verb!\animate! and \verb!\animatevalue!.

\Command{animate}
\Description{
  An overlay specification should be present. The slides specified by
  this overlay specification will be shown only as shortly as
  possible.
}
\Example
\begin{verbatim}
\frame{
  \frametitle{A Five Slide Animation}
  \animate<2-4>

  The first slide is shown normally. When the second slide is shown
  (presumably after pressing a forward key), the second, third, and
  fourth slides ``flash by.'' At the end, the content of the fifth
  slide is shown.

  ... code for creating an animation...
}
\end{verbatim}


\Command{animatevalue}
\Parameters{
\item An overlay specification range given in pointed brackets. Two
  numbers must be given, separated by a dash.
\item Name of a counter or a dimension that will be varied on
  the specified slides.
\item The value the counter or dimension should have on and before the
  first slide of the range.
\item The value the counter or dimension should have on and after the
  last slide of the range.
}
\Description{
  That command allows you to vary a counter or dimension between two
  values. For the slides in the specified range, the counter or
  dimension is set to an interpolated value that depends on the
  current slide number.
}
\Example
\begin{verbatim}
\newcount\opaqueness
\frame{
  \animate<2-10>
  \animatevalue<1-10>{\opaqueness}{100}{0}
  \begin{colormixin}{\the\opaqueness!averagebackgroundcolor}
    \frametitle{Fadeout Frame}

    This text (and all other frame content) will fade out when the
    second slide is shown. This even works with
    {\color{green!90!black}colored} \alert{text}.
  \end{colormixin}
}
\end{verbatim}

\begin{verbatim}
\newcount\opaqueness
\newdimen\offset
\frame{
  \frametitle{Flying Theorems (You Really Shouldn't!)}

  \animate<2-14>

  \animatevalue<1-15>{\opaqueness}{100}{0}
  \animatevalue<1-15>{\offset}{0cm}{-5cm}
  \begin{colormixin}{\the\opaqueness!averagebackgroundcolor}
  \hskip\offset
    \begin{minipage}{\textwidth}
      \begin{theorem}
        This theorem flies out.
      \end{theorem}
    \end{minipage}
  \end{colormixin}

  \animatevalue<1-15>{\opaqueness}{0}{100}
  \animatevalue<1-15>{\offset}{-5cm}{0cm}
  \begin{colormixin}{\the\opaqueness!averagebackgroundcolor}
  \hskip\offset
    \begin{minipage}{\textwidth}
      \begin{theorem}
        This theorem flies in.
      \end{theorem}
    \end{minipage}
  \end{colormixin}
}
\end{verbatim}



\subsection{Slide Transitions}

\textsc{pdf} in general, and the Acrobat Reader in particular, offer a
standardized way of defining \emph{slide transitions}. Such a
transition is a visual effect that is used to show the slide. For
example, instead of just showing the slide immediately, whatever was
shown before might slowly ``dissolve'' and be replaced by the slide's
content.

Slide transitions should be used with great care. Most of the time,
they only distract. However, they can be useful in some situations:
For example, you might show a young boy on a slide and might wish to
dissolve this slide into slide showing a grown man instead. In this
case, the dissolving gives the audience visual feedback that the young
boy ``slowly becomes'' the man.

There are a number of commands that can be used to specify what effect
should be used when the current slide is presented. Consider the
following example:

\begin{verbatim}
\frame{
  \pgfuseimage{youngboy}
}
\frame{
  \transdissolve
  \pgfuseimage{man}
}
\end{verbatim}
The command \verb!\transdissolve! causes the slide of the
second frame to be shown in a ``dissolved way.'' Note that the
dissolving is a property of the second frame, not of the first one. We
could have placed the command anywhere on the frame.

The transition commands are overlay-specification-aware. We could
collapse the two frames into one frame like this:
\begin{verbatim}
\frame{
  \only<1>{\pgfuseimage{youngboy}}
  \only<2>{\pgfuseimage{man}}
  \transdissolve<2>
}
\end{verbatim}
This states that on the first slide the young boy should be shown, on
the second slide the old man should be shown, and when the second
slide is shown, it should be  shown in a ``dissolved way.''

In the following, the different commands for creating transitional
effects are listed.

\Command{transblindshorizontal}
\Description{
  Show the slide as if horizontal blinds where pulled away.
}
\Example\verb!\transblindshorizontal!
  
\Command{transblindsvertical}
\Description{
  Show the slide as if vertical blinds where pulled away.
}
\Example\verb!\transblindsvertical<2,3>!
  
\Command{transboxin}
\Description{
  Show the slide by moving to the center from all four sides.
}
\Example\verb!\transboxin<1>!
  
\Command{transboxout}
\Description{
  Show the slide by showing more and more of a rectangular area that
  is centered on the slide center.
}
\Example\verb!\transboxout!
  
\Command{transdissolve}
\Description{
  Show the slide by slowly dissolving what was shown before.
}
\Example\verb!\transdissolve!
  
\Command{transglitter}
\Parameters{
\item
  a degree (must be a multiple of 90).  
}
\Description{
  Show the slide with a glitter effect that sweeps in the specified
  direction. 
}
\Example\verb!\transglitter<2-3>{90}!
  
\Command{transsplitverticalin}
\Description{
  Show the slide by sweeping two vertical lines from the sides inward.
}
\Example\verb!\transsplitverticalin!
  
\Command{transsplitverticalout}
\Description{
  Show the slide by sweeping two vertical lines from the center outward.
}
\Example\verb!\transsplitverticalout!
  
\Command{transsplithorizontalin}
\Description{
  Show the slide by sweeping two horizontal lines from the sides inward.
}
\Example\verb!\transsplithorizontalin!
  
\Command{transsplithorizontalout}
\Description{
  Show the slide by sweeping two horizontal lines from the center outward.
}
\Example\verb!\transsplithorizontalout!
  
\Command{transwipe}
\Parameters{
\item
  a degree (must be a multiple of 90).  
}
\Description{
  Show the slide by sweeping a single line in the specified direction,
  thereby ``wiping out'' the previous contents.
}
\Example\verb!\transwipe{90}!



You can also specify how \emph{long} a given slide should be shown,
using the following overlay-specification-aware command:

\Command{transduration}
\Parameters{
\item
  a number of seconds
}
\Description{
  In full screen mode, show the slide the specified number of seconds.
  In zero is specified, the slide is shown as short as possible. This
  can be used to create interesting pseudo-animations.
}
\Example\verb!\transduration<2>{1}!




\section{Creating Handouts, Transparencies, and Notes}

The \beamer\ package offers different ways of creating special
versions of your talk that can be used in different contexts. You can
easily create a \emph{handout} version of the presentation that can be
distributed to the audience. You can also create a version that is
more suitable for a presentation using an overhead projector. Finally,
you can add notes for yourself that help you remember what to say
for specific slides. All of these versions coexist in your main
file. They are created by specifying different class
options and rerunning \TeX\ on the main file.


\subsection{Creating Handouts}

\label{handout}

A \emph{handout} is a version of a presentation that is printed on
paper and handed out to the audience before or after the talk. (See
Section~\ref{section-postscript} for how to place numerous frames on one
pages, which is very useful for handouts.)  For the handout you
typically want to produce as few slides as possible per frame. In
particular, you do not want to print a new slide for each slide of a
frame. Rather, only the ``last'' slide should be printed. 

In order to create a handout, specify the class option
\verb!handout!. If you do not specify anything else, this will cause
all overlay specifications to be suppressed. For most cases this will
create exactly the desired result.

In some cases, you may want a more complex behaviour. For example, if
you use many \verb!\only! commands to draw an animation. In this case,
suppressing all overlay specifications is not such a good idea, since
this will cause all steps of the animation to be shown at the same
time. In some cases this is not desirable. Also, it might be desirable
to suppress some \verb!\alert! commands that apply only to specific
slides in the handout.

For a fine-grained control of what is shown on a handout, you can use
\emph{alternate overlay specifications}. They specify which slides
 of a frame should be shown for a special version, for example for the
handout version. An alternate overlay specification is written
alongside the normal overlay specification inside the pointed
brackets. It is separated from the normal specification by a vertical
bar and a space. The version to which the alternate specification
applies is written first, followed by a colon. Here is an example:
\begin{verbatim}
  \only<1-3,5-9| handout:2-3,5>{Text}
\end{verbatim}
This specification says: ``Normally, insert the text on slides 1--3
and 5--9. For the handout version, insert the text only on slides
2,~3, and~5.'' If no alternate overlay specification is given for
handouts, the default is ``always.'' This causes the desirable effect
that if you do not specify anything, the overlay specification is
effectively suppressed for the handout.

An especially useful specification is the following:
\begin{verbatim}
  \only<3| handout:0>{Not shown on handout.}
\end{verbatim}
Since there is no zeroth slide, the text is not shown. Likewise,
\verb!\alert<3| handout:0>{Text}! will not alert the text on a
handout.

You can also use an alternate overlay specification for the optional
argument of the frame command as in the following example.
\begin{verbatim}
\frame[1-| handout:0]{Text...}
\end{verbatim}
This causes the frame to be suppressed in the handout version. Also,
you can restrict the presentation such that only specific slides of
the frame are shown on the handout:
\begin{verbatim}
\frame[1-| handout:4-5]{Text...}
\end{verbatim}

It is also possible to give only an alternate overlay
specification. For example, \verb!\alert<handout:0>{...}! causes the
text to be always hilighted during the presentation, but never on the
handout version. Likewise, \verb!\frame[handout:0]{...}! causes the
frame to be suppressed for the handout.

Finally, note that it is possible to give more than one alternate
overlay specification and in any order. For example, the following
specification states that the text should be inserted on the first
three slides in the presentation, in the first two slides of the
transparency version, and not at all in the handout.
\begin{verbatim}
  \only<trans:1-2| 1-3| handout:0>{Text}
\end{verbatim}

If you wish to give the same specification in all versions, you can do
so by specifying \verb!all:! as the version. For example,
\begin{verbatim}
\frame[all:1-2]
{
  blah...
}
\end{verbatim}
ensures that the frame has two slides in all versions. 


\subsection{Creating Transparencies}

\label{trans}

The main aim of the \beamer\ class is to create presentations for
beamers. However, it is often useful to print transparencies as
backup, in case the hardware fails. A transparencies version of a talk
often has less slides than the main version, since it takes more time
to switch slides, but it may have more slides than the handout
version. For example, while in a handout an animation might be
condensed to a single slide, you might wish to print several slides
for the transparency version.

You can use the same mechanism as for creating handouts: Specify
\verb!trans! as a class option and add alternate transparency
specifications for the \verb!trans! version as needed. An elaborated
example of different overlay specifications for the presentation, the
handout, and the transparencies can be found in the file
\verb!beamerexample.tex!.



\subsection{Adding Notes}

You can add notes to your slides using the command \verb!\note!. A
note is a reminder to yourself of what you should say or should keep in
mind when presenting a frame. The \verb!\note! command should be given
after the frame to which the note applies. Here is a typical example.
\begin{verbatim}
\frame{
  \begin{itemize}
  \item<1-> Eggs
  \item<2-> Plants
  \item<3-> Animals
  \end{itemize}
}
\note{Tell joke about eggs.}
\end{verbatim}
The note command will create a new page that contains your text plus
some information that should make it easier to match the note to the
frame while talking. 

Since you normally do not wish the notes to be part of your
presentation, you must explicitly specify the class option
\verb!notes! to include notes. If this option is not specified, notes
are suppressed. If you specify \verb!notesonly! instead of
\verb!notes!, only notes will be included and all normal frames are
parsed, but not displayed. This is useful for printing the notes.

\Command{note}
\Parameters{
\item a note text.
  }
\Description{
  Creates a note page. Should be given right after a frame.
  }
\Example\verb!\note{Talk no more than 1 minute.}!


\Command{noteitems}
\Parameters{
\item a list of \texttt{item} commands.
  }
\Description{
  Just like the \texttt{note} command, except that an \texttt{itemize}
  environment is setup inside the note.
  }
\Example

\begin{verbatim}
\frame{Bla bla...}
\noteitems{
\item Stress the importance.
\item Use no more than 2 minutes.
}
\end{verbatim}




\section{Customization}


\subsection{Fonts}

By default, the beamer class uses the Computer Modern sans-serif fonts
for typesetting a presentation. The Computer Modern font family is the
original font family designed by Donald Knuth himself for the \TeX\
program. A sans-serif font is a font in which the letters do not have
serifs (from Frensh \emph{sans}, which means ``without''). Serifs are
the little hooks at the ending of the strokes that make up a
letter. The font you are currently reading is a serif font. \textsf{By
  comparison, this text is in a sans-serif font.}

The choice Computer Modern sans-serif had the following reasons:

\begin{itemize}
\item
  The Computer Modern family has a very large number of symbols
  available that go well together.
\item
  Sans-serif fonts are (generally considered to be) easier to read
  when used in a presentation. In low resolution rendering, serifs
  decrease the legibility of a font.
\end{itemize}

While these reasons are pretty good, you still might wish to change the font:

\begin{itemize}
\item
  The Computer Modern fonts are a bit boring if you have seen them too
  often. Using another font (but not Times!) can give a fresh look.
\item
  Other fonts, especially Times, are sometime rendered better since
  they seem to have better internal hinting.
\item
  A presentation typeset in a serif font  creates a conservative
  impression, which might be exactly what you wish to create.
\end{itemize}

There are two ways of changing the document font: First, you must
decide whether the text should be typeset in sans serif or in
serif. To choose this, use either the class option \texttt{sans} or
\texttt{serif}. By default, \texttt{sans} is selected, so you do not
need to specify this. Furthermore, you can specify one of the two
options \texttt{mathsans} or \texttt{mathserif}. These options
override the overall sans-serif/serif choice for math text.

Second, you can independently switch the document font. To do so, you
should use one of the prepared packages of \LaTeX's font
mechanism. For example, to change to Times/Helvetica, simply add
\begin{verbatim}
\usepackage{times}
\end{verbatim}
in your preamble. Note that if you do not specify \texttt{serif} as a
class option, Helvetica (not Times) will be selected as the text
font.

There may be many other fonts available on your
installation. Typically, at least some of the following packages
should be available: \texttt{avant}, \texttt{bookman},
\texttt{chancery}, \texttt{charter},  \texttt{euler}, \texttt{helvet},
\texttt{mathtime}, \texttt{mathptm}, \texttt{newcent},
\texttt{palatino}, \texttt{pifont}, \texttt{times},
\texttt{utopia}.

If you use \texttt{times} together with the \texttt{serif} option, you
may wish to include also the package \texttt{mathptm}. If you use the
\texttt{mathtime} package (you have to buy some of the fonts), you
also need to specify the \texttt{serif} option.



\subsection{Margins and Sizes}

The ``paper size'' of a beamer presentation is fixed to 128mm times
96mm. The aspect ratio of this size is 4:3, which is exactly what most
beamers offer these days. It is the job of the
presentation program (like \verb!acroread!) to display the slides at
full screen size. The main advantage of using a small ``paper size''
is that you can use all your normal fonts at their natural sizes. In
particular, inserting a graphic with 11pt labels will result in
reasonably sized labels during the presentation.

You should refrain from changing the ``paper size.'' However, you
\emph{can} change the size of the left and right margins, which
default to 1cm. To change them, you should use the following two
commands:

\Command{beamersetleftmargin}
\Parameters{
\item a new left margin, \emph{excluding} the left side bar, if present.
  }
\Description{
  Sets a new left margin. This excludes the left side bar. Thus, it is
the distance between the right edge of the left side bar and the left
edge of the text. This command can only be used in the preamble
(before the \texttt{document} environment is used).
  }
\Example \verb!\beamersetleftmargin{1cm}!

\Command{beamersetrightmargin}
\Parameters{
\item a new right margin, excluding the right side bar, if present.
  }
\Description{
  Like \texttt{beamersetleftmargin}, only for the right margin.
  }
\Example \verb!\beamersetleftmargin{1cm}!

For more information on side bars, see Section~\ref{section-sidebar-templates}.



\subsection{Class Options}

Class options are listed right behind the command \verb!\documentclass!
in square brackets. Class options, see the following list, govern
certain global behaviors of the presentation.

\ClassOption{notes}
\Description{
  Include notes in the output file. Normally, notes are not included.
  }

\ClassOption{notesonly}
\Description{
  Include only the notes in the output file. Useful for printing them.
  }

\ClassOption{handout}
\Description{
  Create a version that uses the \texttt{handout} overlay
  specifications. See subsection~\ref{handout}.
  }

\ClassOption{trans}
\Description{
  Create a version that uses the \texttt{trans} overlay
  specifications. See subsection~\ref{trans}.
  }

\ClassOption{inrow}
\Description{
  All small frame representation in the navigation bars for a single
  section are shown alongside each other. Normally, the representation
  for different subsections are shown in different lines.
  }

\ClassOption{slidescentered}
\Description{
  Place text of slides at the (vertical) center of the slides. This is
  the default.
  }

\ClassOption{slidestop}
\Description{
  Place text of slides at the (vertical) top of the slides. This
  corresponds to a vertical ``flush.''
}

\ClassOption{blue, red, gray, brown}
\Description{
  These options change the main color of the navigation and title bars
  to the given colors. Other colors can be setup by redefining the
  color \texttt{structure}.
  }

\ClassOption{bigger}
\Description{
  Makes all fonts a little bigger, which makes the text more
  readable. The downside is that less fits onto each frame.
  }

\ClassOption{smaller}
\Description{
  Makes all fonts a little smaller, which allows you to fit more onto
  frames. Normally, this is not a good idea.
  }

\ClassOption{sans}
\Description{
  Use a sans-serif font during the presentation. (Default.)
  }

\ClassOption{serif}
\Description{
  Use a serif font during the presentation.
}

\ClassOption{mathsans}
\Description{
  Override the math font to be a sans-serif font.
  }

\ClassOption{mathserif}
\Description{
  Override the math font to be a serif font.
}





\subsection{Themes}

Just like \LaTeX\ in general, the \beamer\ class tries to separate the
contents of a text from the way it is typeset (displayed). There are two ways in
which you can change how a presentation is typeset: you can specify a
different theme and you can specify different templates. A theme is
a predefined collection of templates.

There exist a number of different predefined themes that can be used
together with the \beamer\ class. Feel free to add further themes.
Themes are used by including an appropriate \LaTeX\ style file, using
the standard \verb!\usepackage! command.


\subsubsection{Bars Theme}

\Theme{beamerthemebars}
\Parameters{
\item Package option \texttt{headheight}, which specifies the height
of the head line. Specified in a key~=~value fashion.
\item Package option \texttt{footheight}, which specifies the height
of the foot line. Specified in a key~=~value fashion.
}
\Example
\begin{verbatim}
\usepackage[headheight=2cm,footheight=1cm]{beamerthemeboxes}
\end{verbatim}

\Example

\hbox{\pgfuseimage{themebars}\quad\pgfuseimage{themebars2}}


\subsubsection{Boxes Theme}

\vbox{
\Theme{beamerthemeboxes}

\Example

\hbox{\pgfuseimage{themeboxes}\quad\pgfuseimage{themeboxes2}}
}

For this theme, you can specify an arbitrary number of templates for
the boxes in the head line and in the foot line. You can add a
template for another box by using the following commands.

\Command{addheadboxtemplate}
\Parameters{
\item a color command for the background of the box
\item a template for a new box
  }
\Description{
  Each time this command is invoked, a new box is added to the head
  line, with the first added box being shown on the left. All boxes
  will have the same size.
}
\Example
\begin{verbatim}
\addheadboxtemplate{\color{black}}{\color{white}\tiny\quad Left Box}
\addheadboxtemplate{\color{black}}{\color{white}\tiny\quad Right Box}
\end{verbatim}

\Command{addfootboxtemplate}
\Parameters{
\item a color command for the background of the box
\item a template for a new box
  }
\Example
\begin{verbatim}
\addheadfoottemplate{\color{black}}{\color{white}\tiny\quad Big Box}
\end{verbatim}




\subsubsection{Classic Theme}

\vbox{
\Theme{beamerthemeclassic}

\Example

\hbox{\pgfuseimage{themeclassic}\quad\pgfuseimage{themeclassic2}}
}


\subsubsection{Lined Theme}

\vbox{
\Theme{beamerthemelined}

\Example

\hbox{\pgfuseimage{themelined}\quad\pgfuseimage{themelined2}}
}




\subsubsection{Plain Theme}

\vbox{
\Theme{beamerthemeplain}

\Example

\hbox{\pgfuseimage{themeplain}\quad\pgfuseimage{themeplain2}}
}


\subsubsection{Side Bar Themes}

\Theme{beamerthemesidebar}
\Parameters{
\item Package option \texttt{width}, which specifies the width of 
of the side bar. Specified in a key~=~value fashion.
}
\Example
\begin{verbatim}
\usepackage[width=3cm]{beamerthemesidebar}
\end{verbatim}

\Example

\hbox{\pgfuseimage{themesidebar}\quad\pgfuseimage{themesidebar2}}


The following themes take the same parameter as the normal side bar
theme. They only differ in the coloring of the side bar.

\vbox{
\Theme{beamerthemesidebartab}

\Example

\hbox{\pgfuseimage{themesidebartab}\quad\pgfuseimage{themesidebartab2}}
}

\vbox{
\Theme{beamerthemesidebardark}

\Example

\hbox{\pgfuseimage{themesidebardark}\quad\pgfuseimage{themesidebardark2}}
}

\vbox{
\Theme{beamerthemesidebartabdark}

\Example

\hbox{\pgfuseimage{themesidebardarktab}\quad\pgfuseimage{themesidebardarktab2}}
}


\subsubsection{Split Theme}

\vbox{
\Theme{beamerthemesplit}

\Example

\hbox{\pgfuseimage{themesplit}\quad\pgfuseimage{themesplit2}}
}


\subsubsection{Condensed Split Theme}

\vbox{
\Theme{beamerthemesplitcondensed}

\Example

\hbox{\pgfuseimage{themesplitcondensed}\quad\pgfuseimage{themesplitcondensed2}}
}
\vbox{
\Theme{beamerthemetree}



\subsubsection{Tree Themes}

\Example

\hbox{\pgfuseimage{themetree}\quad\pgfuseimage{themetree2}}
}

\vbox{
\Theme{beamerthemetreebars}

\Example

\hbox{\pgfuseimage{themetreebars}\quad\pgfuseimage{themetreebars2}}
}


\subsection{Templates}
\label{section-templates}

If you only wish to modify a small part of how your presentation is
rendered, you do not need to create a whole new theme. Instead, you
can modify an appropriate template.

A template specifies how a part of a presentation is typeset. For
example, the frame title template dictates where the frame title is
put, which font is used, and so on.

As the name suggests, you specify a template by writing the exact
\LaTeX\ code you would also use when typesetting a single frame title
by hand. Only, instead of the actual title, you use the command
\verb!\insertframetitle!.

For example, suppose we would like to have the frame title typeset in
red, centered, and boldface. If we were to typeset a single frame
title by hand, it might be done like this:
\begin{verbatim}
\frame
{
  \begin{centering}
    \color{red}
    \textbf{The Title of This Frame.}
    \par
  \end{centering}

  Blah, blah.
}
\end{verbatim}

In order to typeset the frame title in this way on all slides, we can
change the frame title template as follows:
\begin{verbatim}
\useframetitletemplate{
  \begin{centering}
    \color{red}
    \textbf{\insertframetitle}
    \par
  \end{centering}
}
\end{verbatim}

We can then use the following code to get the desired effect:
\begin{verbatim}
\frame
{
  \frametitle{The Title of This Frame.}

  Blah, blah.
}
\end{verbatim}

When rendering the frame, the \beamer\ class will use the code of the
frame title template to typeset the frame title and it will replace
every occurrence of \verb!\insertframetitle! by the current frame
title.

In the following subsections all commands for changing templates are
listed, like the above-mentioned command
\verb!\useframetitletemplate!. Inside these commands, you should use
the \verb!\insertxxxx! commands listed in 
the next subsection. 

Some of the below subsections start with commands for using
\emph{predefined} templates. These commands are defined in the package
\verb!beamertemplates!. Calling one of them will change a template in
a predefined way. Using them, you can use, for example, your favorite
theme together with a predefined background.



\subsubsection{Title Page}

\Command{usetitlepagetemplate}
\Parameters{
\item a template for the title page
  }
\Example
\begin{verbatim}
\usetitlepagetemplate{
  \vbox{}
  \vfill
  \begin{centering}
    \Large\structure{\inserttitle}
    \vskip1em\par
    \normalsize\insertauthor\vskip1em\par
    {\scriptsize\insertinstitute\par}\par\vskip1em
    \insertdate\par\vskip1.5em
    \inserttitlegraphic
  \end{centering}
  \vfill
}
\end{verbatim}

If you wish to suppress the head and foot line in the title page, use
\verb!\plainframe{\titlepage}!.



\subsubsection{Background}

\label{section-backgrounds}

\paragraph{Predefined Templates}\ 

\Command{beamertemplateshadingbackground}
\Parameters{
\item Name of the color at the page bottom.
\item Name of the color at the page top.
  }
\Description{Installs a vertically shaded background such that the
  specified bottom color changes smoothly to the specified top
  color. \textbf{Use with care: Background shadings are often
    distracting!} However, a very light shading with warm colors can
  make a presentation more lively.}
\Example
\begin{verbatim}
\beamertemplateshadingbackground{red!10}{blue!10}
% Bottom is light red, top is light blue
\end{verbatim}

\Command{beamertemplategridbackground}
\Description{Installs a light grid as background.}

\paragraph{Template Changing Commands}\ 

\Command{usebackgroundtemplate}
\Parameters{
\item a template for the page background
  }
\Example
\begin{verbatim}
\usebackgroundtemplate{%
  \color{red}%
  \vrule  height\paperheight width\paperwidth%
}
\end{verbatim}


\subsubsection{Table of Contents}

\label{section-toc-templates}

\Command{usetemplatetocsection}
\Parameters{
\item A color mix-in specification for grayed sections names. This
parameter is optional and given in square brackets. If present, the
normal parameter for the grayed section name must be omitted.
\item A template for a section name in the table of contents.
\item A template for a grayed section name in the table of contents.
  }
\Example
\begin{verbatim}
\usetemplatetocsection
{\color{structure}\inserttocsection}
{\color{structure!50}\inserttocsection}

\usetemplatetocsection[50!averagebackgroundcolor]
{\color{structure}\inserttocsection}
\end{verbatim}

\Command{usetemplatetocsubsection}
\Parameters{
\item A color mix-in specification for grayed subsections names. This
parameter is optional and given in square brackets. If present, the
normal parameter for the grayed subsection name must be omitted.
\item A template for a subsection name in the table of contents.
\item A template for a grayed subsection name in the table of contents.
  }
\Example
\begin{verbatim}
\usetemplatetocsubsection
{\leavevmode\leftskip=1.5em\color{black}\inserttocsubsection\par}
{\leavevmode\leftskip=1.5em\color{black!50}\inserttocsubsection\par}

\usetemplatetocsection[50!averagebackgroundcolor]
{\leavevmode\leftskip=1.5em\color{black}\inserttocsubsection\par}
\end{verbatim}


\subsubsection{Bibliography}

\label{section-bib-templates}

\paragraph{Predefined Templates}\

\Command{beamertemplatetextbibitems}
\Description{Shows the citation text in front of references in a
  bibliography instead of a small symbol.} 

\Command{beamertemplatearrowbibitems}
\Description{Changes the symbol before references in a bibliography to
  a small arrow.}

\Command{beamertemplatebookbibitems}
\Description{Changes the symbol before references in a bibliography to
  a small book icon.}

\Command{beamertemplatearticlebibitems}
\Description{Changes the symbol before references in a bibliography to
  a small article icon. (Default)}



\paragraph{Template Changes Commands}\ 

\Command{usebibitemtemplate}
\Parameters{
\item a template for the citation text before the entry. (The
  ``label'' of the item.)
}
\Description{
  Use \texttt{insertbiblabel} to insert the label text.
  }
\Example \verb!\usebibitemtemplate{\color{structure}\insertbiblabel}!


\Command{usebibliographyblocktemplate}
\Parameters{
\item a template to be inserted before the first block of the entry
  (the first block is all text before the first occurrence of a
  \texttt{newblock} command).
\item a template to be inserted before the second block (the text
  between the first and second occurrence of \texttt{newblock})
\item a template to be inserted before the third block
\item a template to be inserted before all other blocks
}
\Description{
  The templates are inserted \emph{before} the blocks and you do not
  have access to the blocks themselves via insert commands. In the
  following example, the first \texttt{par} commands ensure that the
  author, the title, and the journal are put on different lines. The
  color commands cause the author (first block) to be typeset using
  the theme color, the second block (title of the paper) to be typeset
  in black, and all other lines to be typeset in a washed-out version
  of the theme color. 
  }
\Example
\begin{verbatim}
  \usebibliographyblocktemplate
  {\color{structure}}
  {\par\color{black}}
  {\par\color{structure!75}}
  {\par\color{structure!75}}
\end{verbatim}




\subsubsection{Frame Titles}

\Command{useframetitletemplate}
\Parameters{
\item a template for the frame title
  }
\Example
\begin{verbatim}
\useframetitletemplate{%
  \begin{centering}
    \structure{\textbf{\insertframetitle}}
    \par
  \end{centering}
}
\end{verbatim}






\subsubsection{Head Lines and Foot Lines}

\label{section-head-templates}

\paragraph{Predefined Templates}\ 

\Command{beamertemplateheadempty}
\Description{Makes the head line empty.}

\Command{beamertemplatefootempty}
\Description{Makes the foot line empty.}

\Command{beamertemplatefootpagenumber}
\Description{Shows only the page number in the foot line.}


\paragraph{Template Changing Commands}\ 

\Command{usefoottemplate}
\Parameters{
\item a template for the foot line
  }
\Example
\begin{verbatim}
\usefoottemplate{\hfil\tiny{\color{black!50}\insertpagenumber}}
\end{verbatim}
or
\begin{verbatim}
\usefoottemplate{%
  \vbox{%
    \tinycolouredline{structure!75}%
      {\color{white}\textbf{\insertshortauthor\hfill\insertshortinstitute}}%
    \tinycolouredline{structure}%
      {\color{white}\textbf{\insertshorttitle}\hfill}%
    }}
\end{verbatim}

\Command{useheadtemplate}
\Parameters{
\item a template for the head line
  }
\Example
\begin{verbatim}
\useheadtemplate{%
  \vbox{%
  \vskip3pt%
  \beamerline{\insertnavigation{\paperwidth}}%
  \vskip1.5pt%
  \insertvrule{0.4pt}{structure!50}}%
}
\end{verbatim}





\subsubsection{Side Bars}

\label{section-sidebar-templates}

Side bars are vertical areas that stretch from the lower end of the
head line to the top of the foot line. There can be a side bar at the
left and one at the right (or even both). Side bars can show a table
of contents, but they could also be added for purely aesthetic
reasons.

When you install a side bar template, you must explicitly specify the
horizontal size of the side bar. The vertical size is determined
automatically. Each side bar can have its own background, which can be
setup using special side background templates.

Adding a sidebar of a certain size, say 1cm, will make the main text
1cm narrower. The distance between the inner side of a side
bar and the outer side of the text, as specified by
the command \verb!\beamersetleftmargin! and its counterpart for the
right margin, is not changed when a side bar is installed.

Internally, the sidebars are typeset by showing them as part of the
headline. The \beamer\ class keeps track of six dimensions, three 
for each side: the variables \verb!\beamer@leftsidebar! and
\verb!\beamer@rightsidebar! store the (horizontal) sizes of the side
bars, the variables \verb!\beamer@leftmargin! and
\verb!\beamer@rightmargin! store the distance between sidebar and
text, and the macros \verb!\Gm@lmargin! and  \verb!\Gm@rmargin! store
the distance from the edge of the paper to the edge of the text. Thus
the sum \verb!\beamer@leftsidebar! and \verb!\beamer@leftmargin! is
exactly  \verb!\Gm@lmargin!. Thus, if you wish to put some text right
next to the left side bar, you might write
\verb!\hskip-\beamer@leftmargin! to get there.

In the following, only the commands for the left side bars are
listed. Each of these commands also exists for the right side bar,
with ``left'' replaced by ``right'' everywhere.


\Command{useleftsidebartemplate}
\Parameters{
\item the horizontal size of the left side bar.
\item a template for formating the left side bar.
  }
\Description{
  When the side bar is typeset, the template is invoked inside a
  \texttt{vbox} of the height of the side bar. Thus, the below example
  will produce a side bar of half a centimeter width, in which the word
  ``top'' is printed just below the head line and ``bottom'' is printed
  just above the foot line.
}
\Example
\begin{verbatim}
\useleftsidebartemplate{1cm}{
  top
  \vfill
  bottom
}
\end{verbatim}


\Command{useleftsidebarbackgroundtemplate}
\Parameters{
\item a template for the background of  the left side bar.
  }
\Description{
  The template is shown behind whatever is shown in the left side
  bar. 
}
\Example
\begin{verbatim}
\useleftsidebarbackgroundtemplate
  {\color{red}\vrule height\paperheight width\beamer@leftsidebar}
\end{verbatim}


\Command{useleftsidebarcolortemplate}
\Parameters{
\item A color command.
  }
\Description{
  Uses the given color as background for the side bar.
}
\Example
\begin{verbatim}
\useleftsidebarcolortemplate{\color{red}}
\useleftsidebarcolortemplate{\color[rgb]{1,0,0.5}}
\end{verbatim}


\Command{useleftsidebarverticalshadingtemplate}
\Parameters{
\item Name of the color at the bottom of the side bar.
\item Name of the color at the top of the side bar.
  }
\Description{
  Installs a smooth vertical transition between the given colors as
  background for the side bar.
}
\Example
\begin{verbatim}
\useleftsidebarverticalshadingtemplate{white}{red}
\end{verbatim}


\Command{useleftsidebarhorizontalshadingtemplate}
\Parameters{
\item Name of the color at the left end of the side bar.
\item Name of the color at the right end of the side bar.
  }
\Description{
  Installs a smooth horizontal transition between the given colors as
  background for the side bar.
}
\Example
\begin{verbatim}
\useleftsidebarhorizontalshadingtemplate{white}{red}
\end{verbatim}


\Command{usesectionsidetemplate}
\Parameters{
\item A template for the current section name in a side navigation
bar. Should be an \texttt{hbox}.
\item a template for a different section name in a side navigation
bar. Should be an \texttt{hbox}.
  }
\Example
\begin{verbatim}
\usesectionsidetemplate
{\setbox\tempbox=\hbox{\color{black}\tiny{\kern3pt\insertsectionhead}}%
  \ht\tempbox=8pt%
  \dp\tempbox=2pt%
  \wd\tempbox=\beamer@sidebarwidth%
  \box\tempbox}
{\setbox\tempbox=\hbox{\color{structure!75}\tiny{\kern3pt\insertsectionhead}}%
  \ht\tempbox=8pt%
  \dp\tempbox=2pt%
  \wd\tempbox=\beamer@sidebarwidth%
  \box\tempbox}
\end{verbatim}


\Command{usesubsectionsidetemplate}
\Parameters{
\item A template for the current subsection name in a side navigation
bar. Should be an \texttt{hbox}.
\item a template for a different subsection name in a side navigation
bar. Should be an \texttt{hbox}.
  }
\Example
\begin{verbatim}
\usesectionsidetemplate
{\setbox\tempbox=\hbox{\color{black}\tiny{\kern3pt\insertsectionhead}}%
  \ht\tempbox=8pt%
  \dp\tempbox=2pt%
  \wd\tempbox=\beamer@sidebarwidth%
  \box\tempbox}
{\setbox\tempbox=\hbox{\color{structure!75}\tiny{\kern3pt\insertsectionhead}}%
  \ht\tempbox=8pt%
  \dp\tempbox=2pt%
  \wd\tempbox=\beamer@sidebarwidth%
  \box\tempbox}
\end{verbatim}











\subsubsection{Navigation Bars}


\paragraph{Predefined Templates}\ 

\Command{beamertemplateboxminiframe}
\Description{Changes the symbols in a navigation bar used to represent
  a frame to a small box.}

\Command{beamertemplateticksminiframe}
\Description{Changes the symbols in a navigation bar used to represent
  a frame to a small vertical bar of varying length.}


\paragraph{Template Changes Commands}\ 

\Command{usesectionheadtemplate}
\Parameters{
\item a template for the current section name in a navigation bar.
\item a template for a different section name in a navigation bar.
  }
\Example
\begin{verbatim}
\usesectionheadtemplate
{\color{structure}\tiny\insertsectionhead}
{\color{structure!50}\tiny\insertsectionhead}
\end{verbatim}


\Command{usesubsectionheadtemplate}
\Parameters{
\item a template for the current subsection name in a navigation bar.
\item a template for a different subsection name in a navigation bar.
  }
\Example
\begin{verbatim}
\usesubsectionheadtemplate
{\color{structure}\tiny\insertsubsectionhead}
{\color{structure!50}\tiny\insertsubsectionhead}
\end{verbatim}


\Command{useminislidetemplate}
\Parameters{
\item a template for the mini frame of the current frame in a navigation bar.
\item a template for the mini frame of a frame of the current
  subsection in a navigation bar.
\item a template for the mini frame of other frames in a navigation bar.
\item horizontal offset between mini frames.
\item vertical offset between mini frames.}
\Example
\begin{verbatim}
\useminislidetemplate
  {
    \color{structure}%
    \hskip-0.4pt\vrule height\boxsize width1.2pt%
  }  
  {%
    \color{structure}%
    \vrule height\boxsize width0.4pt%
  }
  {%
    \color{structure!50}%
    \vrule height\boxsize width0.4pt%
  }
  {.1cm}
  {.05cm}
\end{verbatim}




\subsubsection{Footnotes}

\Command{usefootnotetemplate}
\Parameters{
\item A template for formating a footnote.
  }
\Example
\begin{verbatim}
\usefootnotetemplate{
  \parindent 1em
  \noindent
  \hbox to 1.8em{\hfil\insertfootnotemark}\insertfootnotetext}
\end{verbatim}





\subsubsection{Captions}
\label{section-template-caption}

\paragraph{Predefined Templates}\

\Command{beamertemplatecaptionwithnumber}
\Description{Changes the caption template such that the number of the
table or figure is also shown.}

\Command{beamertemplatecaptionownline}
\Description{Changes the caption template such that the word ``Table''
or ``Figure'' has its own line.}



\paragraph{Template Changes Commands}\

\Command{usecaptiontemplate}
\Parameters{
\item a caption template.
  }
\Example
\begin{verbatim}
\usecaptiontemplate{
  \small
  \structure{\insertcaptionname~\insertcaptionnumber:}
  \insertcaption
}
\end{verbatim}




\subsubsection{Lists (Itemizations, Enumerations, Descriptions)}

\paragraph{Predefined Templates}\

\Command{beamertemplatedotitem}
\Description{Changes the symbols shown in an \texttt{itemize}
  environment to dots.}

\Command{beamertemplateballitem}
\Description{Changes the symbols shown in an \texttt{itemize}
  environment to small plastic balls.}


\paragraph{Template Changes Commands}\

\Command{useenumerateitemtemplate}
\Parameters{
\item a template for the default item in the top level of an
  enumeration. 
  }
\Example \verb!\useenumerateitemtemplate{\insertenumlabel}!

\Command{useitemizeitemtemplate}
\Parameters{
\item a template for the default item in the top level of an
  itemize list.
  }
\Example \verb!\useitemizeitemtemplate{\pgfuseimage{mybullet}}!

\Command{usesubitemizeitemtemplate}
\Parameters{
\item a template for the default item in the second level of an
  itemize list.
  }
\Example \verb!\usesubitemizeitemtemplate{\pgfuseimage{mysubbullet}}!

\Command{useitemizetemplate}
\Parameters{
\item a template for the beginning a top-level itemize list.
\item a template for the end of a top-level itemize list.
  }
\Example \verb!\useitemizetemplate{}{}!

\Command{usesubitemizetemplate}
\Parameters{
\item a template for the beginning a second-level itemize list.
\item a template for the end of a second-level itemize list.
  }
\Example \verb!\usesubitemizetemplate{\begin{small}}{\end{small}}!

\Command{useenumeratetemplate}
\Parameters{
\item a template for the beginning a top-level enumeration.
\item a template for the end of a top-level enumeration.
  }
\Example \verb!\useenumeratetemplate{}{}!

\Command{usesubenumerateitemtemplate}
\Parameters{
\item a template for the default item in the second level of an
  enumeration. 
  }
\Example
\verb!\usesubenumerateitemtemplate{\insertenumlabel-\insertsubenumlabel}!

\Command{usesubenumeratetemplate}
\Parameters{
\item a template for the beginning a second-level enumeration.
\item a template for the end of a second-level enumeration.
  }
\Example \verb!\usesubenumeratetemplate{\begin{small}}{\end{small}}!


\Command{usedescriptiontemplate}
\Parameters{
\item A template for the default item in a description. Use
  \texttt{insertdescriptionitem} to insert the current item text.
\item A default width for the default item, if no other width is
  specified; the width \texttt{labelsep} is automatically added to
  this parameter.
  }
\Example \verb!\usedescriptionitemtemplate{\color{structure}\insertdescriptionitem}{2cm}!



\subsubsection{Hilighting Commands}

\Command{usealerttemplate}
\Parameters{
\item a template for the \texttt{alert} command
  }
\Example \verb!\usealerttemplate{{\color{red}\insertalert}}!

\Command{usestructuretemplate}
\Parameters{
\item a template for the \texttt{structure} command
  }
\Example \verb!\usestructuretemplate{{\color{structure}\insertstructure}}!




\subsubsection{Block Environments}

\Command{useblocktemplate}
\Parameters{
\item a template for the beginning of the block.
\item a template for the end of the block.
  }
\Example
\begin{verbatim}
\useblocktemplate
  {%
   \medskip%
    {\color{blockstructure}\textbf{\insertblockname}}%
    \par%
  }
  {\medskip}
\end{verbatim}

\Command{usealertblocktemplate}
\Parameters{
\item a template for the beginning of the block.
\item a template for the end of the block.
  }
\Example
\begin{verbatim}
\usealertblocktemplate
  {%
    \medskip
    {\alert{\textbf{\insertblockname}}}%
  \par}
  {\medskip}
\end{verbatim}

\Command{useexampleblocktemplate}
\Parameters{
\item a template for the beginning of the block.
\item a template for the end of the block.
  }
\Example
\begin{verbatim}
\useexampleblocktemplate
  {%
    \medskip
    \begingroup\color{darkgreen}{\textbf{\insertblockname}}
    \par}
  {%
     \endgroup
     \medskip
  }
\end{verbatim}





\subsection{Template Inserts}

In the following, an alphabetical list of the different
\verb!\insertxxxx! commands is given. These commands are used inside
templates.


\Command{insertalert}
\Description{
  Inserts the current alerted text into a template.
}

\Command{insertauthor}
\Description{
  Inserts the one-line version of the author names into a template.
}

\Command{insertbiblabel}
\Description{
  Inserts the current citation label into a template.
  }

\Command{insertblockname}
\Description{
  Inserts the name of the current block into a template.
}

\Command{insertcaption}
\Description{
  Inserts the text of the current caption into a template.
}

\Command{insertcaptionname}
\Description{
  Inserts the name of the current caption into a template. This word
  is either ``Table'' or ``Figure'' or, if the \texttt{babel} package is
  used, some translation thereof.
}

\Command{insertcaptionnumber}
\Description{
  Inserts the number of the current figure or table into a template.
}

\Command{insertdate}
\Description{
  Inserts the date into a template.
}

\Command{insertdescriptionitem}
\Description{
  Inserts the current item of a description environment into a
  template.
}

\Command{insertenumlabel}
\Description{
  Inserts the current number of the top-level enumeration (as an
  Arabic number) into a template.
}

\Command{insertfootnotemark}
\Description{
  Inserts the current footnote mark (like a raised number) into a
  template. 
}

\Command{insertfootnotetext}
\Description{
  Inserts the current footnote text into a template. 
}

\Command{insertframenumber}
\Description{
  Inserts the number of the current frame (not slide) into a template.
}

\Command{insertframetitle}
\Description{
  Inserts the current frame title into a template.
}

\Command{insertinstitute}
\Description{
  Inserts the institute into a template.
}

\Command{insertlogo}
\Description{
  Inserts the logo(s) into a template.
}

\Command{insertnavigation}
\Parameters{
\item a width
  }
\Description{
  Inserts a horizontal navigation bar of the given width into a
  template. The bar lists the sections and below them mini frames for
  each frame in that section.
}

\Command{insertpagenumber}
\Description{
  Inserts the current page number into a template.
}

\Command{insertsection}
\Description{
  Inserts the current section into a template.
}

\Command{insertsectionnavigation}
\Parameters{
\item a width
  }
\Description{
  Inserts a vertical navigation bar containing all sections, with the
  current section hilighted.
}

\Command{insertsectionnavigationhorizontal}
\Parameters{
\item a width
\item a text (typical a glue) to be inserted at the left
\item a text (typical a glue) to be inserted at the right
  }
\Description{
  Inserts a horizontal navigation bar containing all sections, with the
  current section hilighted. By inserting a triple fill (a
  \texttt{filll}) you can flush to bar to the left or right.
}
\Example
\begin{verbatim}
\insertsectionnavigationhorizontal{.5\textwidth}{\hskip0pt plus1filll}{}
\end{verbatim}

\Command{insertshortauthor}
\Description{
  Inserts the short version of the author into a template.
}

\Command{insertshortdate}
\Description{
  Inserts the short version of the date into a template.
}

\Command{insertshortinstitute}
\Description{
  Inserts the short version of the institute into a template.
}

\Command{insertshorttitle}
\Description{
  Inserts the short version of the document title into a template.
}

\Command{insertstructure}
\Description{
  Inserts the current structure text into a template.
}

\Command{insertsubenumlabel}
\Description{
  Inserts the current number of the second-level enumeration (as an
  Arabic number) into a template.
}

\Command{insertsubsection}
\Description{
  Inserts the current subsection into a template.
}

\Command{insertsubsectionnavigation}
\Parameters{
\item a width
  }
\Description{
  Inserts a vertical navigation bar containing all subsections of the
  current section, with the current subsection hilighted.
}

\Command{insertsubsectionnavigationhorizontal}
\Parameters{
\item a width
\item a text (typical a glue) to be inserted at the left
\item a text (typical a glue) to be inserted at the right
  }
\Description{
  Inserts a horizontal navigation bar containing all subsections, with the
  current section hilighted. By inserting a triple fill (a
  \texttt{filll}) you can flush to bar to the left or right.
}
\Example
\begin{verbatim}
\insertsubsectionnavigationhorizontal{.5\textwidth}{}{\hskip0pt plus1filll}
\end{verbatim}

\Command{inserttitle}
\Description{
  Inserts a version of the document title into a template that is
  useful for the title page. 
}

\Command{inserttitlegraphic}
\Description{
  Inserts the title graphic into a template.
}

\Command{inserttocsection}
\Description{
  Inserts the version of the current section name into a template that
  is useful for the table of contents.
  }

\Command{inserttocsubsection}
\Description{
  Inserts the version of the current subsection name into a template
  that is useful for the table of contents.
  }

\Command{insertverticalnavigation}
\Parameters{
\item a width
  }
\Description{
  Inserts a vertical navigation bar of the given width into a
  template. The bar shows a little table of contents. The indiviual
  lines are typeset using the templates
  \texttt{usesectionsidetemplate} and \texttt{usesubsectionsidetemplate}.
}

\Command{insertvrule}
\Parameters{
\item a color
\item a thickness
  }
\Description{
  Inserts a rule of the given color and thickness into a
  template. 
  }

\end{document}
