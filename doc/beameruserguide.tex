\documentclass{ltxdoc}

% Copyright 2003, 2004 by Till Tantau <tantau@users.sourceforge.net>.
%
% This program can be redistributed and/or modified under the terms
% of the GNU Public License, version 2.

\usepackage{pgf,xcolor}
\usepackage[left=2.25cm,right=2.25cm,top=2.5cm,bottom=2.5cm,nohead]{geometry}
\usepackage{amsmath,amssymb}
\usepackage[pdfborder={0 0 0}]{hyperref}
\usepackage[latin1]{inputenc}

\pgfdeclareimage[width=6.66666cm,height=5cm]{themebars}{themebars}
\pgfdeclareimage[width=6.66666cm,height=5cm]{themebars2}{themebars2}
\pgfdeclareimage[width=6.66666cm,height=5cm]{themeboxes}{themeboxes}
\pgfdeclareimage[width=6.66666cm,height=5cm]{themeboxes2}{themeboxes2}
\pgfdeclareimage[width=6.66666cm,height=5cm]{themeclassic}{themeclassic}
\pgfdeclareimage[width=6.66666cm,height=5cm]{themeclassic2}{themeclassic2}
\pgfdeclareimage[width=6.66666cm,height=5cm]{themelined}{themelined}
\pgfdeclareimage[width=6.66666cm,height=5cm]{themelined2}{themelined2}
\pgfdeclareimage[width=6.66666cm,height=5cm]{themeplain}{themeplain}
\pgfdeclareimage[width=6.66666cm,height=5cm]{themeplain2}{themeplain2}
\pgfdeclareimage[width=6.66666cm,height=5cm]{themesidebar}{themesidebar}
\pgfdeclareimage[width=6.66666cm,height=5cm]{themesidebar2}{themesidebar2}
\pgfdeclareimage[width=6.66666cm,height=5cm]{themesidebardark}{themesidebardark}
\pgfdeclareimage[width=6.66666cm,height=5cm]{themesidebardark2}{themesidebardark2}
\pgfdeclareimage[width=6.66666cm,height=5cm]{themesidebartab}{themesidebartab}
\pgfdeclareimage[width=6.66666cm,height=5cm]{themesidebartab2}{themesidebartab2}
\pgfdeclareimage[width=6.66666cm,height=5cm]{themesidebardarktab}{themesidebardarktab}
\pgfdeclareimage[width=6.66666cm,height=5cm]{themesidebardarktab2}{themesidebardarktab2}
\pgfdeclareimage[width=6.66666cm,height=5cm]{themesplit}{themesplit}
\pgfdeclareimage[width=6.66666cm,height=5cm]{themesplit2}{themesplit2}
\pgfdeclareimage[width=6.66666cm,height=5cm]{themeshadow}{themeshadow}
\pgfdeclareimage[width=6.66666cm,height=5cm]{themeshadow2}{themeshadow2}
\pgfdeclareimage[width=6.66666cm,height=5cm]{themesplitcondensed}{themesplitcondensed}
\pgfdeclareimage[width=6.66666cm,height=5cm]{themesplitcondensed2}{themesplitcondensed2}
\pgfdeclareimage[width=6.66666cm,height=5cm]{themetree}{themetree}
\pgfdeclareimage[width=6.66666cm,height=5cm]{themetree2}{themetree2}
\pgfdeclareimage[width=6.66666cm,height=5cm]{themetreebars}{themetreebars}
\pgfdeclareimage[width=6.66666cm,height=5cm]{themetreebars2}{themetreebars2}

\def\beamer{\textsc{beamer}}
\def\pdf{\textsc{pdf}}
\def\pgf{\textsc{pgf}}
\def\pstricks{\textsc{pstricks}}
\def\prosper{\textsc{prosper}}
\def\seminar{\textsc{seminar}}
\def\texpower{\textsc{texpower}}
\def\foils{\textsc{foils}}

\def\declare#1{{\color{red!75!black}#1}}

\def\command#1{\list{}{\leftmargin=2em\itemindent-\leftmargin\def\makelabel##1{\hss##1}}%
\item\extractcommand#1@\par\topsep=0pt}
\def\endcommand{\endlist}
\def\extractcommand#1#2@{\strut\declare{\texttt{\string#1}}#2}

\def\example{\par\smallskip\noindent\textit{Example: }}

\def\environment#1{\list{}{\leftmargin=2em\itemindent-\leftmargin\def\makelabel##1{\hss##1}}%
\extractenvironement#1@\par\topsep=0pt}
\def\endenvironment{\endlist}
\def\extractenvironement#1#2@{%
\item{{\ttfamily\char`\\begin\char`\{\declare{#1}\char`\}}#2}%
  {\itemsep=0pt\parskip=0pt\item{\meta{environment contents}}%
  \item{\ttfamily\char`\\end\char`\{\declare{#1}\char`\}}}}

\def\classoption#1{\list{}{\leftmargin=2em\itemindent-\leftmargin\def\makelabel##1{\hss##1}}%
\item{{\ttfamily\char`\\documentclass[\declare{#1}]\char`\{beamer\char`\}}}\par\topsep=0pt}
\def\endclassoption{\endlist}


\def\smallpackage{\vbox\bgroup\package}
\def\endsmallpackage{\egroup\endpackage}
\def\package#1{\list{}{\leftmargin=2em\itemindent-\leftmargin\def\makelabel##1{\hss##1}}%
\extracttheme#1@\par\topsep=0pt}
\def\endpackage{\endlist}
\def\extracttheme#1#2@{%
\item{{{\ttfamily\char`\\usepackage}#2{\ttfamily\char`\{\declare{#1}\char`\}}}}}

\def\class#1{\list{}{\leftmargin=2em\itemindent-\leftmargin\def\makelabel##1{\hss##1}}%
\extractclass#1@\par\topsep=0pt}
\def\endclass{\endlist}
\def\extractclass#1#2@{%
\item{{{\ttfamily\char`\\documentclass}#2{\ttfamily\char`\{\declare{#1}\char`\}}}}}

\newcommand\opt[1]{{\color{black!50!green}#1}}
\renewcommand\oarg[1]{\opt{{\ttfamily[}\meta{#1}{\ttfamily]}}}
\newcommand\sarg[1]{\opt{{\ttfamily\char`\<}\meta{#1}{\ttfamily\char`\>}}}
\newcommand\ssarg[1]{{\ttfamily\char`\<}\meta{#1}{\ttfamily\char`\>}}

\providecommand{\LyX}{L\kern-.1667em\lower.25em\hbox{Y}\kern-.125emX\@}


\newcommand{\beamernote}{\par\noindent\llap{\color{blue}\textsc{beamer}\ \ }}
\newcommand{\articlenote}{\par\smallskip\noindent\llap{\color{blue}\textsc{article}\ \ }}
\newcommand{\lyxnote}{\par\smallskip\noindent\llap{\color{blue}\textsc{lyx}\ \ }}




\begin{document}

\title{User's Guide to the Beamer Class, Version 2.10-CVS\\
\Large\href{http://latex-beamer.sourceforge.net}{\texttt{http://latex-beamer.sourceforge.net}}}
\author{Till Tantau\\
  \href{mailto:tantau@users.sourceforge.net}{\texttt{tantau@users.sourceforge.net}}}

\maketitle

\tableofcontents

\section{Introduction}

This user's guide explains the functionality of the \beamer\ class.
It is a \LaTeX\ class that allows you to create a presentation with a
projector. It can also be used to create slides. It behaves 
similarly to other packages like \textsc{prosper}, but has the
advantage that it works together directly with |pdflatex|, but
also with |dvips|.

\subsection{Getting Started with the Beamer Class and \LaTeX/pdf\LaTeX}

To use the \beamer\ class together with |latex| or |pdflatex|, proceed
as follows: 
\begin{enumerate}
\item
  Specify |beamer| as document class instead of
  |article|.
\item
  Structure your \LaTeX\ text using |section| and
  |subsection| commands.
\item
  Place the text of the individual slides inside |frame|
  commands.
\item
  Run |pdflatex| on the text (or |latex|,
  |dvips|, and |ps2pdf|).
\end{enumerate}

The \beamer\ class has several useful features: You don't need any
external programs to use it other than |pdflatex|, but it works
also with |dvips|. You can easily and intuitively create
sophisticated overlays. Finally, you can easily change the whole slide
theme or only parts of it. The following code shows a typical usage of
the class.

\begin{verbatim}
\documentclass{beamer}

\usepackage{beamerthemesplit}

\title{Example Presentation Created with the Beamer Package}
\author{Till Tantau}
\date{\today}

\begin{document}

\frame{\titlepage}

\section*{Outline}
\frame{\tableofcontents}

\section{Introduction}
\subsection{Overview of the Beamer Class}
\frame
{
  \frametitle{Features of the Beamer Class}

  \begin{itemize}
  \item<1-> Normal LaTeX class.
  \item<2-> Easy overlays.
  \item<3-> No external programs needed.      
  \end{itemize}
}
\end{document}
\end{verbatim}

Run |pdflatex| on this code (twice) and then use, for example, the
Acrobat Reader to present the resulting |.pdf| file in a
presentation. You can also, alternatively, use |dvips|; see
Section~\ref{section-postscript} for details.

As can be seen, the text looks almost like a normal \LaTeX\ text. The
main difference is the usage of the |\frame| command. This
command takes one parameter, which is the text that should be shown on
the frame. Typically, the contents of a frame is shown on a single
slide. However, in case you use overlay commands inside a frame, a
single frame command may produce several slides. An example is the
last frame in the above example. There, the |\item| commands
are followed by \emph{overlay specifications} like |<1->|,
which means ``from slide 1 on.'' Such a specification causes the item
to be shown only on the specified slides of the frame (see
Section~\ref{section-overlay} for details). In the above example, a
total of five slides are produced: a title page slide, an outline
slide, a slide showing only the first of the three items, a slide
showing the first two of them, and a slide showing all three items.
 
To structure your text, you can use the commands |\section| and
|\subsection|. These commands will not only create entries in the
table of contents, but will also in the navigation bars.


\subsection{Getting Started with the Beamer Class and \LyX}

Once installed (see Section~\ref{section-installation}), using the
\beamer\ class  together with \LyX\ is quite easy: You open a new file
and choose |beamer| as the document class. It is often even easier to
choose ``New from template'' and to pick a template from the directory
|beamer/lyx/templates|. 

To reproduce the example from the previous subsection in \LyX, proceed
as follows:

\begin{itemize}
\item
  The command |\usepackage{beamerthemesplit}| must be added to the
  preamble. You can edit the preamble using Layout $\rangle$ Document
  $\rangle$ Preamble.
\item
  Typeset the author and date the usual way, using the styles Author
  and Date. The title page will then
  be created automatically.
\item
  To insert the sections and subsections, use the usual Section and
  Subsection styles.
\item
  To insert the frame containing the table of contents, insert a line
  of style BeginFrame. Since this frame has no title, do not write
  anything on the line with style BeginFrame. Next, insert a line of
  style Standard and use Insert $\rangle$ Insert TOC to insert the
  table of contents. Optionally, end the frame using a line of style
  EndFrame (the following Section style automatically closes the frame).
\item
  To create the last frame, start a new frame using the style
  BeginFrame. Write the frame title on the line having this
  style.
\item
  Use the Itemize style to create the itemized text.
\item
  Add the overlay specifications (the texts like |<1->|) to the items
  by entering \TeX-mode (press on the little \TeX\ icon) and writing
  |<1->|. This \TeX\ text should be placed right at the beginning of
  the item.
\item
  You must end this frame using the style EndFrame (sadly, the end of
  the document and also the beginning of the appendix do not
  automatically end the last frame -- whereas the start of a frame,
  section, part, or subsection does).
\end{itemize}

Now use View $\rangle$ PDF to view the resulting presentation. On a
slow machine, this may take a while. See Section~\ref{section-speedup}
for ways of speeding up the compilation.


\subsection{How to Read this User's Guide}

This user guide is both intended as a tutorial and as a reference
guide. If you have not yet installed the package, please read
Section~\ref{section-installation} first. If you do not have much
experience with preparing presentations, 
Section~\ref{section-workflow} might be especially helpful. The later
sections explain the basic usage of the |beamer| class as well as
advanced features. If you wish to adjust the way your presentations
look (for example, if you wish to add a default logo of your
institution to every presentation in the future), please read the
section on customization. 

In this guide you will find the descriptions of all ``public''
commands provided by the |beamer| class. In each such
description, the described command, environment, or option is printed 
in red. Text shown in green is optional and can be left out.

You will sometimes find one of the words \textsc{beamer},
\textsc{article}, or \textsc{lyx} in blue in some description of a
command or environment. The first indicates that the description
applies only to ``normal beamer operation in \LaTeX.'' The word
\textsc{article} describes some behaviour that is special for the
|article| mode. The word \textsc{lyx} describes behaviour that is
special when you use \LyX\ to prepare your presentation.  



\section{Installation}

\label{section-installation}

To use the beamer class, you just need to put the files of the
\beamer\ package in a directory that is read by \TeX. To uninstall the
class, simply remove these files once more. The same is true of the
\textsc{pgf} package, which you will also need.

Unfortunately, there are different ways of making \TeX\ ``aware'' of
the files in the \beamer\ package. Which way you should choose depends
on how permanently you intend to use the class.


\subsection{Installing Prebundled Packages Like Debian or Red Hat Packages}

Currently, I'm not producing prebundled packages of |beamer|. There
are some out-of-the-box Debian packages and Mik\TeX\ packages around,
but these are possibly outdated. Once |beamer| stabilizes, there
will hopefully also be easy-to-install packages. 



\subsection{Temporary Installation}

If you only wish to install the beamer class for a quick appraisal, do
the following: Obtain the latest source version (ending
|.tar.gz|) of the \beamer\ package from 
\href{http://sourceforge.net/projects/latex-beamer/}{|http://sourceforge.net/projects/latex-beamer/|}
(most likely, you have already done this). Next, you also need at
least version 0.60 of the \textsc{pgf} package, which can be found at 
the same place. Finally, you need at least version 1.06 of the
\textsc{xcolor} package, which can also be found at that place
(although the version on CTAN might be newer).

\lyxnote
For usage with \LyX,  version 1.3.3 of \LyX\ and higher are known to
work. II have not tried earlier versions; they might also work.
\smallskip

In all cases, the packages contain a bunch of files (for the \beamer\
class, |beamer.cls| is one of these files and happens to be the
most important one, for the \textsc{pgf} package |pgf.sty| is
the most important file). Place all files in three directories. For
example, |~/beamer/|, |~/pgf/|, and |~/xcolor/| would work fine for
me. Then setup the environment variable called |TEXINPUTS| to be the
following string (how exactly this is done depends on your operating
system and shell): 

\begin{verbatim}
.:~/beamer/base:~/beamer/art:~/beamer/themes:~/pgf:~/xcolor:
\end{verbatim}

Naturally, if the |TEXINPUTS| variable is already defined
differently, you should \emph{add} the five directories to the
list. Do not forget to place a colon at the end (corresponding to an
empty path), which will include all standard directories.



\subsection{Installation in a texmf Tree}

For a more permanent installation, you can place the files of the
\beamer\ package and of the \textsc{pgf} package (see the previous
subsection on how to obtain them) in an appropriate |texmf|
tree. 

When you ask \TeX\ to use a certain class or package, it usually looks
for the necessary files in so-called |texmf| trees. These trees
are simply huge directories that contain these files. By default,
\TeX\ looks for files in three different |texmf| trees:
\begin{itemize}
\item
  The root |texmf| tree, which is usually located at
  |/usr/share/texmf/|, |c:\texmf\|, or\\
  |c:\Program Files\TeXLive\texmf\|.
\item
  The local  |texmf| tree, which is usually located at
  |/usr/local/share/texmf/|, |c:\localtexmf\|, or\\
  |c:\Program Files\TeXLive\texmf-local\|.
\item
  Your personal  |texmf| tree, which is usually located in your home
  directory at |~/texmf/| or |~/Library/texmf/|.   
\end{itemize}

You should install the packages either in the local tree or in
your personal tree, depending on whether you have write access to the
local tree. Installation in the root tree can cause problems, since an
update of the whole \TeX\ installation will replace this whole tree.

Inside whatever |texmf| directory you have chosen, create
the sub-sub-sub-directories
\begin{itemize}
\item
  |texmf/tex/latex/beamer|,
\item
  |texmf/tex/latex/pgf|, and
\item
  |texmf/tex/latex/xcolor|
\end{itemize}
and place all files in these three directories.

Finally, you need to rebuild \TeX's filename database. This done by
running the command  |texhash| or |mktexlsr| (they are
the same). In Mik\TeX, there is a menu option to do this.

\lyxnote
For usage of the \beamer\ class with \LyX, you have to do all of the
above. Then you also have to make \LyX\ aware of the file
|beamer/lyx/layouts/beamer.layout|. To do so, link (or, not
so good in case of later updates, copy) this file to the directory
|.lyx/layouts/| in your home directory. Then use \LyX's Reconfigure
command to make \LyX\ aware of this file.

\vskip1em
For a more detailed explanation of the standard installation process
of packages, you might wish to consult
\href{http://www.ctan.org/installationadvice/}{|http://www.ctan.org/installationadvice/|}.
However, note that the \beamer\ package does not come with a
|.ins| file (simply skip that part).




\subsection{Updating the Installation}

To update your installation from a previous version, simply replace
everything in the directories like |texmf/tex/latex/beamer| with the
files of the new version. The easiest way to do this is to first
delete the old version and then proceed as described above. Sometimes,
there are changes in the syntax of certain command from version to
version. If things no longer work that used to work, you wish to have
a look at the release notes and at the change log.


\subsection{Testing the Installation}

To test your installation, copy the file |beamerexample1.tex|
from the examples subdirectory to some place where you usually
create presentations. Then run the command |pdflatex| several times on
the file and check whether the resulting |beamerexample1.pdf|
looks correct. If so, you are all set.

\lyxnote
To test the \LyX\ installation, try creating a new file from the
template |beamerpresentation.lyx|, which is located in the directory
|beamer/lyx/templates|.







\section{Workflow}

\label{section-workflow}

This section presents a possible workflow for creating a beamer
presentation and possibly a handout to go along with it. Technical
questions are addressed, like which programs to call with 
which parameters, and hints are given on how to create a
presentation. If you have already created numerous presentations, you
may wish to skip the first of the following steps 
and only have a look at how to convert the |.tex| file into a
|.pdf| or |.ps| file.


\subsection{Step Zero: Know the Time Constraints}

When you start to create a presentation, the very first thing you
should worry about is the amount of time you have for your
presentation. Depending on the occasion, this can
be anything between 2 minutes and two hours. A simple rule for the
number of frames is that you should have at most one frame per
minute.

In most situations, you will have less time for your presentation that
you would like. \emph{Do not try to squeeze more into a
  presentation than time allows for.} No matter how important some
detail seems to you, it is better to leave it out, but get the main
message across, than getting neither the main message nor the detail
across. 

In many situations, a quick appraisal of how much time you have will
show that you won't be able to mention certain details. Knowing this can
save you hours of work on preparing slides that you would have to remove
later anyway.




\subsection{Step One: Setup the Files}

\beamernote
It is advisable that you create a folder for each
presentation. Even though your presentation will usually reside in a
single file, \TeX\ produces so many extra files that things can easily
get very confusing otherwise. The folder's name should ideally start
with the date of your talk in ISO format (like 2003-12-25 for a
Christmas talk), followed by some reminder text of what the talk is
all about. Putting the date at the front in this format causes your
presentation folders to be listed nicely when you have several of them
residing in one directory. If you use an extra directory for each
presentation, you can call your main file |main.tex|. 

To create an initial |main.tex| file for your talk, copy an
existing file (like the file |beamerexample1.tex| that comes along
with the contribution) and delete everything that is not going to be
part of your talk. Adjust the |\author| and other fields as 
appropriate. 

If you wish your talk to reside in the same file as some different,
non-presentation article version of your text, it is advisable to
setup a more elaborate file scheme. See
Section~\ref{section-article-version-workflow} for details.

\lyxnote
You can either open a new file and then select |beamer| as the
document class or you say ``New from template'' and then use a
template from the directory |beamer/lyx/templates|.



\subsection{Step Two: Structure You Presentation}

With the time constraints in mind, make a mental inventory of the
things you can reasonably talk about within the time available. Then
categorize the inventory into sections and subsections. For very long
talks (like a 90 minute lecture), you might also divide your talk into
independent parts (like a ``review of the previous lecture part'' and
a ``main part''). Put |\section| and |\subsection| commands into
the (more or less empty) main file. Do not create any frames until you
have a first working version of a possible table of contents. Do not
feel afraid to change it later on as you work on the talk.

You should not use more than four sections and not less than two per
part. Even four sections are usually too much, unless they follow 
a very easy pattern. Five and more sections are simply too hard to
remember for the audience. After all, when you present the table of
contents, the audience will not yet really be able to grasp the
importance and relevance of the different sections and will most
likely have forgotten them by the time you reach them.

Ideally, a table of contents should be understandable by itself. In
particular, it should be comprehensible \emph{before} someone has
heard your talk. Keep section and subsection titles
self-explaining. Note that each part has its own table of contents. 

Both the sections and the subsections should follow a logical
pattern. Begin with an explanation of what your talk is all about. (Do
not assume that everyone knows this. The Ignorant Audience Law states:
The audience always knows less than you think it should know, even if
you take the Ignorant Audience Law into account.) Then explain what
you or someone else has found out concerning the subject
matter. Always conclude your talk with a summary that repeats the main
message of the talk in a short and simple way. People pay most
attention at the beginning and at the end of talks. The summary is
your ``second chance'' to get across a message.

You can also add an appendix part using the |\appendix| command. Put
everything into this part that you do not actually intend to talk
about, but that might come in handy when questions are asked.



\subsection{Step Three: Creating a PDF or PostScript File}

\beamernote
Once a first version of the structure is finished, you should create a
first PDF or PostScript file of your (still empty) talk. This file
will only contain the title page and the table of contents. The file
might  look like this:

\begin{verbatim}
\documentclass{beamer}
%% This is the file main.tex

\usepackage{beamerthemesplit}

\title{Example Presentation Created with the Beamer Package}
\author{Till Tantau}
\date{\today}

\begin{document}

\frame{\titlepage}

\section*{Outline}
\frame{\tableofcontents}

\section{Introduction}
\subsection{Overview of the Beamer Class}
\subsection{Overview of Similar Classes}

\section{Usage}
\subsection{...}
\subsection{...}

\section{Examples}
\subsection{...}
\subsection{...}

\end{document}
\end{verbatim}

\lyxnote
Use ``View'' to check whether the presentation compiles fine. Note
that you must put the table of contents inside a frame, but that the
title page is created automatically.


\subsubsection{Creating PDF}

\beamernote
To create a |PDF| version of this file, run the program
|pdflatex| on |main.tex| at least twice. Your need to run it twice, so
that \TeX\ can create the table of contents. (It may even be necessary
to run it more often since all sorts of auxiliary files are
created.) In the following example, the greater-than-sign is the prompt. 

\begin{verbatim}
> pdflatex main.tex
    ... lots of output ...
> pdflatex main.tex
    ... lots of output ...
\end{verbatim}

You can next use a program like the Acrobat Reader or |xpdf|
to view the resulting presentation.

\begin{verbatim}
> acroread main.pdf
\end{verbatim}

When printing a presentation using Acrobat, make sure that the option
``expand small pages to paper size'' in the printer dialog is
enabled. This is necessary, because slides are only 128mm times 96mm.

To put several slides onto one page (useful for the handout version)
or to enlarge the slides, you can use the program |pdfnup|. Also, many
commercial programs can perform this task. If you put several slides
on one page and if these slides normally have a white background, it
may be useful to write the following in your preamble:

\begin{verbatim}
\mode<handout>{\beamertemplatesolidbackgroundcolor{black!5}}
\end{verbatim}

This will cause the slides of the the handout version to have a very
light gray background. This makes it easy to discern the slides'
border if several slides are put on one page.

\lyxnote
Choose ``View pdf'' to view your presentation.


\subsubsection{Creating PostScript}
\label{section-postscript}

\beamernote
To create a PostScript version, you should first ascertain that the
\textsc{hyperref} package (which is automatically loaded by the
\beamer\ class) uses the option |dvips| or some compatible
option, see the documentation of the \textsc{hyperref} package for
details. Whether this is the case depends on the contents of your
local |hyperref.cfg| file. You can enforce the usage of this
option by passing |dvips| or a compatible option to the
\beamer\ class (write |\documentclass[dvips]{beamer}|), which
will pass this option on to the \textsc{hyperref} package.

You can then run |latex| twice, followed by |dvips|.

\begin{verbatim}
> latex main.tex
    ... lots of output ...
> latex main.tex
    ... lots of output ...
> dvips -P pdf main.dvi
\end{verbatim}

The option (|-P pdf|) tells |dvips| to use
Type~1 outline fonts instead of the usual Type~3 bitmap fonts. You may
wish to omit this option if there is a problem with it. 

If you wish each slide to completely fill a letter-sized page, use the
following commands instead:

\begin{verbatim}
> dvips -P pdf -tletter main.dvi -o main.temp.ps
> psnup -1 -W128mm -H96mm -pletter main.temp.ps main.ps
\end{verbatim}

For A4-sized paper, use:

\begin{verbatim}
> dvips -P pdf -ta4 main.dvi -o main.temp.ps
> psnup -1 -W128mm -H96mm -pa4 main.temp.ps main.ps
\end{verbatim}

In order to create a white margin around the whole page (which is sometimes
useful for printing), add the option |-m 1cm| to the options of
|psnup|. 

To put two or four slides on one page, use |-2|, respectively
|-4| instead of |-1| as the first parameter for
|psnup|. In this case, you may wish to add the option
|-b 1cm| to add a bit of space around the individual slides. The same
trick as for the \pdf-version can be used to make the borders of
slides more pronounced in the handout version.

You can convert a PostScript file to a pdf file using

\begin{verbatim}
> ps2pdf main.ps main.pdf
\end{verbatim}


\lyxnote
Use ``View Postscript'' to view the PostScript version.


\subsection{Step Four: Create Frames}

Once the table of contents looks satisfactory, start creating frames
for your presentation. In the following, some guidelines that I stick
to are given on what to put on slides and what not to put. You can
certainly ignore any of these guideline, but you should be aware of
it when you ignore a rule and you should be able to justify it to
yourself. 

\lyxnote
To create a frame, use the style ``BeginFrame''. The frame title
is given on the line of this style. The frame ends automatically with
the start of the next frame, with a section or subsection command, and
with an empyt line in the sylte ``EndFrame''. Note that the last frame
of your presentation must be ended using ``EndFrame'' and that the
last frame before the appendix must be ended this way.

\subsubsection{Guidelines on What to Put on a Frame}

\begin{itemize}
\item
  A frame with too little on it is better than a
  frame with too much on it.
\item
  Do not assume that everyone in the audience is an expert on the
  subject matter. (Remember the Ignorant Audience Law.) Even if the
  people listening to you should be experts, they may last have heard
  about things you consider obvious several years ago. You should
  always have the time for a quick reminder of what exactly a
  ``semantical complexity class'' or an ``$\omega$-complete partial
  ordering'' is.
\item
  Never put anything on a slide that you are not going to explain
  during the talk, not even to impress anyone with how
    complicated your subject matter really is. However, you may
  explain things that are not on a slide.
\item
  Keep it simple. Typically, your audience will see a slide for less
  than 50 seconds. They will not have the time to puzzle through long
  sentences or complicated formulas.
\end{itemize}


\subsubsection{Guidelines on Titles}

\begin{itemize}
\item
  Put a title on each frame. The title explains the contents of the
  frame to people who did not follow all details on the slide.
\item
  The title should really \emph{explain} things, not just give a
  cryptic summary that cannot be understood unless one has understood
  the whole slide. For example, a title like ``The Poset'' will have
  everyone puzzled what this slide might be about. Titles like
  ``Review of the Definition of Partially Ordered Sets (Posets)'' or
  ``A Partial Ordering on the Columns of the Genotype Matrix'' are
  \emph{much} more informative.
\item
  Ideally, titles on consecutive frames should ``tell a story'' all by
  themselves.
\item
  In English, you should \emph{either} \emph{always} capitalize all words in
  a frame title except for words like ``a'' or ``the'' (as in a
  title), \emph{or} you \emph{always} use the normal lowercase
  letters. Do \emph{not} mix this; stick to one rule. The same is true
  for block titles. For example, do not use titles like ``A short
  Review of Turing machines.'' Either use ``A Short Review of Turing
  Machines.'' or ``A short review of Turing machines.'' (Turing is
  still spelled with a capital letter since it is a name).
\item
  In English, the title of the whole document should be
  capitalized, regardless of whether you capitalize anything else.
\item
  In German and other languages that have lots of capitalized words,
  always use the correct upper-/lowercase letters. Never capitalize
  anything in addition to what is usually capitalized.
\end{itemize}



\subsubsection{Guidelines on the Body Text}

\begin{itemize}
\item
  \emph{Never} use a smaller font size to ``squeeze more on a frame.''
\item
  Prefer enumerations and itemize environments over plain text. Do not
  use long sentences.
\item
  Do not hyphenate words. If absolutely necessary, hyphenate words
  ``by hand,'' using the command~|\-|.
\item
  Break lines ``by hand'' using the command~|\\|. Do not rely on
  automatic line breaking. Break where there is a logical pause. For 
  example, good breaks in ``the tape alphabet is larger
  than the input alphabet'' are before ``is'' and before the second
  ``the.'' Bad breaks are before either ``alphabet'' and before
  ``larger.''
\item
  Text and numbers in figures should have the \emph{same} size as
  normal text. Illegible numbers on axes usually ruin a chart and its
  message. 
\end{itemize}


\subsubsection{Guidelines on Graphics}

\begin{itemize}
\item
  Put (at least) one graphic on each slide, whenever
  possible. Visualizations help an audience enormously.
\item
  Usually, place graphics to the left of the text. (Use the
  |columns| environment.) 
\item
  Graphics should have the same typographic parameters as the
  text: Use the same fonts (at the same size) in graphics as in the
  main text. A small dot in a graphic should have exactly the same 
  size as a small dot in a text. The line width should be the same as
  the stroke width used in creating the glyphs of the font. For
  example, an 11pt non-bold Computer Modern font has a stroke width of
  0.4pt.
\item
  While bitmap graphics, like photos, can be much more colorful than the
  rest of the text, vector graphics should follow the same ``color
  logic'' as the main text (like black~= normal lines, red~= hilighted
  parts, green~= examples, blue~= structure).
\item
  Like text, you should explain everything that is shown on a
  graphic. Unexplained details make the audience puzzle whether this
  was something important that they have missed. Be careful when
  importing graphics from a paper or some other source. They usually
  have much more detail than you will be able to explain.
\end{itemize}

For technical hints on how to create graphics, see
Section~\ref{section-graphics}.


\subsubsection{Guidelines on Colors}

\begin{itemize}
\item
  Use colors sparsely. The prepared themes are already quite
  colorful (blue~= structure, red~= alert, green~= example). If you
  add more colors, you should have a \emph{very} good reason.
\item
  Be careful when using bright colors on white background,
  \emph{especially} when using green. What looks good on your monitor
  may look bad during a presentation due to the different ways
  monitors, beamers, and printers reproduce colors. Add lots of black
  to pure colors when you use them on bright backgrounds.
\item
  Maximize contrast. Normal text should be black on white or at least
  something very dark on something very bright. \emph{Never} do things
  like ``light green text on not-so-light green background.''
\item
  Background shadings decrease the legibility without increasing the
  information content. Do not add a background shading just because it
  ``somehow looks nicer.'' In the examples that come along with the
  \beamer\ class, the backgrounds are intended as demonstrations, not
  as recommendations.
\item
  Inverse video (bright text on dark background) can be a problem
  during presentations in bright environments since only a small
  percentage of the presentation area is light up by the
  beamer. Inverse video is harder to reproduce on printouts and on
  transparencies. 
\end{itemize}


\subsubsection{Guidelines on Animations and Special Effects}

\begin{itemize}
\item
  Use animations to explain the dynamics of systems, algorithms, etc.
\item
  Do \emph{not} use animations just to attract the attention of your
  audience. This often distracts attention away from the main topic of the
  slide.
\item
  Do \emph{not} use distracting special effects like ``dissolving''
  slides unless you have a very good reason for using them. If you use
  them, use them sparsely. 
\end{itemize}


\subsubsection{Ways of Improving Compilation Speed}
\label{section-speedup}

While working on your presentation, it may sometimes be useful to
\TeX\ your |.tex| file quickly and have the presentation contain only
the most important information. This is especially true if you have a
slow machine. In this case, you can do several things to speedup the
compilation. First, you can use the |draft| class option.

\begin{classoption}{{draft}}
  Causes the head lines, foot lines, and sidebars to be replaced by
  gray rectangles (their sizes are still computed, though). Many
  other packages, including |pgf| and |hyperref|, also ``speedup''
  when this option is given.
\end{classoption}

Second, you can use the following command:

\begin{command}{{\includeonlyframes}\marg{frame label list}}
  This command behaves a little bit like the |\includeonly| command:
  Only the frames mentioned in the list are included. All other frames
  are suppressed. Nevertheless, the section and subsection commands
  are still executed, so that you still have the correct navigation
  bars. By labeling the current frame as, say, |current| and then
  saying |\includeonlyframes{current}|, you can work on a single frame
  quickly.  

  The \meta{frame label list} is a comma-separated list (without
  spaces) of the names of frames that have been labeled. To label a
  frame, you must pass the option |label=|\meta{name} to the |\frame|
  command.

  \example
\begin{verbatim}
\includeonlyframes{example1,example3}

\frame[label=example1]
{This frame will be included. }

\frame[label=example2]
{This frame will not be included. }

\frame{This frame will not be included.}

\againframe{example1} % Will be included
\end{verbatim}
\end{command}



\subsection{Step Five: Test Your Presentation}

\emph{Always} test your presentation. For this, you should
vocalize or subvocalize your talk in a quiet environment. Typically,
this will show that your talk is too long. You should then remove
parts of the presentation, such that it fits into the allotted time
slot. Do \emph{not} attempt to talk faster in order to squeeze the
talk into the given amount of time. You are almost sure to loose your
audience this way.

Do not try to create the ``perfect'' presentation immediately. Rather,
test and retest the talk and modify it as needed. 




\subsection{Step Six: Optionally Create a Handout or an Article Version}

Once your talk is fixed, you can create a handout, if this seems
appropriate. For this, use the class option |handout| as
explained in Section~\ref{handout}. Typically, you might wish
to put several handout slides on one page. See
Section~\ref{section-postscript} on how to do this.

You may also wish to create an article version of your talk. An
``article version'' of your presentation is a normal \TeX\ text
typeset using, for example, the document class |article| or perhaps
|llncs| or a similar document class. The \beamer\ class offers
facilities to have this version coexist with your presentation version
in one file and to share code. Also, you can include slides of your
presentation as figures in your article version. Details on how to
setup the article version can be found in
Section~\ref{section-article}.  

\lyxnote
Creating an article version is not really possible in \LyX. You
can \emph{try}, but I would not advise it.





\section{Frames and Overlays}

\label{section-overlay}

This section explains how you can create frames and overlays. It
starts with a description of a general concept, calls \emph{overlay
specifications}. Nearly all of \beamer's commands for creating frames
and overlays are based on this concept, except for the simple |\pause|
command (though it is internally also mapped to this concept).


\subsection{The Concept of Overlay Specifications}

\label{section-concept-overlays}

\subsubsection{The General Concept}

When creating overlays, how do you specify on which slides of a
series of slides a certain text should be shown? (Such a series is
called a \emph{frame} in \beamer.) The approach taken by most
presentation classes is to introduce new commands, which get a certain
slide number as input and which affect the text on the slide following
this command in a certain way. For example, \textsc{prosper}'s
|\FromSlide{2}| command causes all text following this command to be
shown only from the second slide on.

The \beamer\ class uses a different approach (though the
abovementioned command is also available: |\onslide<2->| will have the
same effect as |\FromSlide{2}|, expect that |\onslide| trancedes
environments). The idea is to add \emph{overlay specifications} to
certain commands. These specifications are always given in pointed
brackets and follow the command ``as soon as possible,'' though in
certain cases \beamer\ also allows overlay specification to be
given a little later. In the simplest case, the specification contains
just a number. A command with an overlay
specification following it will only have ``effect'' on the slide(s)
mentioned in the specification. What exactly ``having an effect''
means, depends on the command. Consider the following example.

\begin{verbatim}
\frame
{
  \textbf{This line is bold on all three slides.}
  \textbf<2>{This line is bold only on the second slide.}
  \textbf<3>{This line is bold only on the third slide.}
}
\end{verbatim}

For the command |\textbf|, the overlay specification causes the
text to be set in boldface only on the specified slides. On all other
slides, the text is set in a normal font.

For a second example, consider the following frame:
\begin{verbatim}
\frame
{
  \only<1>{This line is inserted only on slide 1.}
  \only<2>{This line is inserted only on slide 2.}
}
\end{verbatim}

The command |\only|, which is introduced by \beamer, normally simply
inserts its parameter into the current frame. However, if an
overlay-specification is present, it ``throws away'' its parameter on
slides that are not mentioned. 

Overlay specifications can only be written behind certain commands,
not every command. Which commands you can use and which effects this
will have is explained in Section~\ref{section-overlay-commands}. However, it
is quite easy to redefine an existing command such that it becomes
``overlay specification aware,'' see also Section~\ref{section-overlay-commands}.

The syntax of (basic) overlay specifications is the following: They
are comma-separated lists of slides and ranges. Ranges are specified
like this: |2-5|, which means slide two through to five. The start or
the end of a range can be omitted. For example, |3-| means
``slides three, four, five, and so on'' and |-5| means the same as
|1-5|. A complicated example is |-3,6-8,10,12-15|, which selects the 
slides 1, 2, 3, 6, 7, 8, 10, 12, 13, 14, and 15.


\lyxnote
Overlay specifications can also be given in \LyX. You must give them
in \TeX-mode (otherwise the pointed brackets may be ``escaped'' by
\LyX, though this will not happen in all versions). For example, to
add an overlay specification to an item, simply insert a \TeX-mode
text like |<3>| as the first thing in that item. Likewise, you can add
an overlay specification to environments like |theorem| by giving
them in \TeX-mode right at the start of the environment. 



\subsubsection{Mode Specifications}

This subsection is only important if you use \beamer's mode mechanism
to create different versions of your presentation. If you are not
familiar with \beamer's modes, please skip this section or read
Section~\ref{section-modes} first.

In certain cases you may wish to have different overlay specifications
to apply to a command in different modes.
For example, you might wish a certain text to appear only from the
third slide on during your presentation, but in a handout for the
audience there should be no second slide and the text should appear
already on the second slide. In this case you could write
\begin{verbatim}
\only<3| handout:2>{Some text}
\end{verbatim}
                                %\begin{verbatim}

The vertical bar, which must be followed by a (white) space, separates
the two different specifications |3| and |handout:2|. By writing a
mode name before a colon, you specify that the following specification
only applies to that mode. If no mode is given, as in |3|, the mode
|beamer| is automatically added. For this reason, if you write
|\only<3>{Text}| and you are in |handout| mode, the text will be shown
on all slides since there is no restriction specified for handouts and
since the |3| is the same as |beamer:3|.

It is also possible to give an overlay specification that contains
only a mode name (or several, separated by vertical bars):
\begin{verbatim}
\only<article>{This text is shown only in article mode.}
\end{verbatim}
An overlay specification that does not contain any slide numbers is
called a (pure) \emph{mode specification}. If a mode specification is
given, all modes that are not mentioned are automatically
suppressed. Thus |<beamer:1->| means ``on all slides in |beamer| mode
and also on all slides in all other modes, since nothing special is
specified for them,'' whereas |<beamer>| means ``on all slides in
|beamer| mode and not on any other slide.''

Mode specifications can also be used outside frames as in the following
examples:
\begin{verbatim}
\section<presentation>{This section exists only in the presentation modes}
\section<article>{This section exists only in the article mode}
\end{verbatim}

You can also mix pure mode specifications and overlay specifications,
although this can get confusing: 
\begin{verbatim}
\only<article| beamer:1>{Riddle}
\end{verbatim}

This will cause the text |Riddle| to be inserted in |article| mode and
on the first slide of a frame in |beamer| mode, but not at all in
|handout| or |trans| mode. (Try to find out how
\verb/<beamer| beamer:1>/ differs from |<beamer>| and from
|<beamer:1>|.)


\subsubsection{Action Specifications}
\label{section-action-specifications}


This subsection also introduces a rather advanced concept. You may
also wish to skip it on first reading.

Some overlay-specification-aware commands cannot only handle normal
overlay specifications, but also so called \emph{action
specifications}. In an action specification, the list of slide numbers
and ranges is prefixed by \meta{action}|@|, where \meta{action} is the
name of a certain action to be taken on the specified slides:
\begin{verbatim}
\item<3-| alert@3> Shown from slide 3 on, alerted on slide 3. 
\end{verbatim}
In the above example, the |\item| command, which allows actions to be
specified, will uncover the item text from slide three on and it will,
additionally, alert this item exactly on slide 3.

Not all commands can take an action specification. Currently, only
|\item| (though not in |article| mode currently), |\action|, the
environment |actionenv|, and the block environments (like |block| or
|theorem|) handle them. 

By default, the following actions are available:
\begin{itemize}
\item \declare{|alert|} alters the item or block.
\item \declare{|uncover|} uncovers the item or block (this is
  the default, if no action is specified).
\item \declare{|only|} causes the whole item or block
  to be inserted only on the specified slides.
\item \declare{|visible|} causes the text to become visible only on
  the specified slides (the difference between |uncover| and
  |visible| is the same as between |\uncover| and |\visible|).
\item \declare{|invisible|} causes the text to become invisble on the
  specified slides.
\end{itemize}

The rest of this section explains how you can add your own actions and
make commands action-specification-aware. You may wish to skip it upon
first reading.

You can easily add your own actions: An action specification like
\meta{action}|@|\meta{slide numbers} simply inserts an environment
called \meta{action}|env| around the |\item| or parameter of
|\action| with |<|\meta{slide numbers}|>| as overlay
specification. Thus, by defining a new overlay-specification-aware
environment named \meta{my action name}|env|, you can add your own
action:
\begin{verbatim}
\newenvironment{checkenv}{\only{\useitemizeitemtemplate{X}}}{}
\end{verbatim}
You can then  write
\begin{verbatim}
\item<beamer:check@2> Text.
\end{verbatim}
This will change the itemization symbol before |Text.| to |X| on
slide~2 in |beamer| mode. The definition of |checkenv| used the fact
that |\only| also accepts an overlay-specification given after its
argument. 

The whole action mechanism is base on the following environment:

\begin{environment}{{actionenv}\sarg{action specification}}
  This environment extracts all actions from the \meta{action
    specification} for the current mode. For each action of the form
  \meta{action}|@|\meta{slide numbers}, it inserts the following text:
  |\begin{|\meta{action}|env}<|\meta{slide number}|>| and the
  beginning of the environment and the text |\end{|\meta{action}|env}|
  at the end. If there several action specifications, several
  environments are opened (and closed in the appropriate order). An
  \meta{overlay specification} without an action is promoted to
  |uncover@|\meta{overlay specification}.

  If the so called \emph{default overlay specification} is not empty,
  it will be used in case no \meta{action specification} is given. The
  default overlay specification is usually just empty, but it may be
  set either by providing an additional optional argument to the
  command |\frame| or to the environments |itemize|, |enumerate|, or
  |description| (see these for details). Also, the default action
  specification can be set using the command
  |\beamerdefaultoverlayspecification|, see below.
 
  \example 

\begin{verbatim} 
\frame 
{
  \begin{actionenv}<2-| alert@3-4,6>
    This text is shown the same way as the text below.
  \end{actionenv}

  \begin{uncoverenv}<2->
    \begin{alertenv}<3-4,6>
      This text is shown the same way as the text above.
    \end{alertenv}
  \end{uncoverenv}
}
\end{verbatim}
                                %\begin{verbatim}
\end{environment} 
 
\begin{command}{\action\sarg{action specification}\marg{text}}
  This has the same effect as putting \meta{text} in an |actionenv|.

  \example |\action<alert@2>{Could also have used \alert<2>{}.}|
\end{command}

\begin{command}{\beamerdefaultoverlayspecification\marg{default
      overlay specification}}
  Locally sets the default overlay specification to the given
  value. This overlay specification will be used in every |actionenv|
  environment and every |\item| that does not have its own overlay
  specification. The main use of this command is to install an
  incremental overlay specification like |<+->| or
  \verb/<+-| alert@+>/, see Section~\ref{section-incremental}.

  Usually, the default overlay specification is installed
  automatically by the optional arguments to |\frame|, |itemize|,
  |enumerate|, and |description|. You will only have to use this
  command if you wish to do funny things.

  \example |\beamerdefaultoverlayspecification{<+->}|

  \example |\beamerdefaultoverlayspecification{}| clears the default
  overlay specification. (Actually, it installs the default overlay
  specification |<*>|, which just means ``always,'' but the
  ``portable'' way of clearing the default overlay specification is
  this call.)
\end{command}



\subsubsection{Incremental Specifications}
\label{section-incremental}

This subsection is mostly important for people who have already used
overlay specifications a lot and have grown tired of writing things
like |<1->|, |<2->|, |<3->|, and so on again and again. You should
skip this section on first reading.

Often you want to have overlay specifications that follow a pattern
similar to the following:
\begin{verbatim}
\begin{itemize}
\item<1-> Apple
\item<2-> Peach
\item<3-> Plum
\item<4-> Orange
\end{itemize}
\end{verbatim}
The problem starts if you decide to insert a new fruit, say, at the
beginning. In this case, you would have to adjust all of the overlay
specifications. Also, if you add a |\pause| command before the
|itemize|, you would also have to update the overlay specifications.

\beamer\ offers a special syntax to make creating lists as the one
above more ``robust.'' You can replace it by the following list of
\emph{incremental overlay specifications}:
\begin{verbatim}
\begin{itemize}
\item<+-> Apple
\item<+-> Peach
\item<+-> Plum
\item<+-> Orange
\end{itemize}
\end{verbatim}
The effect of the |+|-sign is the following: You can use it in any
overlay specification at any point where you would usually use a
number. If a |+|-sign is encountered, it is replaced by the current
value of the \LaTeX\ counter |beamerpauses|, which is 1 at the
beginning of the frame. Then the counter is increased by 1, though it
is only increased once for every overlay specification, even if the
specification contains multiple |+|-signs (they are replaced by the
same number).

In the above example, the first specification is replaced by
|<1->|. Then the second is replaced by |<2->| and so forth. We can now
easily insert new entries, without having to change anything. We might
also write the following:
\begin{verbatim}
\begin{itemize}
\item<+-| alert@+> Apple
\item<+-| alert@+> Peach
\item<+-| alert@+> Plum
\item<+-| alert@+> Orange
\end{itemize}
\end{verbatim}
This will alert the current item when it is uncovered. For example,
the first specification \verb/<+-| alert@+>/ is replaced by
\verb/<1-| alert@1>/, the second is replaced by \verb/<2-| alert@2>/, and so on.
Since the |itemize| environment also allows you to specify a default
overlay specification, see the documentation of that environment, the
above example can be written even more economically as follows:
\begin{verbatim}
\begin{itemize}[<+-| alert@+>]
\item Apple
\item Peach
\item Plum
\item Orange
\end{itemize}
\end{verbatim}

The |\pause| command also updates the counter |beamerpauses|. You can
change this counter yourself using the normal \LaTeX\ commands
|\setcounter| or |\addtocounter|.

Any occurence of a |+|-sign may be followed by an \emph{offset} in
round brackets. This offset will be added to the value of
|beamerpauses|. Thus, if |beamerpauses| is 2, then |<+(1)->| expands to
|<3->| and |<+(-1)-+>| expands to |<1-2>|. 



\subsection{Frames}

\subsubsection{Frame Creation}

A presentation consists of a series of frames. Each frame consists of
a series of slides. You create a frame using the command
|\frame|. This command takes one parameter, namely the
contents of the frame. All of this text that is not tagged by overlay
specifications is shown on all slides of the frame. If a frame
contains commands that have an overlay specification, the frame will
contain multiple slides; otherwise it contains only one slide.

\begin{command}{\frame\sarg{overlay specification}%
    \opt{|[<|\meta{default overlay specification}|>]|}\oarg{options}\marg{frame text}}
  The \meta{overlay specification} dictates which slides of a frame are
  to be shown. If left out, the number is calculated automatically.
  The \meta{frame text} can be normal \LaTeX\ text, but may not contain
  |\verb| commands or |verbatim| environments, unless the
  |containsverbatim| options is given, see also
  Section~\ref{section-verbatim}. 
  
  \example
\begin{verbatim}
\frame
{
  \frametitle{A title}
  Some content.
}
\end{verbatim}
  
  \example
\begin{verbatim}
\frame<beamer>  % frame is only shown in beamer mode
{
  \frametitle{Outline}
  \tabelofcontent[current]
}
\end{verbatim}

  The \meta{default overlay specification} is an optional argument
  that is ``detected'' according to the following rule: If the first
  optional argument in square brackets starts with a |<|, then this
  argument is a \meta{default overlay specification}, otherwise it is
  a normal \meta{options} argument. Thus |\frame[<+->][plain]| would
  be legal, but also |\frame[plain]|.

  The effect of the \meta{default overlay specification} is the
  following: Every command or environment \emph{inside the frame} that
  accepts an action specification, see
  Section~\ref{section-action-specifications}, (this includes the
  |\item| command, the |actionenv| environment, |\action|, and all
  block environments) and that is not followed by 
  an overlay specification gets the \meta{default overlay
    specification} as its specification. By providing an incremental
  specification like |<+->|, see Section~\ref{section-incremental},
  this will essentially cause all blocks and all enumerations to be
  uncovered piece-wise (blocks internally employ action
  specifications). 
  
  \example In this frame, the theorem is shown from the first slide
  on, the proof from the second slide on, with the first two itemize
  points shown one after the other; the last itemize point is shown
  together with the first one. In total, this frame will contain four
  slides.
\begin{verbatim}
\frame[<+->]
{
  \begin{theorem}
    $A = B$.
  \end{theorem}
  \begin{proof}
    \begin{itemize}
    \item Clearly, $A = C$.
    \item As shown earlier,  $C = B$.
    \item<3-> Thus $A = B$.
    \end{itemize}
  \end{proof}
}
\end{verbatim}
 
  The following \meta{options} may be given:
  \begin{itemize}
  \item
    \declare{|b|}, \declare{|c|}, \declare{|t|} will cause the frame
    to be vertically aligned at the bottom/center/top. This overrides
    the global placement policy, which is governed by the class
    options |slidestop| and |slidescentered|.
    
  \item
    \declare{|containsverbatim|} tells \beamer\ that the frame
    contains verbatim commands. In this case, only one slide of
    the frame is typeset (unless all slides are suppressed by the
    \meta{overlay specification}). If you wish to use verbatim text in
    a frame with several slides, a more roundabout approach is
    necessary, see Section~\ref{section-verbatim}. This option cannot
    be used together with the |label| option.
    
  \item
    \declare{|label=|\meta{name}} causes the frame's contents to
    be stored under the name \meta{name} for later resumption using
    the command |\againframe|. If this option is given, you cannot
    include verbatim text in the frame, even if you specify an overlay
    specification like |<1>|. The frame is still rendered
    normally. See also |\againframe|.

    Furthermore, on each slide of the frame a label with the name
    \meta{name}|<|\meta{slide number}|>| is created. On the
    \emph{first} slide, furthermore, a label with the name \meta{name}
    is created (so the labels \meta{name} and \meta{name}|<1>| point
    to the same slide). Note that labels in general, and these labels
    in particular, can be used as targets for hyperlinks.
  \item
    \declare{|plain|} causes  the head lines, foot lines,
  and side bars to be suppressed. This is useful for creating single
  frames with different head and foot lines or for creating frames
  showing big pictures that completely fill the frame.

  \example A frame with a picture completely filling the frame:  
\begin{verbatim}
\frame[plain]{\hfill\pgfimage[height=9.6cm]{bigimagefilename}\hfill}
\end{verbatim}
  
  \example A title page, in which the head and foot lines are replaced
  by two graphics.
\begin{verbatim}
\usetitlepagetemplate{
  \beamerline{\pgfuseimage{toptitle}}
  \vskip0pt plus 1filll

  \begin{centering}
    \Large{\textbf{\inserttitle}}
    
    \insertdate
  \end{centering}

  \vskip0pt plus 1filll
  \beamerline{\pgfuseimage{bottomtitle}}
}
\frame[plain]{\titlepage}
\end{verbatim}
  \end{itemize}

  \lyxnote
  Use the style ``BeginFrame'' to start a frame and the style
  ``EndFrame'' to end it. A frame is automatically ended by the start
  of a new frame and by the start of a new section or subsection (but
  not by the end of the document!).

  \lyxnote
  You can pass options and an overlay specification to a frame by
  giving these in \TeX-mode as the first thing in the frame
  title. (Some magic is performed to extract them in \LyX\ mode from
  there.)

  \lyxnote
  The style ``BeginPlainFrame'' is included as a convenience. It
  passes the |plain| option to the frame. To pass further options to a
  plain frame, you should use the normal ``BeginFrame'' style and
  specify all options (include |plain|).

  \lyxnote
  In \LyX, you can insert verbatim text directly even in overlayed
  frames. The reason is that \LyX\ uses a different internal mechanism
  for typesetting verbatim text, that is easier to handle for \beamer.

  \articlenote
  In |article| mode, the |\frame| command does not create any visual
  reference to the original frame (no frame is drawn). Rather, the
  frame text is inserted into the normal text. To change this, you can
  modify the frame template, see
  Section~\ref{section-frame-template}. To suppress a frame in
  |article| mode, you can, for example, specify |<presentation>| as
  overlay specification. 
\end{command}


For compatibility with earlier versions, you can also give an overlay
specification in square brackets. If the sole argument to the |\frame|
command is an argument in square brackets, the \beamer\ class will try
to check whether this argument ``looks like'' an overlay
specification. If so, it is assumed to be an overlay specification.

Note that there is \emph{no} environment for creating frames. The
reason is that I simply have not been able to come up with an idea of
how to implement it in the presence of multiple overlays.


\begin{command}{\againframe\sarg{overlay
      specification}\opt{|[<|\meta{default overlay specification}|>]|}\oarg{options}\marg{name}}
  \beamernote
  Resumes a frame that was previously created using |\frame|
  with the option |label=|\meta{name}. You must have used this option,
  just placing a label inside a frame ``by hand'' is not enough. You
  can use this command to ``continue'' a frame that has been
  interrupted by another frame. The effect of this command is to call
  the |\frame| command with the given \meta{overlay specification},
  \meta{default overlay specification} (if present), and
  \meta{options} (if present) and with the original frame's contents. 

  \example
\begin{verbatim}
\frame<1-2>[label=myframe]
{
  \begin{itemize}
  \item<alert@1> First subject.
  \item<alert@2> Second subject.
  \item<alert@3> Third subject.
  \end{itemize}
}

\frame
{
  Some stuff explaining more on the second matter.
}

\againframe<3>{myframe}
\end{verbatim}
  The effect of the above code is to create four slides. In the first
  two, the items 1 and~2 are hilighted. The third slide contains the
  text ``Some stuff explaining more on the second matter.'' The fourth
  slide is identical to the first two slides, except that the third
  point is now hilighted.

  \example
\begin{verbatim}
\frame<1>[label=Cantor]
{
  \frametitle{Main Theorem}

  \begin{Theorem}
    $\alpha < 2^\alpha$ for all ordinals~$\alpha$.
  \end{Theorem}

  \begin{overprint}
  \onslide<1>
    \hyperlink{Cantor<2>}{\beamergotobutton{Proof details}}

  \onslide<2->
    % this is only shown in the appendix, where this frame is resumed.
    \begin{proof}
      As shown by Cantor, ...
    \end{proof}

    \hfill\hyperlink{Cantor<1>}{\beamerreturnbutton{Return}}
  \end{overprint}
}

...
\appendix

\againframe<2>{Cantor}
\end{verbatim}
  In this example, the proof details are deferred to a slide in the
  appendix. Hyperlinks are setup, so that one can jump to the proof
  and go back.

  \articlenote
  This command is ignored.

  \lyxnote
  Use the style ``AgainFrame'' to insert an |\againframe| command. The
  \meta{label name} is the text on following the style name
  and is \emph{not} put in \TeX-mode. However, an overlay specification
  must be given in \TeX-mode and it must preceed the label name.
\end{command}





\subsubsection{Components of a Frame}

Each frame consists of several components:
\begin{enumerate}\itemsep=0pt\parskip=0pt
\item a head line,
\item a foot line,
\item a left side bar,
\item a right side bar,
\item navigation symbols,
\item a logo,
\item a frame title, and
\item some frame contents.
\end{enumerate}

A frame need not have all of these components. Usually, the first six
components are automatically setup by the theme you are
using. To change them, you must install an appropriate template, see
Section~\ref{section-head-templates} for the head and foot lines and
Section~\ref{section-sidebar-templates} for the side bars. To install
a logo, invoke the following command in the preamble, \emph{after}
having loaded the theme:

\begin{command}{\logo\marg{logo text}}
  The \meta{logo text} is usually a command for including a
  graphic.
  \example
\begin{verbatim}
\pgfdeclareimage[height=0.5cm]{logo}{tu-logo}
\logo{\pgfuseimage{logo}}
\end{verbatim}

  \articlenote This command has no effect.
\end{command}

The frame title is shown prominently at the top of the frame and can
be specified with the following command:

\begin{command}{\frametitle\sarg{overlay specification}\oarg{short
  frame title}\marg{frame title text}} 
  You should end the \meta{frame title text} with a period, if the title is a
  proper sentence. Otherwise, there should not be a period. The
  \meta{short frame title} is normally not shown, but its available
  via the |\insertshortframetitle| command. The \meta{overlay
  specification} is mostly useful for suppressing the frame title in
  |article| mode.
\example
\begin{verbatim}
\frame{
  \frametitle{A Frame Title is Important.}

  Frame contents.
}
\end{verbatim}

  \articlenote
  By default, this command creates a new paragraph in |article| mode,
  entitled \meta{frame title text}. Using the \meta{overlay
    specification} makes it easy to suppress the a frame title once in
  a while. If you generally wish to suppress \emph{all} frame
  titles in |article| mode, say |\useframetitletemplate{}|.

  \lyxnote
  The frame title is the text that follows on the line of the
  ``BeginFrame'' style.
\end{command}

Be default, all material for a slide is vertically centered. You can
change this using the following class options:

\begin{classoption}{slidestop}
  Place text of slides at the (vertical) top of the slides. This
  corresponds to a vertical ``flush.'' You can override this for
  individual frames using the |c| or |b| option.
\end{classoption}

\begin{classoption}{slidescentered}
  Place text of slides at the (vertical) center of the slides. This is
  the default. You can override this for
  individual frames using the |t| or |b| option.
\end{classoption}



\subsubsection{Restricting the Slides of a Frame}
\label{subsection-restriction}

The number of slides in a frame is automatically
calculated. If the largest number mentioned in any
overlay specification inside the frame is 4, four slides are
introduced (despite the fact that a specification like |<4->|
might suggest that more than four slides would be possible).

You can also specify the number of slides in the frame ``by hand.'' To
do so, you pass an overlay specification the |\frame| command. The
frame will contain only the slides specified in this
argument. Consider the following example. 

\begin{verbatim}
\frame<1-2,4->
{
  This is slide number \only<1>{1}\only<2>{2}\only<3>{3}%
  \only<4>{4}\only<5>{5}.
}
\end{verbatim}
This command will create a frame containing four slides. The first
will contain the text ``This is slide number~1,'' the second ``This is
slide number~2,'' the third ``This is slide number~4,'' and the fourth
``This is slide number~5.''

A useful specification is just |<0>|, which causes the frame to
have to no slides at all. For example, |\frame<handout:0>| causes
the frame to be suppressed in the handout version, but to be shown
normally in all other versions. Another useful specification is
|<beamer>|, which causes the frame to be shown normally in |beamer|
mode, but to be suppressed in all other versions.


\subsubsection{Verbatim Commands and Listings inside Frames}
\label{section-verbatim}

The |\verb| command, the |verbatim| environment, the
|lstlisting| environment, and related environments that allow
you to typeset arbitrary text work only in
frames that contain a single slide or that are suppressed
altogether. Furthermore, you must explicitly specify that the frame
contains verbatim text using the |containsverbatim| commans:
\begin{verbatim}
\frame[containsverbatim]
{
  \frametitle{Our Search Procedure}

\begin{verbatim}
  int find(int* a, int n, int x)
  {
    for (int i = 0; i<n; i++)
      if (a[i] == x)
        return i;
  }
\end{verbatim}
\unskip{\MacroFont|\end{verbatim}|}
                                %\begin{verbatim}
\begin{verbatim}
}
\end{verbatim}

You may \emph{not} use the |label=|\meta{label name} option if you
have a verbatim text on a slide.

If you need to use verbatim commands in frames that contain several
slides or on a frame that uses the |label| option, you must
\emph{declare} your verbatim texts before the frame starts. This is
done using two special commands:


\begin{command}{\defverb\marg{command name}\opt{|*|}%
    \meta{delimiter symbol}\meta{verbatim text}\meta{delimiter symbol}}
  Declares a verbatim text for later use. The declaration should be
  done outside the frame. Once declared, the text can be used
  in overlays like normal text. The one-line \meta{verbatim text} must
  be delimited by a special \meta{delimiter symbol} (works like the
  |\verb| command). Adding a star makes spaces visible.

\example
\begin{verbatim}
\defverb\mytext!int main (void) { ...!
\defverb\mytextspaces*!int  main  (void ){  ...!

\frame
{
  \begin{itemize}
  \item<1-> In C you need a main function.
  \item<2-> It is declare like this: \mytext
  \item<3-> Spaces are not important: \mytextspaces
  \end{itemize}
}
\end{verbatim}
\end{command}


\begin{command}{\defverbatim\oarg{options}\marg{command name}\marg{text}}
  The \meta{text} may contain a |verbatim|,  |verbatim*|,
  |lstlisting|, or a related environment. The command \marg{command
    name} can be used later inside frames. The declaration
  should be done outside the frame. Once declared, the text can be
  used in overlays like normal text.

  The following \meta{options} may be given:
  \begin{itemize}
  \item
    \declare{|colored|} declares that the verbatim text will have
    its ``own'' colors. Normally, the verbatim text is typeset using
    the current color, which allows you to use commands like |\alert|
    to make verbatim text red on certain slides. However, if the
    verbatim text has, say, a special background color or different
    parts of it a colored differently (the |lstlisting| environment
    does this), then you do \emph{not} want the verbatim text to
    inherit its color from the ``outside.'' In this case, you should
    give the |colored| option.
  \end{itemize}  

  \example
\begin{verbatim}
\defverbatim\algorithmA{
\begin{verbatim}
int main (void)
{
  cout << "Hello world." << endl;
  return 0;
}
\end{verbatim}
\unskip{\MacroFont|\end{verbatim}|}
\begin{verbatim}
}

\defverbatim[colored]\algorithmB{
\begin{lstlisting}[language={C++},backgroundcolor=\color{yellow}]
int main (void)
{
  cout << "Hello world." << endl;
  return 0;
}
\end{lstlisting}
}

\frame
{
  Our algorithm:
  \alert<1>{\algorithmA}
  \uncover<2>{Note the return value.}
}

\frame
{
  Same algorithm typeset using the lstlisting environment:
  \algorithmB
}
\end{verbatim}
%\begin{verbatim}
\end{command}



\subsection{Creating Overlays}

\subsubsection{The Pause Commands}

The |pause| command offers an easy, but not very flexible
way of creating frames that are uncovered piecewise. If you say
|\pause| somewhere in a frame, only the text on the frame up to the
|\pause| command is shown on the first slide. On the 
second slide, everything is shown up to the second |\pause|, and
so forth. You can also use |\pause| inside environments; its effect
will last after the environment. However, taking this to
extremes and use |\pause| deeply within a nested environment may not
have the desired result.

A much more fine-grained control over what is shown on each slide can
be attained using overlay specifications, see the next
subsections. However, for many simple cases the |\pause|
command is sufficient.
 
The effect of |\pause| lasts till the next |\pause|, |\onslide|, or
the end of the frame.   

\begin{verbatim}
\frame{
  \begin{itemize}
  \item
    Shown from first slide on.
  \pause
  \item
    Shown from second slide on.
    \begin{itemize}
    \item
      Shown from second slide on.
    \pause
    \item
      Shown from third slide on.
    \end{itemize}
  \item
    Shown from third slide on.
  \pause
  \item
    Shown from fourth slide on.
  \end{itemize}

  Shown from fourth slide on.

  \begin{itemize}
  \unpause
  \item
    Shown from first slide on.
  \pause
  \item
    Shown from fifth slide on.
  \end{itemize}
}
\end{verbatim}

\begin{command}{\pause\oarg{number}}
  This command causes the text following it to be shown only from the
  next slide on, or, if the optional \meta{number} is given,
  from the slide with the number \meta{number}. If the optional
  \meta{number} is given, the counter |beamerpauses| is set to this
  number. This command uses the |\onslide| command, internally.
  This command does \emph{not} work inside |amsmath| environments like
  |align|, since these do really wicked things.

  \example
\begin{verbatim}
\frame
{
  \begin{itemize}
  \item
    A    
  \pause
  \item
    B
  \pause
  \item
    C
  \end{itemize}
}
\end{verbatim}

  \articlenote
  This command is ignored.

  \lyxnote
  Use the ``Pause'' style with an empty line to insert a pause.
\end{command}

To ``unpause'' some text, that is, to temporarily suspend pausing, use
the command |\onslide|, see below.


\subsubsection{Commands with Overlay Specifications}
\label{section-overlay-commands}
\label{subsection-overlay}

A much more powerful and flexible way of specifying overlays uses
overlay specifications, see Section~\ref{section-concept-overlays} for
an introduction to this concept. In this subsection, the
basic commands that take overlay specifications are described.

For the following commands, adding an overlay specification causes the
command to be simply ignored on slides that are not included in the
specification: |\textbf|, |\textit|, |\textsl|,
|\textrm|, |\textsf|, |\color|, |\alert|,
|\structure|. If a command takes several arguments, like
|\color|, the specification should directly follow the command as in
the following example (but there are exceptions to this rule):
\begin{verbatim}
\frame
{
  \color<2-3>[rgb]{1,0,0} This text is red on slides 2 and 3, otherwise black.
}
\end{verbatim}

For the following commands, the effect of an overlay specification is
special:

\begin{command}{\onslide\sarg{overlay specification}}
  All text following this command will only be shown  (uncovered) on
  the specified slides. On non-specified slides, the text still
  occupies space. If no slides are specified, the following
  text is always shown. You need not call this command in the same
  \TeX\ group, its effect transcedes block groups. However, this
  command has a \emph{different} effect inside an |overprint|
  environment, see the description of |overprint|.
  
  \example
\begin{verbatim}
\frame
{
  Shown on first slide.
  \onslide<2-3>
  Shown on second and third slide.
  \begin{itemize}
  \item
    Still shown on the second and third slide.
  \onslide<4->
  \item
    Shown from slide 4 on.
  \end{itemize}
  Shown from slide 4 on.
  \onslide
  Shown on all slides.
}
\end{verbatim}
\end{command}


\begin{command}{\only\sarg{overlay
      specification}\marg{text}\sarg{overlay specification}}
  If either \meta{overlay specification} is present (though only one
  may be present), the \meta{text} is inserted only into the specified
  slides. For other slides, the text is simply thrown away. In
  particular, it occupies no space.
  
  \example |\only<3->{Text inserted from slide 3 on.}|

  Since the overlay specification may also be given after the text,
  you can often use |\only| to make other commands
  overlay-specification-aware in a simple manner:

  \example
\begin{verbatim}
\newcommand{\myblue}{\only{\color{blue}}}
\frame
{
  \myblue<2> This text is blue only on slide 2.
}
\end{verbatim}
\end{command}


\begin{command}{\uncover\sarg{overlay specification}\marg{text}}
  If the \meta{overlay specification} is present, the \meta{text} is
  shown (``uncovered'') only on the specified slides. On other slides, the
  text still occupies space and it is still typeset, but it is not
  shown or only shown as if transparent. For details on how to specify
  whether the text is invisible or just transparent, see
  Section~\ref{section-transparent}. 
  \example |\uncover<3->{Text shown from slide 3 on.}|

  \articlenote
  This command has the same effect as |\only|.
\end{command}

\begin{command}{\visible\sarg{overlay specification}\marg{text}}
  This command does almost the same as |\uncover|. The only difference
  is that if the text is not shown, it is never shown in a transparent
  way, but rather it is not shown at all. Thus for this command the
  transparency settings have no effect.
  
  \example |\visible<2->{Text shown from slide 2 on.}|

  \articlenote
  This command has the same effect as |\only|.
\end{command}

\begin{command}{\invisible\sarg{overlay specification}\marg{text}}
  This command is the opposite of |\visible|.
  
  \example |\invisible<-2>{Text shown from slide 3 on.}|
\end{command}

\begin{command}{\alt\sarg{overlay specification}%
    \marg{default text}\marg{alternative text}\sarg{overlay specification}}
  Only one \meta{overlay specification} may be given. 
  The default text is shown on the specified slides, otherwise the
  alternative text. The specification must always be present.
  \example |\alt<2>{On Slide 2}{Not on slide 2.}|

  Once more, giving the overlay specification at the end is useful
  when the command is used inside other commands.
  
  \example Here is the definition of |\uncover|:
\begin{verbatim}
\newcommand{\uncover}{\alt{\@firstofone}{\makeinvisible}}
\end{verbatim}
\end{command}

\begin{command}{\temporal\ssarg{overlay specification}%
    \marg{before slide text}\marg{default text}\marg{after slide text}}
  This command alternates between three different texts, depending on
  whether the current slide is temporally before the specified
  slides, is one of the specified slides, or comes after them. If the
  \meta{overlay specification} is not an interval (that is, if it has
  a ``hole''), the ``hole'' is considered to be part of the before slides.
  \example
\begin{verbatim}
  \temporal<3-4>{Shown on 1, 2}{Shown on 3, 4}{Shown 5, 6, 7, ...}
  \temporal<3,5>{Shown on 1, 2, 4}{Shown on 3, 5}{Shown 6, 7, 8, ...}
\end{verbatim}

  As a possible application of the |\temporal| command consider the
  following example: 
  \example
\begin{verbatim}
\def\colorize<#1>{%
  \temporal<#1>{\color{structure!50}}{\color{black}}{\color{black!50}}}

\frame{
  \begin{itemize}
    \colorize<1> \item First item.
    \colorize<2> \item Second item.
    \colorize<3> \item Third item.
    \colorize<4> \item Fourth item.
  \end{itemize}
}
\end{verbatim}
\end{command}


\begin{command}{\item\sarg{alert specification}\oarg{item
      label}\sarg{alert specification}}
  \beamernote
  Only one \meta{alert specification} may be given. The effect of
  \meta{alert specification} is described in
  Section~\ref{section-action-specifications}. 
  
  \example
\begin{verbatim}
\frame
{
  \begin{itemize}
  \item<1-> First point, shown on all slides.
  \item<2-> Second point, shown on slide 2 and later.
  \item<2-> Third point, also shown on slide 2 and later.
  \item<3-> Fourth point, shown on slide 3.
  \end{itemize}
}

\frame
{
  \begin{enumerate}
  \item<3-| alert@3>[0.] A zeroth point, shown at the very end.
  \item<1-| alert@1> The first an main point.
  \item<2-| alert@2> The second point.
  \end{enumerate}
}
\end{verbatim}

  \articlenote
  The \meta{action specification} is currently completely ignored.

  \lyxnote
  The \meta{action specification} must be given in \TeX-mode and it
  must be given at the very start of the item.
\end{command}

The related command |\bibitem| is also overlay-specification-aware
in the same way as |\item|.

\begin{command}{\label\sarg{overlay specification}\marg{label name}}
  If the \meta{overlay specification} is present, the label is only
  inserted on the specified slide. Inserting a label on more than one
  slide will cause a `multiple labels' warning. \emph{However}, if no
  overlay specification is present, the specification is automatically
  set to just `1' and the label is thus inserted only on the first
  slide. This is typically the desired behaviour since it does not
  really matter on which slide the label is inserted, \emph{except} if
  you use an |\only| command and \emph{except} if you wish to use that
  lable as a hyperjump target. Then you need to specifiy a slide.

  Labels can be used as target of hyperjumps. A convenient way of
  labelling a frame is to use the |label=|\meta{name} option of the
  |\frame| command. However, this will cause the whole frame to be
  kept in memory till the end of the compilation, which may pose a
  problem. 
  \example
\begin{verbatim}
\frame
{
  \begin{align}
    a &= b + c   \label{first}\\ % no specification needed
    c &= d + e   \label{second}\\% no specification needed
  \end{align}

  Blah blah, \uncover<2>{more blah blah.}

  \only<3>{Specification is needed now.\label<3>{mylabel}}
}
\end{verbatim}
\end{command}



\subsubsection{Environments with Overlay Specifications}

Environments can also be equipped with overlay specifications. For
most of the predefined environments, see Section~\ref{predefined},
adding an overlay specifications causes the whole environment to be
uncovered only on the specified slides. This is useful for showing
things incrementally as in the following example.

\begin{verbatim}
\frame
{
  \frametitle{A Theorem on Infinite Sets}

  \begin{theorem}<1->
    There exists an infinite set.
  \end{theorem}

  \begin{proof}<3->
    This follows from the axiom of infinity.
  \end{proof}

  \begin{example}<2->
    The set of natural numbers is infinite.
  \end{example}
}
\end{verbatim}
In the example, the first slide only contains the theorem, on the
second slide an example is added, and on the third slide the proof is
also shown.

For each of the basic commands |\only|, |\alt|, |\visible|,
|\uncover|, and |\invisible| there exists 
``environment versions'' |onlyenv|, |altenv|, |visibleenv|,
|uncoverenv|, and |invisibleenv|. Except for |altenv|
and |onlyenv|, these environments do the same as the commands.

\begin{environment}{{onlyenv}\sarg{overlay specification}}
  If the \meta{overlay specification} is given, the contents of the
  environment is inserted into the text only on the specified
  slides. The difference to |\only| is, that the text is actually
  typeset inside a box that is then thrown away, whereas |\only|
  immediately throws away its contents. If the text is not
  ``typesettable,'' the |onlyenv| may produce an error where |\only|
  would not.
  \example
\begin{verbatim}
\frame
{
  This line is always shown.
  \begin{onlyenv}<2>
    This line is inserted on slide 2.
  \end{onlyenv}
}
\end{verbatim}
\end{environment}


\begin{environment}{{altenv}\sarg{overlay specification}\marg{begin
text}\marg{end text}\marg{alternate begin text}\marg{alternate end
text}\sarg{overlay specification}}
  Only one \meta{overlay specification} may be given. On the specified
  slides, \meta{begin text} will be inserted at the beginning of the
  environment and \meta{end text} will be inserted at the end. On all
  other slides, \meta{alternate begin text} and \meta{alternate end
    text} will be used.
  
  \example
\begin{verbatim}
\frame
{
  This 
  \begin{altenv}<2>{(}{)}{[}{]}
    word
  \end{uncoverenv}
  is in round brackets on slide 2 and in square brackets on slide 1.
}
\end{verbatim}
\end{environment}


\subsubsection{Dynamically Changing Text}

You may sometimes wish to have some part of a frame change dynamically
from slide to slide. On each slide of the frame, something different
should be shown inside this area. You could achieve the effect of
dynamically changing text by giving a list of |\only| commands like this:
\begin{verbatim}
  \only<1>{Initial text.}
  \only<2>{Replaced by this on second slide.}
  \only<3>{Replaced again by this on third slide.}
\end{verbatim}
The trouble with this approach is that it may lead to slight, but
annoying differences in the heights of the lines, which may cause the
whole frame to ``whobble'' from slide to slide. This problem becomes
much more severe if the replacement text is several lines long.

To solve this problem, you can use two environments:
|overlayarea| and |overprint|. The first is more flexible,
but less user-friendly.

\begin{environment}{{overlayarea}\marg{area width}\marg{area height}}
  Everything within the environment will be placed in a rectangular
  area of the specified size. The area will have the same size on all
  slides of a frame, regardless of its actual contents. 
  \example
\begin{verbatim}
\begin{overlayarea}{\textwidth}{3cm}
  \only<1>{Some text for the first slide.\\Possibly several lines long.}
  \only<2>{Replacement on the second slide.}
\end{overlayarea}
\end{verbatim}

  \lyxnote
  Use the style ``OverlayArea'' to insert an overlay area.
\end{environment}

\begin{environment}{{overprint}\oarg{area width}}
  The \meta{area width} defaults to the text width.
  Inside the environment, use |\onslide| commands to specify
  different things that should be shown for this environment on
  different slides. The |\onslide| commands are used like
  |\item| commands. Everything within the environment will be
  placed in a rectangular area of the specified width. The height and
  depth of the area are chosen large enough to accommodate the largest
  contents of the area. The overlay specifications of the
  |\onslide| commands must be disjoint. This may be a problem for
  handouts, since, there, all overlay specifications defaul to |1|. If
  you use the option |handout|, you can disable all but one
  |\onslide| by setting the others to |0|.
  \example
\begin{verbatim}
\begin{overprint}
  \onslide<1| handout:1>
    Some text for the first slide.\\
    Possibly several lines long.
  \onslide<2| handout:0>
    Replacement on the second slide. Supressed for handout.
\end{overprint}
\end{verbatim}

  \lyxnote
  Use the style ``Overprint'' to insert an |overprint|
  environment. You have to use \TeX-mode to insert the |\onslide|
  commands. 
\end{environment}




\subsection{Making Commands and Environments Overlay-Specification-Aware}

This subsection explains how to define new commands that are 
overlay-specification-aware. Also, it explains how to setup counters
correctly that should be increased from frame to frame (like equation
numbering), but not from slide to slide. You may wish to skip this
section, unless you  want to write your own extensions to the \beamer\
class. 

\beamer\ extends the syntax of \LaTeX's standard command
|\newcommand|:


\begin{command}{\newcommand\declare{|<>|}\marg{command name}%
    \oarg{argument number}\oarg{default optional value}\marg{text}}
  Declares the new command named \meta{command name}. The \meta{text}
  should contain the body of this command and it may contain
  occurences of parameters like |#|\meta{number}. Here \meta{number}
  may be between 1 and $\mbox{\meta{argument number}}+1$. The
  additionally allowed argument is the overlay specification.

  When \meta{command name} is used, it will scan as many as
  \meta{argument number} arguments. While scanning them, it will look
  for an overlay specification, which may be given between any two
  arguments, before the first argument, or after the last argument. If
  it finds an overlay specification like |<3>|, it will call
  \meta{text} with arguments 1 to \meta{argument number} set to the
  normal arguments and the argument number $\mbox{\meta{argument
      number}}+1$ set to |<3>| (including the pointed brackets). If no
  overlay specification is found, the extra argument is empty.

  If the \meta{default optional value} is provided, the first argument
  of \meta{command name} is optional. If no optional argument is
  specified in square brackets, the \meta{default optional value} is
  used.
  
  \example The following command will typeset its argument in red on
  the specified slides:
\begin{verbatim}
\newcommand<>{\makered}[1]{{\color#2{red}#1}}
\end{verbatim}
  
  \example Here is \beamer's definition of |\emph|:
\begin{verbatim}
\newcommand<>{\emph}[1]{{\only#2{\itshape}#1}}
\end{verbatim}

  
  \example Here is \beamer's definition of |\transdissolve| (the
  command |\beamer@dotrans| mainly passes its argument to |hyperref|):
\begin{verbatim}
\newcommand<>{\transdissolve}[1][]{\only#2{\beamer@dotrans[#1]{Dissolve}}}
\end{verbatim}
\end{command}

\begin{command}{\renewcommand\declare{|<>|}\marg{existing command name}%
    \oarg{argument number}\oarg{default optional value}\marg{text}}
  Redeclares a command that already exists in the same way as
  |\newcommand<>|. Inside \meta{text}, you can 
  still access to original definitions using the command
  |\beameroriginal|, see the example.
  \example This command is used in \beamer\ to make |\hyperlink| overlay-specification-aware:
\begin{verbatim}
\renewcommand<>{\hyperlink}[2]{\only#3{\beameroriginal{\hyperlink}{#1}{#2}}}
\end{verbatim}
\end{command}


\begin{command}{\newenvironment\declare{|<>|}\marg{environment name}%
    \oarg{argument number}\oarg{default optional value}\\\marg{begin
    text}\marg{end text}}
  Declares a new environment that is overlay-specification-aware. If
  this environment encountered, the same algorithm as for
  |\newcommand<>| is used to parse the arguments and the overlay
  specification.

  Note that, as always, the \meta{end text} may not contain any
  arguments like |#1|. In particular, you do not have access to the
  overlay specification. In this case, it is usually a good idea to
  use |altenv| environment in the \meta{begin text}.
  
  \example Declare your own action block:
\begin{verbatim}
\newenvironment<>{myboldblock}[1]{%
  \begin{actionenv}#2%
    \textbf{#1}
    \par}
  {\par%
  \end{actionenv}}

\frame
{
  \begin{myboldblock}<2>
    This theorem is shown only on the second slide.
  \end{myboldblock}
}
\end{verbatim}

  \example Text in the following environment is normally bold and
  italic on non-specified slides: 
\begin{verbatim}
\newenvironment<>{boldornormal}
  {\begin{altenv}#1
    {\begin{bfseries}}{\end{bfseries}}
    {}{}}
  {\end{altenv}}
\end{verbatim}

  Incidentally, since |altenv| also accepts its argument at the end,
  the same effect could have been achieved using just 
  \begin{verbatim}
\newenvironment{boldornormal}
  {\begin{altenv}
    {\begin{bfseries}}{\end{bfseries}}
    {}{}}
  {\end{altenv}}
\end{verbatim}
\end{command}

\begin{command}{\renewenvironment\declare{|<>|}\marg{existing environment name}%
    \oarg{argument number}\oarg{default optional value}\\
    \marg{begin
    text}\marg{end text}}
  Redefines an existing environment. The original environment is still
  available under the name |original|\meta{existing environment name}.

  \example
\begin{verbatim}
\renewenvironment<>{verse}
{\begin{actionenv}#1\begin{originalverse}}
{\end{originalverse}\end{actionenv}}
\end{verbatim}
\end{command}

The following two commands can be used to ensure that a certain
counter is automatically reset on subsequent slides of a frame. This
is necessary for example for the equation count. You might want this
count to be increased from frame to frame, but certainly not from
overlay slide to overlay slide. For equation counters and footnote
counters (you should not use footnotes), these commands have already
been invoked.

\begin{command}{\resetcounteronoverlays\marg{counter name}}
  After you have invoked this command, the value of the specified
  counter will be the same on all slides of every frame. 
  \example |\resetcounteronoverlays{equation}|
\end{command}
 
\begin{command}{\resetcountonoverlays\marg{count register name}}
  The same as |\resetcounteronoverlays|, except that this
  command should be used with counts that have been created using the
  \TeX\ primitive |\newcount| instead of \LaTeX's  |\definecounter|. 
  \example
\begin{verbatim}
\newcount\mycount
\resetcountonoverlays{mycount}
\end{verbatim}
\end{command}






\section{Structuring a Presentation}

\subsection{Global Structure of Presentations}

Ideally, during most presentations you would like to present your
slides in a perfectly linear fashion, presumably by pressing the
page-down-key once for each slide. However, there are different
reasons why you might have to deviate from this linear order:
\begin{itemize}
\item
  Your presentation may contain ``different levels of detail'' that
  may or may not be skipped or expanded, depending on the audience's
  reaction.
\item
  You are asked questions and wish to show supplementary slides.
\item
  You are asked questions about an earlier slide, which forces you to 
  find and then jump to that slide.
\end{itemize}
You cannot really prepare against the last kind of questions. In this
case, you can use the navigation bars and symbols to find the slide
you are interested in, see \ref{section-navigation-bars}.

Concerning the first two kinds of deviations, the \beamer\ class
offers several ways of preparing such ``planned detours'' or ``planned
short cuts''.
\begin{itemize}
\item
  You can easily add predefined ``skip buttons.'' When such a button
  is pressed, you jump over a well-defined part of your talk. Skip
  button have two advantages over just pressing the forward key
  is rapid succession: first, you immediately end up at the correct
  position and, second, the button's label can give the audience a
  visual feedback of what exactly will be skipped. For example, when
  you press a skip button labeled ``Skip proof'' nobody will start
  puzzling over what he or she has missed.
\item
  You can add an appendix to your talk. The appendix is kept
  ``perfectly separated'' from the main talk. Only once you ``enter''
  the appendix part (presumably by hyperjumping into it), does the
  appendix structure become visible. You can put all frames that you
  do not intend to show during the normal course of your talk, but
  which you would like to have handy in case someone asks, into this
  appendix.
\item
  You can add ``goto buttons'' and ``return buttons'' to create
  detours. Pressing a goto button will jump to a certain part of the
  presentation where extra details can be shown. In this part, there
  is a return button present on each slide that will jump back to the
  place where the goto button was pressed.
\item
  You can use the |\againframe| command to ``continue'' frames that
  you previously started somewhere, but where certain details have
  been suppressed. You can use the |\againframe| command at a
  much later point, for example only in the appendix to show to
  additional slides there.
\end{itemize}


\subsection{Commands for Creating the Global Structure}

\subsubsection{Adding a Title Page}

You can use the |\titlepage| command to insert a title page into
a frame. 

The |\titlepage| command will arrange the following elements on
the title page: the document title, the author(s)'s names, their
affiliation, a title graphic, and a date.

\begin{command}{\titlepage}
  Inserts the text of a title page into the current frame.
  \example |\frame{\titlepage}|

  \lyxnote
  If you use the ``Title'' style in your presentation, a title page is
  automatically inserted.
\end{command}

For compatibility with other classes in article mode, the following
command is also provided: 

\begin{command}{\maketitle}
  \beamernote
  Same as |\titlepage|.
\end{command}


Before you invoke the title page command, you must specify all
elements you wish to be shown. This is done using the following
commands: 

\begin{command}{\title\oarg{short title}\marg{title}}
  The \meta{short tile} is used in head lines and foot lines. Inside
  the \meta{title} line breaks can be inserted using the
  double-backslash command.
  \example
\begin{verbatim}
\title{The Beamer Class}
\title[Short Version]{A Very Long Title\\Over Several Lines}
\end{verbatim}

  \articlenote
  The short form is ignored in |article| mode.
\end{command}

\begin{command}{\author\oarg{short author names}\marg{author names}}
  The names should be separated using the
  command |\and|. In case authors have different affiliations,
  they should be suffixed by the command |\inst| with different
  parameters.
  \example|\author[Hemaspaandra et al.]{L. Hemaspaandra\inst{1} \and T. Tantau\inst{2}}|

  \articlenote
  The short form is ignored in |article| mode.
\end{command}

\begin{command}{\institute\oarg{short institute}\marg{institute}}
  If more than one institute is given, they should be separated using
  the command |\and| and they should be prefixed by the command
  |\inst| with different parameters.
  \example
\begin{verbatim}
\institute[Universities of Rochester and Berlin]{
  \inst{1}Department of Computer Science\\
  University of Rochester
  \and
  \inst{2}Fakult\"at f\"ur Elektrotechnik und Informatik\\
  Technical University of Berlin}
\end{verbatim}

  \articlenote
  The short form is ignored in |article| mode. The long form is also
  ignored, except if the document class (like |llncs|) defines it.
\end{command}

\begin{command}{\date\oarg{short date}\marg{date}}
  \example|\date{\today}| or |\date[STACS 2003]{STACS Conference, 2003}|.

  \articlenote
  The short form is ignored in |article| mode.
\end{command}

\begin{command}{\titlegraphic\marg{text}}
  The \meta{text} is shown as title graphic. Typically, a picture
  environment is used as \meta{text}.
  \example|\titlegraphic{\pgfuseimage{titlegraphic}}|

  \articlenote
  The command is  ignored in |article| mode.
\end{command}




\subsubsection{Adding Sections and Subsections}

You can structure your text using the commands |\section| and
|\subsection|. Unlike standard \LaTeX, these commands will not
create a heading at the position where you use them. Rather, they will
add an entry to the table of contents and also to the navigation
bars.

In order to create a line break in the table of contents (usually not
a good idea), you can use the command |\breakhere|. Note that the
standard command |\\| does not work (actually, I do not really know
why; comments would be appreciated).

\begin{command}{\section\sarg{mode specification}\oarg{short section name}\marg{section name}}
  Starts a section. No heading is created. The \meta{section name}
  is shown in the table of contents and in the navigation bars, except
  if \meta{short section name} is specified. In this case, \meta{short
    section name} is used in the navigation bars instead. If a
    \meta{mode specification} is given, the command only has an effect
    for the specified modes.
    
  \example|\section[Summary]{Summary of Main Results}|

  \articlenote
  Then \meta{mode specification} allows you to provide an alternate
  section command in |article| mode. This is necessary for example if
  the \meta{short section name} is unsuitable for the table of
  contents:

  \example
\begin{verbatim}
\section<presentation>[Results]{Results on the Main Problem}
\section<article>{Results on the Main Problem}
\end{verbatim}
\end{command}

\begin{command}{\section\sarg{mode specification}\declare{|*|}\marg{section name}}
  Starts a section without an entry in the table of contents. No
  heading is created, but the \meta{section name} is shown in the
  navigation bar. 
  \example|\section*{Outline}|
  \example|\section<beamer>*{Outline}|
\end{command}

\begin{command}{\subsection\sarg{mode specification}\oarg{short
  subsection name}\marg{subsection name}} 
  This command works the same way as the |\section| command.
  \example|\subsection[Applications]{Applications to the Reduction of Pollution}|
\end{command}

\begin{command}{\subsection\sarg{mode specification}\declare{|*|}\marg{subsection name}} 
  Starts a subsection without an entry in the table of contents. No
  heading is created, but the \meta{subsection name} is shown in the
  navigation bar, \emph{except} if \meta{subsection name} is empty. In
  this case, neither a table of contents entry nor a navigation bar
  entry is created, \emph{but} any frames in this ``empty'' subsection
  are shown in the navigation bar.

  \example
\begin{verbatim}
\section{Summary}

  \frame{This frame is not shown in the navigation bar}

  \subsection*{}

  \frame{This frame is shown in the navigation bar, but no subsection
    entry is shown.}

  \subsection*{A subsection}

  \frame{Normal frame, shown in navigation bar. The subsection name is
    also shown in the navigation bar, but not in the table of contents.} 
\end{verbatim}
\end{command}

Often, you may want a certain type of frame to be shown directly after
a section or subsection starts. For example, you may wish every
subsection to start with a frame showing the table of contents with
the current subsection hilighted. To facilitate this, you can use the
following two commands.


\begin{command}{\AtBeginSection\oarg{special star text}\marg{text}}
  The given text will be inserted at the beginning of every
  section. If the \meta{special star text} parameter is specified,
  this text will be used for starred sections instead. Different calls
  of this command will not ``add up'' the given texts (like the
  |\AtBeginDocument| command does), but will  overwrite any previous
  text. 
  
  \example
\begin{verbatim}
\AtBeginSection[] % Do nothing for \section*
{
  \frame<beamer>
  {
    \frametitle{Outline}
    \tableofcontents[current]
  }
}
\end{verbatim}

  \articlenote
  This command has no effect in |article| mode.

  \lyxnote
  You have to insert this command using a \TeX-mode text.
\end{command}


\begin{command}{\AtBeginSubsection\oarg{special star text}\marg{text}}
  The given text will be inserted at the beginning of every
  subsection. If the \meta{special star text} parameter is specified,
  this text will be used for starred subsections instead. Different calls
  of this command will not ``add up'' the given texts.
  
  \example
\begin{verbatim}
\AtBeginSubsection[] % Do nothing for \subsection*
{
  \frame<beamer>
  {
    \frametitle{Outline}
    \tableofcontents[current,currentsubsection]
  }
}
\end{verbatim}
\end{command}




\subsubsection{Adding Parts}

If you give a long talk (like a lecture), you may wish to break up
your talk into several parts. Each such part acts like a little ``talk
of its own'' with its own table of contents, its own navigation bars,
and so on. Inside one part, the sections and subsections of the other
parts are not shown at all.

To create a new part, use the |\part| command. All sections and
subsections following this command will be ``local'' to that part.
Like the |\section| and |\subsection| command, the |\part| command
does not cause any frame or special text to be produced. However,
it is often advisable for the start of a new part to use the command
|\partpage| to insert some text into a frame that ``advertises'' the
beginning of a new part. See |beamerexample3.tex| for an example.

\begin{command}{\part\sarg{mode specification}\oarg{short part name}\marg{part name}}
  Starts a part. The \meta{part name} will be shown when the
  |\partpage| command is used. The \meta{shown part name} is not shown
  anywhere by default, but it is accessible via the command
  |\insertshortpart|.
  \example
\begin{verbatim}
\begin{document}
  \frame{\titlepage}

  \section*{Outlines}
  \subsection{Part I: Review of Previous Lecture}
  \frame{
    \frametitle{Outline of Part I}
    \tableofcontents[part=1]}
  \subsection{Part II: Today's Lecture}
  \frame{
    \frametitle{Outline of Part II}
    \tableofcontents[part=2]}

  \part{Review of Previous Lecture}
  \frame{\partpage}
  \section[Previous Lecture]{Summary of the Previous Lecture}
  \subsection{Topics}
  \frame{...}
  \subsection{Learning Objectives}
  \frame{...}
  
  \part{Today's Lecture}
  \frame{\partpage}
  \section{Topic A}
  \frame{\tableofcontents[current]}
  \subsection{Foo}
  \frame{...}
  \section{Topic B}
  \frame{\tableofcontents[current]}
  \subsection{bar}
  \frame{...}
\end{document}
\end{verbatim}
\end{command}

\begin{command}{\partpage}
  Works like |\titlepage|, only that the current part, not the current
  presentation is ``advertised.'' The appearance can be changed by
  adjusting the part page template, see
  Section~\ref{section-part-page-template}. 
  \example |\frame{\partpage}|
\end{command}

\begin{command}{\AtBeginPart\marg{text}}
  The given text will be inserted at the beginning of every
  part.
  
  \example
\begin{verbatim}
\AtBeginPart{\frame{\partpage}}
\end{verbatim}
\end{command}


\subsubsection{Splitting a Course Into Lectures}

When using \beamer\ with the |article| mode, you may wish to have the
lecture notes of a whole course reside in one file. In this case, only
a few frames are actually part of any particular lecture.

The |\lecture| command makes it easy to select only a certain set of
frames from a file to be presented. This command takes (among other
things) a label name. If you say |\includeonlylecture| with this label
name, then only the frames following the |\lecture| command are
shown. The frames following other |\lecture| commands are suppressed.

By default, the |\lecture| command has no other effect. It does not
create any frames or introduce entries in the table of
contents. However, you can use |\AtBeginLecture| to have \beamer\
insert, say, a title page at the beginning of (each) lecture.

\begin{command}{\lecture\oarg{short lecture name}\marg{lecture
  name}\marg{lecture label}} 
  Starts a lecture. The \meta{lecture name} will be available via the
  |\insertlecture| command. The \meta{short lecture name} is available
  via the |\insertshortlecture| command.
  
  \example
\begin{verbatim}
\begin{document}
\lecture{Vector Spaces}{week 1}

\section{Introduction}
...
\section{Summary}

\lecture{Scalar Products}{week 2}

\section{Introduction}
...
\section{Summary}

\end{document}
\end{verbatim}

  \articlenote
  This command has no effect in |article| mode.
\end{command}

\begin{command}{\includeonlylecture\meta{lecture label}}
  Causes all |\frame|, |\section|, |\subsection|, and |\part| commands
  following a |\lecture| command to be suppressed, except if the
  lecture's label matches the \meta{lecture label}. Frames before any
  |\lecture| commands are always included. This command should be
  given in the preamble.

  \example |\includeonlylecture{week 1}|

  \articlenote
  This command has no effect in |article| mode.
\end{command}

\begin{command}{\AtBeginLecture\marg{text}}
  The given text will be inserted at the beginning of every
  lecture.
  
  \example
\begin{verbatim}
\AtBeginLecture{\frame{\Large Today's Lecture: \insertlecture}}
\end{verbatim}

  \articlenote
  This command has no effect in |article| mode.
\end{command}


\subsubsection{Adding a Table of Contents}

You can create a table of contents using the command
|\tableofcontents|. Unlike the normal \LaTeX\ table of contents
command, this command takes an optional parameter in square brackets
that can be used to create certain special effects.

\begin{command}{\tableofcontents\oarg{comma-separated option list}}
  Inserts a table of contents into the current frame. To change how
  the table of contents is typeset, you need to modify the appropriate
  templates, see Section~\ref{section-toc-templates}. 
  \example
\begin{verbatim}
\section*{Outline}
\frame{\tableofcontents}

\section{Introduction}
\frame{\tableofcontents[current]}
\subsection{Why?}
\frame{...}
\frame{...}
\subsection{Where?}
\frame{...}

\section{Results}
\frame{\tableofcontents[current]}
\subsection{Because}
\frame{...}
\subsection{Here}
\frame{...}
\end{verbatim}

  The following options can be given:
  \begin{itemize}
  \item
    \declare{|part=|\meta{part number}} causes the table of contents
    of part \meta{part number} to be shown, instead of the table of
    contents of the current part (which is the default). This option
    can be combined with the other options, although combining it with
    the |current| option obviously makes no sense.
  \item
    \declare{|sections=|\marg{overlay specification}} causes only the
    sections mentioned in the \meta{overlay specification} to be
    shown. For example, \verb/sections={<2-4| handout:0>}/ causes only the second,
    third, and fourth section to be shown in the normal version,
    nothing to be shown in the handout version, and everything to be
    shown in all other versions. For convenience, if you omit the
    pointed brackets, the specification is assumed to apply to all
    versions. Thus |sections={2-4}| causes sections two, three, and
    four to be shown in all versions.
  \item
    \declare{|firstsection=|\meta{section number}} specifies which
    section should be numbered as section~``1.''  This is useful if
    you have a first section (like an overview section) that should
    not receive a number. Section numbers are not shown by default. To
    show them, you must install a different table of contents
    templates.
  \item
    \declare{|current|} causes all sections but the current to
    be shown in a semi-transparent way. Also, all subsections but
    those in the current section are shown in the semi-transparent way.
  \item
    \declare{|currentsubsection|} causes all subsections but the
    current subsection in the current section to be shown in a
    semi-transparent way.
  \item
    \declare{|pausesections|} causes a |\pause| command to
    be issued before each section. This is useful if you wish to show
    the table of contents in an incremental way.
  \item
    \declare{|pausesubsections|} causes a |\pause| command to
    be issued before each subsection.
  \item
    \declare{|hidesubsections|} causes the subsections to be
    omitted. However, if used together with the |current| option,
    the subsections of the current section are not omitted.
  \item
    \declare{|shadesubsections|} causes the subsections to
    be shown in a semi-transparent way.
  \end{itemize}
  The last two commands are useful if you do not wish to show too many
  details when presenting the talk outline.

  \articlenote
  The options are ignored in |article| mode.

  \lyxnote
  You can give options to the |\tableofcontents| command by 
  inserting a \TeX-mode text with the options in square brackets
  directly after the table of contents.
\end{command}



\subsubsection{Adding a Bibliography}

You can use the bibliography environment and the |\cite| commands
of \LaTeX\ in a \beamer\ presentation. However, there are a few things
to keep in mind:

\begin{itemize}
\item
  It is a bad idea to present a long bibliography in a 
  presentation. Present only very few references.
\item
  Present references only if they are intended as ``further reading,''
  for example at the end of a lecture.
\item
  Using the |\cite| commands can be confusing since the audience
  has little chance of remembering the citations. If you cite the
  references, always cite them with full author name and year like
  ``[Tantau, 2003]'' instead of something like ``[2,4]'' or
  ``[Tan01,NT02]''.
\item
  If you want to be modest, you can abbreviate your name when citing
  yourself as in ``[Nickelsen and T., 2003]'' or ``[Nickelsen and T,
  2003]''. However, this can be confusing for the audience since it is
  often not immediately clear who exactly ``T.'' might be. I recommend
  using the full name.
\end{itemize}

Keeping the above warnings in mind, proceed as follows to create the
bibliography: 

For a beamer presentation, you will typically have to typeset your
bibliography items partly ``by hand.'' Nevertheless, you \emph{can}
use |bibtex| to create a ``first approximation'' of the
bibliography. Copy the content of the file |main.bbl| into your
presentation. If you are not familiar with |bibtex|, you may wish
to consult its documentation. It is a  powerful tool for
creating high-quality citations.

Using |bibtex| or your editor, place your bibliographic
references in the environment |thebibliography|. This
(standard \LaTeX) environment takes one parameter, which should be the
longest |\bibitem| label in the following list of bibliographic
entries.

\begin{environment}{{thebibliography}\marg{longest label text}}
  Inserts a bibliography into the current frame. The \meta{longest
    label text} is used to determine the indent of the list. However,
  several templates for the typesetting of the bibliography (see
  Section~\ref{section-bib-templates}) ignore this 
  parameter since they replace the references by a symbol.

  Inside the environment, use a (standard \LaTeX) |\bibitem| command
  for each reference item. Inside each item, use a (standard \LaTeX)
  |\newblock| command to separate the authors's names, the title, the
  book/journal reference, and any notes. Each of these commands may
  introduce a new line or color or other formatting, as specified by
  the template for bibliographies.

  The environment must be placed inside a frame. If the bibliography
  does not fit on one frame, you should 
  split it (create a new frame and a second |thebibliography|
  environment). Even better, you should reconsider whether it is a good
  idea to present so many references.
  \example
\begin{verbatim}
\frame{
  \frametitle{For Further Reading}

  \begin{thebibliography}{Dijkstra, 1982}
  \bibitem[Solomaa, 1973]{Solomaa1973}
    A.~Salomaa.
    \newblock {\em Formal Languages}.
    \newblock Academic Press, 1973.

  \bibitem[Dijkstra, 1982]{Dijkstra1982}
    E.~Dijkstra.
    \newblock Smoothsort, an alternative for sorting in situ.
    \newblock {\em Science of Computer Programming}, 1(3):223--233, 1982.
  \end{thebibliography}
 }
\end{verbatim}
\end{environment}

\begin{command}{\bibitem\sarg{overlay specification}%
    \oarg{citation text}\marg{label name}}
  The \meta{citation text} is inserted into the text when the item is
  cited using |\cite{|\meta{label name}|}| in the main presentation
  text. For a \beamer\ presentation, this should usually be as long as
  possible.  

  Use |\newblock| commands to separate the authors's names, the title, the
  book/journal reference, and any notes. If the \meta{overlay specification}
  is present, the entry will only be shown on the
  specified slides.
  \example
\begin{verbatim}
\bibitem[Dijkstra, 1982]{Dijkstra1982}
  E.~Dijkstra.
  \newblock Smoothsort, an alternative for sorting in situ.
  \newblock {\em Science of Computer Programming}, 1(3):223--233, 1982.
\end{verbatim}
\end{command}

Unlike normal \LaTeX, the default template for the
bibliography does not repeat the citation text (like ``[Dijkstra,
1982]'') before each item in the bibliography. Instead, a cute, small
article symbol is drawn. The rationale is that the audience will not be
able to remember any abbreviated citation texts till the end of the
talk. If you really insist on using abbreviations, you can use the
command |\beamertemplatetextbibitems| to restore the default
behavior, see also Section~\ref{section-bib-templates}.




\subsubsection{Adding an Appendix}

You can add an appendix to your talk by using the |\appendix|
command. You should put frames and perhaps whole subsections into the
appendix that you do not intend to show during your presentation, but
which might be useful to answer a question. The |\appendix| command
essentially just starts a new part named |\appendixname|. However, it
also sets up certain hyperlinks. 
Like other parts, the appendix is kept separate of your actual
talk.

\begin{command}{\appendix\sarg{mode specification}}
  Starts the appendix in the specified modes. All frames, all
  |\subsection| commands, and all |\section| commands used after this
  command will not be shown as part of the normal navigation bars.
  \example
\begin{verbatim}
\begin{document}
\frame{\titlepage}
\section*{Outline}
\frame{\tableofcontents}
\section{Main Text}
\frame{Some text}
\section*{Summary}
\frame{Summary text}

\appendix
\section{\appendixname}
\frame{\tableofcontents}
\subsection{Additional material}
\frame{Details}
\frame{Text omitted in main talk.}
\subsection{Even more additional material}
\frame{More details}
\end{document}
\end{verbatim}
\end{command}





\subsubsection{Adding Hyperlinks and Buttons}

To create an anticipated nonlinear jumps in your talk structure, you
can add hyperlinks to your presentation. A hyperlink is a text
(usually rendered as a button) that, when you click on it, jumps the
presentation to some other slide. Creating such a button is a
three-step process: 
\begin{enumerate}
\item
  You specify a target using the command |\hypertarget| or (easier)
  the command |\label|. In some cases, see below, this step may be
  skipped. 
\item
  You render the button using |\beamerbutton| or a similar
  command. This will \emph{render} the button, but clicking it will
  not yet have any effect. 
\item
  You put the button inside a |\hyperlink| command. Now clicking it
  will jump to the target of the link.  
\end{enumerate}

\begin{command}{\hypertarget\sarg{overlay specification}%
    \marg{target name}\marg{text}}
  If the \meta{overlay specification} is present, the \meta{text} is
  the target for hyper jumps to \meta{target name} only on the
  specified slide. On all other slides, the text is shown
  normally. Note that you \emph{must} add an overlay specification to
  the |\hypertarget| command whenever you use it on frames that have
  multiple slides (otherwise |pdflatex| rightfully complains
  that you have defined the same target on different slides).
  \example
\begin{verbatim}
\frame{
  \begin{itemize}
  \item<1-> First item.
  \item<2-> Second item.
  \item<3-> Third item.
  \end{itemize}

  \hyperlink{jumptosecond}{\beamergotobutton{Jump to second slide}}
  \hypertarget<2>{jumptosecond}{}
}
\end{verbatim}

  \articlenote
  You must say |\usepackage{hyperref}| in your preamble to use this
  command in |article| mode.
\end{command}

The |\label| command creates a hypertarget as a side-effect and the
|label=|\meta{name} option of the |\frame| command creates a label
named \meta{name}|<|\meta{slide number}|>| for each slide of the frame
as a side-effect. Thus the above example could be written more easily
as: 
\begin{verbatim}
\frame[label=threeitems]{
  \begin{itemize}
  \item<1-> First item.
  \item<2-> Second item.
  \item<3-> Third item.
  \end{itemize}

  \hyperlink{threeitems<2>}{\beamergotobutton{Jump to second slide}}
}
\end{verbatim}



The following commands can be used to specify in an abstract way what
a button will be used for. How exactly these buttons are rendered is
governed by a template, see Section~\ref{section-navigation-buttons}.

\begin{command}{\beamerbutton\marg{button text}}
  Draws a button with the given \meta{button text}.
  \example |\hyperlink{somewhere}{\beamerbutton{Go somewhere}}|

  \articlenote
  This command (and the following) just insert their argument in
  |article| mode.
\end{command}

\begin{command}{\beamergotobutton\marg{button text}}
  Draws a button with the given \meta{button text}. Before the text, a
  small symbol (usually a right-pointing arrow) is inserted that
  indicates that pressing this button will jump to another ``area'' of
  the presentation.

  \example |\hyperlink{detour}{\beamergotobutton{Go to detour}}|
\end{command}

\begin{command}{\beamerskipbutton\marg{button text}}
  The symbol drawn for this button is usually a double right
  arrow. Use this button if pressing it will skip over a
  well-defined part of your talk.

  \example
\begin{verbatim}
\frame{
  \begin{theorem}
    ...
  \end{theorem}

  \begin{overprint}
  \onslide<1>
    \hfill\hyperlinkframestartnext{\beamerskipbutton{Skip proof}}
  \onslide<2>
    \begin{proof}
      ...
    \end{proof}
  \end{overprint}
}
\end{verbatim}
\end{command}

\begin{command}{\beamerreturnbutton\marg{button text}}
  The symbol drawn for this button is usually a left pointing
  arrow. Use this button if pressing it will return from a detour. 

  \example
\begin{verbatim}
\frame<1>[label=mytheorem]
{
  \begin{theorem}
    ...
  \end{theorem}

  \begin{overprint}
  \onslide<1>
    \hfill\hyperlink{mytheorem<2>}{\beamergotobutton{Go to proof details}}
  \onslide<2>
    \begin{proof}
      ...
    \end{proof}
    \hfill\hyperlink{mytheorem<1>}{\beamerreturnbutton{Return}}
  \end{overprint}
}
\appendix
\againframe<2>{mytheorem}
\end{verbatim}
\end{command}

To make a button ``clickable'' you must place it in a command like
|\hyperlink|. The command |\hyperlink| is a standard command of the
|hyperref| package. The \beamer\ class defines a whole bunch of other
hyperlink commands that you can also use.

\begin{command}{\hyperlink\sarg{overlay specification}\marg{target
      name}\marg{link text}\sarg{overlay specification}}
  Only one \meta{overlay specification} may be given.
  The \meta{link text} is typeset in the usual way. If you click
  anywhere on this text, you will jump to the slide on which the
  |\hypertarget| command was used with the parameter \meta{target
    name}. If an \meta{overlay specification} is present, the
    hyperlink (including the \meta{link text}) is completely
    suppressed on the non-specified slides.
\end{command}

The following commands have a predefined target; otherwise they behave
exactly like |\hyperlink|. In particular, they all also accept an
overlay specification and they also accept it at the end, rather than
at the beginning.

\begin{command}{\hyperlinkslideprev\sarg{overlay specification}\marg{link text}}
  Clicking the text jumps one slide back.
\end{command}

\begin{command}{\hyperlinkslidenext\sarg{overlay specification}\marg{link text}}
  Clicking the text jumps one slide forward.
\end{command}
  
\begin{command}{\hyperlinkframestart\sarg{overlay specification}\marg{link text}}
  Clicking the text jumps to the first slide of the current frame.
\end{command}

\begin{command}{\hyperlinkframeend\sarg{overlay specification}\marg{link text}}
  Clicking the text jumps to the last slide of the current frame.
\end{command}

\begin{command}{\hyperlinkframestartnext\sarg{overlay specification}\marg{link text}}
  Clicking the text jumps to the first slide of the next frame.
\end{command}

\begin{command}{\hyperlinkframeendprev\sarg{overlay specification}\marg{link text}}
  Clicking the text jumps to the last slide of the previous frame.
\end{command}

The previous four command exist also with ``|frame|'' replaced by
``|subsection|'' everywhere, and also again with  ``|frame|'' replaced
by ``|section|''.

\begin{command}{\hyperlinkpresentationstart\sarg{overlay specification}\marg{link text}}
  Clicking the text jumps to the first slide of the presentation.
\end{command}

\begin{command}{\hyperlinkpresentationend\sarg{overlay specification}\marg{link text}}
  Clicking the text jumps to the last slide of the presentation. This
  \emph{excludes} the appendix.
\end{command}

\begin{command}{\hyperlinkappendixstart\sarg{overlay specification}\marg{link text}}
  Clicking the text jumps to the first slide of the appendix. If there
  is no appendix, this will jump to the last slide of the document.
\end{command}

\begin{command}{\hyperlinkappendixend\sarg{overlay specification}\marg{link text}}
  Clicking the text jumps to the last slide of the appendix.
\end{command}

\begin{command}{\hyperlinkdocumentstart\sarg{overlay specification}\marg{link text}}
  Clicking the text jumps to the first slide of the presentation.
\end{command}

\begin{command}{\hyperlinkdocumentend\sarg{overlay specification}\marg{link text}}
  Clicking the text jumps to the last slide of the presentation or, if
  an appendix is present, to the last slide of the appendix.
\end{command}





\subsection{Navigation Bars and Symbols}
\label{section-navigation-bars}

Navigation bars and symbols are two independent concepts that can be
used to navigate through a presentation. They are created
automatically.


\subsubsection{Using the Navigation Bars}

Most themes that come along with the \beamer\ class show some kind of
navigation bar during your talk. Although these navigation bars take
up quite a bit of space, they are often useful for two reasons:

\begin{itemize}
\item
  They provide the audience with a visual feedback of how much of your
  talk you have covered and what is yet to come. Without such
  feedback, an audience will often puzzle whether something you are
  currently introducing will be explained in more detail later on or
  not.
\item
  You can click on all parts of the navigation bar. This will directly
  ``jump'' you to the part you have clicked on. This is particularly
  useful to skip certain parts of your talk and during a ``question
  session,'' when you wish to jump back to a particular frame someone
  has asked about.
\end{itemize}

Some navigation bars can be ``compressed'' using the following option:

\begin{classoption}{compress}
  Tries to make all navigation bars as small as possible. For example,
  all small frame representations in the navigation bars for a single
  section are shown alongside each other. Normally, the representations
  for different subsections are shown in different lines. Furthermore,
  section and subsection navigations are compressed into one line.
\end{classoption}

When you click on one of the icons representing a frame in a
navigation bar (by default this is icon is a small circle), the
following happens: 
\begin{itemize}
\item
  If you click on (the icon of) any frame other than the current frame, the
  presentation will jump to the first slide of the frame you clicked
  on.
\item
  If you click on the current frame and you are not on the last slide
  of this frame, you will jump to the last slide of the frame.
\item
  If you click on the current frame and you are on the last slide, you
  will jump to the first slide of the frame.
\end{itemize}
By the above rules you can:
\begin{itemize}
\item
  Jump to the beginning of a frame from somewhere else by clicking on
  it once.  
\item
  Jump to the end of a frame from somewhere else by clicking on it
  twice.
\item
  Skip the rest of the current frame by clicking on it once.
\end{itemize}

I also tried making a jump to an already-visited frame jump
automatically to the last slide of this frame. However, this turned
out to be more confusing than helpful. With the current implementation
a double-click always brings you to the end of a slide, regardless
from where you ``come.''

By clicking on a section or subsection in the navigation bar, you will
jump to that section. Clicking on a section is particularly useful if
the section starts with a |\tableofcontents[current]|, since you
can use it to jump to the different subsections.

By clicking on the document title in a navigation bar (not all themes
show it), you will jump to the first slide of your presentation
(usually the title page) \emph{except} if you are already at the first
slide. On the first slide, clicking on the document title will jump to
the end of the presentation, if there is one. Thus by \emph{double}
clicking the document title in a navigation bar, you can jump to the end.



\subsubsection{Using the Navigation Symbols}
\label{section-navigation-symbols}

Navigation symbols are small icons that are shown on every slide
by default. The following symbols are shown: 
\begin{enumerate}
\item
  A slide icon, which is depicted as  a single rectangle. To the left and
  right of this symbol, a left and right arrow are shown.
\item
  A frame icon, which is depicted as three slide icons ``stacked on top of
  each other''. This symbols is framed by arrows.
\item
  A subsection icon, which is depicted as a highlighted subsection
  entry in a table of contents. This  symbols is framed by arrows.
\item
  A section icon, which is depicted as a highlighted section entry
  (together with all subsections) in a table of contents. This symbols
  is framed by arrows.
\item
  A presentation icon, which is depicted as a completely highlighted
  table of contents.
\item
  An appendix icon, which is depicted as a completely highlighted
  table of contents consisting of only one section. (This icon is only
  shown if there is an appendix.
\item
  Back and forward icons, depicted as circular arrows.
\item
  A ``search'' or ``find'' icon, depicted as a detective's
  magnifying glass.
\end{enumerate}

Clicking on the left arrow next to an icon always jumps to (the
last slide of) the previous slide, frame, subsection, or
section. Clicking on the right arrow next to an icon always jump to
(the first slide of) the next slide, frame, subsection, or section. 

Clicking \emph{on} any of these icons has different effects:
\begin{enumerate}
\item
  If supported by the viewer application, clicking on a slide icon
  pops up a window that allows you to enter a slide number to which
  you wish to jump.
\item
  Clicking on the left side of a frame icon will jump to the first
  slide of the frame, clicking on the right side will jump to the last
  slide of the frame (this can be useful for skipping overlays).
\item
  Clicking on the left side of a subsection icon will jump to the
  first slide of the subsection, clicking on the right side will jump
  to the last slide of the subsection.
\item
  Clicking on the left side of a section icon will jump to the
  first slide of the section, clicking on the right side will jump
  to the last slide of the section.
\item
  Clicking on the left side of the presentation icon will jump to the
  first slide, clicking on the right side will jump to the last slide
  of the presentation. However, this does \emph{not} include the
  appendix. 
\item
  Clicking on the left side of the appendix icon will jump to the
  first slide of the appendix, clicking on the right side will jump to
  the last slide of the appendix.
\item
  If supported by the viewer application, clicking on the back and
  forward symbols jumps to the previously visited slides.
\item
  If supported by the viewer application, clicking on the search icon
  pops up a window that allows you to enter a search string. If found,
  the viewer application will jump to this string.
\end{enumerate}

You can reduce the number of icons that are shown or their layout by
adjusting the navigation symbols template, see
Section~\ref{section-navigation-symbols-template}. 







\subsection{Command for Creating the Local Structure}

Just like your whole presentation, each frame should also be
structured. A frame that is solely filled with some long text is very
hard to follow. It is your job to structure the contents of each frame
such that, ideally, the audience immediately seems which information
is important, which information is just a detail, how the presented
information is related, and so on.

\LaTeX\ provides different commands for structuring text ``locally,''
for example, via the |itemize| environment. These environments
are also available in the beamer class, although their appearance has
been slightly changed. Furthermore, the \beamer\ class also defines
some new commands and environments, see below, that may help you to
structure your text.


\subsubsection{Itemizations, Enumerations, and Descriptions}

\label{section-enumerate}

There are three predefined environments for creating lists, namely
|enumerate|, |itemize|, and |description|. The first
two can be nested to depth two, but not further (this would
create totally unreadable slides).

The |\item| command is overlay-specification-aware. If an overlay
specification is provided, the item will only be shown on the
specified slides, see the following example. If the |\item|
command is to take an optional argument and an overlay specification,
the overlay specification can either come first as in |\item<1>[Cat]|
or come last as in |\item[Cat]<1>|.

\begin{verbatim}
\frame
{
  There are three important points:
  \begin{enumerate}
  \item<1-> A first one,
  \item<2-> a second one with a bunch of subpoints,
    \begin{itemize}
    \item first subpoint. (Only shown from second slide on!).
    \item<3-> second subpoint added on third slide.
    \item<4-> third subpoint added on fourth slide.
    \end{itemize}
  \item<5-> and a third one.
  \end{enumerate}
}
\end{verbatim}


\begin{environment}{{itemize}\opt{|[<|\meta{default overlay specification}|>]|}}
  Used to display a list of items that do not have a special
  ordering. Inside the environment, use an |\item| command for
  each topic. The appearance of the items can be changed using
  templates, see Section~\ref{section-templates}.

  If the optional parameter \meta{default overlay specification} is
  given, in every occurence of an |\item| command that does not have
  an overlay specification attached to it, the \meta{default overlay
    specification} is used. By setting this specificaiton to be an
  incremental overlay specification, see
  Section~\ref{section-incremental}, you can implement, for example, a
  step-wise uncovering of the items. The \meta{default overlay
    specification} is inherited by subenvironments. Naturally, in a
  subenvironment you can reset it locally by setting it to |<1->|.
  \example
\begin{verbatim}
\begin{itemize}
\item This is important.
\item This is also important.
\end{itemize}
\end{verbatim}
  
  \example
\begin{verbatim}
\begin{itemize}[<+->]
\item This is shown from the first slide on.
\item This is shown from the second slide on.
\item This is shown from the third slide on.
\item<1-> This is shown from the first slide on.
\item This is shown from the fourth slide on.    
\end{itemize}
\end{verbatim}
  
  \example
\begin{verbatim}
\begin{itemize}[<+-| alert@+>]
\item This is shown from the first slide on and alerted on the first slide.
\item This is shown from the second slide on and alerted on the second slide.
\item This is shown from the third slide on and alerted on the third slide.
\end{itemize}
\end{verbatim}
  
  \example
\begin{verbatim}
\newenvironment{mystepwiseitemize}{\begin{itemize}[<+-| alert@+>]}{\end{itemize}}
\end{verbatim}

  \lyxnote
  Unfortunately, currently you cannot specify optional arguments with
  the |itemize| environment. You can, however, use the command
  |\beamerdefaultoverlayspecification| before the environment to get
  the desired effect. 
\end{environment}


\begin{environment}{{enumerate}\opt{|[<|\meta{default overlay specification}|>]|}\oarg{mini template}} 
  Used to display a list of items that are ordered.  Inside the
  environment, use an |\item| command for each topic. By default,
  before each item increasing Arabic numbers  followed by a dot are
  printed (as in ``1.'' and ``2.''). This can be changed by specifying
  a different template,  see
  Section~\ref{section-template-enumerate}.

  The first optional argument \meta{default overlay specification} has
  exactly the same effect as for the |itemize| environment. It is
  ``detected'' by the opening |<|-sign in the \meta{default overlay
    specification}. Thus, if there is only one optional argument and
  if this argument does not start with |<|, then it is considered to
  be a \meta{mini template}. 

  The syntex of the \meta{mini template} is the same as
  the syntax of mini templates in the |enumerate| package (you do not
  need to include the 
  |enumerate| package, this is done automatically). Roughly spoken,
  the text of the \meta{mini template} is printed before each item,
  but any occurrence of a |1| in the mini template is replaced by the
  current item number, an occurrence of the letter |A| is replaced by
  the $i$th letter of the alphabet (in uppercase) for the $i$th item,
  and the letters |a|, |i|, and |I| are replaced by the corresponding
  lowercase letters, lowercase Roman letters, and uppercase Roman
  letters, respectively. So the mini template |(i)| would yield the
  items (i), (ii), (iii), (iv), and so on. The mini template |A.)|
  would yield the items A.), B.), C.), D.) and so on. For more details
  on the possible mini templates, see the documentation of the
  |enumerate| package. Note that there is also a template that governs
  the appearance of the mini template (for example, to change its
  color), see Section~\ref{section-template-enumerate}.
  
  \example
\begin{verbatim}
\begin{enumerate}
\item This is important.
\item This is also important.
\end{enumerate}

\begin{enumerate}[(i)]
\item First Roman point.
\item Second Roman point.
\end{enumerate}

\begin{enumerate}[<+->][(i)]
\item First Roman point.
\item Second Roman point, uncovered on second slide.
\end{enumerate}
\end{verbatim}

  \articlenote
  To use the \meta{mini template}, you have to include the package
  |enumerate|.  

  \lyxnote
  The same constraints as for |itemize| apply.
\end{environment}


\begin{environment}{{description}\opt{|[<|\meta{default overlay specification}|>]|}\oarg{long text}} 
  Like |itemize|, but used to display an list that explains or defines
  labels. The width of \meta{long text} is used to set the indent. Normally,
  you choose the widest label in the description and copy it here.

  As for |enumerate|, the \meta{default overlay specification} is
  detected by an opening~|<|. The effect is the same as for
  |enumerate| and |itemize|.
  \example
\begin{verbatim}
\begin{description}
\item[Lion] King of the savanna.
\item[Tiger] King of the jungle.
\end{description}

\begin{description}[longest label]
\item<1->[short] Some text.
\item<2->[longest label] Some text.
\item<3->[long label] Some text.
\end{description}
\end{verbatim}

  \example The following has the same effect as the previous example:
\begin{verbatim}
\begin{description}[<+->][longest label]
\item[short] Some text.
\item[longest label] Some text.
\item[long label] Some text.
\end{description}
\end{verbatim}

  \lyxnote
  Since you cannot specify the optional argument in \LyX, if you wish
  to specify the width, you must use the command
  |\usedescriptionitemofwidthas|, which you must insert in \TeX-mode
  shortly before the environment.
\end{environment}


\begin{command}{\usedescriptionitemofwidthas\marg{long text}}
  This command overrides the default width of the
  description label by the width of \meta{long text} for the current
  \TeX\ group. You should only use this command if, for some reason or
  another, you cannot give the \meta{long text} as an argument to the
  |description| environment. This happens, for example, if you create a
  |description| environment in \LyX.

  \example
\begin{verbatim}
\usedescriptionitemofwidthas{longest label}
\begin{description}
\item<1->[short] Some text.
\item<2->[longest label] Some text.
\item<3->[long label] Some text.
\end{description}
\end{verbatim}
\end{command}



\subsubsection{Hilighting}

The \beamer\ class predefines a commands and environments for
hlighting text. Using these commands makes is easy to change the
appearance of a document by changing the theme. 

\begin{command}{\alert\sarg{overlay specification}\marg{hilighted text}}
  The given text is hilighted, typically be coloring the text red. If
  the \meta{overlay specification} is present, the command only has an
  effect on the specified slides.
  \example |This is \alert{important}.|

  \articlenote
  Alerted text is typeset as emphasized text. This can be changed by
  specifying another template.

  \lyxnote
  You need to use \TeX-mode to insert this command.
\end{command}

\begin{environment}{{alertenv}\sarg{overlay specification}}
  Environment version of the |\alert| command.
\end{environment}



\begin{command}{\structure\sarg{overlay specification}\marg{text}}
  The given text is marked as part of the structure, typically by
  coloring it in the |structure| color. If the \meta{overlay
    specification} is present, the command only has an effect on the
  specified slides.
  \example|\structure{Paragraph Heading.}|

  \articlenote
  Structure text is typeset as bold text. This can be changed by
  specifying another template.

  \lyxnote
  You need to use \TeX-mode to insert this command.
\end{command}

\begin{environment}{{structureenv}\sarg{overlay specification}}
  Environment version of the |\structure| command.
\end{environment}




\subsubsection{Block Environments}
\label{predefined}

The \beamer\ class predefines an environment for typesetting a
``block'' of text that has a heading. The appearence of the block is
governed by a template.

\begin{environment}{{block}\sarg{action specification}\marg{block
      title}\sarg{action specification}}
  Only one \meta{action specification} may be given.
  Inserts a block, like a definition or a theorem, with the title
  \meta{block title}. If the \meta{action specification} is present,
  the given actions are taken on the specified slides, see
  Section~\ref{section-action-specifications}. In the example, the 
  definition is shown only from slide 3 onward.
  \example
\begin{verbatim}
  \begin{block}<3->{Definition}
    A \alert{set} consists of elements.
  \end{block}
\end{verbatim}

  \articlenote
  The block name is typeset in bold.

  \lyxnote
  The argument of the block must (currently) be given in
  \TeX-mode. More precisely, there must be an opening brace in
  \TeX-mode and a closing brace in \TeX-mode around it. The text
  in between can also be typeset using \LyX. I hope to get rid of this
  some day.
\end{environment}


\begin{environment}{{alertblock}\sarg{action specification}\marg{block
title}\sarg{action specification}} 
  Inserts a block whose title is hilighted. Behaves like the |block|
  environment otherwise.
  \example
\begin{verbatim}
  \begin{alertblock}{Wrong Theorem}
    $1=2$.
  \end{alertblock}
\end{verbatim}

  \articlenote
  The block name is typeset in bold and is emphasized.

  \lyxnote
  Same applies as for |block|.
\end{environment}

\begin{environment}{{exampleblock}\sarg{action
specification}\marg{block title}\sarg{overlay specification}} 
  Inserts a block that is supposed to be an example. Behaves like the
  |block| environment otherwise.
  
  \example In the following example, the block is completely
  suppressed on the first slide (it does not even occupy any space).
\begin{verbatim}
  \begin{exampleblock}{Example}<only@2->
    The set $\{1,2,3,5\}$ has four elements.
  \end{exampleblock}
\end{verbatim}

  \articlenote
  The block name is typeset in italics.

  \lyxnote
  Same applies as for |block|.
\end{environment}

\lyxnote
Overlay specifications must be given right at the beginning of the
environments and in \TeX-mode.



\subsubsection{Theorem Environments}
\label{section-theorems}

The \beamer\ class predefines several environments, like |theorem| or
|definition| or |proof|, that you can use to typeset things like,
well, theorems, definitions, or proofs. The complete list is the
following:  |theorem|, |corollary|, |definition|,
|definitions|, |fact|, |example|, and |examples|. The following German
block environments are also predefined: |Problem|, |Loesung|,
|Definition|, |Satz|, |Beweis|, |Folgerung|, |Lemma|, |Fakt|,
|Beispiel|, and |Beispiele|.

Here is a typical example on how to use them:

\begin{verbatim}
\frame
{
  \frametitle{A Theorem on Infinite Sets}

  \begin{theorem}<1->
    There exists an infinite set.
  \end{theorem}

  \begin{proof}<2->
    This follows from the axiom of infinity.
  \end{proof}

  \begin{example}<3->[Natural Numbers]
    The set of natural numbers is infinite.
  \end{example}
}
\end{verbatim}

In the following, only the English versions are discussed. The German
ones behave  analogously.

\begin{environment}{{theorem}\sarg{action
  specification}\oarg{additional text}\sarg{action specification}}
  Inserts a theorem. Only one \meta{action specification} may be
  given. If present, the \meta{additional text} is shown behind the
  word ``Theorem'' is rounded brackets (although this can be changed by
  the template).

  The appearance of the theorem is governed by
  templates, see Section~\ref{section-theorems-templates} for details
  on how to change these. Every theorem is put into a |block|
  environment, thus the templates for blocks also apply.

  The theorem style (a concept from |amsthm|) used for this
  environment is |plain|. In this style, the body of a theorem should
  be typeset in italics. The head of the theorem should be typeset in
  a bold font, but this is usually overruled by the templates.

  If the option |envcountsect| is given either as class option in one
  of the |presentation| modes or as an option to the package
  |beamerbasearticle| in |article| mode, then the numbering of the
  theorems is local to each section with the section number prefixing
  the theorem number; otherwise they are numbered consecutively
  throughout the presentation or article. I recommend using this
  option in |article| mode.

  By default, no theorem numbers are shown in the |presentation| modes.

  \example
\begin{verbatim}
\begin{theorem}[Kummer, 1992]
  If $\#^_A^n$ is $n$-enumerable, then $A$ is recursive.
\end{theorem}

\begin{theorem}<2->[Tantau, 2002]
  If $\#_A^2$ is $2$-fa-enumerable, then $A$ is regular.
\end{theorem}
\end{verbatim}

  \lyxnote
  Is present, the optional argument and the action specification must
  be given in \TeX\ mode at the beginning of the environment.
\end{environment}

The environments \declare{|corollary|}, \declare{|fact|}, and
\declare{|lemma|} behave exactly the same way.

\begin{classoption}{{envcountsect}}
  Causes theorems, definitions, and so on to be numbered locally to
  each section. Thus, the first theorem of the second section would be
  Theorem~2.1 (assuming that there are no definitions, lemmas, or
  corollaries earlier in the section).
\end{classoption}

\begin{environment}{{defintion}\sarg{action
      specification}\oarg{additional text}\sarg{action specification}}
  Behaves like the |theorem| environment, except that the theorem
  style |definition| is used. In this style, the body of a theorem is
  typeset in an upright font.
\end{environment}

The environment \declare{|definitions|} behaves exactly the same way.

\begin{environment}{{example}\sarg{action
      specification}\oarg{additional text}\sarg{action specification}}
  Behaves like the |theorem| environment, except that the theorem
  style |example| is used. A side-effect of using this theorem style
  is that the \meta{environment contents} is put in an |exampleblock|
  instead of a |block|.
\end{environment}

The environment \declare{|examples|} behaves exactly the same way.

Some remarks on numbered theorems:

\beamernote
The default template for typesetting theorems suppresses the theorem
number, even if this number is ``available'' for typesetting (which it
is by default in all predefined environments; but if you define your
own environment using |\newtheorem*| no number will be available). I would 
like to discourage using numbered theorems in a presentation. The
audience has no chance of remembering these numbers. \emph{Never} say
things like ``now, by Theorem~2.5 that I showed you earlier, we have \dots''
It would be much better to refer to, say, Kummer's
Theorem instead of Theroem~2.5. If Theorem~2.5 is some obscure theorem
that does not have its own name (like Kummer's Theorem or Main Theorem
or Second Main Theorem or Key Lemma), then the audience will have
forgotten about it anyway by the time you refer to it again.

In my opinion, the only situtation in which numbered theorems make
sense in a presentation is in a lecture, in which the students can read
lecture notes in parallel to the lecture where the theorems are
numbered in exactly the same way. 

\articlenote
In |article| mode, theorems are automatically numbered. By specifying
the option |envcountsect|, theorems will be numbered locally to each
section, which is usually a good idea, except for very short
articles.

The predefined environments number everything consecutively. Thus if
there are one theorem, one lemma, and one definition, you would have
Theorem~1, Lemma~2, and Definition~3. Some people prefer all three to
be numbered~1. I would \emph{strongly} like to discourage this. The
problem is that this makes it virtually impossible to find anything
since Theorem~2 might come after Definition~10 or the other way
round. Papers and, worse, books that have a Theorem~1 and a
Definition~1 are a pain. Do not inflict pain on other people.

\begin{environment}{{proof}\sarg{action specification}\oarg{proof
name}\sarg{action specification}}
  Typesets a proof. If the optional \meta{proof name} is given, it
  completely replaces the word ``Proof.'' This is different from
  normal theorems, where the optional argument is shown in brackets.

  At the end of the theorem, a |\qed| symbol is shown, except if you
  say |\qedhere| earlier in the proof (this is exactly as in
  |amsthm|). The default |\qed| symbol is a filled rectangle. To
  completely suppress the symbol, write |\def\qedsymbol{}| in 
  your preamble. To get an open rectangle, say
  |\def\qedsymbol{\openbox}|. Adding |\color{beamerstructure}| might
  also be a good idea.

  If you use |babel| and a different language, the text ``Proof'' is
  replaced by whatever is appropriate in the selected language.

  \example
\begin{verbatim}
\begin{proof}<2->[Sketch of proof]
  Suppose ...
\end{proof}
\end{verbatim}
\end{environment}

You can define new environments using the following command:

\begin{command}{\newtheorem\opt{|*|}\marg{environment name}\oarg{numbered same
      as}\marg{head text}\oarg{number within}}
  This command is used exactly the same way as in the |amsthm| package
  (as a matter of fact, it is the command from that package), see its
  documentation. The only difference is that environments declared using
  this command are overlay-specification-aware in \beamer\ and that,
  when typeset, are typeset according to \beamer's templates.

  \articlenote
  Environments declared using this command are also
  overlay-specification-aware in |article| mode.

  \example |\newtheorem{observation}[theorem]{Observation}|
\end{command}

You can also use |amsthm|'s command |\newtheoremstyle| to define new
theorem styles. Note that the default template for theorems will
ignore any head font setting, but will honor the body font setting.

If you wish to define the environments like |theorem| differently (for
example, have it numbered within each subsection), you can use the
following class option to disable the definition of the predefined
environments: 

\begin{classoption}{{notheorems}}
  Switches off the definition of default blocks like |theorem|, but
  still loads |amsthm| and makes theorems  
  overlay-specificiation-aware.
\end{classoption}

The option is also available as a package option for
|beamerbasearticle| and has the same effect.

\articlenote
In the |article| version, the package |amsthm| sometimes clashes with
the document class. In this case you can use the following option,
which is once more available as a class option for \beamer\ and as a
package option for |beamerbasearticle|, to switch off the loading of
|amsthm| altogether. 

\begin{classoption}{{noamsthm}}
  Does not load |amsthm| and also not |amsmath|. Environments like
  |theorem| or |proof| will not be available.
\end{classoption}




\subsubsection{Framed Text}

In order to draw a frame (a rectangle) around some text, you can use
\LaTeX s standard command |\fbox|. More frame types are offered by the
package |fancybox|, which defines the following commands:
|\shadowbox|, |\doublebox|, |\ovalbox|, and |\Ovalbox|. Please consult
the \LaTeX\ Companion for details on how to use these commands.

The \beamer\ class also defines an environment for creating boxes:

\begin{environment}{{beamerboxesrounded}\oarg{options}\marg{head}}
  The text inside the environment is framed by a rectangular area with
  rounded corners. The background of the rectangular area is filled
  with a certain color, which depends on the current color scheme (see
  below). If the \meta{head} is not empty, \meta{head} is drawn in the
  upper part of the box in a different color, which also depends on
  the scheme. The following options can be given:
  \begin{itemize}
  \item \declare{|scheme=|\meta{name}} causes the color scheme \meta{name} to be
    used. A color scheme must previously be defined using the command
    |\beamerboxesdeclarecolorscheme|.
  \item \declare{|width=|\meta{dimension}} causes the width of the text inside
    the box to be the specified \meta{dimension}. By default, the
    |\textwidth| is used. Note that the box will protrude 4pt to the
    left and right.
  \item \declare{|shadow=|\meta{true or false}}. If set to |true|, a shadow will
    be drawn.    
  \end{itemize}
  A color scheme dictates the background colors used in the head part
  and in the body of the box. If no \meta{head} is given, the head
  part is completely suppressed.
  \example
\begin{verbatim}
\begin{beamerboxesrounded}[scheme=alert,shadow=true]{Theorem}
  $A = B$.
\end{beamerboxesrounded}
\end{verbatim}

  \articlenote
  This environment is not available in |article| mode.
\end{environment}

\begin{command}{\beamerboxesdeclarecolorscheme\marg{scheme
      name}\marg{head color}\marg{body color}}
  Declares a color scheme for later use in a |beamerboxesrounded|
  environment.
  \example |\beamerboxesdeclarecolorscheme{alert}{red}{red!15!averagebackgroundcolor}|

  \articlenote
  This environment is not available in |article| mode.
\end{command}



\subsubsection{Figures and Tables}

You can use the standard \LaTeX\ environments |figure| and
|table| much the same way you would normally use them. However,
any placement specification will be ignored. Figures and tables are
immediately inserted where the environments start. If there are too
many of them to fit on the frame, you must manually split them among
additional frames.

\example
\begin{verbatim}
\frame{
  \begin{figure}
    \pgfuseimage{myfigure}
    \caption{This caption is placed below the figure.}
  \end{figure}

  \begin{figure}
    \caption{This caption is placed above the figure.}
    \pgfuseimage{myotherfigure}
  \end{figure}
}
\end{verbatim}

You can adjust how the figure and table captions are typeset by
changing the corresponding template, see
Section~\ref{section-template-caption}.





\subsubsection{Splitting a Frame into Multiple Columns}

The \beamer\ class offers several commands and environments for
splitting (perhaps only part of) a frame into multiple columns. These
commands have nothing to do with \LaTeX's commands for creating
columns. Columns are especially useful for placing a graphic next to a
description/explanation.

The main environment for creating columns is called |columns|. Inside
this environment, you can either place several |column| environments,
each of which creates a new column, or use the |\column| command to
create new columns.

\begin{environment}{{columns}\oarg{options}}
  A multi-column area. Inside the environment you should place only
  |column| environments or |\column| commands (see below). The
  following \meta{options} may be given: 
  \begin{itemize}
  \item
    \declare{|b|} will cause the bottom lines of the columns to be
    vertically aligned.
  \item
    \declare{|c|} will cause the columns to be centered vertically
    relative to each other. Default, unless the global option
    |slidestop| is used. 
  \item
    \declare{|onlytextwidth|} is the same as |totalwidth=\textwidth|.
  \item
    \declare{|t|} will cause the first lines of the columns to be
    aligned. Default if global option |slidestop| is used.
  \item
    \declare{|totalwidth=|\meta{width}} will cause the columns to occupy
    not the whole page width, but only \meta{width}, all told.
  \end{itemize}
    
  \example
\begin{verbatim}
\begin{columns}[t]
  \begin{column}{5cm}
    Two\\lines.
  \end{column}
  \begin{column}{5cm}
    One line (but aligned).
  \end{column}
\end{columns}
\end{verbatim}
  
  \example
\begin{verbatim}
\begin{columns}[t]
  \column{5cm}
    Two\\lines.

  \column{5cm}
    One line (but aligned).
\end{columns}
\end{verbatim}

  \articlenote
  This environment is ignored in |article| mode.
  
  \lyxnote
  Use ``Columns'' or ``ColumnsTopAligned'' to create a |columns|
  environment. To pass options, insert then in \TeX-mode right at the
  beginning of the environment in square brackets.
\end{environment}

To create a column, you can either use the |column| environment or the
|\column| command. 

\begin{environment}{{column}\oarg{placement}\marg{column width}}
  Creates a single column of width \meta{column width}. The vertical
  placement of the enclosing |columns| environment can be overruled by
  specifying a specific \meta{placement} (|t| for top, |c| for
  centered, and |b| for bottom). 

  \example The following code has the same effect as the above examples:
\begin{verbatim}
\begin{columns}
  \begin{column}[t]{5cm}
    Two\\lines.
  \end{column}
  \begin{column}[t]{5cm}
    One line (but aligned).
  \end{column}
\end{columns}
\end{verbatim}
  \articlenote
  This command is ignored in |article| mode.

  \lyxnote
  The ``Column'' styles insert the command version, see below.
\end{environment}

\begin{command}{{\column}\oarg{placement}\marg{column width}}
  Starts a single column. The parameters and options are the same as
  for the |column| environment. The column automatically ends with the
  next occurrence of |\column| or of a |column| environment or of the
  end of the current |columns| environment.

  \example 
\begin{verbatim}
\begin{columns}
  \column[t]{5cm}
    Two\\lines.
  \column[t]{5cm}
    One line (but aligned).
\end{columns}
\end{verbatim}
  \articlenote
  This command is ignored in |article| mode.

  \lyxnote
  In a ``Column'' style, the width of the column must be given as
  normal text, not in \TeX-mode.
\end{command}



\subsubsection{Positioning Text and Graphics Absolutely}

Normally, \beamer\ uses \TeX's normal typesetting mechanism to
position text and graphics on the page. In certain situation you may
instead wish a certain text or graphic to appear at a
page position that is specified \emph{absolutely}. This means that the
position is specified relative to the upper left corner of the slide.

The package |textpos| provides several commands for positioning text
absolutely and it works together with \beamer. When using this
package, you will typically have to specify the options |overlay| and
perhaps |absolute|. For details on how to use the package, please see
its documentation.




\subsubsection{Verse, Quotations, Quotes}

\LaTeX\ defines three environments for typesetting quotations and
verses: |verse|, |quotation|, and |quote|. These environments are also
available in the \beamer\ class, where they are
overlay-specification-aware. If an overlay specification is given, the
verse or quotation is shown only on the specified slides and is
covered otherwise. The difference between a |quotation| and a |quote|
is that the first has paragraph indentation, whereas the second does
not. 

Unlike the standard \LaTeX\ environments, in \beamer\ these
environments do not only change the left and right margins, but also
the font: A verse is typeset using an italic serif font, quotations
and quotes are typeset using an italic font (whether serif or
sans-serif depends on the standard document font). To change this, you
can adjust the templates for these environments. 


\subsubsection{Footnotes}

First a word of warning: Using footnote is usually not a good
idea. They disrupt the flow of reading.

You can use the usual |\footnote| command. It has been augmented to
take by an additional option, for placing footnotes at the frame
bottom instead of at the bottom of the current minipage.

\begin{command}{\footnote\oarg{options}\marg{text}}
  Inserts a footnote into the current frame. Footnotes will always be
  shown at the bottom of the current frame; they will never be
  ``moved'' to other frames. As usual, one can give a number as
  \meta{options}, which will cause the footnote to use that
  number. The \beamer\ class adds one additional option:
  \begin{itemize}
  \item \declare{|frame|} causes the footnote to be shown at the
    bottom of the frame. This is normally the default behavior anyway,
    but in minipages and certain blocks it makes a difference. In a
    minipage, the footnote is usually shown as part of the minipage
    rather than as part of the frame.
  \end{itemize}

  \example |\footnote{On a fast machine.}|
  \example |\footnote[frame,2]{Not proved.}|
\end{command}

You can change the way footnotes are typeset by changing the footnote
templates, see Section~\ref{section-templates-footnotes}



\section{Graphics, Colors, Animations, and Special Effects}

\subsection{Graphics}
\label{section-graphics}

Graphics often convey concepts or ideas much more efficiently than
text: A picture can say more than a thousand words. (Although,
sometimes a word can say more than a thousand pictures.) In the
following, the advantages and disadvantages of different possible ways
of creating graphics for beamer presentations are discussed.



\subsubsection{Including External Graphic Files}

One way of creating graphics for a presentation is to  use an 
external program, like |xfig| or the Gimp. These programs
have an option to \emph{export} graphic files in a format that can
then be inserted into the presentation.

The main advantage is:
\begin{itemize}
\item
  You can use a powerful program to create a high-quality graphic.
\end{itemize}

The main disadvantages are:
\begin{itemize}
\item
  You  have to worry about many files. Typically there are at least
  two for each presentation, namely the program's graphic data file and the
  exported graphic file in a format that can be read by \TeX.
\item
  Changing the graphic using the program does not automatically change
  the graphic in the presentation. Rather, you must reexport the
  graphic and rerun \LaTeX.
\item
  It may be difficult to get the line width, fonts, and font sizes
  right.
\item
  Creating formulas as part of graphics is often difficult or
  impossible.
\end{itemize}

You can use all the standard \LaTeX\ commands for inserting graphics,
like |\includegraphics| (be sure to use the package
|graphics|). Also, the |pgf| package offers commands for including
graphics. Either will work fine in most situations, so choose
whichever you like. Like |\pgfdeclareimage|,
|\includegraphics| also includes an image only once in a |.pdf| file,
even if it used several times (as a matter of fact, the |graphics|
package is even a bit smarter about this than |pgf|). However,
currently only |pgf| offers the ability to include images that are
partly transparent.

There are few things to note about the format of graphics you can
include:
\begin{itemize}
\item
  When using |latex| and |dvips|, you can only include external
  graphic files ending with the extension |.eps| (Encapsulated
  PostScript). This is true both for the normal |graphics| package and
  for |pgf|. When using |pgf|, do \emph{not} add the extension
  |.eps|. When using |graphics|, do add the extension. If your graphic
  file has a different format (like a |.jpg| file), you must first
  convert it to an |.eps| file using some conversion program.
\item
  When using |pdflatex|, you can only include external graphic files
  ending with one of the extensions |.pdf|, |.jpg|, |.jpeg|, or
  |.png|. As before, do not add these extension when using |pgf|, but
  do add them when using |graphics|. If your graphic file has a
  different format, you have to convert it.
\end{itemize}

Note that, most frustratingly, there is no graphic format that can be
read by \emph{both} |pdflatex| \emph{and} |dvips|.


\lyxnote
You can use the usual ``Insert Graphic'' command to insert a graphic.

The commands |\includegraphics|, |\pgfuseimage|, and |\pgfimage| are
overlay-specification-aware in \beamer. If the overlay specification
does not apply, the command has no effect. This is useful for creating
a simple animation where each picture of the animation resides in a
different file:

\begin{verbatim}
\frame{
  \includegraphics<1>[height=2cm]{step1.pdf}
  \includegraphics<2>[height=2cm]{step2.pdf}
  \includegraphics<3>[height=2cm]{step3.pdf}
}
\end{verbatim}



\subsubsection{Inlining Graphic Commands}

A different way of creating graphics is to insert
graphic drawing commands directly into your \LaTeX\ file. There are numerous
packages that help you do this. They have various degrees of
sophistication. Inlining graphics suffers from none of the
disadvantages mentioned above for including external graphic files,
but the main disadvantage is that it is often hard to use
these packages. In some sense, you ``program'' your graphics, which 
requires a bit of practice.

When choosing a graphic package, there are a few things to keep in
mind:
\begin{itemize}
\item
  Many packages produce poor quality graphics. This is especially true
  of the standard |picture| environment of \LaTeX.
\item
  Powerful packages that produce high-quality graphics often do not
  work together with |pdflatex|.
\item
  The most powerful and easiest-to-use package around, namely
  |pstricks|, does not work together with |pdflatex| and
  this is a fundamental problem. Due to the fundamental differences
  between \textsc{pdf} and PostScript, it is not possible to write a
  ``|pdflatex| back-end for |pstricks|.''
\end{itemize}

A solution to the above problem (though not necessarily the best) is
to use the \textsc{pgf} package. It produces high-quality graphics and
works together with |pdflatex|, but also with normal
|latex|. It is not as powerful as |pstricks| (as pointed
out above, this is because of rather fundamental reasons) and not as
easy to use, but it should be sufficient in most cases.

\lyxnote
Inlined graphics must currently by inserted in a large \TeX-mode
box. This is not very convenient.




\subsection{Color Management}

The color management of the \beamer\ class relies on the packages
|xcolor|, which is a stand-alone extension of the |color| package, 
and on |xxcolor|, which in turn is an extension of
|xcolor| and is part of \pgf. Hopefully, in the future |xxcolor| and
|xcolor| will merge into one package and perhaps they will
someday also merge together with |color|.

Since the |color| package and the |xcolor| package are loaded already
by the \beamer\ class, in order to pass options to these classes, you
need to use the class options |color={|\meta{options for color}|}| or
|xcolor={|\meta{options for xcolor}|}| to pass options to these
classes.

\subsubsection{Colors of Main Text Elements}

By default, the following colors are used in a presentation:
\begin{itemize}
\item
  Normal text is typeset in |black|.
\item
  All ``structural'' elements, like titles, navigation bars, block
  titles, and so on, are typeset using the color
  |beamerstructure|. By default, this color is bluish. Using one of
  the class options |red|, |blackandwhite|, or |brown|
  changes this. You can also change this color simply be redefining
  the color |beamerstructure|.
\item
  All ``alert'' text is typeset by setting the default color and the
  structure color to 85\% of red. To change this, you can either
  redefine the color |beameralert|, or you can change the whole alert
  template. 
\item
  All examples are typeset using 50\% of green. To change this, you
  must change the example templates.
\end{itemize}

\begin{classoption}{brown}
  Changes the main color of the navigation and title bars
  to a brownish color.
\end{classoption}

\begin{classoption}{red}
  Changes the main color of the navigation and title bars
  to a reddish color.
\end{classoption}

\begin{classoption}{blackandwhite}
  Changes the main color of the navigation and title bars
  to monochrome.
\end{classoption}



\subsubsection{Average Background Color}

\label{section-average}

In some situations, for example when creating a transparency effect,
it is useful to have access to the current background
color. One can then, for example, mix a color with the background
color to create a ``transparent'' color.

Unfortunately, it is not always clear what exactly the background
color is. If the background is a shading or a picture, different parts
of a slide have different background colors. In these cases, one can
at least try to mix-in an \emph{average} background color, called
|averagebackgroundcolor|. If a shading or picture is not too
colorful, this works fairly well.

To specify the average background color, use the following command:

\begin{command}{\beamersetaveragebackground\marg{color expression}}
  Installs the given color as the average background color. See the
  |xcolor| package for the syntax of color expressions.
  \example |\beamersetaveragebackground{red!10}|
\end{command}

If you use the commands from Section~\ref{section-backgrounds} for
installing a background coloring, the average background color is
computed automatically for you. When you directly use the command
|\usebackgroundtemplate|, you should must set the average
background color afterward.




\subsubsection{Transparency Effects}
\label{section-transparent}

By default, \emph{covered} items are not shown during a
presentation. Thus if you write |\uncover<2>{Text.}|, the text
is not shown on any but the second slide. On the other slide, the text
is not simply printed using the background color -- it is not shown at
all. This effect is most useful if your background does not have a
uniform color.

Sometimes however, you might prefer that covered items are not
completely covered. Rather, you would like them to be shown already in
a very dim or shaded way. This allows your audience to get a feeling
for what is yet to come, without getting distracted by it. Also, you
might wish text that is covered ``once more'' still to be visible to
some degree.

Ideally, there would be an option to make covered text
``transparent.'' This would mean that when covered text is shown, it
would instead be mixed with the background behind it. Unfortunately,
|pgf| does not support real transparency yet.
Nevertheless, one can come ``quite close'' to transparent text using
the special command
\begin{verbatim}
\beamersetuncovermixins{#1}{#2}
\end{verbatim}
This commands allows you to specify in a quite general way how a
covered item should be rendered. You can even specify different ways
of rendering the item depending on how long it will take before this
item is shown or for how long it has already been covered once
more. The transparency effect will automatically apply to all colors,
\emph{except} for the colors in images and shadings. For images and
shadings there is a workaround, see the documentation of the
\pgf\ package. 

As a convenience, several commands install a predefined uncovering
behavior.

\begin{command}{\beamertemplatetransparentcovered}
  Makes all covered text quite transparent. 
\end{command}

\begin{command}{\beamertemplatetransparentcoveredmedium}
  Makes all covered text even more transparent. 
\end{command}

\begin{command}{\beamertemplatetransparentcoveredhigh}
  Makes all covered text highly transparent. 
\end{command}

\begin{command}{\beamertemplatetransparentcoveredhigh}
  Makes all covered text extremely transparent, but not totally. 
\end{command}

\begin{command}{\beamertemplatetransparentcovereddynamic}
  Makes all covered text quite transparent, but is a dynamic way. The
  longer it will take till the text is uncovered, the stronger the
  transparency. 
\end{command}

\begin{command}{\beamertemplatetransparentcovereddynamicmedium}
  Like the previous command, only it the ``range'' of dynamics is
  smaller. 
\end{command}

\begin{command}{\beamersetuncovermixins\marg{not yet list}%
    \marg{once more list}}
  The \meta{not yet list} specifies  how to render covered items that
  have not  yet been uncovered. The \meta{once more list} specifies
  how to render covered items that have once more been covered. 
  If you leave one of the specifications empty, the corresponding
  covered items are completely covered, that is, they are invisible.
  \example
\begin{verbatim}
\beamersetuncovermixins
  {\opaqueness<1>{15}\opaqueness<2>{10}\opaqueness<3>{5}\opaqueness<4->{2}}
  {\opaqueness<1->{15}}
\end{verbatim}
  The \meta{not yet list} and the  \meta{once more list} can
  contain any number of |\opaqueness| commands.
\end{command}

%\begin{command}{\mixinon\ssarg{overlay specification}\marg{mix-in specification}}
%  The \meta{overlay specification} specifies on which slides the
%  \meta{mix-in specification} should be applied to all colors. Unlike
%  other overlay specifications, this \meta{overlay specification} is a
%  ``relative'' overlay specification. For example, the specification
%  ``3'' here means ``things that will be uncovered three slides
%  ahead,'' respectively ``things that have once more been covered for
%  three slides.'' More precisely, if an item is uncovered for more
%  than one slide and then covered once more, only the ``first moment
%  of uncovering'' is used for the calculation of how long the item has
%  been covered once more.

%  \emph{Mix-in} specifications are a concept introduced by the
%  |xcolor| package. The \meta{mix-in specification} specifies how colors
%  should be altered by adding another color to them. The specification
%  consists of two parts, separated by an exclamation mark. The first
%  part is a number between 0 and 100, where 0 means ``do not mix in the
%  text color at all'' and 100 means ``use only the text color''. The
%  second part is the color that should be mixed in. This second part may
%  be omitted (along with the exclamation mark), in which case ``white''
%  is used as mix-in color. Any color that has been defined using the
%  |\definecolor| command is permissible as a mix-in color.

%  The mix-in specifications is added to the \pgf\ alternate
%  extension for shadings and images (see the \pgf\
%  documentation). Nested uses of mix-in accumulate. 
%  \example
%\begin{verbatim}
%\beamersetuncovermixins{\mixinon<1>{15!blue}{\mixinon<1->{15!white}}
%\pgfdeclareimage{book}{book}
%\pgfdeclareimage{book.!15!averagebackgroundcolor}{filenameforbooknearlyblue}
%\pgfdeclareimage{book.!15!white}{filenameforbooknearlywhite}
%\end{verbatim}
%  For all items that become uncovered on the next slide or that have
%  just been covered on the previous slide (depending on whether this
% command is used as part of the first or second parameter of the command
%  |\beamersetuncovermixins|), use only 15\% of the actual color and
%  85\% of the average background color.
%\end{command}

\begin{command}{\opaqueness\ssarg{overlay
      specification}\marg{percentage of opaqueness}}
  The \meta{overlay specification} specifies on which slides covered
  text should have which \meta{percentage of opaqueness}. Unlike
  other overlay specifications, this \meta{overlay specification} is a
  ``relative'' overlay specification. For example, the specification
  ``3'' here means ``things that will be uncovered three slides
  ahead,'' respectively ``things that have once more been covered for
  three slides.'' More precisely, if an item is uncovered for more
  than one slide and then covered once more, only the ``first moment
  of uncovering'' is used for the calculation of how long the item has
  been covered once more.

  An opaqueness of 100 is fully opaque and 0 is fully
  transparent. Currently, since real transparency is not yet
  implemented, this command causes all colors to get a mixing of
  \meta{percentage of opaqueness} of the current
  |averagebackgroundcolor|. At some future point this command might
  result in real transparency.

  The alternate \pgf\ extension used inside an opaque area is
  \meta{percentage of opaqueness}|opaque|. In case of nested calls,
  only the innermost opaqueness specification is used. 
  \example
\begin{verbatim}
\beamersetuncovermixins{\opaqueness<1->{15}{\opaqueness<1->{15}}
\pgfdeclareimage{book}{book}
\pgfdeclareimage{book.15opaque}{filenameforbooknearlytransparent}
\end{verbatim}
  Makes everything that is uncovered in two slides only 15 percent
  opaque. 
\end{command}



\subsection{Animations}

A word of warning first: Animations can be very distracting. No matter
how cute a rotating, flying theorem seems to look and no matter
how badly you feel your audience needs some action to keep it happy,
most people in the audience will typically feel you are making fun of
them. 

\subsubsection{Using an External Viewer}

If you have created an animation using some external
program (like a renderer), you can use the capabilities of the
presentation program (like the Acrobat Reader) to show the
animation. Unfortunately, currently there is no portable way of doing
this and even the Acrobat Reader does not support this feature on all
platforms.


\subsubsection{Animations Created by Showing Slides in Rapid Succession}

You can create an animation in a portable way by using the
overlay commands of the \beamer\ package to create a series of slides
that, when shown in rapid succession, present an animation. This is a
flexible approach, but such animations will typically be rather static
since it will take some time to advance from one slide to the
next. This approach is mostly useful for animations where you want
to explain each ``picture'' of the animation.
When you advance slides ``by hand,'' that is, by pressing a forward
button, it typically takes at least a second for the next slide to
show.

More ``lively'' animations can be created by relying on a capability
of the viewer program. Some programs support
showing slides only for a certain number of seconds during a
presentation (for the Acrobat Reader this works only in full-screen
mode). By setting the number of seconds to zero, you can create a
rapid succession of slides.

To facilitate the creation of animations using this feature, the
following commands can be used: |\animate| and |\animatevalue|.

\begin{command}{\animate\ssarg{overlay specification}}
  The slides specified by \meta{overlay specification} will be shown
  only as shortly as  possible.
\example
\begin{verbatim}
\frame{
  \frametitle{A Five Slide Animation}
  \animate<2-4>

  The first slide is shown normally. When the second slide is shown
  (presumably after pressing a forward key), the second, third, and
  fourth slides ``flash by.'' At the end, the content of the fifth
  slide is shown.

  ... code for creating an animation with five slides ...
}
\end{verbatim}

  \articlenote
  This command is ignored in |article| mode.
\end{command}

\begin{command}{\animatevalue|<|\meta{start slide}|-|\meta{end slide}|>|%
    \marg{name}\marg{start value}\marg{end value}}
  The \meta{name} must be the name of a counter or a dimension.
  It will be varied between two values. For the slides in the
  specified range, the counter or dimension is set to an interpolated
  value that depends on the current slide number. On slides before the
  \meta{start slide}, the counter or dimension is set to \meta{start
    value}; on the slides after the \meta{end slide} it is set to
  \meta{end value}.
  \example
\begin{verbatim}
\newcount\opaqueness
\frame{
  \animate<2-10>
  \animatevalue<1-10>{\opaqueness}{100}{0}
  \begin{colormixin}{\the\opaqueness!averagebackgroundcolor}
    \frametitle{Fadeout Frame}

    This text (and all other frame content) will fade out when the
    second slide is shown. This even works with
    {\color{green!90!black}colored} \alert{text}.
  \end{colormixin}
}

\newcount\opaqueness
\newdimen\offset
\frame{
  \frametitle{Flying Theorems (You Really Shouldn't!)}

  \animate<2-14>

  \animatevalue<1-15>{\opaqueness}{100}{0}
  \animatevalue<1-15>{\offset}{0cm}{-5cm}
  \begin{colormixin}{\the\opaqueness!averagebackgroundcolor}
  \hskip\offset
    \begin{minipage}{\textwidth}
      \begin{theorem}
        This theorem flies out.
      \end{theorem}
    \end{minipage}
  \end{colormixin}

  \animatevalue<1-15>{\opaqueness}{0}{100}
  \animatevalue<1-15>{\offset}{-5cm}{0cm}
  \begin{colormixin}{\the\opaqueness!averagebackgroundcolor}
  \hskip\offset
    \begin{minipage}{\textwidth}
      \begin{theorem}
        This theorem flies in.
      \end{theorem}
    \end{minipage}
  \end{colormixin}
}
\end{verbatim}

  \articlenote
  This command is ignored in |article| mode.
\end{command}



\subsection{Slide Transitions}

\textsc{pdf} in general, and the Acrobat Reader in particular, offer a
standardized way of defining \emph{slide transitions}. Such a
transition is a visual effect that is used to show the slide. For
example, instead of just showing the slide immediately, whatever was
shown before might slowly ``dissolve'' and be replaced by the slide's
content.

Slide transitions should be used with great care. Most of the time,
they only distract. However, they can be useful in some situations:
For example, you might show a young boy on a slide and might wish to
dissolve this slide into slide showing a grown man instead. In this
case, the dissolving gives the audience visual feedback that the young
boy ``slowly becomes'' the man.

There are a number of commands that can be used to specify what effect
should be used when the current slide is presented. Consider the
following example:

\begin{verbatim}
\frame{
  \pgfuseimage{youngboy}
}
\frame{
  \transdissolve
  \pgfuseimage{man}
}
\end{verbatim}
The command |\transdissolve| causes the slide of the
second frame to be shown in a ``dissolved way.'' Note that the
dissolving is a property of the second frame, not of the first one. We
could have placed the command anywhere on the frame.

The transition commands are overlay-specification-aware. We could
collapse the two frames into one frame like this:
\begin{verbatim}
\frame{
  \only<1>{\pgfuseimage{youngboy}}
  \only<2>{\pgfuseimage{man}}
  \transdissolve<2>
}
\end{verbatim}
This states that on the first slide the young boy should be shown, on
the second slide the old man should be shown, and when the second
slide is shown, it should be  shown in a ``dissolved way.''

In the following, the different commands for creating transitional
effects are listed. All of them take an optional argument that may
contain a list of \meta{key}|=|\meta{value} pairs. The following
options are possible:

\begin{itemize}
\item
  |duration=|\meta{seconds}. Specifies the number of \meta{seconds}
  the transition effect needs. Default is one second, but often a
  shorter one (like 0.2 seconds) is more appropriate. Viewer
  applications, especially Acrobat, may interpret this option in
  slightly strange ways.
\item
  |direction=|\meta{degree}. For ``directed'' effects, this option
  specifies the effect's direction. Allowed values are |0|, |90|,
  |180|, |270|, and for the glitter effect also |315|.
\end{itemize}

\articlenote
All of these commands are ignored in |article| mode.

\lyxnote
You must insert these commands using \TeX-mode.

\begin{command}{\transblindshorizontal\sarg{overlay specification}\oarg{options}}
  Show the slide as if horizontal blinds where pulled away.
  \example|\transblindshorizontal|
\end{command}
  
\begin{command}{\transblindsvertical\sarg{overlay specification}\oarg{options}}
  Show the slide as if vertical blinds where pulled away.
  \example|\transblindsvertical<2,3>|
\end{command}
  
\begin{command}{\transboxin\sarg{overlay specification}\oarg{options}}
  Show the slide by moving to the center from all four sides.
  \example|\transboxin<1>|
\end{command}
  
\begin{command}{\transboxout\sarg{overlay specification}\oarg{options}}
  Show the slide by showing more and more of a rectangular area that
  is centered on the slide center.
  \example|\transboxout|
\end{command}
 
\begin{command}{\transdissolve\sarg{overlay specification}\oarg{options}}
  Show the slide by slowly dissolving what was shown before.
  \example|\transdissolve[duration=0.2]|
\end{command}
  
\begin{command}{\transglitter\sarg{overlay specification}\oarg{options}}
  Show the slide with a glitter effect that sweeps in the specified
  direction.
  \example|\transglitter<2-3>[direction=90]|
\end{command}
  
\begin{command}{\transsplitverticalin\sarg{overlay specification}\oarg{options}}
  Show the slide by sweeping two vertical lines from the sides inward.
  \example|\transsplitverticalin|
\end{command}
  
\begin{command}{\transsplitverticalout\sarg{overlay specification}\oarg{options}}
  Show the slide by sweeping two vertical lines from the center outward.
  \example|\transsplitverticalout|
\end{command}
  
\begin{command}{\transsplithorizontalin\sarg{overlay specification}\oarg{options}}
  Show the slide by sweeping two horizontal lines from the sides inward.
  \example|\transsplithorizontalin|
\end{command}
  
\begin{command}{\transsplithorizontalout\sarg{overlay specification}\oarg{options}}
  Show the slide by sweeping two horizontal lines from the center outward.
  \example|\transsplithorizontalout|
\end{command}
 
\begin{command}{\transwipe\sarg{overlay specification}\oarg{options}}
  Show the slide by sweeping a single line in the specified direction,
  thereby ``wiping out'' the previous contents.
  \example|\transwipe[direction=90]|
\end{command}


You can also specify how \emph{long} a given slide should be shown,
using the following overlay-specification-aware command:

\begin{command}{\transduration\sarg{overlay specification}\marg{number of seconds}}
  In full screen mode, show the slide for \meta{number of seconds}.
  In zero is specified, the slide is shown as short as possible. This
  can be used to create interesting pseudo-animations.
  \example|\transduration<2>{1}|
\end{command}






\section{Managing Non-Presentation Versions and Material}

\label{section-modes}

The \beamer\ package offers different ways of creating special
versions of your talk and adding material that are not shown during
the presentation. You can create a \emph{handout} version of the
presentation that can be distributed to the audience. You can also
create a version that is more suitable for a presentation using an
overhead projector. You can add notes for yourself that help
you remember what to say for specific slides. Finally, you can have a
completely independent ``article'' version of your presentation 
coexist in your main file. All special versions are created by
specifying different class options and rerunning \TeX\ on the main
file. 






\subsection{Creating Handouts}

\label{handout}

A \emph{handout} is a version of a presentation in which the slides
are printed on paper and handed out to the audience before or after
the talk. (See Section~\ref{section-postscript} for how to place
numerous frames on one pages, which is very useful for handouts.)  For
the handout you typically want to produce as few slides as possible
per frame. In particular, you do not want to print a new slide for
each slide of a frame. Rather, only the ``last'' slide should be
printed.  

In order to create a handout, specify the class option
|handout|. If you do not specify anything else, this will cause
all overlay specifications to be suppressed. For most cases this will
create exactly the desired result.

\begin{classoption}{handout}
  Create a version that uses the |handout| overlay specifications.
\end{classoption}

In some cases, you may want a more complex behaviour. For example, if
you use many |\only| commands to draw an animation. In this case,
suppressing all overlay specifications is not such a good idea, since
this will cause all steps of the animation to be shown at the same
time. In some cases this is not desirable. Also, it might be desirable
to suppress some |\alert| commands that apply only to specific
slides in the handout.

For a fine-grained control of what is shown on a handout, you can use
\emph{mode specifications}. They specify which slides
 of a frame should be shown for a special version, for example for the
handout version. As explained in
Section~\ref{section-concept-overlays}, a mode specification is written
alongside the normal overlay specification inside the pointed
brackets. It is separated from the normal specification by a vertical
bar and a space. Here is an example:
\begin{verbatim}
  \only<1-3,5-9| handout:2-3,5>{Text}
\end{verbatim}
This specification says: ``Normally (in |beamer| mode), insert the
text on slides 1--3 and 5--9. For the handout version, insert the text
only on slides 2,~3, and~5.'' If no special mode specification is
given for handouts, the default is ``always.'' This causes the
desirable effect that if you do not specify anything, the overlay
specification is effectively suppressed for the handout.

An especially useful specification is the following:
\begin{verbatim}
  \only<3| handout:0>{Not shown on handout.}
\end{verbatim}
Since there is no zeroth slide, the text is not shown. Likewise,
\verb!\alert<3| handout:0>{Text}! will not alert the text on a
handout.

You can also use a mode specification for the overlay specification
of the |\frame| command as in the following example.
\begin{verbatim}
\frame<1-| handout:0>{Text...}
\end{verbatim}
This causes the frame to be suppressed in the handout version. Also,
you can restrict the presentation such that only specific slides of
the frame are shown on the handout:
\begin{verbatim}
\frame<1-| handout:4-5>{Text...}
\end{verbatim}

It is also possible to give only an alternate overlay
specification. For example, |\alert<handout:0>{...}| causes the
text to be always hilighted during the presentation, but never on the
handout version. Likewise, |\frame<handout:0>{...}| causes the
frame to be suppressed for the handout.

Finally, note that it is possible to give more than one alternate
overlay specification and in any order. For example, the following
specification states that the text should be inserted on the first
three slides in the presentation, in the first two slides of the
transparency version, and not at all in the handout.
\begin{verbatim}
  \only<trans:1-2| 1-3| handout:0>{Text}
\end{verbatim}

If you wish to give the same specification in all versions, you can do
so by specifying |all:| as the version. For example,
\begin{verbatim}
\frame<all:1-2>
{
  blah...
}
\end{verbatim}
ensures that the frame has two slides in all versions. 




\subsection{Creating Transparencies}

\label{trans}

The main aim of the \beamer\ class is to create presentations for
beamers. However, it is often useful to print transparencies as
backup, in case the hardware fails. A transparencies version of a talk
often has less slides than the main version, since it takes more time
to switch slides, but it may have more slides than the handout
version. For example, while in a handout an animation might be
condensed to a single slide, you might wish to print several slides
for the transparency version.

You can use the same mechanism as for creating handouts: Specify
|trans| as a class option and add alternate transparency
specifications for the |trans| version as needed. An elaborated
example of different overlay specifications for the presentation, the
handout, and the transparencies can be found in the file
|beamerexample1.tex|.

\begin{classoption}{trans}
  Create a version that uses the |trans| overlay
  specifications. 
\end{classoption}

When printing a presentation using Acrobat, make sure that the option
``expand small pages to paper size'' in the printer dialog is
enabled. This is necessary, because slides are only 128mm times 96mm.



\subsection{Adding Notes}

A \emph{note} is a small piece of paper that is intended as a reminder
to yourself of what you should say or should keep in 
mind when presenting a slide.



\subsubsection{Specifying Note Contents}

To add a note to a slide or a frame, you should use the |\note|
command. This command can be used both inside and outside frames, but
it has quite different behaviors then: Inside frames, |\note| commands
accumulate and append a single note page after the current slide;
outside frames each |\note| directly inserts a single note page with
the given parameter as contents. Using the |\note| command inside
frames is usually preferably over using them outside, since only
commands issued inside frames profit from the class option
|onlyslideswithnotes|, see below.

\lyxnote
In \LyX, only the inside-frame |\note| command with the option
|[item]| is available in the form of the NoteItem style. 

Inside a frame, the effect of |\note|\meta{text} is the following:
When you use it somewhere inside the frame on a specific slide, a note
page is created after the slide, containing the \meta{text}. Since you
can add an overlay specification to the |\note| command, you can
specify after which slide the note should be shown. If you use
multiple |\note| commands on one slide, they ``accumulate'' and are
all shown on the same note.

To make the accumulation of notes more convenient, you can use the
|\note| command with the option |[item]|. The notes added with this
option are accumulated in an |enumerate| list that follows any text
inserted using |\note|. 

The following example will produce one note page that follows the
second slide and has two entries. 

\begin{verbatim}
\frame{
  \begin{itemize}
  \item<1-> Eggs
  \item<2-> Plants
    \note[item]<2>{Tell joke about plants.}
    \note[item]<2>{Make it short.}
  \item<3-> Animals
  \end{itemize}
}
\end{verbatim}


Outside frames, the command |\note|. It create a single note page. It
is ``independent'' of any usage of the |\note| commands
inside the previous frame. If you say |\note| inside a frame and
|\note| right after it, \emph{two} note pages are created.

In the following, the syntax and effects of the |\note| command
\emph{inside} frames is described:

\begin{command}{\note\sarg{overlay
      specification}\oarg{options}\marg{note text}}
  \emph{Effects inside frames:}
  
  This command addends to \meta{note text} to
  the note that follows the current slide. Multiple uses of this
  command on a slide accumulate. If you 
  do not specify an \meta{overlay specification}, the 
  note will be added to \emph{all} slides of the current frame. This
  often not what you want, so adding a specification like |<1>| is
  usually a good idea.

  The following \meta{options} may be given:
  \begin{itemize}
  \item \declare{|item|} causes the note to be put as an item in a
    list that is shown at the end of the note page.
  \end{itemize}
    
  \example|\note<2>{Do not talk longer than 2 minutes about this.}|

  \articlenote
  Notes are ignored in |article| mode.

  \lyxnote
  Use the NoteItem style to insert a note item. 
\end{command}


Next, the syntax and effects of the |\note| command
\emph{outside} frames is described:

\begin{command}{\note\oarg{options}\marg{note text}}
  Outside frames, this command creates a note page. This command is  
  \emph{not} affected by the option |notes=onlyframeswithnotes|, see
  below.

  The following \meta{options} may be given:
  \begin{itemize}
  \item \declare{|itemize|} will enclose the whole note page in an
    |itemize| environment. This is just a convenience.
  \item \declare{|enumerate|} will enclose the whole note page in an
    |enumerate| environment.
  \end{itemize}
  
  \example
\begin{verbatim}
\frame{some text}
\note{Talk no more than 1 minute.}

\note[enumerate]
{
\item Stress this first.
\item Then this.
}
\end{verbatim}

  \articlenote
  Notes are ignored in |article| mode.
\end{command}



\subsubsection{Specifying Which Notes and Frames Are Shown}

Since you normally do not wish the notes to be part of your
presentation, you must explicitly specify the class option
|notes| to include notes. If this option is not specified, notes
are suppressed.

The |notes| class option takes several parameters whose effects are
explained in the following.


\begin{classoption}{notes=hide}
  Notes are not shown. This is the default in a presentation.
\end{classoption}

\begin{classoption}{notes=show}
  Include notes in the output file. Normal slides are also included. 
\end{classoption}

\begin{classoption}{notes=only}
  Include only the notes in the output file and suppresses all
  frames. Useful for printing them. If you specify this command, the
  |.aux| and |.toc| files are \emph{not} updated. So, if you add a
  section and re\TeX\ your presentation, this will not be reflected in
  the navigation bars (which you do not see anyway since only notes
  are output).
\end{classoption}

\begin{classoption}{notes=onlyslideswithnotes}
  This includes all notes and those slides that \emph{contain} a
  |\note|. Frames that are just ``followed'' by a |\note| command will
  not be included. 

  If you use only |\note| commands, this option will cause the frames
  and the notes that apply to them to be nicely paired. This is useful
  for printing.
\end{classoption}



\subsubsection{Changing the Appearance of Notes}

By default, notes are put on a page that contains your text,
some information that should make it easier to match the note to the
frame while talking, and a little ``mini version'' of the slide coming
before the note (this mini version contains only the body of the
frame, the head line, foot line, and sidebars are not shown). 

You can change this appearance by specifying a different template for
note rendering, see Section~\ref{section-note-templates} for
details. In most cases it will be sufficient to say either
|\beamertemplatenoteplain| in your preamble, which will give you
``plain'' notes without anything on them but your text, or
|\beamertemplatenotecompress|, which will give you notes with more
space on them.





\subsection{Creating an Article Version}

\label{section-article}

In the following, the ``article version'' of your presentation refers
to a normal \TeX\ text typeset using, for example, the document class
|article| or perhaps |llncs| or a similar document
class. This version of the presentation will typically follow
different typesetting rules and may even have a different
structure. Nevertheless, you may wish to have this version coexist
with your presentation in one file and you may wish to share some part
of it (like a figure or a formula) with your presentation.



\subsubsection{Starting the Article Mode}

The class option |class=|\meta{class name}, where
\meta{class name} is the name of another document class like
|article| or |report|, causes the |beamer| class to transfer control
almost immediately to the class named \meta{class name}. None of the
normal commands defined by the beamer class will be 
defined, except for one: |\mode|. All
class options passed to the beamer class will be passed on to the
class \meta{class name}, \emph{except}, naturally, for the option
|class=|\meta{class name} itself.

\begin{classoption}{class={\normalfont\meta{another class
        name}}{\opt{,{\normalfont\meta{options for another class}}}}}
  Transfer control to document class \meta{another class name} with
  the options \meta{options for another class}.
  
  \example
\begin{verbatim}
\documentclass[class=article,a4paper]{beamer}
\end{verbatim}
  This will cause the rest of the text to be typeset using the
  |article| class with the only class option being
  |a4paper|.
\end{classoption}

Since \beamer\ gives over control to another class almost immediately,
none of the usual commands like |\frame| and so on are defined in
|article| mode initially. The only command that is guaranteed to be
defined is |\mode|; which can be used to ``comment out'' all of
\beamer's commands. For example, in your preamble you might write
things like
\begin{verbatim}
\mode<presentation>{\usepackage{beamerthemeshadow}}
\mode<article>{\usepackage{fullpage}}
\mode<all>{
  \usepackage{times}
  \newcommand{\myfavoritecommand}{...}
}
\end{verbatim}

However, for the main text this is very bothersome and there is a much
better way: 
You can include the package |beamerbasearticle|. This package will
define virtually all of \beamer's commands in a way that is sensible
for the |article| mode. Also, overlay specifications can be given to
commands like |\textbf| or |\item| once |beamerbasearticle| has been
loaded. Note that, except for |\item| these overlay specifications
also work: by writing |\section<presentation>{Name}| you will suppress
this section command in the article version. For the exact effects
overlay specifications have in |article| mode, please see the
descriptions of the commands to which you wish to apply them.

\begin{package}{{beamerbasearticle}\opt{|[|\meta{options}|]|}}
  Makes most \beamer\ commands available for an article.

  The following \meta{options} may be given:
  \begin{itemize}
  \item
    \declare{|activeospeccharacters|} will leave the character code of
    the characters used in overlay specifications as specified by
    other packages. Normally, \beamer\ will turn off the special
    behaviour of characters like |:| or |!| in styles like |french|
    since they clash with the implementation of overlay
    specifications. Using this option, you can reinstall the original
    behaviour at the price of possible problems when using overlay
    specifications in the |article| mode.
  \item
    \declare{|noamsthm|} will suppress the loading of the |amsthm|
    package. No theorems will be defined.
  \item
    \declare{|notheorem|} will suppress the definition of standard
    environments like |theorem|, but |amsthm| is still loaded and the
    |\newtheorem| command still makes the defined environments
    overlay-specification-aware. Using this option allows you to
    define the standard environments in whatever way you like while
    retaining the power of the extensions to |amsthm|.    
  \item
    \declare{|envcountsect|} causes theorem, definitions and the like
    to be numbered with each section. Thus instead of Theorem~1 you
    get Theorem~1.1. I recommend using this option.
  \item
    \declare{|noxcolor|} will suppress the loading of the |xcolor|
    package. No colors will be defined.
  \end{itemize}
\end{package}

There is one remaining problem: While the |article| version can easily
\TeX\ the whole file, even in the presence of commands like
|\frame<2>|, we do not want the special article text to be inserted
into out original beamer presentations. That means, we would like all
text \emph{between} frames to be suppressed. More precisely, we want
all text except for commands like |\section| and so on to be
suppressed. This behaviour can be enforced by specifying the option
|ignorenonframetext| in the presentation version. The option will
insert a |\mode*| at the beginning of your presentation. 

The following example shows a simple usage of the article mode:

\begin{verbatim}
\documentclass[class=article,a4paper]{beamer}
%%\documentclass[ignorenonframetext,red]{beamer}

\mode<article>{\usepackage{fullpage}}
\mode<presentation>{\usepackage{beamerthemesplit}}

%% everyone:
\usepackage[english]{babel}
\usepackage{pgf}

\pgfdeclareimage[height=1cm]{myimage}{filename}

\begin{document}

\section{Introduction}

This is the introduction text. This text is not shown in the
presentation, but will be part of the article.

\frame{
  \begin{figure}
    % In the article, this is a floating figure,
    % In the presentation, this figure is shown in the first frame
    \pgfuseimage{myimage}
  \end{figure}
}

This text is once more not shown in the presentation.

\section{Main Part}

While this text is not shown in the presentation, the section command
also applies to the presentation.

We can add a subsection that is only part of the article like this:

\subsection<article>{Article-Only Section}

With some more text.

\frame{
  This text is part both of the article and of the presentation.
  \begin{itemize}
  \item This stuff is also shown in both version.
  \item This too.
  \only<article>{\item This particular item is only part
      of the article version.}
  \item<presentation:only@0> This text is also only part of the article.
  \end{itemize}
}
\end{document}
\end{verbatim}

There is one command, whose behaviour is a bit special in the article
mode: The line break command |\\|. Inside frames, this command has no
effect in article mode, except if an overlay specification is
present. Then it has the normal effect dictated by the
specification. The reason for this behaviour is that you will
typically inserts lots of |\\| commands in a presentation in order to
get control over all line breaks. These line breaks are mostly
superfluous in the article mode. If you really want a line break to
apply in all versions, say |\\<all>|. Note that the command |\\| is
often redefined by certain environments, so it may not always be
overlay-specification-aware. In such a case you have to write
something like |\only<presentation>{\\}|.




\subsubsection{Workflow}
\label{section-article-version-workflow}
The following workflow steps are optional, but they can simplify the
creation of the article version.

\begin{itemize}
\item 
  In the main file |main.tex|, delete the first line, which sets the
  document class.
\item
  Create a file named, say, |main.beamer.tex| with the
  following content:
\begin{verbatim}
\documentclass[ignorenonframetext]{beamer}
\input{main.tex}
\end{verbatim} 
\item
  Create an extra file named, say, |main.article.tex| with the
  following content:
\begin{verbatim}
\documentclass[class=article]{beamer}
\usepackage{beamerbasearticle}
\setjobnamebeamerversion{main.beamer}
\input{main.tex}
\end{verbatim}
\item
  You can now run |pdflatex| or |latex| on the two files
  |main.beamer.tex| and |main.article.tex|. 
\end{itemize}

The command |\setjobnamebeamerversion| tells the article version where
to find the presentation version. This is necessary if you wish to include
slides from the presentation version in an article as figures.

\begin{command}{\setjobnamebeamerversion\marg{filename without extension}}
  Tells the beamer class where to find the presentation version of the
  current file.  
\end{command}

An example of this workflow approach can be found in the |examples|
subdirectory for files starting with |beamerexample2|.



\subsubsection{Including Slides from the Presentation Version in the
  Article Version}

If you use the package |beamerbasearticle|, the |\frame| command
becomes available in |article| mode. By adjusting the frame template,
you can ``mimic'' the appearance of frames typeset by \beamer\ in
your articles. However, sometimes you may wish to insert ``the real
thing'' into the |article| version, that is, a precise ``screenshot''
of a slide from the presentation. The commands introduced in the
following help you do exactly this.

In order to include a slide from your presentation in your article
version, you must do two things: First, you must place a normal
\LaTeX\ label on the slide using the |\label| command. Since this
command is overlay-specification-aware, you can also select specific
slides of a frame. Also, by adding the option |label=|\meta{name} to
a frame, a label \meta{name}|<|\meta{slide number}|>| is automatically
added to each slide of the frame.

Once you have labeled a slide, you can use the following command in
your article version to insert the slide into it:

\begin{command}{\includeslide\oarg{options}\marg{label name}}
  This command calls |\pgfimage| with the given \meta{options} for
  the file specified by
  \begin{quote}
    |\setjobnamebeamerversion|\meta{filename}
  \end{quote}
  Furthermore, the option |page=|\meta{page of label name} is passed
  to |\pgfimage|, where the \meta{page of label name} is read
  internally from the file \meta{filename}|.snm|.
  \example

\begin{verbatim}
\article
  \begin{figure}
    \begin{center}
      \includeslide[height=5cm]{slide1}
    \end{center}
    \caption{The first slide (height 5cm). Note the partly covered second item.}
  \end{figure}
  \begin{figure}
    \begin{center}
      \includeslide{slide2}
    \end{center}
    \caption{The second slide (original size). Now the second item is also shown.}
  \end{figure}
\end{verbatim}  
\end{command}

The exact effect of passing the option |page=|\meta{page of label
  name} to the command |\pgfimage| is explained in the documentation
of |pgf|. In essence, the following happens:
\begin{itemize}
\item
  For old version of |pdflatex| and for any version of |latex|
  together with |dvips|, the |pgf| package will look for a file named
  \begin{quote}
    \meta{filename}|.page|\meta{page of label name}|.|\meta{extension}
  \end{quote}
  For each page of your |.pdf| or |.ps| file that is to be included in
  this way, you must create such a file by hand. For example, if the
  PostScript file of your presentation version is named
  |main.beamer.ps| and you wish to include the slides with page
  numbers 2 and~3, you must create (single page) files
  |main.beamer.page2.ps| and |main.beamer.page3.ps| ``by hand'' (or
  using some script). If these files cannot be found, |pgf| will
  complain.
\item
  For new versions of |pdflatex|, |pdflatex| also looks for the files
  according to the above naming scheme. However, if it fails to find
  them (because you have not produced them), it uses a special
  mechanism to directly extract the desired page from the presentation
  file |main.beamer.pdf|.
\end{itemize}




\subsection{Details on Modes}

\label{section-mode-details}

This subsection describes how modes work exactly and how you can use
the |\mode| command to control what part of your text belongs to which
mode. 

When \beamer\ typesets your text, it is always in one of the following
four modes:
\begin{itemize}
\item
  \declare{|beamer|} is the default mode.
\item
  \declare{|handout|} is the mode for creating handouts.
\item
  \declare{|trans|} is the mode for creating transparencies.
\item
  \declare{|article|} is the mode when control has been transferred to
  another class, like |article.cls|. Note that the mode is also
  |article| if control is transferred to, say, |book.cls|.
\end{itemize}

In addition to these modes, \beamer\ recognizes the following
names for modes sets:

\begin{itemize}
\item
  \declare{|all|} refers to all modes.
\item
  \declare{|presentation|} refers to the first three modes, that is,
  to all modes except for the |article| mode.
\end{itemize}

Depending on the current mode, you may wish to have certain text
inserted only in that mode. For example, you might wish a certain
frame or a certain table to be left out of your article version. In
some situations, you can use the |\only| command for this
purpose. However, the command |\mode|, which is described in the
following, is much more powerful than |\only|.

The command actually comes in three ``flavors,'' which only slightly
differ in syntax. The first, and simplest, is the version that takes
one argument. It behaves essentially the same way as |\only|.

\begin{command}{\mode\sarg{mode specification}\marg{text}}
  Causes the \meta{text} to be inserted only for the specified
  modes. Recall that a \meta{mode specification} is just an overlay
  specification in which no slides are mentioned.

  \example
\begin{verbatim}
\mode<article>{Extra detail mentioned only in the article version.}

\mode
<beamer| trans>
{\frame{\tableofcontents[current]}}
\end{verbatim}
\end{command}

The second flavor of the |\mode| command takes no argument. ``No
argument'' means that it is not followed by an opening brace, but any
other symbol.

\begin{command}{\mode\sarg{mode specification}}
  In the specified mode, this command actually has no effect. The
  interesting part is the effect in the non-specified modes: In these
  modes, the command causes \TeX\ to enter a kind of ``gobbling''
  state. It will now ignore all following lines until the next
  line that has a sole occurrence of one of the following commands:
  |\mode|, |\mode*|, |\begin{document}|, |\end{document}|. Even a comment on
  this line will make \TeX\ skip it.

  When \TeX\ encounters a single |\mode| command, it will execute this
  command. If the command is |\mode| command of the first flavor,
  \TeX\ will resume its ``gobbling'' state after having inserted (or
  not inserted) the argument of the |\mode| command. If the |\mode|
  command is of the second flavor, it takes over.

  Using this second flavor of |\mode| is less convenient than the
  first, but there are two reasons why you might need to use it:
  \begin{itemize}
  \item
    The line-wise gobbling is much faster than the gobble of the third
    flavor, explained below.
  \item
    The first flavor reads its argument completely. This means, it
    cannot contain any verbatim text.
  \item
    If the text mainly belongs to one mode with only small amounts of
    text from another mode inserted, this second flavor is nice to
    use. 
  \end{itemize}

  \emph{Note:} When searching line-wise for a |\mode| command to shake
  it out of its gobbling state, \TeX\ will not recognize a |\mode|
  command if a mode specification follows on the same line. Thus, such
  a specification must be given on the next line.

  \emph{Note:} When a \TeX\ file ends, \TeX\ must not be in the
  gobbling state. Switch this state off using |\mode| on one line and
  |<all>| on the next.
  
  \example
\begin{verbatim}
\mode<article>

This text is typeset only in article mode.
\verb!verabtim text is ok {!

\mode
<presentation>
{ % this text is inserted only in presentation mode
\frame{\tableofcontents[current]}}

Here we are back to article mode stuff. This text
is not inserted in presentation mode

\mode
<presentation>

This text is only inserted in presentation mode.
\end{verbatim}
\end{command}

The last flavor of the mode command behaves quite differently.

\begin{command}{\mode\declare{|*|}}
  The effect of this mode is to ignore all text outside frames in the
  |presentation| modes. In article mode it has no effect.

  This mode should only be entered outside of frames. Once entered, if
  the current mode is a |presentation| mode, \TeX\ will enter a
  gobbling state similar to the gobbling state of the second
  ``flavor'' of the |\mode| command. The difference is that the text
  is now read token-wise, not line-wise. The text is gobbled token by
  token until one of the following tokens is found: |\mode|, |\frame|,
  |\againframe|, |\part|, |\section|, |\subsection|, |\appendix|,
  |\note|, and |\end{document}| (this is not really a
  token, but it is recognized anyway).

  Once one of these commands is encountered, the gobbling stops and
  the command is executed. However, all of these commands restore the
  mode that was in effect when they started. Thus, once the command is
  finished, \TeX\ returns to its gobbling.

  Normally, |\mode*| is exactly what you want \TeX\ to do outside of
  frames: ignore everything except for the above-mentioned commands
  outside frames in |presentation| mode. However, there are  cases
  in which you have to use the second flavor of the |\mode| command
  instead: If you have verbatim text that contains one of the commands,
  if you have very long text outside frames, or if you wish some text
  outside a frame (like a definition) to be executed also in
  |presentation| mode.

  The class option |ignorenonframetext| will switch on |\mode*| at the
  beginning of the document.

  \example
\begin{verbatim}
\begin{document}
\mode*

This text is not shown in the presentation.

\frame
{
  This text is shown both in article and presentation mode.
}

this text is not shown in the presentation again.

\section{This command also has effect in presentation mode}

Back to article stuff again.

\frame<presentation>
{ this frame is shown only in the presentation. }
\end{document}
\end{verbatim}
  
\end{command}









\section{Customization}

\subsection{Fonts}

\subsubsection{Serif Fonts and Sans-Serif Fonts}

By default, the \beamer\ class uses the Computer Modern sans-serif fonts
for typesetting a presentation. The Computer Modern font family is the
original font family designed by Donald Knuth himself for the \TeX\
program. A sans-serif font is a font in which the letters do not have
serifs (from French \emph{sans}, which means ``without''). Serifs are
the little hooks at the ending of the strokes that make up a
letter. The font you are currently reading is a serif font. \textsf{By
  comparison, this text is in a sans-serif font.}

The choice Computer Modern sans-serif had the following reasons:

\begin{itemize}
\item
  The Computer Modern family has a very large number of symbols
  available that go well together.
\item
  Sans-serif fonts are (generally considered to be) easier to read
  when used in a presentation. In low resolution rendering, serifs
  decrease the legibility of a font.
\end{itemize}

While these reasons are pretty good, you still might wish to change the font:

\begin{itemize}
\item
  The Computer Modern fonts are a bit boring if you have seen them too
  often. Using another font (but not Times!) can give a fresh look.
\item
  Other fonts, especially Times, are sometime rendered better since
  they seem to have better internal hinting.
\item
  A presentation typeset in a serif font creates a conservative
  impression, which might be exactly what you wish to create.
\item
  On projections with a very high resolutions serif text is just as
  readable as sans serif text.
\end{itemize}

You must decide whether the text should be typeset in sans-serif or in
serif. To choose this, use either the class option |sans| or
|serif|. By default, |sans| is selected, so you do not
need to specify this.

\begin{classoption}{sans}
  Use a sans-serif font during the presentation. (Default.)
\end{classoption}

\begin{classoption}{serif}
  Use a serif font during the presentation.
\end{classoption}




\subsubsection{Fonts in Mathematical Text}

By default, if a sans-serif font is used for the main text,
mathematical formulas are also typeset using sans-serif letters. In
most cases, this is visually the pleasing and easily readable way of
typesetting mathematical formulas. However, in mathematical texts the
font used to render, say, a variable is sometimes used to
differentiate between different meanings of this variable. In such
case, it may be necessary to typeset mathematical text using serif
letters. Also, if you have a lot of mathematical text, the audience
may be quicker to ``parse'' it, if it typeset in the way people
usually read mathematical text: in a serif font.

You can use the two options |mathsans| and |mathserif| to override the
overall sans-serif/serif choice for math text. However, using the option
|mathsans| in a |serif| environment makes little sense in my opinion.

\begin{classoption}{mathsans}
  Override the math font to be a sans-serif font.
\end{classoption}

\begin{classoption}{mathserif}
  Override the math font to be a serif font.
\end{classoption}

The command |\mathrm| will always produce upright (not slanted), serif
text and the command |\mathsf| will always produce upright, sans-serif
text. The command |\mathbf| will produce upright, bold-face,
sans-serif or serif text, depending on whether |mathsans| or
|mathserif| is used.

To produce an upright, sans-serif or serif text, depending on
whether |mathsans| or |mathserif| is used, you can use for instance
the command |\operatorname| from the |amsmath| package. Using this
command instead of |\mathrm| or |\mathsf| directly will  automatically
adjust  upright mathematical text if you switch from sans-serif to
serif or back.



\subsubsection{Font Families}

\label{section-substition}

Independently of the serif/sans-serif choice, you can switch the
document font. To do so, you should use one of the prepared packages
of \LaTeX's font mechanism. For example, to change to Times/Helvetica,
simply add 
\begin{verbatim}
\usepackage{times}
\end{verbatim}
in your preamble. Note that if you do not specify |serif| as a
class option, Helvetica (not Times) will be selected as the text
font.

There may be many other fonts available on your
installation. Typically, at least some of the following packages
should be available: |avant|, |bookman|, |chancery|, |charter|,
|euler|, |helvet|, |mathtime|, |mathptm|, |newcent|, |palatino|,
|pifont|, |times|, |utopia|.

If you use |times| together with the |serif| option, you
may wish to include also the package |mathptm|. If you use the
|mathtime| package (you have to buy some of the fonts), you
also need to specify the |serif| option.

If you use professional fonts (fonts that you buy and that come with a
complete set of every symbol in all modes), you may need to specify the
class option |professionalfont|. This will tell \beamer\ that it
should not meddle with the fonts you use. The reason is that \beamer\ 
normally replaces certain character glyphs in mathematical text by
more appropriate versions. For example, \beamer\ will normally replace
glyphs such that the italic characters from the main font are used for
variables in mathematical text. If your professional font package
takes care of this already, \beamer's meddling should be switched
off. Note that \beamer's substitution is automatically turned off if
one of the following packages is loaded: |mathtime|, |mathpmnt|,
|lucidabr|, |mtpro|, and |hvmath|. If your favorite professional font
package is not among these, use the |professionalfont| option (and
write me an email, so that the package can be added).

\begin{classoption}{professionalfont}
  Deactivates \beamer's internal font replacements for mathematical
  text. This option should be used if you use a professional font
  package that sets up all mathematical fonts correctly.
\end{classoption}


\subsubsection{Font Sizes}

The default sizes of the fonts are chosen in a way that makes it
difficult to fit ``too much'' onto a slide. Also, it will ensure that 
your slides are readable even under bad conditions like a large
room and a small only a small projection area. However, you may wish
to enlarge or shrink the fonts a bit if you know this to be more
appropriate in your presentation environment.

The default font size is 11pt. This may seem surprisingly small, but
the actual size of each frame is just 128mm times 96mm and the viewer
application enlarges the font. By specifying a default font size
smaller than 11pt you can put more onto each slide, by specifying a
larger font size you can fit on less.

To specify the the font size, you can use the following class options:

\begin{classoption}{8pt}
  This is way too small. Requires that the package |extsize|
  is installed.
\end{classoption}

\begin{classoption}{9pt}
  This is also too small. Requires that the package |extsize|
  is installed.
\end{classoption}

\begin{classoption}{10pt}
  If you really need to fit much onto each frame, use this
  option. Works without |extsize|.
\end{classoption}

\begin{classoption}{smaller}
  Same as the |10pt| option.
\end{classoption}

\begin{classoption}{11pt}
  The default font size. You need not specify this option.
\end{classoption}

\begin{classoption}{12pt}
  Makes all fonts a little bigger, which makes the text more
  readable. The downside is that less fits onto each frame.
\end{classoption}

\begin{classoption}{bigger}
  Same as the |12pt| option.
\end{classoption}

\begin{classoption}{14pt}
  Makes all fonts somewhat bigger. Requires |extsize| to be installed.
\end{classoption}

\begin{classoption}{17pt}
  This is about the default size of PowerPoint. Requires |extsize| to
  be installed. 
\end{classoption}

\begin{classoption}{20pt}
  This is really huge. Requires |extsize| to be installed.
\end{classoption}


\subsubsection{Font Encodings}
\label{section-font-encoding}

The same font can come in different encodings, which are (very roughly
spoken) the ways the characters of a text are mapped to glyphs (the
actual shape of a particular character in a particular font at a
particular size). In \TeX\ two encodings are often used: the
T1~encoding and the OT1~encoding (old T1~encoding).

Conceptually, the newer T1~encoding is preferable over the old
OT1~encoding. For example, hyphenation of words containing umlauts
(like the famous German word Fr\"aulein) will work only if you use the
T1~encoding. Unfortunately, only the bitmapped version of the Computer
Modern fonts are available in this encoding in a standard
installation. For this reason, using the T1~encoding will produce
\pdf\ files that render very poorly.

Most standard PostScript fonts are available in T1~encoding. For
example, you can use Times in the T1~encoding. The package |lmodern|
makes the standard Computer Modern fonts available in the
T1~encoding. Furthermore, if you use |lmodern| several extra fonts
become available (like a sans-serif boldface math) and extra symbols
(like proper guillemots).

To select the T1 encoding, use |\usepackage[T1]{fontenc}|. Thus, if
you have the |lmodern| fonts installed, you could write
\begin{verbatim}
\usepackage{lmodern}
\usepackage[T1]{fontenc}
\end{verbatim}
to get beautiful outline fonts and correct hyphenation.


\subsection{Margin Sizes}

The ``paper size'' of a beamer presentation is fixed to 128mm times
96mm. The aspect ratio of this size is 4:3, which is exactly what most
beamers offer these days. It is the job of the
presentation program (like |acroread|) to display the slides at
full screen size. The main advantage of using a small ``paper size''
is that you can use all your normal fonts at their natural sizes. In
particular, inserting a graphic with 11pt labels will result in
reasonably sized labels during the presentation.

You should refrain from changing the ``paper size.'' However, you
\emph{can} change the size of the left and right margins, which
default to 1cm. To change them, you should use the following two
commands:

\begin{command}{\beamersetleftmargin\marg{left margin dimension}}
  Sets a new left margin. This excludes the left side bar. Thus, it is
  the distance between the right edge of the left side bar and the left
  edge of the text. This command can only be used in the preamble
  (before the |document| environment is used).
  \example |\beamersetleftmargin{1cm}|

  \articlenote
  This command has no effect in |article| mode.
\end{command}

\begin{command}{\beamersetrightmargin\marg{left margin dimension}}
  Like |\beamersetleftmargin|, only for the right margin.
\end{command}

For more information on side bars, see
Section~\ref{section-sidebar-templates}. 



\subsection{Themes}

Just like \LaTeX\ in general, the \beamer\ class tries to separate the
contents of a text from the way it is typeset (displayed). There are two ways in
which you can change how a presentation is typeset: you can specify a
different theme and you can specify different templates. A theme is
a predefined collection of templates.

There exist a number of different predefined themes that can be used
together with the \beamer\ class. Feel free to add further themes.
Themes are used by including an appropriate \LaTeX\ style file, using
the standard |\usepackage| command.


\begin{smallpackage}{{beamerthemebars}}
  \example

  \pgfuseimage{themebars}\quad\pgfuseimage{themebars2}
\end{smallpackage}


\begin{package}{{beamerthemeboxes}\opt{|[headheight=|\meta{head height}|,footheight=|\meta{foot height}|]|}}
  \example

  \pgfuseimage{themeboxes}\quad\pgfuseimage{themeboxes2}

  \example
\begin{verbatim}
\usepackage[headheight=12pt,footheight=12pt]{beamerthemeboxes}
\end{verbatim}

  For this theme, you can specify an arbitrary number of templates for
  the boxes in the head line and in the foot line. You can add a
  template for another box by using the following commands.
\end{package}

\begin{command}{\addheadboxtemplate%
    \marg{background color command}\marg{box template}}
  Each time this command is invoked, a new box is added to the head
  line, with the first added box being shown on the left. All boxes
  will have the same size.
  \example
\begin{verbatim}
\addheadboxtemplate{\color{black}}{\color{white}\tiny\quad 1. Box}
\addheadboxtemplate{\color{structure}}{\color{white}\tiny\quad 2. Box}
\addheadboxtemplate{\color{structure!50}}{\color{white}\tiny\quad 3. Box}
\end{verbatim}
\end{command}

\begin{command}{\addfootboxtemplate%
    \marg{background color command}\marg{box template}}
  \example
\begin{verbatim}
\addfootboxtemplate{\color{black}}{\color{white}\tiny\quad 1. Box}
\addfootboxtemplate{\color{structure}}{\color{white}\tiny\quad 2. Box}
\end{verbatim}
\end{command}


\begin{smallpackage}{{beamerthemeclassic}}
  \example

  \pgfuseimage{themeclassic}\quad\pgfuseimage{themeclassic2}
\end{smallpackage}


\begin{smallpackage}{{beamerthemelined}}
  \example

  \pgfuseimage{themelined}\quad\pgfuseimage{themelined2}
\end{smallpackage}


\begin{smallpackage}{{beamerthemeplain}}
  \example

  \pgfuseimage{themeplain}\quad\pgfuseimage{themeplain2}
\end{smallpackage}


\begin{package}{{beamerthemesidebar}\opt{|[width=|\meta{sidebar
        width}|,dark,tab]|}}
  The option |width| sets the width of the sidebar to \meta{sidebar
    width}. The option |dark| makes the side bar and the whole theme
  darked. The option |tab| causes the current section or subsection to
  be hilighted by changing the background behind the entry, rather
  than hilighting the entry itself.
  
  \example |\usepackage{beamerthemesidebar}|

  \pgfuseimage{themesidebar}\quad\pgfuseimage{themesidebar2}

  \example |\usepackage[tab]{beamerthemesidebar}|

  \pgfuseimage{themesidebartab}\quad\pgfuseimage{themesidebartab2}

  \example |\usepackage[dark]{beamerthemesidebar}|

  \pgfuseimage{themesidebardark}\quad\pgfuseimage{themesidebardark2}

  \example |\usepackage[dark,tab]{beamerthemesidebar}|

  \pgfuseimage{themesidebardarktab}\quad\pgfuseimage{themesidebardarktab2}
\end{package}


\begin{smallpackage}{{beamerthemeshadow}}
  \example

  \pgfuseimage{themeshadow}\quad\pgfuseimage{themeshadow2}
\end{smallpackage}

\begin{smallpackage}{{beamerthemesplit}}
  \example

  \pgfuseimage{themesplit}\quad\pgfuseimage{themesplit2}
\end{smallpackage}

\begin{smallpackage}{{beamerthemetree}}
  \example

  \pgfuseimage{themetree}\quad\pgfuseimage{themetree2}
\end{smallpackage}


\begin{smallpackage}{{beamerthemetree}\declare{|[bar]|}}
  \example

  \pgfuseimage{themetreebars}\quad\pgfuseimage{themetreebars2}
\end{smallpackage}



\subsection{Templates}
\label{section-templates}

\subsubsection{Introduction to Templates}

If you only wish to modify a small part of how your presentation is
rendered, you do not need to create a whole new theme. Instead, you
can modify an appropriate template.

A template specifies how a part of a presentation is typeset. For
example, the frame title template dictates where the frame title is
put, which font is used, and so on.

As the name suggests, you specify a template by writing the exact
\LaTeX\ code you would also use when typesetting a single frame title
by hand. Only, instead of the actual title, you use the command
|\insertframetitle|.

For example, suppose we would like to have the frame title typeset in
red, centered, and boldface. If we were to typeset a single frame
title by hand, it might be done like this:
\begin{verbatim}
\frame
{
  \begin{centering}
    \color{red}
    \textbf{The Title of This Frame.}
    \par
  \end{centering}

  Blah, blah.
}
\end{verbatim}

In order to typeset the frame title in this way on all slides, we can
change the frame title template as follows:
\begin{verbatim}
\useframetitletemplate{
  \begin{centering}
    \color{red}
    \textbf{\insertframetitle}
    \par
  \end{centering}
}
\end{verbatim}

We can then use the following code to get the desired effect:
\begin{verbatim}
\frame
{
  \frametitle{The Title of This Frame.}

  Blah, blah.
}
\end{verbatim}

When rendering the frame, the \beamer\ class will use the code of the
frame title template to typeset the frame title and it will replace
every occurrence of |\insertframetitle| by the current frame
title.

In the following subsections all commands for changing templates are
listed, like the above-mentioned command
|\useframetitletemplate|. Inside these commands, you should use
the |\insertxxxx| commands, which are listed following the template
changing commands. Although the |\insertxxxx| commands are listed
alongside the templates for which they make the most sense, you can
(usually) also use them in all other templates.

\articlenote
In |article| mode, most of the template mechanism is switched off and
has no effect. However, a few templates are also available. If this is
the case, it is specially indicated.
\smallskip

Some of the below subsections start with commands for using
\emph{predefined} templates. Calling one of them will change a template in
a predefined way, making it unnecessary to worry about how exactly one
creates, say, these cute little balls in different sizes. Using them,
you can use, for example, your favorite theme together with a
shading background and a numbered table of contents.

Here are a few hints that might be helpful when you wish to redefine a
template: 
\begin{itemize}
\item
  Usually, you might wish to copy code from an existing template. The
  code often takes care of some things that you may not yet have
  thought about. The file |beamerbasetemplates| might be useful
  starting point.
\item
  When copying code from another template and when inserting this code
  in the preamble of your document (not in another style file), you may
  have to ``switch on'' the at-character (|@|). To do so, add the
  command |\makeatletter| before the |\usexxxtemplate| command and the
  command |\makeatother| afterward.
\item
  Most templates having to do with the frame components (head lines,
  side bars, etc.)\ can only be changed in the preamble. Other
  templates can be changed during the document.
\item
  The height of the head line and foot line templates is calculated
  automatically. This is done by typesetting the templates and then
  ``having a look'' at their heights. This recalculation is done right
  at the beginning of the document, \emph{after} all packages have
  been loaded and even \emph{after} these have executed their
  |\AtBeginDocument| initialization. 
\item
  The left and right margins of the head and foot line templates are
  the same as of the normal text. In order to start the head line and
  foot line at the page margin, you must insert a negative
  horizontal skip using |\hskip-\Gm@lmargin|. You may wish to add a
  |\hskip-\Gm@rmargin| at the end to avoid having \TeX\ complain about
  overfull boxes.
\item
  Getting the boxes right inside any template is often a bit of a
  hassle. You may wish to consult the \TeX\ book for the glorious
  details on ``Making Boxes.'' If your headline is simple, you might
  also try putting everything into a |pgfpicture| environment, which
  makes the placement easier.
\end{itemize}



\subsubsection{Title Page}

\paragraph{Predefined Templates}\ 

\begin{command}{\beamertemplatelargetitlepage}
  Causes the title page to be typeset with a large font for the title.
\end{command}

\begin{command}{\beamertemplateboldtitlepage}
  Causes the title page to be typeset with a bold font for the title.
\end{command}



\paragraph{Template Installation Commands}\ 

\begin{command}{\usetitlepagetemplate\marg{title page template}}
  \example
\begin{verbatim}
\usetitlepagetemplate{
  \vbox{}
  \vfill
  \begin{centering}
    \Large\structure{\inserttitle}
    \vskip1em\par
    \normalsize\insertauthor\vskip1em\par
    {\scriptsize\insertinstitute\par}\par\vskip1em
    \insertdate\par\vskip1.5em
    \inserttitlegraphic
  \end{centering}
  \vfill
}
\end{verbatim}
\end{command}

If you wish to suppress the head and foot line in the title page, use
|\frame[plain]{\titlepage}|.




\paragraph{Inserts for this Template}\ 

\begin{command}{\insertauthor}
  Inserts the author names into a template.
\end{command}

\begin{command}{\insertdate}
  Inserts the date into a template.
\end{command}

\begin{command}{\insertinstitute}
  Inserts the institute into a template.
\end{command}

\begin{command}{\inserttitle}
  Inserts a version of the document title into a template that is
  useful for the title page. 
\end{command}

\begin{command}{\inserttitlegraphic}
  Inserts the title graphic into a template.
\end{command}



\subsubsection{Part Page}

\label{section-part-page-template}

\paragraph{Predefined Templates}\ 

\begin{command}{\beamertemplatelargepartpage}
  Causes the part pages to be typeset with a large font for the part name.
\end{command}

\begin{command}{\beamertemplateboldpartpage}
  Causes the part pages to be typeset with a bold font for the part name.
\end{command}


\paragraph{Template Installation Commands}\ 

\begin{command}{\usepartpagetemplate\marg{part page template}}
  \example
\begin{verbatim}
\usepartpagetemplate{
  \begin{centering}
    \Large\structure{\partname~\insertromanpartnumber}
    \vskip1em\par
    \insertpart\par
  \end{centering}
  }
\end{verbatim}
\end{command}


\paragraph{Inserts for this Template}\ 

\begin{command}{\insertpart}
  Inserts the current part name.
\end{command}

\begin{command}{\insertpartnumber}
  Inserts the current part number as an Arabic number into a template.
\end{command}

\begin{command}{\insertpartromannumber}
  Inserts the current part number as a Roman number into a template.
\end{command}



\subsubsection{Frames}

\label{section-frame-template}

\paragraph{Template Installation Commands}\ 

\begin{command}{\useframetemplate\marg{begin of frame}\marg{end of
      frame}}
  \beamernote
  This command is currently \emph{not} available in the presentation
  modes.

  \articlenote
  The \meta{begin of frame} text is inserted at the beginning of each
  frame, when it is inserted into the article. The text \meta{end of
  frame} is inserted at the end. You can use this template to put,
  say, lines around a frame or to put the whole frame into  a
  minipage. By default, nothing is inserted.
  
  \example
\begin{verbatim}
\useframetemplate{\par\medskip\hrule\smallskip}{\par\smallskip\hrule\medskip}
\end{verbatim}
\end{command}





\subsubsection{Background}

\label{section-backgrounds}

\paragraph{Predefined Templates}\ 

\begin{command}{\beamertemplatesolidbackgroundcolor\marg{color}}
  Installs the given color as the background color for every frame.
  
  \example |\beamertemplatesolidbackgroundcolor{white!90!red}|
\end{command}

\begin{command}{\beamertemplateshadingbackground%
    \marg{color expression page bottom}\marg{color expression page top}}
  Installs a vertically shaded background such that the
  specified bottom color changes smoothly to the specified top
  color. \emph{Use with care: Background shadings are often
    distracting!} However, a very light shading with warm colors can 
  make a presentation more lively.
  \example
\begin{verbatim}
\beamertemplateshadingbackground{red!10}{blue!10}
%% Bottom is light red, top is light blue
\end{verbatim}
\end{command}


\begin{command}{\beamertemplategridbackground\oarg{spacing}}
  Installs a light grid as background with lines spaced apart by
  \meta{spacing}. Default is half a centimeter.

  \example |\beamertemplategridbackground[0.2cm]|
\end{command}


\paragraph{Template Installation Commands}\ 

\begin{command}{\usebackgroundtemplate\marg{background template}}
  Installs a new background template. Call
  |\beamersetaveragebackground| after you have called this macro, see
  Section~\ref{section-average} for details.
  \example
\begin{verbatim}
\usebackgroundtemplate{%
  \color{red}%
  \vrule  height\paperheight width\paperwidth%
}
\end{verbatim}
\end{command}







\subsubsection{Table of Contents}

\label{section-toc-templates}

\paragraph{Predefined Templates}\ 

\begin{command}{\beamertemplateplaintoc}
  Installs a simple table of contents template with indented subsections. 
  \example |\beamertemplateplaintoc|
\end{command}

\begin{command}{\beamertemplateballtoc}
  Installs a table of contents template in which small balls are shown
  before each section and subsection.
  \example |\beamertemplateballtoc|
\end{command}

\begin{command}{\beamertemplatenumberedsectiontoc}
  Installs a table of contents template in which the sections are
  numbered. 
  \example |\beamertemplatenumberedsectiontoc|
\end{command}

\begin{command}{\beamertemplatenumberedcirclesectiontoc}
  Installs a table of contents template in which the sections are
  numbered and the numbers are drawn on a small circle. 
  \example |\beamertemplatenumberedcirclesectiontoc|
\end{command}

\begin{command}{\beamertemplatenumberedballsectiontoc}
  Installs a table of contents template in which the sections are
  numbered and the numbers are drawn on a small ball. 
  \example |\beamertemplatenumberedballsectiontoc|
\end{command}

\begin{command}{\beamertemplatenumberedsubsectiontoc}
  Installs a table of contents template in which the subsections are
  numbered. 
  \example |\beamertemplatenumberedsubsectiontoc|
\end{command}



\paragraph{Template Installation Commands}\ 

\begin{command}{\usetemplatetocsection\oarg{mix-in specification}%
    \marg{template}\opt{\marg{grayed template}}}
  Installs a \meta{template} for rendering sections in the table of
  contents. If the \meta{mix-in specification} is present, the
  \meta{grayed template} may not be present and the grayed sections
  names are obtained by mixing in the  \meta{mix-in specification}. 
  If \meta{mix-in specification} is not present,  \meta{grayed
    template} must be present and is used to render grayed section
  names. 
  \example
\begin{verbatim}
\usetemplatetocsection
{\color{structure}\inserttocsection}
{\color{structure!50}\inserttocsection}

\usetemplatetocsection[50!averagebackgroundcolor]
{\color{structure}\inserttocsection}
\end{verbatim}
\end{command}

\begin{command}{\usetemplatetocsubsection\oarg{mix-in specification}%
    \marg{template}\opt{\marg{grayed template}}}
  See |\usetemplatetocsection|.
  \example
\begin{verbatim}
\usetemplatetocsubsection
{\leavevmode\leftskip=1.5em\color{black}\inserttocsubsection\par}
{\leavevmode\leftskip=1.5em\color{black!50!white}\inserttocsubsection\par}

\usetemplatetocsection[50!averagebackgroundcolor]
{\leavevmode\leftskip=1.5em\color{black}\inserttocsubsection\par}
\end{verbatim}
\end{command}



\paragraph{Inserts for this Template}\ 

\begin{command}{\inserttocsection}
  Inserts the table of contents version of the current section name
  into a template.
\end{command}

\begin{command}{\inserttocsectionnumber}
  Inserts the number of the current section (in the table of contents)
  into a template. 
\end{command}

\begin{command}{\inserttocsubsection}
  Inserts the table of contents version of the current subsection name
  into a template. 
\end{command}

\begin{command}{\inserttocsubsectionnumber}
  Inserts the number of the current subsection (in the table of
  contents) into a template. 
\end{command}





\subsubsection{Bibliography}

\label{section-bib-templates}

\paragraph{Predefined Templates}\

\begin{command}{\beamertemplatetextbibitems}
  Shows the citation text in front of references in a
  bibliography instead of a small symbol.
\end{command} 

\begin{command}{\beamertemplatearrowbibitems}
  Changes the symbol before references in a bibliography to
  a small arrow.
\end{command}

\begin{command}{\beamertemplatebookbibitems}
  Changes the symbol before references in a bibliography to
  a small book icon.
\end{command}

\begin{command}{\beamertemplatearticlebibitems}
  Changes the symbol before references in a bibliography to
  a small article icon. (Default)
\end{command}



\paragraph{Template Installation Commands}\ 

\begin{command}{\usebibitemtemplate\marg{citation template}}
  Installs a template for the citation text before the entry. (The 
  ``label'' of the item.)
  \example |\usebibitemtemplate{\color{structure}\insertbiblabel}|
\end{command}


\begin{command}{\usebibliographyblocktemplate%
    \marg{template 1}\marg{template 2}%
    \marg{template 3}\marg{template 4}}
  The text \meta{template~1} is inserted before the first block of the
  entry (the first block is all text before the first occurrence of a 
  |\newblock| command). The text \meta{template~2} is inserted before
  the second block (the text between the first and second occurrence
  of |\newblock|). Likewise for \meta{template~3} and \meta{template~4}. 

  The templates are inserted \emph{before} the blocks and you do not
  have access to the blocks themselves via insert commands. In the
  following example, the first |\par| commands ensure that the
  author, the title, and the journal are put on different lines. The
  color commands cause the author (first block) to be typeset using
  the theme color, the second block (title of the paper) to be typeset
  in black, and all other lines to be typeset in a washed-out version
  of the theme color. 
  \example
\begin{verbatim}
  \usebibliographyblocktemplate
  {\color{structure}}
  {\par\color{black}}
  {\par\color{structure!75}}
  {\par\color{structure!75}}
\end{verbatim}
\end{command}


\paragraph{Inserts for these Templates}\ 

\begin{command}{\insertbiblabel}
  Inserts the current citation label into a template.
\end{command}



\subsubsection{Frame Titles}

\paragraph{Predefined Templates}\

\begin{command}{\beamertemplateboldcenterframetitle}
  Typesets the frame title using a bold face and centers it.
\end{command}

\begin{command}{\beamertemplatelargeframetitle}
  Typesets the frame title using a large face and flushes it left.
\end{command}


\paragraph{Template Installation Commands}\ 

\begin{command}{\useframetitletemplate\marg{frame title template}}
  \example
\begin{verbatim}
\useframetitletemplate{%
  \begin{centering}
    \structure{\textbf{\insertframetitle}}
    \par
  \end{centering}
}
\end{verbatim}

  \articlenote
  This command is also available in |article| mode. By default, a new
  paragraph is created. You may wish to install a template that will
  simply suppress the frame title.
\end{command}

\paragraph{Inserts for this Template}\ 

\begin{command}{\insertframetitle}
  Inserts the current frame title into a template.
\end{command}




\subsubsection{Head Lines and Foot Lines}

\label{section-head-templates}

\paragraph{Predefined Templates}\ 

\begin{command}{\beamertemplateheadempty}
  Makes the head line empty.
\end{command}

\begin{command}{\beamertemplatefootempty}
  Makes the foot line empty.
\end{command}

\begin{command}{\beamertemplatefootpagenumber}
  Shows only the page number in the foot line.
\end{command}



\paragraph{Template Installation Commands}\ 

\begin{command}{\usefoottemplate\marg{foot line template}}
  The final height of the foot line is calculated by invoking this
  template just before the beginning of the document and by setting
  the foot line height to the height of the template.
  \example
\begin{verbatim}
\usefoottemplate{\hfil\tiny{\color{black!50}\insertpagenumber}}
\end{verbatim}
or
\begin{verbatim}
\usefoottemplate{%
  \vbox{%
    \tinycolouredline{structure!75}%
      {\color{white}\textbf{\insertshortauthor\hfill\insertshortinstitute}}%
    \tinycolouredline{structure}%
      {\color{white}\textbf{\insertshorttitle}\hfill}%
    }}
\end{verbatim}
\end{command}


\begin{command}{\useheadtemplate\marg{head line template}}
  See |\usefoottemplate|.
  \example
\begin{verbatim}
\useheadtemplate{%
  \vbox{%
  \vskip3pt%
  \beamerline{\insertnavigation{\paperwidth}}%
  \vskip1.5pt%
  \insertvrule{0.4pt}{structure!50}}%
}
\end{verbatim}
\end{command}



\paragraph{Inserts for these Templates}\ 

\begin{command}{\insertframenumber}
  Inserts the number of the current frame (not slide) into a template.
\end{command}

\begin{command}{\inserttotalframenumber}
  Inserts the total number of the frames (not slides) into a
  template. The number is only correct on the second run of \TeX\ on
  your document.
\end{command}

\begin{command}{\insertlogo}
  Inserts the logo(s) into a template.
\end{command}

\begin{command}{\insertnavigation\marg{width}}
  Inserts a horizontal navigation bar of the given \meta{width} into a
  template. The bar lists the sections and below them mini frames for
  each frame in that section.
\end{command}

\begin{command}{\insertpagenumber}
  Inserts the current page number into a template.
\end{command}

\begin{command}{\insertsection}
  Inserts the current section into a template.
\end{command}

\begin{command}{\insertsectionnavigation\marg{width}}
  Inserts a vertical navigation bar containing all sections, with the
  current section hilighted.
\end{command}

\begin{command}{\insertsectionnavigationhorizontal\marg{width}%
    \marg{left insert}\marg{right insert}}
  Inserts a horizontal navigation bar containing all sections, with
  the current section hilighted. The \meta{left insert} will be
  inserted to the left of the sections, the \marg{right insert} to the
  right. By inserting a triple fill (a
  |filll|) you can flush to bar to the left or right.
  \example
\begin{verbatim}
\insertsectionnavigationhorizontal{.5\textwidth}{\hskip0pt plus1filll}{}
\end{verbatim}
\end{command}

\begin{command}{\insertshortauthor\oarg{options}}
  Inserts the short version of the author into a template. The text
  will be printed in one long line, line breaks introduced using the
  |\\| command are suppressed.  The
  following \meta{options} may be given:
  \begin{itemize}
  \item
    \declare{|width=|\meta{width}}
    causes the text to be put into a multi-line minipage of the given
    size. Line breaks are still suppressed by default.
  \item
    \declare{|center|}
    centers the text inside the minipage created using the |width|
    option, rather than having it left aligned.
  \item
    \declare{|respectlinebreaks|}
    causes line breaks introduced by the |\\| command to be honored.    
  \end{itemize}

  \example |\insertauthor[width={3cm},center,respectlinebreaks]|
\end{command}

\begin{command}{\insertshortdate\oarg{options}}
  Inserts the short version of the date into a template. The same
  options as for |\insertshortauthor| may be given. 
\end{command}

\begin{command}{\insertshortinstitute\oarg{options}}
  Inserts the short version of the institute into a template. The same
  options as for |\insertshortauthor| may be given. 
\end{command}

\begin{command}{\insertshortpart\oarg{options}}
  Inserts the short version of the part name into a template. The same
  options as for |\insertshortauthor| may be given. 
\end{command}

\begin{command}{\insertshorttitle\oarg{options}}
  Inserts the short version of the document title into a template. Same
  options as for |\insertshortauthor| may be given. 
\end{command}


\begin{command}{\insertsubsection}
  Inserts the current subsection into a template.
\end{command}

\begin{command}{\insertsubsectionnavigation\marg{width}}
  Inserts a vertical navigation bar containing all subsections of the
  current section, with the current subsection hilighted.
\end{command}

\begin{command}{\insertsubsectionnavigationhorizontal\marg{width}%
    \marg{left insert}\marg{right insert}}
  See |\insertsectionnavigationhorizontal|.
\end{command}


\begin{command}{\insertverticalnavigation\marg{width}}
  Inserts a vertical navigation bar of the given \meta{width} into a
  template. The bar shows a little table of contents. The individual
  lines are typeset using the templates
  |\usesectionsidetemplate| and |\usesubsectionsidetemplate|.
\end{command}

\begin{command}{\insertvrule\marg{color expression}\marg{thickness}}
  Inserts a rule of the given color and \meta{thickness} into a
  template. 
\end{command}





\subsubsection{Side Bars}

\label{section-sidebar-templates}

Side bars are vertical areas that stretch from the lower end of the
head line to the top of the foot line. There can be a side bar at the
left and one at the right (or even both). Side bars can show a table
of contents, but they could also be added for purely aesthetic
reasons.

When you install a side bar template, you must explicitly specify the
horizontal size of the side bar. The vertical size is determined
automatically. Each side bar can have its own background, which can be
setup using special side background templates.

Adding a sidebar of a certain size, say 1cm, will make the main text
1cm narrower. The distance between the inner side of a side
bar and the outer side of the text, as specified by
the command |\beamersetleftmargin| and its counterpart for the
right margin, is not changed when a side bar is installed.

Internally, the sidebars are typeset by showing them as part of the
headline. The \beamer\ class keeps track of six dimensions, three 
for each side: the variables |\beamer@leftsidebar| and
|\beamer@rightsidebar| store the (horizontal) sizes of the side
bars, the variables |\beamer@leftmargin| and
|\beamer@rightmargin| store the distance between sidebar and
text, and the macros |\Gm@lmargin| and  |\Gm@rmargin| store
the distance from the edge of the paper to the edge of the text. Thus
the sum |\beamer@leftsidebar| and |\beamer@leftmargin| is
exactly  |\Gm@lmargin|. Thus, if you wish to put some text right
next to the left side bar, you might write
|\hskip-\beamer@leftmargin| to get there.

In the following, only the commands for the left side bars are
listed. Each of these commands also exists for the right side bar,
with ``left'' replaced by ``right'' everywhere.


\begin{command}{\useleftsidebartemplate\marg{horizontal size}\marg{template}}
  When the side bar is typeset, the \meta{template} is invoked inside a
  |\vbox| of the height of the side bar. Thus, the below example
  will produce a side bar of half a centimeter width, in which the word
  ``top'' is printed just below the head line and ``bottom'' is printed
  just above the foot line.
  \example
\begin{verbatim}
\useleftsidebartemplate{1cm}{
  top
  \vfill
  bottom
}
\end{verbatim}
\end{command}

\begin{command}{\useleftsidebarbackgroundtemplate\marg{template}}
  The template is shown behind whatever is shown in the left side
  bar. 
  \example
\begin{verbatim}
\useleftsidebarbackgroundtemplate
  {\color{red}\vrule height\paperheight width\beamer@leftsidebar}
\end{verbatim}
\end{command}


\begin{command}{\useleftsidebarcolortemplate\marg{color expression}}
  Uses the given color as background for the side bar.
  \example
\begin{verbatim}
\useleftsidebarcolortemplate{\color{red}}
\useleftsidebarcolortemplate{\color[rgb]{1,0,0.5}}
\end{verbatim}
\end{command}

\begin{command}{\useleftsidebarverticalshadingtemplate\marg{bottom
      color expression}\marg{top color expression}}
  Installs a smooth vertical transition between the given colors as
  background for the side bar.
  \example
\begin{verbatim}
\useleftsidebarverticalshadingtemplate{white}{red}
\end{verbatim}
\end{command}


\begin{command}{\useleftsidebarhorizontalshadingtemplate\marg{left end
      color expression}\marg{right end color expression}}
  Installs a smooth horizontal transition between the given colors as
  background for the side bar.
  \example
\begin{verbatim}
\useleftsidebarhorizontalshadingtemplate{white}{red}
\end{verbatim}
\end{command}


\begin{command}{\usesectionsidetemplate\marg{current section
      template}\marg{other section template}}
  Both parameters should be |\hbox|es. The templates are used to
  typeset a section name inside a side navigation bar.
  \example
\begin{verbatim}
\usesectionsidetemplate
{\setbox\tempbox=\hbox{\color{black}\tiny{\kern3pt\insertsectionhead}}%
  \ht\tempbox=8pt%
  \dp\tempbox=2pt%
  \wd\tempbox=\beamer@sidebarwidth%
  \box\tempbox}
{\setbox\tempbox=\hbox{\color{structure!75}\tiny{\kern3pt\insertsectionhead}}%
  \ht\tempbox=8pt%
  \dp\tempbox=2pt%
  \wd\tempbox=\beamer@sidebarwidth%
  \box\tempbox}
\end{verbatim}
\end{command}



\begin{command}{\usesubsectionsidetemplate\marg{current subsection
      template}\marg{other subsection template}}
  See |\usesectionsidetemplate|.
  \example
\begin{verbatim}
\usesectionsidetemplate
{\setbox\tempbox=\hbox{\color{black}\tiny{\kern3pt\insertsectionhead}}%
  \ht\tempbox=8pt%
  \dp\tempbox=2pt%
  \wd\tempbox=\beamer@sidebarwidth%
  \box\tempbox}
{\setbox\tempbox=\hbox{\color{structure!75}\tiny{\kern3pt\insertsectionhead}}%
  \ht\tempbox=8pt%
  \dp\tempbox=2pt%
  \wd\tempbox=\beamer@sidebarwidth%
  \box\tempbox}
\end{verbatim}
\end{command}










\subsubsection{Buttons}
\label{section-navigation-buttons}

\paragraph{Predefined Templates}\ 

\begin{command}{\beamertemplateoutlinebuttons}
  Renders buttons as rectangles with rounded left and right
  border. Only the border (outline) is painted.
\end{command}

\begin{command}{\beamertemplatesolidbuttons}
  Renders buttons as filled rectangles with rounded left and right
  border.
\end{command}


\paragraph{Template Installation Commands}\ 

\begin{command}{\usebuttontemplate\marg{button template}}
  Installs a new button template. This template is invoked whenever a
  button should be rendered.
  \example
\begin{verbatim}
\usebuttontemplate{\color{structure}\insertbuttontext}
\end{verbatim}
\end{command}


\paragraph{Inserts}\ 

Inside the button template, the button text can be accessed via the
following command:

\begin{command}{\insertbuttontext}
  Inserts the text of the current button into a template. When called
  by  button creation commands, like |\beamerskipbutton|, the symbol
  will be part of this text.
\end{command}

The button creation commands automatically add the following three
inserts to the text to be rendered by |\insertbuttontext|:

\begin{command}{\insertgotosymbol}
  Inserts a small right-pointing arrow.
\end{command}

\begin{command}{\insertskipsymbol}
  Inserts a double right-pointing arrow.
\end{command}

\begin{command}{\insertreturnsymbol}
  Inserts a small left-pointing arrow.
\end{command}

You can redefine these commands to change these symbols.




\subsubsection{Navigation Bars}

\paragraph{Predefined Templates}\ 

\begin{command}{\beamertemplatecircleminiframe}
  Changes the symbols in a navigation bar used to represent
  a frame to a small circle.
\end{command}

\begin{command}{\beamertemplatecircleminiframeinverted}
  Changes the symbols in a navigation bar used to represent
  a frame to a small circle, but with the colors inverted. Use this if
  the navigation bar is shown on a dark background.
\end{command}

\begin{command}{\beamertemplatesphereminiframe}
  Changes the symbols in a navigation bar used to represent
  a frame to a small spheres.
\end{command}

\begin{command}{\beamertemplatesphereminiframeinverted}
  Changes the symbols in a navigation bar used to represent
  a frame to a small spheres, but with the colors inverted. Use this if
  the navigation bar is shown on a |structure| background.
\end{command}

\begin{command}{\beamertemplateboxminiframe}
  Changes the symbols in a navigation bar used to represent
  a frame to a small box.
\end{command}

\begin{command}{\beamertemplateticksminiframe}
  Changes the symbols in a navigation bar used to represent
  a frame to a small vertical bar of varying length.
\end{command}


\paragraph{Template Installation Commands}\ 

\begin{command}{\usesectionheadtemplate\marg{current section
      template}\marg{other section template}}
  The templates are used to render the section names in a navigation
  bar. 
  \example
\begin{verbatim}
\usesectionheadtemplate
{\color{structure}\tiny\insertsectionhead}
{\color{structure!50}\tiny\insertsectionhead}
\end{verbatim}
\end{command}
  

\begin{command}{\usesubsectionheadtemplate\marg{current subsection
      template}\marg{other subsection template}}
  See |\usesectionheadtemplate|.
  \example
\begin{verbatim}
\usesubsectionheadtemplate
{\color{structure}\tiny\insertsubsectionhead}
{\color{structure!50}\tiny\insertsubsectionhead}
\end{verbatim}
\end{command}

\begin{command}{\useminislidetemplate%
    \marg{template current frame icon}%
    \marg{template current subsection frame icon}\\%
    \marg{template other frame icon}%
    \marg{horizontal offset}%
    \marg{vertical offset}}
  The templates are used to draw frame icons in navigation bars. The
  offsets describe the offset between icons.
  \example
\begin{verbatim}
\useminislidetemplate
  {
    \color{structure}%
    \hskip-0.4pt\vrule height\boxsize width1.2pt%
  }  
  {%
    \color{structure}%
    \vrule height\boxsize width0.4pt%
  }
  {%
    \color{structure!50}%
    \vrule height\boxsize width0.4pt%
  }
  {.1cm}
  {.05cm}
\end{verbatim}
\end{command}




\subsubsection{Navigation Symbols}
\label{section-navigation-symbols-template}

\paragraph{Predefined Templates}\ 

\begin{command}{\beamertemplatenavigationsymbolsempty}
  Suppresses all navigation symbols.
\end{command}

\begin{command}{\beamertemplatenavigationsymbolsframe}
  Shows only the frame symbol as navigation symbol.
\end{command}

\begin{command}{\beamertemplatenavigationsymbolsvertical}
  Organizes the navigation symbols vertically.
\end{command}

\begin{command}{\beamertemplatenavigationsymbolshorizontal}
  Organizes the navigation symbols horizontally.
\end{command}



\paragraph{Template Installation Commands}\ 

\begin{command}{\usenavigationsymbolstemplate\marg{symbols template}}
  Installs a new symbols template. This template is invoked by themes
  at the place where the navigation symbols should be shown.
  \example
\begin{verbatim}
\usenavigationsymbolstemplate{\vbox{%
  \hbox{\insertslidenavigationsymbols}
  \hbox{\insertframenavigationsymbols}
  \hbox{\insertsubsectionnavigationsymbols}
  \hbox{\insertsectionnavigationsymbols}
  \hbox{\insertdocnavigationsymbols}
  \hbox{\insertbackfindforwardnavigationsymbols}}}
\end{verbatim}
\end{command}


\paragraph{Inserts for this Template}\ 

The following inserts are useful for the navigation symbols template:

\begin{command}{\insertslidenavigationsymbols}
  Inserts the slide navigation symbol, see
  Section~\ref{section-navigation-symbols}.
\end{command}

\begin{command}{\insertframenavigationsymbols}
  Inserts the frame navigation symbol, see
  Section~\ref{section-navigation-symbols}.
\end{command}

\begin{command}{\insertsubsectionnavigationsymbols}
  Inserts the subsection navigation symbol, see
  Section~\ref{section-navigation-symbols}.
\end{command}

\begin{command}{\insertsectionnavigationsymbols}
  Inserts the section navigation symbol, see
  Section~\ref{section-navigation-symbols}.
\end{command}

\begin{command}{\insertdocnavigationsymbols}
  Inserts the presentation navigation symbol and (if necessary) the
  appendix navigation symbol, see
  Section~\ref{section-navigation-symbols}.
\end{command}

\begin{command}{\insertbackfindforwardnavigationsymbols}
  Inserts a back, a find, and a forward navigation symbol, see
  Section~\ref{section-navigation-symbols}.
\end{command}





\subsubsection{Footnotes}

\label{section-templates-footnotes}

\paragraph{Template Installation Commands}\

\begin{command}{\usefootnotetemplate\marg{footnote template}}
  \example
\begin{verbatim}
\usefootnotetemplate{
  \parindent 1em
  \noindent
  \hbox to 1.8em{\hfil\insertfootnotemark}\insertfootnotetext}
\end{verbatim}
\end{command}


\paragraph{Inserts for these Templates}\

\begin{command}{\insertfootnotemark}
  Inserts the current footnote mark (like a raised number) into a
  template. 
\end{command}

\begin{command}{\insertfootnotetext}
  Inserts the current footnote text into a template. 
\end{command}





\subsubsection{Captions}
\label{section-template-caption}

\paragraph{Predefined Templates}\

\begin{command}{\beamertemplatecaptionwithnumber}
  Changes the caption template such that the number of the
  table or figure is also shown.
\end{command}

\begin{command}{\beamertemplatecaptionownline}
  Changes the caption template such that the word ``Table''
  or ``Figure'' has its own line.
\end{command}



\paragraph{Template Installation Commands}\

\begin{command}{\usecaptiontemplate\marg{caption template}}
  \example
\begin{verbatim}
\usecaptiontemplate{
  \small
  \structure{\insertcaptionname~\insertcaptionnumber:}
  \insertcaption
}
\end{verbatim}
\end{command}



\paragraph{Inserts for these Templates}\
 
\begin{command}{\insertcaption}
  Inserts the text of the current caption into a template.
\end{command}

\begin{command}{\insertcaptionname}
  Inserts the name of the current caption into a template. This word
  is either ``Table'' or ``Figure'' or, if the |babel| package is
  used, some translation thereof.
\end{command}

\begin{command}{\insertcaptionnumber}
  Inserts the number of the current figure or table into a template.
\end{command}






\subsubsection{Lists (Itemizations, Enumerations, Descriptions)}

\label{section-template-enumerate}

\paragraph{Predefined Templates}\

\begin{command}{\beamertemplateballitem}
  Changes the symbols shown in an |itemize| and an |enumerate|
  environment to small plastic balls.
\end{command}

\begin{command}{\beamertemplatedotitem}
  Changes the symbols shown in an |itemize|
  environment to dots.
\end{command}

\begin{command}{\beamertemplatetriangleitem}
  Changes the symbols shown in an |itemize|
  environment to triangles.
\end{command}

\begin{command}{\beamertemplateenumeratealpha}
  Changes the labels of first-level enumerations to ``1.'', ``2.'',
  ``3.'', and so on, and to ``1.1'', ``1.2'', ``1.3'', and so on for
  the second level.
\end{command}



\paragraph{Template Installation Commands}\

\begin{command}{\useenumerateitemtemplate\marg{template}}
  The \meta{template} is used to render the default item in the top
  level of an enumeration. 
  \example |\useenumerateitemtemplate{\insertenumlabel}|
\end{command}


\begin{command}{\useitemizeitemtemplate\marg{template}}
  The \meta{template} is used to render the default item in the top
  level of an itemize list.
  \example |\useitemizeitemtemplate{\pgfuseimage{mybullet}}|
\end{command}


\begin{command}{\usesubitemizeitemtemplate\marg{template}}
  The \meta{template} is used to render the default item in the
  second level of an itemize list.
  \example |\usesubitemizeitemtemplate{\pgfuseimage{mysubbullet}}|
\end{command}

\begin{command}{\usesubsubitemizeitemtemplate\marg{template}}
  The \meta{template} is used to render the default item in the
  third level of an itemize list.
  \example |\usesubsubitemizeitemtemplate{\pgfuseimage{mysubsubbullet}}|
\end{command}

\begin{command}{\useitemizetemplate\marg{begin text}\marg{end text}}
  The \meta{begin text} is inserted at the beginning of a top-level
  itemize list, the \meta{end text} at its end.
  \example |\useitemizetemplate{}{}|
\end{command}

\begin{command}{\usesubitemizetemplate\marg{begin text}\marg{end text}}
  The \meta{begin text} is inserted at the beginning of a second-level
  itemize list, the \meta{end text} at its end.
  \example |\usesubitemizetemplate{\begin{small}}{\end{small}}|
\end{command}

\begin{command}{\usesubsubitemizetemplate\marg{begin text}\marg{end text}}
  The \meta{begin text} is inserted at the beginning of a third-level
  itemize list, the \meta{end text} at its end.
  \example |\usesubitemizetemplate{\begin{footnotesize}}{\end{footnotesize}}|
\end{command}


\begin{command}{\useenumerateitemtemplate\marg{template}}
  The \meta{template} is used to render the default item in the
  top-level of an enumeration.  
  \example
  |\useenumerateitemtemplate{\insertenumlabel}|
\end{command}

\begin{command}{\useenumerateitemminitemplate\marg{template}}
  The \meta{template} is used to render the items in an enumeration
  where the optional argument \meta{mini template} is used (see
  Section~\ref{section-enumerate}). 
  \example
  |\useenumerateitemminitemplate{\color{structure}\insertenumlabel}|
\end{command}

\begin{command}{\usesubenumerateitemtemplate\marg{template}}
  The \meta{template} is used to render the default item in the second
  level of an enumeration. 
  \example
  |\usesubenumerateitemtemplate{\insertenumlabel-\insertsubenumlabel}|
\end{command}

\begin{command}{\usesubsubenumerateitemtemplate\marg{template}}
  The \meta{template} is used to render the default item in the third
  level of an enumeration. 
  \example
\begin{verbatim}
\usesubsubenumerateitemtemplate
{\insertenumlabel-\insertsubenumlabel-\insertsubsubenumlabel}
\end{verbatim}
\end{command}


\begin{command}{\useenumeratetemplate\marg{begin text}\marg{end text}}
  The \meta{begin text} is inserted at the beginning of a top-level
  enumeration, the \meta{end text} at its end.
  \example |\useenumeratetemplate{}{}|
\end{command}

\begin{command}{\usesubenumeratetemplate\marg{begin text}\marg{end text}}
  The \meta{begin text} is inserted at the beginning of a second-level
  enumeration, the \meta{end text} at its end.
  \example |\usesubenumeratetemplate{\begin{small}}{\end{small}}|
\end{command}

\begin{command}{\usesubsubenumeratetemplate\marg{begin text}\marg{end text}}
  The \meta{begin text} is inserted at the beginning of a third-level
  enumeration, the \meta{end text} at its end.
  \example |\usesubsubenumeratetemplate{\begin{footnotesize}}{\end{footnotesize}}|
\end{command}

\begin{command}{\usedescriptiontemplate\marg{description
      template}\marg{default width}}
  The \meta{default width} is used as width of the default item, if no
  other width is specified; the width |\labelsep| is
  automatically added to this parameter.
  \example
  |\usedescriptionitemtemplate{\color{structure}\insertdescriptionitem}{2cm}|
\end{command}


\paragraph{Inserts for these Templates}\

\begin{command}{\insertdescriptionitem}
  Inserts the current item of a description environment into a
  template.
\end{command}

\begin{command}{\insertenumlabel}
  Normally, this command inserts the current number of the top-level
  enumeration (as an Arabic number) into a template. However, in an
  enumeration where the optional \meta{mini template} option is used,
  this command inserts the current number rendered by this mini
  template. For example, if the \meta{mini template} is |(i)| and this
  command is used in the fourth item, |\insertenumlabel| would yield
  |(iv)|. 
\end{command}

\begin{command}{\insertsubenumlabel}
  Inserts the current number of the second-level enumeration (as an
  Arabic number) into a template.
\end{command}

\begin{command}{\insertsubsubenumlabel}
  Inserts the current number of the third-level enumeration (as an
  Arabic number) into a template.
\end{command}





\subsubsection{Hilighting Commands}


\paragraph{Template Installation Commands}\

\begin{command}{\usealerttemplate\marg{alert template
      begin}\marg{alert template end}}
  In an |\alert| command and in an |alertenv| environment, the text
  \meta{alert template begin} is inserted at the beginning, the text
  \meta{alert template end} at the end.
  
  \example |\usealerttemplate{\color{red}}{}|

  \articlenote
  This command is also available in |article| mode.
\end{command}

\begin{command}{\usestructuretemplate\marg{structure template
      begin}\marg{structure template end}}
  Same as for alerts.
  
  \example |\usestructuretemplate{\color{blue}}{}|

  \articlenote
  This command is also available in |article| mode.
\end{command}




\subsubsection{Block Environments}

\paragraph{Predefined Templates}\

\begin{command}{\beamertemplateboldblocks}
  Block titles are printed in bold.
\end{command}

\begin{command}{\beamertemplatelargeblocks}
  Block titles are printed slightly larger.
\end{command}

\begin{command}{\beamertemplateroundedblocks}
  Changes the block templates such that they are printed on a
  rectangular area with rounded corners.
\end{command}

\begin{command}{\beamertemplateshadowblocks}
  Changes the block templates such that they are printed on a
  rectangular area with rounded corners and a shadow.
\end{command}



\paragraph{Template Installation Commands}\

\begin{command}{\useblocktemplate\marg{block beginning
      template}\marg{block end template}}
  \example
\begin{verbatim}
\useblocktemplate
  {%
   \medskip%
    {\color{blockstructure}\textbf{\insertblockname}}%
    \par%
  }
  {\medskip}
\end{verbatim}

  \articlenote
  This command is also available in |article| mode.
\end{command}


\begin{command}{\usealertblocktemplate\marg{block beginning
      template}\marg{block end template}}
  \example
\begin{verbatim}
\usealertblocktemplate
  {%
    \medskip
    {\alert{\textbf{\insertblockname}}}%
  \par}
  {\medskip}
\end{verbatim}

  \articlenote
  This command is also available in |article| mode.
\end{command}


\begin{command}{\useexampleblocktemplate\marg{block beginning
      template}\marg{block end template}}
  \example
\begin{verbatim}
\useexampleblocktemplate
  {%
    \medskip
    \begingroup\color{darkgreen}{\textbf{\insertblockname}}
    \par}
  {%
     \endgroup
     \medskip
  }
\end{verbatim}

  \articlenote
  This command is also available in |article| mode.
\end{command}


\paragraph{Inserts for these Templates}\

\begin{command}{\insertblockname}
  Inserts the name of the current block into a template.
\end{command}



\subsubsection{Theorem Environments}

\label{section-theorems-templates}

\paragraph{Predefined Templates}\

\begin{command}{\beamertemplatetheoremssimple}
  Causes the theorem head and text to be directly passed to the
  |block| or |exampleblock| environment. All font specifications for
  theorems are ignored. 
\end{command}

\begin{command}{\beamertemplatetheoremsunnumbered}
  Causes theorems to be typeset as follows: The font specification for
  the body is honored, the font specification for the head is
  ignored. No theorem number is printed. This is the default.
\end{command}

\begin{command}{\beamertemplatetheoremsnumbered}
  Like |\beamertemplatetheoremsunnumbered|, except that the theorem
  number is printed for environments that are numbered.
\end{command}

\begin{command}{\beamertemplatetheoremsamslike}
  This causes theorems to be put in a |block| or |exampleblock|, but
  to be otherwise typeset as is normally done in |amsthm|. Thus the
  head font and body font depend on the setting for the theorem to be
  typeset and theorems are numbered. 
\end{command}



\paragraph{Template Installation Commands}\

\begin{command}{\usetheoremtemplate\marg{block beginning
      template}\marg{block end template}}
  \beamernote
  Whenever an environment declared using the command |\newtheorem| is
  to be typeset, the \meta{block beginning template} is inserted at
  the beginning and the \meta{block end template} at the end. If there
  is a overlay specifciation when an environment like |theorem| is
  used, this overlay specifciation will directly follow the
  \meta{block beginning template} upon invocation. This is even true
  if there was an optional argument to the |theorem| environment. This
  optional argument is available via the insert |\inserttheoremaddition|.

  Numerous inserts are available in this template, see below.  

  Before the template starts, the font is set to the body font
  prescribed by the environment to be typeset.
  
  \example The following typesets theorems like |amsthm|:
\begin{verbatim}
\usetheoremtemplate{\begin{\inserttheoremblockenv}
  {%
    \inserttheoremheadfont
    \inserttheoremname
    \inserttheoremnumber
    \ifx\inserttheoremaddition\@empty\else\ (\inserttheoremaddition)\fi%
    \inserttheorempunctuation
  }%
}{\end{\inserttheoremblockenv}}
\end{verbatim}

  \example In the following example, all font ``suggestions'' for the
  environment are suppressed or ignored; and the theorem number is
  suppressed.
\begin{verbatim}
\usetheoremtemplate{%
  \normalfont% ignore body font
  \begin{\inserttheoremblockenv}
  {%
    \inserttheoremname
    \ifx\inserttheoremaddition\@empty\else\ (\inserttheoremaddition)\fi%
  }%
}{\end{\inserttheoremblockenv}}
\end{verbatim}
  
  \articlenote
  This command is not available in |article| mode.
\end{command}


\paragraph{Inserts for these Templates}\

\begin{command}{\inserttheoremblockenv}
  This will normally expand to |block|, but if a theorem that has
  theorem style |example| is typeset, it will expand to
  |exampleblock|. Thus you can use this insert to decide which
  environment should be used when typesetting the theorem.
\end{command}

\begin{command}{\inserttheoremheadfont}
  This will expand to a font chainging command that switches to the
  font to be used in the head of the theorem. By not inserting it, you
  can ignore the head font.
\end{command}

\begin{command}{\inserttheoremname}
  This will expand to the name of the environment to be typeset (like
  ``Theorem'' or ``Corollary''). 
\end{command}


\begin{command}{\inserttheoremnumber}
  This will expand to the number of the current theorem preceeded by a
  space or to nothing, if the current theorem does not have a number.
\end{command}

\begin{command}{\inserttheoremaddition}
  This will expand to the optional argument given to the environment
  or will be empty, if there was no optional argument.
\end{command}

\begin{command}{\inserttheorempunctuation}
  This will expand to the punctuation character for the current
  environment. This is usually a period.
\end{command}





\subsubsection{Verse, Quotation and Quote Environments}


\paragraph{Template Installation Commands}\

\begin{command}{\usetemplateverse\marg{block beginning
      template}\marg{block end template}}
  In a |verse| environment, the \meta{block beginning template} is
  inserted before the verse, the \meta{block end template} after the
  verse. The margins are not setup in these templates; this is done in
  the |verse| environment and cannot be changed.
  \example |\usetemplateverse{\rmfamily\itshape}{}|
\end{command}


\begin{command}{\usetemplatequotation\marg{block beginning
      template}\marg{block end template}}
  Both in |quotation| and in |quote| environments, the \meta{block
    beginning template} is inserted before the quotation, the
  \meta{block end template} after the quotation. As for verses, the
  margins are not setup in these templates and cannot be changed.
  
  \example |\usetemplatequotation{\itshape}{}|
\end{command}




\subsubsection{Typesetting Notes}

\label{section-note-templates}

\paragraph{Predefined Templates}\

\begin{command}{\beamertemplatenoteplain}
  Causes all note pages to contain only the note text.
\end{command}

\begin{command}{\beamertemplatenotecompress}
  Causes the ``routing information'' at the top of a note to be
  smaller. 
\end{command}


\paragraph{Template Installation Commands}\

\begin{command}{\usetemplatenote\marg{note template}}
  Each note is typeset by inserting the \meta{note template}. The
  template should contain a mentioning of the insert |\insertnote|,
  which will contain the note text.
  
  \example |\usetemplatenote{\tiny\insertnote}|
\end{command}



\paragraph{Inserts for these Templates}\

\begin{command}{\insertnote}
  Inserts the text of the current note into the template.
\end{command}


\begin{command}{\insertslideintonotes\marg{magnification}}
  Inserts a ``mini picture'' of the last slide into the current
  note. The slide will be scaled by the given magnification.

  \example |\insertslideintonotes{0.25}|

  This will give a mini slide whose width and height are one fourth of
  the usual size.
\end{command}




\section{Tips and (Dirty) Tricks}

The aim of this section is to collect some hints and tricks that make
use of the basic \beamer-class concepts.



\subsection{Piecewise Uncovering}

\subsubsection{Uncovering an Enumeration Piecewise}

A common usage of overlays is to show a list of points in an
enumeration in a piecewise fashion. The easiest and most flexible way
to do this is the following:

\begin{verbatim}
\begin{itemize}
\item<1-> First point.
\item<2-> Second point.
\item<3-> Third point.
\end{itemize}
\end{verbatim}

The advantage of this approach is that you retain total control over
the order in which items are shown. By chaning, for example, the last
specification to |<2->|, you can have the last two points uncovered at
the same time.

A disadvantage of the approach is that you will have to renumber
everything if you add a new item. This is usually not such a big
problem, but it can be a nuiseance.

To automize the uncovering, you can use the following code:

\begin{verbatim}
\begin{itemize}[<+->]
\item First point.
\item Second point.
\item Third point.
\end{itemize}
\end{verbatim}

The effect of the |[<+->]| is to install a \emph{default overlay
  specification}, see the definition of |itemize| for details.

Now, suppose you wish the second and third point to be shown at the
same time. You could achieve this by adding the specificaiton |<2->|
to either the second or third |\item| command. However, then you still
have to do some renumbering if you add a new item at the beginning. A
better, though more cumbersome, approach is to decrease the counter
|beamerpause| before the last item:

\begin{verbatim}
\begin{itemize}[<+->]
\item First point.
\item Second point.
  \addtocounter{beamerpause}{-1}
\item Third point.
\end{itemize}
\end{verbatim}

This does not look so nice, but it works. Also, you might wish to
build your own macros based on these ideas (like an |itemstep|
environment or a |\itemlikeprevious| command).



\subsubsection{Hilighting the Current Point in an Enumeration}

If you uncover an enumeration piecewise, it is sometimes a good idea
to hilight the last uncovered point to draw the audience's attention
to it. This is best achieved as follows:


\begin{verbatim}
\begin{itemize}
\item<1-| alert@1> First point.
\item<2-| alert@2> Second point.
\item<3-| alert@3> Third point.
\end{itemize}
\end{verbatim}

or

\begin{verbatim}
\begin{itemize}[<+-| alert@+>]
\item First point.
\item Second point.
\item Third point.
\end{itemize}
\end{verbatim}

Note that this will draw the little item symbol also in red.



\subsubsection{Changing Symbol Before an Enumeration}

When uncovering a list of tasks or problems, you may desire that the
symbol in front of the last uncovered symbol is, say, an ballot X,
while for the previous items it is a check mark (you'll find these
characters in some Dingbats fonts).

The best way to achieve this is to implement a new action
environment. If this action is activated, it temporarily changes the
item symbol template to the other symbol:

\begin{verbatim}
\newenvironment{ballotenv}
{\only{%
  \useitemizeitemtemplate{code for showing a ballot}%
  \usesubitemizeitemtemplate{code for showing a smaller ballot}%
  \usesubsubitemizeitemtemplate{code for showing a smaller ballot}}}
{}

\useitemizeitemtemplate{code for showing a check mark}
\usesubitemizeitemtemplate{code for showing a smaller check mark}
\usesubsubitemizeitemtemplate{code for showing a smaller check mark}
\end{verbatim}

The effect of the code is to install a check mark as the default
template. If the action |ballot| is now requested for some item, this
template will temporarily be replaced by the ballot templates.

Note that the |ballotenv| is invoked with the overlay specification
given for the action directly following it. This causes the |\only| to
be invoked exactly for the specified overlays.

Here are example usages:

\begin{verbatim}
\begin{itemize}
\item<1-| ballot@1> First point.
\item<2-| ballot@2> Second point.
\item<3-| ballot@3> Third point.
\end{itemize}
\end{verbatim}

and

\begin{verbatim}
\begin{itemize}[<+-| ballot@+>]
\item First point.
\item Second point.
\item Third point.
\end{itemize}
\end{verbatim}

In the following example, more and more items become ``checked'' from
slide to slide:

\begin{verbatim}
\begin{itemize}[<ballot@+-| visible@1-,+(1)>]
\item First point.
\item Second point.
\item Third point.
\end{itemize}
\end{verbatim}

The important point is |ballot@+|. The funny |visible@1-,+(1)| has the
following effect: Although it has no effect with respect to what is
shown (after all, it applies to all slides), it ensures that in
the enumeration the slide number 4 is mentioned. Thus there will also
be a slide in which all three points are checked.


\subsubsection{Uncovering Tagged Formulas Piecewise}

Suppose you have a three line formula as the following:
\begin{verbatim}
\begin{align}
  A &= B \\
    &= C \\
    &= D
\end{align}
\end{verbatim}

Uncovering this formula line-by-line is a little tricky. A first idea
is to use the |\pause| or |\onslide| commands. Unfortunately, these do
not work since |align| internally reprocesses its input several times,
which messes up the delicate internals of the commands. The next idea
is the following, which works a little better:
\begin{verbatim}
\begin{align}
  A &= B \\
    \uncover<2->{&= C \\}
    \uncover<3->{&= D}
\end{align}
\end{verbatim}
Unfortunately, this does not work in the presence of tags (so it works
for the |align*| environment). What happens is that the tag of the
last line is shown on all slides. The problem here is that the tag is
created when |\\| is encountered or when |\end{align}| is
encountered. In the last line these are already ``behind'' the
|\uncover|.

To solve this problem, you can add an empty line without a tag and then
insert a negative vertical skip to undo the last line:
\begin{verbatim}
\begin{align}
  A &= B \\
    \uncover<2->{&= C \\}
    \uncover<3->{&= D \\}
    \notag
  \end{align}
\vskip-1.5em
\end{verbatim}


\subsubsection{Uncovering a Table Linewise}

When you wish to uncover a table line-by-line, you will run into all
sorts of problems if there are vertical and horizontal lines in the
table. The reason is that the first vertical line at the left end is
drawn before the line is even read (and thus, in particular, before
any |\onslide| command can be read). However, placing a |\pause| or
|\uncover| at the end of the line before is also not helpful since it
will then suppress the horizontal line below the last uncovered line.

A possible way to solve this problem is not to use either horizontal
or vertical lines. Instead, colouring the lines using the |colortbl|
package is a good alternative to structure the table. Here is an
optically pleasing example, where the table is uncovered line-wise:

\begin{verbatim}
\rowcolors[]{1}{blue!20}{blue!10}
\begin{tabular}{l!{\vrule}cccc}
  Class & A & B & C & D \\\hline
  X     & 1 & 2 & 3 & 4 \pause\\
  Y     & 3 & 4 & 5 & 6 \pause\\
  Z     & 5 & 6 & 7 & 8
\end{tabular}
\end{verbatim}

By using |\onslide| instead of |\pause|, you can get more fine-grained
control over which line is shown on which slide.


\subsubsection{Uncovering a Table Columnwise}

The same problems as for uncovering a table linewise arise for
uncovering it columnwise.

Once more, using the |colortbl| package offers a solution. In the
following example, the |tabular| header is used to insert |\onslide|
commands, one for each column, that cover the entries in the colomn
from a certain slide on. At the end of the last column, the |\onslide|
without a specification ensures that the first column on the next row
is once more shown normally.

Inserting a horizontal line is tricky since it will protrude over the
full width of the table also in the covered version. The best idea is
just not to use horizontal bars.

\begin{verbatim}
\rowcolors[]{1}{blue!20}{blue!10}
\begin{tabular}{l!{\vrule}c<{\onslide<2->}c<{\onslide<3->}c<{\onslide<4->}c<{\onslide}c}
  Class & A & B & C & D \\
  X     & 1 & 2 & 3 & 4 \\
  Y     & 3 & 4 & 5 & 6 \\
  Z     & 5 & 6 & 7 & 8
\end{tabular}
\end{verbatim}




\section{Compatibility and Emulation}


\subsection{Compatibility with Other Packages and Classes}

When using certain packages or classes together with the |beamer|
class, extra options or precautions may be necessary.

\begin{package}{{amsthm}}
  This package is automatically loaded since \beamer\ uses it for
  typesetting theorems. If you do not wish it to be loaded, which can
  be necessary especially in |article| mode if the package is
  incompatible with the document class, you can use the class option
  |noamsthm| to suppress its loading. See
  Section~\ref{section-theorems} for more details.
\end{package}

\begin{package}{{babel}|[|\declare{|french|}|]|}
  When using the French style, certain features that clash with the
  functionality of the beamer class will be turned off. For example,
  enumerations are still produced the way the theme dictates, not the
  way the French style does. Also, the characters |:| and |!| will not
  be a active characters. This means, that the little space that is
  inserted before them in the |french| style is not inserted. You have
  to do this  ``by hand.''

  \articlenote
  To make the characters |:| and |!| active in |article| mode, pass
  the option |activeospeccharacters| to the package
  |beamerbasearticle|. However, this may lead to problems with overlay
  specifications.
\end{package}

\begin{package}{{babel}|[|\declare{|spanish|}|]|}
  \beamernote
  When using the Spanish style, certain features that clash with the
  functionality of the beamer class will be turned off. In particular,
  the special behaviour of the pointed brackets |<| and |>| is
  deactivated. 

  \articlenote
  To make the characters |<| and |>| active in |article| mode, pass
  the option |activeospeccharacters| to the package
  |beamerbasearticle|. As for the |french| package, this may lead to
  problems with overlay specifications.
\end{package}

\begin{package}{{color}}
  \beamernote
  The |color| package is automatically loaded by |beamer.cls|. This
  makes it impossible to pass options to |color| in the preamble of
  your document. To pass a \meta{list of options} to |color|, you must
  use the following class option:

  \begin{classoption}{{color={\normalfont\meta{list of options}}}}
    Causes the \meta{list of options} to be passed on to the |color|
    package. If the \meta{list of options} contains more than one
    option you must enclose it in curly brackets.
  \end{classoption}

  \articlenote
  The |color| package is not loaded automatically if
  |beamerbasearticle| is loaded with the |noxcolor| option.
\end{package}

\begin{package}{{CJK}}
  \beamernote
  When using the |CJK| package for using Asian fonts, you must use the
  class option \declare{|CJK|}. See |beamerexample4.tex| for an
  example. 
\end{package}

\begin{package}{{deluxetable}}
  \beamernote
  The caption generation facilities of |deluxetable| are
  deactivated. Instead, the caption template is used.
\end{package}

\begin{package}{{enumerate}}
  \articlenote
  This package is loaded automatically in the |presentation| modes, but not
  in the |article| mode. If you use its features, you have to load the
  package ``by hand'' in the |article| mode.
\end{package}

\begin{class}{{foils}}
  If you wish to emulate the |foils| class using \beamer, please see
  Section~\ref{section-foiltex}.
\end{class}

\begin{package}{{fontenc}|[|\declare{|T1|}|]|}
  Use this option only with fonts that have outline fonts available in
  the T1 encoding like Times or the |lmodern| fonts. In a standard
  installation the standard Computer Modern fonts (the fonts Donald
  Knuth originally designed and which are used by default) are
  \emph{not} available in the T1 encoding. Using this  option with
  them will result in very poor rendering of your presentation when
  viewed with \pdf\ viewer applications like Acrobat or |xpdf|. To use
  the Computer Modern fonts with the T1 encoding, use the package
  |lmodern|.  See also Section~\ref{section-font-encoding}.
\end{package}

\begin{package}{{fourier}}
  The package switches to a T1~encoding, but it does not redefine all
  fonts such that outline fonts (non-bitmapped fonts) are used by
  default. For example, the sans-serif text and the typewriter text
  are not replaced. To use outline fonts for these, write
  |\usepackage{lmodern}| \emph{before} including the |fourier|
  package. 
\end{package}

\begin{package}{{HA-prosper}}
  You cannot use this package with \beamer. However, you might try to
  use the package |beamerprosper| instead, see
  Section~\ref{section-prosper}. 
\end{package}

\begin{package}{{hyperref}}
  \beamernote
  The |hyperref| package is automatically loaded by |beamer.cls| and
  certain options are setup. In order pass additional options to
  |hyperref| or to override options, you can use the following class
  option: 

  \begin{classoption}{{hyperref={\normalfont\meta{list of options}}}}
    Causes the \meta{list of options} to be passed on to the |hyperref|
    package.

    \example |\documentclass[hyperref={bookmarks=false}]{beamer}|
  \end{classoption}

  Alternatively, you can also use the |\hypersetup| command.

  \articlenote
  In the |article| version, you must include |hyperref| manually if
  you want to use it. It is not included automatically.
\end{package}

\begin{package}{{inputenc}|[|\declare{|utf8|}|]|}
  \beamernote
  When using Unicode, you may wish to use one of the following class
  options: 
  \begin{classoption}{{ucs}}
    Loads the package |ucs| and passes the correct Unicode options to
    |hyperref|. Also, it preloads the Unicode code pages zero and
    one.
  \end{classoption}
  
  \begin{classoption}{{utf8}}
    Same as the option |ucs|, but also sets the input encoding to
    |utf8|. You could also use the option |ucs| and say
    |\usepackage[utf8]{inputenc}| in the preamble.
  \end{classoption}

  If you use a Unicode character outside the first two code pages
  (which includes the Latin alphabet and the extended Latin alphabet)
  in a section or subsection heading, you have to use the command
  |\PreloadUnicodePage{|\meta{code  page}|}| to give |ucs| a chance to
  preload these code pages. You will know that a character has not
  been preloaded, if you get a message like ``Please insert into
  preamble.'' The code page of a character is given by the unicode
  number of the character divided by 256.
\end{package}

\begin{package}{{listings}}
  \beamernote
  Note that you must treat |lstlisting| environments exactly the same
  way as you would treat |verbatim| environments. When using
  |\defverbatim| that contains a colored |lstlisting|, use the
  |colored| option of |\defverbatim|.
\end{package}

\begin{package}{{\normalfont\meta{professional font package}}}
  \beamernote
  If you use a professional font package, \beamer's internal
  redefinition of how variables are typeset may interfere with the
  font package's superior way of typesetting them. In this case, you
  should use the class option |professionalfont| to suppress any font
  substitution. See Section~\ref{section-substition} for details.
\end{package}

\begin{class}{{prosper}}
  If you wish to (partly) emulate the |prosper| class using \beamer,
  please see Section~\ref{section-prosper}.
\end{class}

\begin{package}{{pstricks}}
  You should add the option |xcolor=pst| to make |xcolor| aware of the
  fact that you are using |pstricks|.
\end{package}

\begin{class}{{seminar}}
  If you wish to emulate the |seminar| class using \beamer, please see
  Section~\ref{section-seminar}.
\end{class}

\begin{package}{{texpower}}
  You cannot use this package with \beamer. However, you might try to
  use the package |beamertexpower| instead, see
  Section~\ref{section-texpower}.  
\end{package}

\begin{package}{{textpos}}
  \beamernote
  \beamer\ automatically installs a white background behind
  everything, unless you install a different background
  template. Because of this, you must use the |overlay| option when
  using |textpos|, so that it will place boxes \emph{before}
  everything. Alternatively, you can install an empty background
  template, but this may result in an incorrect display in certain
  situtations with older versions of the Acrobat Reader. 
\end{package}

\begin{package}{{ucs}}
  See |\usepackage[utf8]{inputenc}|.
\end{package}


\begin{package}{{xcolor}}
  \beamernote
  The |xcolor| package is automatically loaded by |beamer.cls|. The
  same applies as to |color|.

  \begin{classoption}{{xcolor={\normalfont\meta{list of options}}}}
    Causes the \meta{list of options} to be passed on to the |xcolor|
    package.
  \end{classoption}

  When using \beamer\ together with the |pstricks| package, be sure to
  pass the |xcolor=pst| option to \beamer\ (and hence to |xcolor|).

  \articlenote
  The |color| package is not loaded automatically if
  |beamerbasearticle| is loaded with the |noxcolor| option.
\end{package}



\subsection{Emulation of other Classes}

\beamer\ is a powerful class. In some ways it is more powerful
than other classes around, in others it is less powerful or harder to
use. If \beamer\ implements all or most of the features some other
class also offers, albeit under a different name, it is possible to
\emph{emulate} that class using \beamer. Such an emulation is not a
perfect substitute for the original (emulations seldomly are), but it
can help speed up creating \beamer\ presentations that use  parts of
old presentations. You can simply copy these parts in verbatim,
without having to worry about the subtle differences in syntax.

A useful effect of using an emulation layer is that you get access to
all the features of \beamer\ while using the syntax of another
class. For example, you can use the |article| mode to create a
nice article version of a \prosper\ talk.



\subsubsection{Prosper and HA-Prosper}
\label{section-prosper}

The package |beamerprosper| maps the commands of the \prosper\
package, developped by Fr\'ed\'eric Goualard, to \beamer\
commands. Also, some commands of the \textsc{ha}-\prosper\ package,
developped by Hendri Adriaens, are mapped to \beamer\ commands. 
\emph{These mappings cannot perfectly emulate all of Prosper!} Rather,
these mappings are intended as an aide when porting presentations
created using \prosper\ to \beamer. \emph{No styles are implemented
that mimick Prosper styles.} Rather, the normal \beamer\ themes must
be used (although, one could implement \beamer\ themes that mimicks
existing \prosper\ styles; I have not done that and do not intend to).

The workflow for the migration is the following:
\begin{enumerate}
\item
  Replace the document class |prosper| by |beamer|. Most options
  passed to |prosper| do not apply to |beamer| and should be omitted.
\item
  Add a |\usepackage{beamerprosper}| to start the emulation.
\item
  If you are using \textsc{ha}-\prosper, delete the
  |\usepackage{HA-prosper}|. You may wish to add the option
  |framesassubsections| to |beamerprosper|, though I do not recommend
  it (use the |\subsection| command instead; it gives you more
  fine-grained control).
\item
  Possibly add commands to install themes and templates.
\item
  It may be necessary to adjust the content of commands like |\title|
  or |\author|. Note that in \prosper\ the |\email| command is given
  outside the |\author| command, whereas in \beamer\ and also in
  \textsc{ha}-\prosper\ it is given inside.
\item
  In the main text, you will almost surely have to adjust usages of
  |\includegraphics|. If you use pdf\LaTeX\ to typeset the
  presentation, than you cannot include PostScript file. You should
  convert them to |.pdf| or to |.png| and adjust any usage of
  |\includegraphics| accordingly.
\item
  When starting to change things, you can use all of \beamer's
  commands and even mix them with \prosper\ commands.
\end{enumerate}

An example can be found in the file
|beamerexample-haprosper.tex|. Note that this file, except for the
changes at the beginning, is due to Hendri Adriaens.

There are, unfortunately, quite a few places where you may run into
problems:
\begin{itemize}
\item
  In \beamer, the command |\PDForPS| will do exactly what the name
  suggests: insert the first argument when run by |pdflatex|, insert
  the second argument when run by |latex|. However, in \prosper, the
  code inserted for the \pdf\ case is acutally PostScript code, which
  is only later converted to \pdf\ by some external program. You will
  need to adjust this PostScript code such that it works with
  |pdflatex| (which is not always possible).
\item
  If you used fine-grained spacing commands, like adding a little
  horizontal skip   here and a big negative vertical skip there, the
  typesetting of the text may be poor. It may be a good idea to just
  remove these spacing commands.
\item
  If you use |pstricks| commands, you will either have to stick to
  using |latex| and |dvips| or will have to work around them using,
  for example, |pgf|. Porting lot's of |pstricks| code is bound to be
  difficult, if you wish to switch over to |pdflatex|, so be warned.
\item
  If the file cannot be compiled because some \prosper\ command is not
  implemented, you will have to delete this command and try to mimick
  its behaviour using some \beamer\ command.
\end{itemize}


\begin{package}{{beamerprosper}}
  Include this package in a |beamer| presentation to get access to
  \prosper\ commands. Use |beamer| as the document class, not
  |prosper|. Most of the options passed to the class |prosper| make no
  sense in |beamer|, so just delete them.

  This package takes the following options:
  \begin{itemize}
  \item
    \declare{|framesassubsections|} causes each frame to create its
    own subsection with the frame title as subsection name. This
    behaviour mimicks \textsc{ha}-\textsc{prosper}'s behaviour. In a
    long talk this will create way too many subsections. 
  \end{itemize}

  \articlenote
  The |framesassubsections| option has no effect in |article| mode.

  \example
  Consider the following original \prosper\ file (adapted from an
  example by Hendri Adriaens):
\begin{verbatim}
\documentclass[pdf]{prosper}
\usepackage[toc,highlight,HA,notes,portrait,hlsections]{HA-prosper}

\title{Example for the HA-prosper package}
\subtitle{A package for use with prosper}

\author{Hendri Adriaens}

\DefaultTransition{Wipe}
\TitleSlideNav{FullScreen}
\NormalSlideNav{ShowBookmarks}
\LeftFoot{\href{http://center.uvt.nl/phd_stud/adriaens}{Hendri Adriaens}, \today}
\RightFoot{Example for the HA-prosper package}

\begin{document}
\maketitle

\begin{slide}{Introduction}
\begin{itemize}
\item Welcome to the example for the HA-prosper package.
\item This example demonstrates some of the possibilities of HA-prosper.
\item See the style-specific examples for a demonstration of
features implemented by a style.
\end{itemize}
\end{slide}

\overlays{2}{
\begin{slide}{Numbering and overlays}
\begin{itemstep}
\item This overlay contains an equation:
\begin{equation}
\label{eq:1}
(a+b)^n=\sum_{k=0}^n\left(\begin{array}{l}n\\k\end{array}\right)a^{n-k}b^k
\end{equation}
\item It is equation number~\ref{eq:1}.
\end{itemstep}
\end{slide}
}

\begin{notes}{Notes for these slides}
My notes for these slides.
\end{notes}
\end{document}
\end{verbatim}

  To port this example to \beamer, the first two lines should be
  replaced as follows:
\begin{verbatim}
\documentclass{beamer}
\usepackage{beamerprosper}
\end{verbatim}

  Everything else can stay the same. You can now run, for example,
  pdf\LaTeX\ on the file to get a \beamer\ presentation with
  overlays. Adding the |notes| option will also show the note. Certain
  commands, like |\LeftFoot|, are ignored. You can change the theme
  using the usual commands; for example |beamerthemesidebar| is quite
  ``near'' to Adriaens' original theme. You can also use all normal
  \beamer\ commands and concepts, like overlay-specifications, in
  the file. You can also create an |article| version by adding the
  class option |class=article| and including the package
  |beamerbasearticle|. 
\end{package}

In the following, the effect of \prosper\ commands in \beamer\ are
listed.

\begin{command}{\subtitle\marg{title}}
  Adds a subtitle by adding a new line to an existing title with the
  given \meta{title} typeset in a smaller font. 
\end{command}

\begin{command}{\email\marg{text}}
  Simply typesets its argument in typewriter text. Should hence be
  given \emph{inside} the |\author| command.
\end{command}

\begin{command}{\institution\marg{text}}
  This command is mapped to \beamer's |\institute|
  command if given \emph{outside} the |\author| command, otherwise it
  typesets its argument in a smaller font.
\end{command}

\begin{command}{\Logo\opt{|(|\meta{x}|,|\meta{y}|)|}\marg{logo text}}
  This is mapped to |\logo{|\meta{logo text}|}|. The coordinates are ignored.
\end{command}

\begin{environment}{{slides}\oarg{options}\marg{frame title}}
  Inserts a frame with the |containsverbatim| option set. The
  \meta{frame title} will be enclosed in a |\frametitle| command.

  The following \meta{options} may be given:
  \begin{itemize}
  \item
    \declare{|trans=|\meta{prosper transition}} installs the specified
    \meta{prosper transition} as the transition effect when showing
    the slide.
  \item
    \declare{\meta{prosper transition}} has the same effect as
    |trans=|\meta{prosper transition}.
  \item
    \declare{|toc=|\meta{entry}} overrides the subsection table of
    contents entry created by this slide by \meta{entry}. Note that a
    subsection entry is create for a slide only if the
    |framesassubsections| options is specified.
  \item
    \declare{|template|=\meta{text}} is ignored.
  \end{itemize}

  \example The following two texts have the same effect:

\begin{verbatim}
\begin{slide}[trans=Glitter,toc=short]{A Title}
  Hi!
\end{slide}
\end{verbatim}

  and

\begin{verbatim}
\subsection{short} % omitted, if framesassubsections is not specified
\frame[containsverbatim]
{
  \transglitter
  \frametitle{A Title}
  Hi!
}
\end{verbatim}
\end{environment}


\begin{command}{\overlays\marg{number}\marg{slide environment}}
  This will put the \meta{slide environment} into a frame that does
  not have the |containsverbatim| option and which can hence contain
  overlayed text. The \meta{number} is ignored since the number of
  necessary overlays is computed automatically by \beamer.

  \example The following code fragments have the same effect:
  
\begin{verbatim}
\overlays{2}{
\begin{slide}{A Title}
  \begin{itemstep}
  \item Hi!
  \item Ho!
  \end{itemstep}
\end{slide}}
\end{verbatim}

  and

\begin{verbatim}
\subsection{A Title} % omitted, if framesassubsections is not specified
\frame
{
  \frametitle{A Title}
  \begin{itemstep}
  \item Hi!
  \item Ho!
  \end{itemstep}
}
\end{verbatim}
\end{command}


\begin{command}{\fromSlide\marg{slide number}\marg{text}}
  This is mapped to |\uncover<|\meta{slide number}|->{|\meta{text}|}|.
\end{command}

\begin{command}{\fromSlide|*|\marg{slide number}\marg{text}}
  This is mapped to |\only<|\meta{slide number}|->{|\meta{text}|}|.
\end{command}

\begin{command}{\onlySlide\marg{slide number}\marg{text}}
  This is mapped to |\uncover<|\meta{slide number}|>{|\meta{text}|}|.
\end{command}

\begin{command}{\onlySlide|*|\marg{slide number}\marg{text}}
  This is mapped to |\only<|\meta{slide number}|>{|\meta{text}|}|.
\end{command}

\begin{command}{\untilSlide\marg{slide number}\marg{text}}
  This is mapped to |\uncover<-|\meta{slide number}|>{|\meta{text}|}|.
\end{command}

\begin{command}{\untilsSlide|*|\marg{slide number}\marg{text}}
  This is mapped to |\only<-|\meta{slide number}|>{|\meta{text}|}|.
\end{command}

\begin{command}{\FromSlide\marg{slide number}}
  This is mapped to |\onslide<|\meta{slide number}|->|.
\end{command}

\begin{command}{\OnlySlide\marg{slide number}}
  This is mapped to |\onslide<|\meta{slide number}|>|.
\end{command}

\begin{command}{\UntilSlide\marg{slide number}}
  This is mapped to |\onslide<-|\meta{slide number}|>|.
\end{command}

\begin{command}{\slideCaption\marg{text}}
  This is mapped to |\date{|\meta{text}|}|.
\end{command}

\begin{command}{\fontTitle\marg{text}}
  Simply inserts \meta{text}.
\end{command}

\begin{command}{\fontText\marg{text}}
  Simply inserts \meta{text}.
\end{command}

\begin{command}{\PDFtransition\marg{prosper transition}}
  Maps the \meta{prosper transition} to an appropriate |\transxxxx|
  command.
\end{command}

\begin{environment}{{Itemize}}
  This is mapped to |itemize|.
\end{environment}

\begin{environment}{{itemstep}}
  This is mapped to |itemize| with the option |[<+->]|.
\end{environment}

\begin{environment}{{enumstep}}
  This is mapped to |enumerate| with the option |[<+->]|.
\end{environment}

\begin{command}{\hiddenitem}
  This is mapped to |\addtocounter{beamerpauses}{1}|.
\end{command}

\begin{command}{\prosperpart\oarg{options}\marg{text}}
  This command has the same effect as \prosper's |\part|
  command. \beamer's normal |\part| command retains its normal
  sematics. Thus, you might wish to replace all occurences of |\part|
  by |\prosperpart|.
\end{command}


\begin{command}{\tsection\opt{|*|}\marg{section name}}
  Creates a section names \meta{section name}. The star, if present,
  is ignored.
\end{command}

\begin{command}{\tsectionandpart\opt{|*|}\marg{part text}}
  Mapped to a |\section| command followed by a |\prosperpart|
  command.

  \articlenote
  In |article| mode, no part page is added.
\end{command}

\begin{command}{\dualslide\oarg{x}\oarg{y}\oarg{z}\marg{options}\marg{left
      column}\marg{right column}}
  This command is mapped to a |columns| environment. The \meta{left
    column} text is shown in the left column, the \meta{right column}
  text is shown in the right column. The options \meta{x}, \meta{y},
  and \meta{z} are ignored. Also, all \emph{options} are ignored,
  except for \declare{|lcolwidth=|} and \declare{|rcolwidth=|}. These
  set the width of the left or right column, respectively.
\end{command}

\begin{command}{\PDForPS\marg{PostScript text}\marg{PDF text}}
  Inserts eight the \meta{PostScript text} or the \meta{PDF text},
  depending on whether |latex| or |pdflatex| is used. When porting,
  the \meta{PDF text} will most likely be \emph{incorrect}, since in
  \prosper\ the \meta{PDF text} is actually PostScript text that is
  later transformed to \pdf\ by some external program.

  If the \meta{PDF text} contains an |\includegraphics| command (which
  is its usual use), you should change the name of the graphic file
  that is included to a name ending |.pdf|, |.png|, or
  |.jpg|. Typically, you will have to convert your graphic to this
  format.
\end{command}

\begin{command}{\onlyInPDF\meta{PDF text}}
  The \meta{PDF text} is only included if |pdflatex| is used. The same
  as for the command |\PDForPS| applies here.
\end{command}

\begin{command}{\onlyInPS\meta{PS text}}
  The \meta{PS text} is only included if |latex| is used. 
\end{command}

\begin{environment}{{notes}\marg{title}}
  Mapped to |\note{\textbf{|\meta{title}|}|\meta{environment contents}|}|
  (more or less). 
\end{environment}

The following commands are parsed by \beamer, but have no effect:
\begin{itemize}\itemsep=0pt\parskip=0pt
\item |\myitem|,
\item |\FontTitle|,
\item |\FontText|,
\item |\ColorFoot|,
\item |\DefaultTransition|,
\item |\NoFrenchBabelItemize|,
\item |\TitleSlideNav|,
\item |\NormalSlideNav|,
\item |\HAPsetup|,
\item |\LeftFoot|, and
\item |\RightFoot|.
\end{itemize}




\subsubsection{Seminar}
\label{section-seminar}

The package |beamerseminar| maps a subset of the commands of the \seminar\
package to \beamer. As for \prosper, the emulation cannot be
perfect. For example, no portrait slides are supported, no
automatic page braking, the framing of slides is not
emulated. Unfortunately, for all frames (|slide| environments) that
contain overlays, you have to put the environment into a |\frame| ``by
hand'' and must remove all occurences of |\newslide| inside the
environment by closing the slide and opening a new one (and them
putting these into |\frame| commands).

The workflow for the migration is the following:
\begin{enumerate}
\item
  Replace the document class |seminar| by |beamer|. Most options
  passed to |prosper| do not apply to |beamer| and should be
  omitted. If the presentation is mixed with normal text, add the
  |ignorenonframetext| option and place \emph{every} |slide|
  environment inside a |\frame| since \beamer\ will not recognize the
  |\begin{slide}| as the beginning of a frame.
\item
  Add a |\usepackage{beamerseminar}| to start the emulation. Add the
  option |accumulate| if you wish to create a presentation to be held
  with a video projector.
\item
  Possibly add commands to install themes and templates.
\item
  Remove most commands in the preamble having to do with page and
  slide styles. They do not apply to |beamer|.
\item
  If a |\newslide| command is used in a |slide| (or similarly
  |slide*|) environment that contains an overlay, you must replace it
  by a closing |\end{slide}| and an opening |\begin{slide}|. 
\item
  Next, for each |slide| or |slide*| environment that contains an
  overlay, you must place a |\frame| command around it. You can remove
  the environment, unless you use the |accumulate| option.
\item
  If you use |\section| or |\subsection| commands inside slides, you
  will have to move them \emph{outside} the frames. It may then be
  necessary to add a |\frametitle| command to the slide.
\item
  If you use pdf\LaTeX\ to typeset the presentation, than you cannot
  include PostScript files. You should convert them to |.pdf| or to
  |.png| and adjust any usage of |\includegraphics| accordingly.
\item
  When starting to change things, you can use all of \beamer's
  commands and even mix them with \prosper\ commands.
\end{enumerate}

An example can be found in the file
|beamerexample-seminar.tex|.

There are, unfortunately, numerous places where you may run into
problems:
\begin{itemize}
\item
  The whole |note| management of |seminar| is so different from
  |beamer|'s, that you will have to edit notes ``by hand.'' In
  particular, commands like |\ifslidesonly| and |\ifslide| may not do
  exactly what you expect.
\item
  If you use |pstricks| commands, you will either have to stick to
  using |latex| and |dvips| or will have to work around them using,
  for example, |pgf|. Porting lot's of |pstricks| code is bound to be
  difficult, if you wish to switch over to |pdflatex|, so be warned.
\item
  If the file cannot be compiled because some \seminar\ command is not
  implemented, you will have to delete this command and try to mimick
  its behaviour using some \beamer\ command.
\end{itemize}

\begin{package}{{beamerseminar}}
  Include this package in a |beamer| presentation to get access to
  \seminar\ commands. Use |beamer| as the document class, not
  |seminar|. Most of the options passed to the class |seminar| make no
  sense in |beamer|, so just delete them.

  This package takes the following options:
  \begin{itemize}
  \item
    \declare{|accumulate|} causes overlays to be accumulated. The
    original behaviour of the \seminar\ package is that in each
    overlay only the really ``new'' part of the overlay is shown. This
    makes sense, if you really print out the overlays on
    transparencies and then really stack overlays on top of each
    other. For a presentation with a video projector, you rather
    want to present an ``accumulated'' version of the overlays. This
    is what this option does: When the new material of the $i$th
    overlay is shown, the material of all previous overlays is also
    shown. 
  \end{itemize}

  \example
  The following example is an extract of |beamerexample-seminar.tex|:
\begin{verbatim}
\documentclass[ignorenonframetext]{beamer}
\usepackage[accumulated]{beamerseminar}
\usepackage{beamerthemeclassic}

\title{Example for seminar.sty}
\author{Policarpa Salabarrieta}
\date{July 21, 1991}

\newcommand{\sref}[1]{SLIDE \ref{#1}}

%% CHANGED: different definition of \heading
%%\newcommand{\heading}[1]{\begin{center}\large\bf #1\end{center}}
\let\heading=\frametitle

%% CHANGED: Commented:
%%\newpagestyle{MH}%
%%  {University of Guaduas, March 13, 1998\hfil\thepage}{}
%%\pagestyle{MH}

\begin{document}

%% CHANGED: Added \frame
\frame{
\maketitle         
}

This is a lot of gobbledy-gook intended only to illustrate some of the
features of seminar.sty.

%% CHANGED: Added \frame
\frame{
\begin{slide}\label{too_much}%
\begin{center}
  \large\bf
 Information overload = ``Too much'' information
\end{center}
\smallskip

\begin{verse} \bf\tt
  You have 134 unread messages:\\
  Do you want to read them now?
\end{verse}

\begin{enumerate}
  {\overlay2
  \item People {\overlay1 cannot process all} the information they receive.}
  \item People {\em should} receive less information.
 \end{enumerate}
\end{slide}
}
\end{document}
\end{verbatim}

  You can use all normal \beamer\ commands and concepts, like
  overlay-specifications, in the file. You can also create an
  |article| version by adding the class option |class=article| and
  including the package |beamerbasearticle|. 
\end{package}

In the following, the effect of \seminar\ commands in \beamer\ are
listed.

\begin{command}{\overlay\marg{number}}
  Shows the material till the end of the current \TeX\ group only on
  overlay numbered $\hbox{\meta{number}}+1$ or, if the |accumulate|
  option is given, from that overlay on. Usages of this command may be
  nested (as in \seminar). If an |\overlay| command is given inside
  another, it temporarily ``overrules'' the outer one as demonstrated
  in the following example, where it is assumed, that the |accumulate|
  option is given.

  \example
\begin{verbatim}
\frame{
\begin{slide}
  This is shown from the first slide on.
  {\overlay{2}
    This is shown from the third slide on.
    {\overlay{1}
      This is shown from the second slide on.
    }
    This is shown once more from the third slide on.
  }
\end{slide}
}
\end{verbatim}
\end{command}


\begin{environment}{{slide\opt{|*|}}}
  Mainly installs an |\overlay{0}| around the \meta{environment
    contents}. If the |accumulate| option is given, this has no
  effect, but otherwise it will cause the main text of the slide to be
  shown \emph{only} on the first slide. This is useful if you really
  wish to physically place slides on top of each other.

  The starred version does the same as the nonstarred one.

  If this command is not issued inside a |\frame|, it sets up a frame
  with the |containsverbatim| option set. Thus, this frame will
  contain only a single slide.

  \example
\begin{verbatim}
\begin{slide}
  Some text.
\end{slide}

\frame{
\begin{slide}
  Some text. And an {\overlay{1} overlay}.
\end{slide}
}
\end{verbatim}
\end{environment}

\begin{command}{\red}
  Mapped to |\color{red}|.
\end{command}

\begin{command}{\blue}
  Mapped to |\color{blue}|.
\end{command}

\begin{command}{\green}
  Mapped to |\color{green}|.
\end{command}

\begin{command}{\ifslide}
  True in the |presentation| modes, false in the |article| mode.
\end{command}

\begin{command}{\ifslidesonly}
  Same as |\ifslide|.
\end{command}

\begin{command}{\ifarticle}
  False in the |presentation| modes, true in the |article| mode.
\end{command}

\begin{command}{\ifportrait}
  Always false.
\end{command}

The following commands are parsed by \beamer, but have no effect:
\begin{itemize}\itemsep=0pt\parskip=0pt
\item |\ptsize|.
\end{itemize}


\subsubsection{Foil\TeX}
\label{section-foiltex}

The package |beamerfoils| maps a subset of the commands of the \foils\
package to \beamer. Since this package defines only few non-standard
\TeX\ commands and since \beamer\ implements all the standard
commands, the emulation layer is pretty simple. The main problem is
the fact in foil\TeX\ page breaks are performed automatically, which
is not the case in \beamer.

A copyright notice: The Foil\TeX\ package has a restricted
license. For this reason, no example from the \foils\ package is
included in the \beamer\ class. The emulation itself does not use the
code of the \foils\ package (rather, it just maps \foils\ commands to
\beamer\ commands). For this reason, my understanding is that the
\emph{emulation} offered by the \beamer\ class is ``free'' and legally
so. IBM has a copyright on the \foils\ class, not on the effect the
commands of this class have. (At least, that's my understanding of
things.) 

The workflow for the migration is the following:
\begin{enumerate}
\item
  Replace the document class |foils| by |beamer|.
\item
  Add a |\usepackage{beamerfoils}| to start the emulation.
\item
  Possibly add commands to install themes and templates.
\item
  You may need to insert |\frame| commands to specify how the text
  should be broken up into frames. If there is an automatic page break
  in foil\TeX, you will have to manually insert a |\frame| for the
  first frame and a second one for the second frame.
\item
  If the command |\foilhead| is used inside a |\frame| command, it
  behaves like |\frametitle|. If it used outside a frame, it will
  start a new frame (with the |containsverbatim| option, thus no
  overlays are allowed). This frame will persist till the next occurence
  of |\foilhead| or of the new command |\endfoil|. Note that a
  |\frame| command will \emph{not} end a frame started using
  |\foilhead|. 
\item
  If you rely on automatic frame creation based on |\foilhead|, you
  will need to insert an |\endfoil| before the end of the document to
  end the last frame.
\item
  If you use pdf\LaTeX\ to typeset the presentation, than you cannot
  include PostScript files. You should convert them to |.pdf| or to
  |.png| and adjust any usage of |\includegraphics| accordingly.
\item
  Sizes of objects are different in \beamer, since the scaling is done
  by the viewer, not by the class. Thus a framebox of size 6 inches
  will be way too big in a \beamer\ presentation. You will have to
  manually adjust explicit dimension occuring in a foil\TeX\ presentation.
\end{enumerate}

\begin{package}{{beamerfoils}}
  Include this package in a |beamer| presentation to get access to
  \foils\ commands. Use |beamer| as the document class, not
  |foils|.

  \example In the following example, frames are automatically
  created. The |\endfoil| at the end is needed to close the last
  frame. 
\begin{verbatim}
\documentclass{beamer}
\usepackage{beamerfoils}
  
\begin{document}

\maketitle

\foilhead{First Frame}

This is on the first frame.

\foilhead{Second Frame}

This is on the second frame.

\endfoil
\end{document}
\end{verbatim}
  
  \example In this example, frames are manually inserted. No
  |\endfoil| is needed.
\begin{verbatim}
\documentclass{beamer}
\usepackage{beamerfoils}
  
\begin{document}

\frame{\maketitle}

\frame{
\foilhead{First Frame}
This is on the first frame.
}

\frame{
\foilhead{Second Frame}
This is on the second frame.
}
\end{document}
\end{verbatim}
\end{package}


In the following, the effect of \foils\ commands in \beamer\ are
listed.

\begin{command}{\MyLogo\marg{logo text}}
  This is mapped to |\logo|, though the logo is internally stored,
  such that it can be switched on and off using |\LogoOn| and |\LogoOff|.
\end{command}

\begin{command}{\LogoOn}
  Makes the logo visible.
\end{command}

\begin{command}{\LogoOff}
  Makes the logo invisible.
\end{command}

\begin{command}{\foilhead\oarg{dimension}\marg{frame title}}
  If used inside a |\frame| command, this is mapped to
  |\frametitle{|\meta{frame title}|}|. If used outside any frames, a
  new frame is started. If a frame was previously started using this
  command, it will be closed before the next frame is started. The
  \meta{dimension} is ignored.
\end{command}

\begin{command}{\rotatefoilhead\oarg{dimension}\marg{frame title}}
  This command has exactly the same effect as |\foilhead|.
\end{command}

\begin{command}{\endfoil}
  This is a command that is \emph{not} available in \foils. In
  \beamer, it can be used to end a frame that has automatically been
  opened using |\foildhead|. This command must be given before the end
  of the document if the last frame was opened using |\foildhead|.  
\end{command}

\begin{environment}{{boldequation\opt{|*|}}}
  This is mapped to the |equation| or the |equation*| environment,
  with |\boldmath| switched on.
\end{environment}

\begin{command}{\FoilTeX}
  Typests the foil\TeX\ name as in the \foils\ package.
\end{command}

\begin{command}{\bm\marg{text}}
  Implemented as in the \foils\ package.
\end{command}

\begin{command}{\bmstyle\marg{text}\marg{more text}}
  Implemented as in the \foils\ package.
\end{command}

The following additional theorem-like environments are predefined:
\begin{itemize}
\item |Theorem*|,
\item |Lemma*|,
\item |Corollary*|,
\item |Proposition*|,
\item |Definition*|.
\end{itemize}
For example, the first is defined using |\newtheorem*{Theorem*}{Theorem}|. 

The following commands are parsed by \beamer, but have not effect:
\begin{itemize}
\item |\leftheader|,
\item |\rightheader|,
\item |\leftfooter|,
\item |\rightfooter|,
\item |\Restriction|, and
\item |\marginpar|.
\end{itemize}


\subsubsection{\TeX Power}
\label{section-texpower}

The package |beamertexpower| maps a subset of the commands of the
\texpower\ package, due to Stephan Lehmke, to \beamer. This subset is
currently rather small, so a lot of adaptions may be necessary. Note
that \texpower\ is not a full class by itself, but a package that
needs another class, like |seminar| or |prosper| to do the actualy
typesetting. It may thus be necessary to additionally load an
emulation layer for these also. Indeed, it \emph{might} be possible to
directly use \texpower\ inside \beamer, but I have not tried that.

Currently, the package |beamertexpower| mostly just maps the
|\stepwise| and related commands to appropriate \beamer\ commands. The
|\pause| command need not be mapped since it is directly implemented
by \beamer\ anyway.

The workflow for the migration is the following:
\begin{enumerate}
\item
  Replace the document class by |beamer|. If the document class is
  |seminar| or |prosper|, you can use the above emulation layers, that
  is, you can include the files |beamerseminar| or |beamerprosper| to
  emulate the class.

  All notes on what to do for the emulation of \seminar\ or \prosper\
  also apply here.
\item
  Additionally, add |\usepackage{beamertexpower}| to start the
  emulation.
\end{enumerate}


\begin{package}{{beamertexpower}}
  Include this package in a |beamer| presentation to get access to the
  \texpower\ commands having to do with the |\stepwise| command.
\end{package}

A note on the |\pause| command: Both \beamer\ and \texpower\ implement
this command and they have the same semantics; so there is no need to
map this command to anything different in |beamertexpower|. However, a
difference is that |\pause| can be used almost anywhere in \beamer,
whereas is may only be used in nested situtations in \texpower. Since
\beamer\ is only more flexible than \texpower\ here, this will not
cause problems when porting.

In the following, the effect of \texpower\ commands in \beamer\ are
listed.


\begin{command}{\stepwise\marg{text}}
  As in \TeX Power, this initiates text in which commands like |\step|
  or |\switch| may be given. Text contained in a |\step| command will
  be enclosed in an |\only| command with the overlay specification
  |<+(1)->|. This means that the text of the first |\step| is inserted
  from the second slide onward, the text of the second |\step| is
  inserted from the third slide onward, and so on.
\end{command}

\begin{command}{\parstepwise\marg{text}}
  Same as |\stepwise|, only |\uncover| is used instead of |\only| when
  mapping the |\step| command.
\end{command}

\begin{command}{\liststepwise\marg{text}}
  Same as |\stepwise|, only an invisible horizontal line is inserted
  before the \meta{text}. This is presumable useful for solving some
  problems related to vertical spacing in \texpower. 
\end{command}

\begin{command}{\step\marg{text}}
  This is either mapped to |\only<+(1)->|\meta{text} or to
  |\uncover<+(1)->|\meta{text}, depending on whether this command is
  used inside a |\stepwise| environment or inside a |\parstepwise|
  environment. 
\end{command}

\begin{command}{\steponce\marg{text}}
  This is either mapped to |\only<+(1)>|\meta{text} or to
  |\uncover<+(1)|\meta{text}, depending on whether this command is
  used inside a |\stepwise| environment or inside a |\parstepwise|
  environment. 
\end{command}

\begin{command}{\switch\marg{alternate text}\marg{text}}
  This is mapped to |\alt<+(1)->{|\meta{text}|}{|\meta{alternate
      text}|}|. Note that the arguments are swapped.
\end{command}

\begin{command}{\bstep\marg{text}}
  This is always mapped to |\uncover<+(1)->|\meta{text}.
\end{command}

\begin{command}{\dstep}
  This just proceeds the counter |beamerpauses| by one. It has no
  other effect.
\end{command}

\begin{command}{\vstep}
  Same as |\dstep|.
\end{command}

\begin{command}{\restep\marg{text}}
  Same as |\step|, but the \meta{text} is shown one the same slide as
  the previous |\step| command. This is implemented by first
  decreasing the countern  |beamerpauses| by one before calling
  |\step|. 
\end{command}

\begin{command}{\reswitch\marg{alternate text}\meta{text}}
  Like |\restep|, only for the |\switch| command.
\end{command}

\begin{command}{\rebstep\meta{text}}
  Like |\restep|, only for the |\bstep| command.
\end{command}

\begin{command}{\redstep}
  This command has no effect.
\end{command}

\begin{command}{\revstep}
  This command has no effect.
\end{command}

\begin{command}{\boxedsteps}
  Temporarily (for the current \TeX\ group) changes the effect of
  |\step| to issue an |\uncover|, even if used inside a |\stepwise|
  environment. 
\end{command}

\begin{command}{\nonboxedsteps}
  Temporarily (for the current \TeX\ group) changes the effect of
  |\step| to issue an |\only|, even if used inside a |\parstepwise|
  environment. 
\end{command}

\begin{command}{\code\marg{text}}
  Typesets the argument using a boldface typewriter font.
\end{command}

\begin{command}{\codeswitch}
  Switches to a boldface typewriter font.
\end{command}


\section{License: The GNU Public License, Version 2}

The \beamer\ class is distributed under the \textsc{gnu} public
license, version 2. In detail, this means the following (the following
text is copyrighted by the Free Software Foundation):

\subsection{Preamble}

The licenses for most software are designed to take away your freedom to
share and change it.  By contrast, the \textsc{gnu} General Public License is
intended to guarantee your freedom to share and change free software---to
make sure the software is free for all its users.  This General Public
License applies to most of the Free Software Foundation's software and to
any other program whose authors commit to using it.  (Some other Free
Software Foundation software is covered by the \textsc{gnu} Library General Public
License instead.)  You can apply it to your programs, too.

When we speak of free software, we are referring to freedom, not price.
Our General Public Licenses are designed to make sure that you have the
freedom to distribute copies of free software (and charge for this service
if you wish), that you receive source code or can get it if you want it,
that you can change the software or use pieces of it in new free programs;
and that you know you can do these things.

To protect your rights, we need to make restrictions that forbid anyone to
deny you these rights or to ask you to surrender the rights.  These
restrictions translate to certain responsibilities for you if you
distribute copies of the software, or if you modify it.

For example, if you distribute copies of such a program, whether gratis or
for a fee, you must give the recipients all the rights that you have.  You
must make sure that they, too, receive or can get the source code.  And
you must show them these terms so they know their rights.

We protect your rights with two steps: (1) copyright the software, and (2)
offer you this license which gives you legal permission to copy,
distribute and/or modify the software.

Also, for each author's protection and ours, we want to make certain that
everyone understands that there is no warranty for this free software.  If
the software is modified by someone else and passed on, we want its
recipients to know that what they have is not the original, so that any
problems introduced by others will not reflect on the original authors'
reputations.

Finally, any free program is threatened constantly by software patents.
We wish to avoid the danger that redistributors of a free program will
individually obtain patent licenses, in effect making the program
proprietary.  To prevent this, we have made it clear that any patent must
be licensed for everyone's free use or not licensed at all.

The precise terms and conditions for copying, distribution and
modification follow.

\subsection{Terms and Conditions For Copying, Distribution and
  Modification}

\begin{enumerate}

\addtocounter{enumi}{-1}

\item 

This License applies to any program or other work which contains a notice
placed by the copyright holder saying it may be distributed under the
terms of this General Public License.  The ``Program'', below, refers to
any such program or work, and a ``work based on the Program'' means either
the Program or any derivative work under copyright law: that is to say, a
work containing the Program or a portion of it, either verbatim or with
modifications and/or translated into another language.  (Hereinafter,
translation is included without limitation in the term ``modification''.)
Each licensee is addressed as ``you''.

Activities other than copying, distribution and modification are not
covered by this License; they are outside its scope.  The act of
running the Program is not restricted, and the output from the Program
is covered only if its contents constitute a work based on the
Program (independent of having been made by running the Program).
Whether that is true depends on what the Program does.

\item You may copy and distribute verbatim copies of the Program's source
  code as you receive it, in any medium, provided that you conspicuously
  and appropriately publish on each copy an appropriate copyright notice
  and disclaimer of warranty; keep intact all the notices that refer to
  this License and to the absence of any warranty; and give any other
  recipients of the Program a copy of this License along with the Program.

You may charge a fee for the physical act of transferring a copy, and you
may at your option offer warranty protection in exchange for a fee.

\item

You may modify your copy or copies of the Program or any portion
of it, thus forming a work based on the Program, and copy and
distribute such modifications or work under the terms of Section 1
above, provided that you also meet all of these conditions:

\begin{enumerate}

\item 

You must cause the modified files to carry prominent notices stating that
you changed the files and the date of any change.

\item

You must cause any work that you distribute or publish, that in
whole or in part contains or is derived from the Program or any
part thereof, to be licensed as a whole at no charge to all third
parties under the terms of this License.

\item
If the modified program normally reads commands interactively
when run, you must cause it, when started running for such
interactive use in the most ordinary way, to print or display an
announcement including an appropriate copyright notice and a
notice that there is no warranty (or else, saying that you provide
a warranty) and that users may redistribute the program under
these conditions, and telling the user how to view a copy of this
License.  (Exception: if the Program itself is interactive but
does not normally print such an announcement, your work based on
the Program is not required to print an announcement.)

\end{enumerate}


These requirements apply to the modified work as a whole.  If
identifiable sections of that work are not derived from the Program,
and can be reasonably considered independent and separate works in
themselves, then this License, and its terms, do not apply to those
sections when you distribute them as separate works.  But when you
distribute the same sections as part of a whole which is a work based
on the Program, the distribution of the whole must be on the terms of
this License, whose permissions for other licensees extend to the
entire whole, and thus to each and every part regardless of who wrote it.

Thus, it is not the intent of this section to claim rights or contest
your rights to work written entirely by you; rather, the intent is to
exercise the right to control the distribution of derivative or
collective works based on the Program.

In addition, mere aggregation of another work not based on the Program
with the Program (or with a work based on the Program) on a volume of
a storage or distribution medium does not bring the other work under
the scope of this License.

\item
You may copy and distribute the Program (or a work based on it,
under Section 2) in object code or executable form under the terms of
Sections 1 and 2 above provided that you also do one of the following:

\begin{enumerate}

\item

Accompany it with the complete corresponding machine-readable
source code, which must be distributed under the terms of Sections
1 and 2 above on a medium customarily used for software interchange; or,

\item

Accompany it with a written offer, valid for at least three
years, to give any third party, for a charge no more than your
cost of physically performing source distribution, a complete
machine-readable copy of the corresponding source code, to be
distributed under the terms of Sections 1 and 2 above on a medium
customarily used for software interchange; or,

\item

Accompany it with the information you received as to the offer
to distribute corresponding source code.  (This alternative is
allowed only for noncommercial distribution and only if you
received the program in object code or executable form with such
an offer, in accord with Subsection b above.)

\end{enumerate}


The source code for a work means the preferred form of the work for
making modifications to it.  For an executable work, complete source
code means all the source code for all modules it contains, plus any
associated interface definition files, plus the scripts used to
control compilation and installation of the executable.  However, as a
special exception, the source code distributed need not include
anything that is normally distributed (in either source or binary
form) with the major components (compiler, kernel, and so on) of the
operating system on which the executable runs, unless that component
itself accompanies the executable.

If distribution of executable or object code is made by offering
access to copy from a designated place, then offering equivalent
access to copy the source code from the same place counts as
distribution of the source code, even though third parties are not
compelled to copy the source along with the object code.

\item
You may not copy, modify, sublicense, or distribute the Program
except as expressly provided under this License.  Any attempt
otherwise to copy, modify, sublicense or distribute the Program is
void, and will automatically terminate your rights under this License.
However, parties who have received copies, or rights, from you under
this License will not have their licenses terminated so long as such
parties remain in full compliance.

\item
You are not required to accept this License, since you have not
signed it.  However, nothing else grants you permission to modify or
distribute the Program or its derivative works.  These actions are
prohibited by law if you do not accept this License.  Therefore, by
modifying or distributing the Program (or any work based on the
Program), you indicate your acceptance of this License to do so, and
all its terms and conditions for copying, distributing or modifying
the Program or works based on it.

\item
Each time you redistribute the Program (or any work based on the
Program), the recipient automatically receives a license from the
original licensor to copy, distribute or modify the Program subject to
these terms and conditions.  You may not impose any further
restrictions on the recipients' exercise of the rights granted herein.
You are not responsible for enforcing compliance by third parties to
this License.

\item
If, as a consequence of a court judgment or allegation of patent
infringement or for any other reason (not limited to patent issues),
conditions are imposed on you (whether by court order, agreement or
otherwise) that contradict the conditions of this License, they do not
excuse you from the conditions of this License.  If you cannot
distribute so as to satisfy simultaneously your obligations under this
License and any other pertinent obligations, then as a consequence you
may not distribute the Program at all.  For example, if a patent
license would not permit royalty-free redistribution of the Program by
all those who receive copies directly or indirectly through you, then
the only way you could satisfy both it and this License would be to
refrain entirely from distribution of the Program.

If any portion of this section is held invalid or unenforceable under
any particular circumstance, the balance of the section is intended to
apply and the section as a whole is intended to apply in other
circumstances.

It is not the purpose of this section to induce you to infringe any
patents or other property right claims or to contest validity of any
such claims; this section has the sole purpose of protecting the
integrity of the free software distribution system, which is
implemented by public license practices.  Many people have made
generous contributions to the wide range of software distributed
through that system in reliance on consistent application of that
system; it is up to the author/donor to decide if he or she is willing
to distribute software through any other system and a licensee cannot
impose that choice.

This section is intended to make thoroughly clear what is believed to
be a consequence of the rest of this License.

\item
If the distribution and/or use of the Program is restricted in
certain countries either by patents or by copyrighted interfaces, the
original copyright holder who places the Program under this License
may add an explicit geographical distribution limitation excluding
those countries, so that distribution is permitted only in or among
countries not thus excluded.  In such case, this License incorporates
the limitation as if written in the body of this License.

\item
The Free Software Foundation may publish revised and/or new versions
of the General Public License from time to time.  Such new versions will
be similar in spirit to the present version, but may differ in detail to
address new problems or concerns.

Each version is given a distinguishing version number.  If the Program
specifies a version number of this License which applies to it and ``any
later version'', you have the option of following the terms and conditions
either of that version or of any later version published by the Free
Software Foundation.  If the Program does not specify a version number of
this License, you may choose any version ever published by the Free Software
Foundation.

\item
If you wish to incorporate parts of the Program into other free
programs whose distribution conditions are different, write to the author
to ask for permission.  For software which is copyrighted by the Free
Software Foundation, write to the Free Software Foundation; we sometimes
make exceptions for this.  Our decision will be guided by the two goals
of preserving the free status of all derivatives of our free software and
of promoting the sharing and reuse of software generally.

\end{enumerate}

\subsection{No Warranty}

\begin{enumerate}

\addtocounter{enumi}{9}

\item
Because the program is licensed free of charge, there is no warranty
for the program, to the extent permitted by applicable law.  Except when
otherwise stated in writing the copyright holders and/or other parties
provide the program ``as is'' without warranty of any kind, either expressed
or implied, including, but not limited to, the implied warranties of
merchantability and fitness for a particular purpose.  The entire risk as
to the quality and performance of the program is with you.  Should the
program prove defective, you assume the cost of all necessary servicing,
repair or correction.

\item
In no event unless required by applicable law or agreed to in writing
will any copyright holder, or any other party who may modify and/or
redistribute the program as permitted above, be liable to you for damages,
including any general, special, incidental or consequential damages arising
out of the use or inability to use the program (including but not limited
to loss of data or data being rendered inaccurate or losses sustained by
you or third parties or a failure of the program to operate with any other
programs), even if such holder or other party has been advised of the
possibility of such damages.
\end{enumerate}


\end{document}
