% This file is included by beamerexample2.article.tex and
% beamerexample2.beamer.tex 

% Copyright 2003 by Till Tantau <tantau@cs.tu-berlin.de>.
%
% This program can be redistributed and/or modified under the terms
% of the LaTeX Project Public License Distributed from CTAN
% archives in directory macros/latex/base/lppl.txt.

%
% The purpose of this example is to demonstrate the usage of the
% nameslide command
%
\common

  \usepackage[english]{babel}
 
\article

  \usepackage{fullpage}
  \usepackage{pgf}
  \setjobnamebeamerversion{beamerexample2.beamer}

\presentation

  \usepackage{beamertemplates}
  \usepackage{beamerthemetree}

  \beamertemplatetransparentcovereddynamic
  \beamertemplateballitem
  \beamertemplatesolidbuttons

  \hypersetup{%
    pdftitle={Second Beamer Example},%
    pdfauthor={Till Tantau},
    pdfsubject={Presentation Programs}}

\common

  \title{Second Beamer Example}
  \author{Till~Tantau}

\presentation
  
  \institute{
    Fakult�t f�r Elektrotechnik und Informatik\\
    Technical University of Berlin}


\begin{document}

\article

  \maketitle

\presentation

  \frame{\titlepage}

\common

  \section{The first section}

\article

  This is the first section of the article version. In the
  presentation, there is a frame containing an overlay. The two slides
  of this overlay are shown in Figures~\ref{figure-example1}
  and~\ref{figure-example2}.

  \begin{figure}[ht]
    \begin{center}
      \includeslide{example1}
    \end{center}
    \caption{The first slide. Note the partly covered second item.}
    \label{figure-example1}
  \end{figure}

  \begin{figure}[ht]
    \begin{center}
      \includeslide{example2}
    \end{center}
    \caption{The second slide. Now the second item is also shown.}
    \label{figure-example2}
  \end{figure}
  
\presentation

  \frame{
    \nameslide<1>{example1}
    \nameslide<2>{example2}
  
    \frametitle{This is a frame with two overlays.}

    \begin{itemize}
    \item The first item$\dots$
      \pause
    \item $\dots$ and the second one.
    \end{itemize}
  }

\end{document}



%%% Local Variables: 
%%% mode: latex
%%% TeX-master: "beamerexample2.article"
%%% End: 
