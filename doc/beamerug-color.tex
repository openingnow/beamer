% Copyright 2003, 2004 by Till Tantau <tantau@users.sourceforge.net>.
%
% This program can be redistributed and/or modified under the terms
% of the GNU Public License, version 2.

\section{Colors}

\label{section-colors}


The color management of the \beamer\ class relies on the packages
|xcolor|, which is a stand-alone extension of the |color| package, 
and on |xxcolor|, which in turn is an extension of
|xcolor| and is part of \pgf. Hopefully, in the future |xxcolor| and
|xcolor| will merge into one package and perhaps they will
someday also merge together with |color|.

Since the |color| package and the |xcolor| package are loaded already
by the \beamer\ class, in order to pass options to these classes, you
need to use the class options |color={|\meta{options for color}|}| or
|xcolor={|\meta{options for xcolor}|}| to pass options to these
classes.




\subsection{Color Themes}

\subsubsection{Default and Special-Purpose Color Themes}

\begin{colorthemeexample}{default}
  The |default| color theme is very sober. It installs little special
  colors and even less backgrounds. The default color theme sets up
  the default parent relations between the different \beamer-colors.

  The main colors set in the |default| color theme are the following: 
  \begin{itemize}
  \item
    |normal text| is black on white.
  \item
    |alerted text| is red.
  \item
    |example text| is a dark green (green with 50\% black).
  \item
    |structure| is set to a light version of MidnightBlue
    (more precisely, 20\% red, 20\% green, and 70\% blue).  
  \end{itemize}
  Use this theme for a no-nonsense presentation. Note, however, that
  since this theme is loaded by default, you cannot ``reload'' it
  after having loaded another color theme.
\end{colorthemeexample}

\begin{colorthemeexample}[\oarg{options}]{structure}
  The example was created using |\usecolortheme[named=SeaGreen]{structure}|. 
  
  This theme offers a convenient way of changing the color used for
  structural elements. More precisely, it just changes the foreground
  of the \beamer-color |structure|. You can also achieve this by
  directly invoking the function |\setbeamercolor|, but this color
  theme makes things a bit easier.

  The theme offers several \meta{options}, which can be used to
  specify the color to be used for structural elements:
  \begin{itemize}
  \item
    \declare{|rgb=|\marg{rgb tuple}} sets the |structure| foreground
    to the specified red-green-blue tuple. The numbers are given as
    decimals between 0 and 1. For example, |rgb={0.5,0,0}| yields a
    dark red.
  \item
    \declare{|RGB=|\marg{rgb tuple}} does the same as |rgb|, except
    that the numbers range between 0 and 255. For example,
    |RGB={128,0,0}|  yields a dark red.
  \item
    \declare{|cmyk=|\marg{cymk tuple}} sets the |structure| foreground
    to the specified cyan-magenta-yellow-black tuple. The numbers are
    given as decimals between 0 and 1. For example, |cymk={0,1,1,0.5}|
    yields a dark red.
  \item
    \declare{|cmy=|\marg{cym tuple}} is similar to |cmyk|, except that
    the black component is not specified.
  \item
    \declare{|hsb=|\marg{hsb tuple}}  sets the |structure| foreground
    to the specified hue-saturation-brightness tuple. The numbers are
    given as decimals between 0 and 1. For example, |hsb={0,1,.5}|
    yields a dark red.
  \item
    \declare{|named=|\marg{color name}} sets the |structure| foreground
    to a named color. This color must previously have been defined
    using the |\DefineNamedColor| command. Adding the class option
    |xcolor=dvipsnames| will install a long list of standard
    names. See the file |dvipsnam.def| for the list.
  \end{itemize}
\end{colorthemeexample}

\begin{colorthemeexample}{sidebartab}
  This theme changes the colors in a sidebar such that the current
  entry in a table of contents shown there gets hilighted by showing a
  different background behind it.
\end{colorthemeexample}



\subsubsection{Complete Color Themes}

A ``complete'' color theme is a color theme that completely specifies
all colors for all parts of a frame. It installs specific colors and
does not derive the colors from, say, the |structure| \beamer-color.
Complete complete themes happen to have bird names.

\begin{colorthemeexample}{albatross}
  The color theme is a ``dark'' or ``inverted'' theme using yellow on
  blue as the main colors. The color theme also installs a slightly
  darker background color for blocks, which is necessary for
  presentation themes that use shadows, but which (in my opinion) is
  undesirable for all other presentation themes. By using the |lily|
  color theme together with this theme, the backgrounds for blocks can
  be removed.

  When using a light-on-dark theme like this one, be aware that there
  are certain disadvantages:
  \begin{itemize}
  \item
    If the room in which the talk is given has been ``darkened,''
    using such a theme makes it more difficult for the audience to
    take or read notes.
  \item
    Since the room becomes darker, the pupil becomes larger, thus
    making it harder for the eye to focus. This \emph{can} make text
    harder to  read.
  \item
    Printing such slides is difficult at best.
  \end{itemize}

  On the other hand, a light-on-dark presentation often appears to be
  more ``stylish''  than a plain black-on-white one.
\end{colorthemeexample}


\begin{colorthemeexample}{crane}
  This theme uses the colors of the Lufthansa, whose logo is a crane.
\end{colorthemeexample}


\subsubsection{Block Color Themes}

Block themes only specify the colors of block environments. They can
be used together with other (color) themes. If they are used to change the
block colors installed by a presentation theme or another color theme,
they should obviously be specified \emph{after} the other theme has
been loaded. Block themes happen to have flower names.

\begin{colorthemeexample}{lily}
  This theme is mainly used to \emph{uninstall} any block colors setup
  by another theme, restoring the colors used in the |default|
  theme. In particular, using this theme will remove all background
  colors for blocks.
\end{colorthemeexample}

\begin{colorthemeexample}{orchid}
  This theme installs white-on-dark block titles. The background of
  the title of a normal block is set to the foreground of the
  structure color, the foreground is set to white. The background of
  alerted blocks are set to red and of example blocks to green. The
  body of blocks get a nearly transparent background.
\end{colorthemeexample}

\begin{colorthemeexample}{rose}
  This theme installs nearly transparent backgrounds for both block
  titles and block bodies. This theme is much less ``aggressive'' than
  the |orchid| theme. The background colors are derived from the
  foreground of the structure \beamer-color.
\end{colorthemeexample}


\subsubsection{Palette Color Themes}

A palette color theme  changes the palette colors, on which the colors
used in the headline, footline, and sidebar 
are based by default. Palette color themes do not change block
titles. They have happen to sea-animal names.

\begin{colorthemeexample}{whale}
  Installs a white-on-dark palette for the headline, footline, and
  sidebar. The backgrounds used there are set to shades between the
  structure \beamer-color and black. The foreground is set to
  white.

  While this color theme can appear to be agressive, you should note
  that a dark bar at the border of a frame will have a somewhat
  different appearance during a presentation than it has on paper:
  During a presentation the projection on the 
  wall is usually surrounded by blackness. Thus, a dark bar will
  not create a contrast as opposed to the way it does on
  paper. Indeed, using this theme will cause the main part of the
  frame to be more at the focus of attention.

  The counterpart to the theme with respect to blocks is the |orchid|
  theme. However, pairing it with the |rose| color theme is also
  interesting. 
\end{colorthemeexample}

\begin{colorthemeexample}{seahorse}
  Installs a near-transparent backgrounds for the headline, footline,
  and sidebar. Using this theme will cause navigational elements to be
  much less ``dominant'' than when using the |whale| theme (see the
  dicussion on contrast there, though).

  It goes well with the |rose| or the |lily| color theme. Pairing it
  with the |orchid| overemphasizes blocks (in my opinion).
\end{colorthemeexample}




\subsection{Transparency Effects}
\label{section-transparent}

By default, \emph{covered} items are not shown during a
presentation. Thus if you write |\uncover<2>{Text.}|, the text
is not shown on any but the second slide. On the other slide, the text
is not simply printed using the background color -- it is not shown at
all. This effect is most useful if your background does not have a
uniform color.

Sometimes however, you might prefer that covered items are not
completely covered. Rather, you would like them to be shown already in
a very dim or shaded way. This allows your audience to get a feeling
for what is yet to come, without getting distracted by it. Also, you
might wish text that is covered ``once more'' still to be visible to
some degree.

Ideally, there would be an option to make covered text
``transparent.'' This would mean that when covered text is shown, it
would instead be mixed with the background behind it. Unfortunately,
|pgf| does not support real transparency yet. Instead, transparency is
created by mixing the color of the object you want to show with the
current background color (the color |bg|, which has hopefully been
setup such that it is the average color of the background on which the
object should be placed). To specify the ``degree of
transparency'', you can use the special command
\begin{verbatim}
\beamersetuncovermixins{#1}{#2}
\end{verbatim}
This commands allows you to specify in a quite general way how a
covered item should be rendered. You can even specify different ways
of rendering the item depending on how long it will take before this
item is shown or for how long it has already been covered once
more. The transparency effect will automatically apply to all colors,
\emph{except} for the colors in images. For images there is a
workaround, see the documentation of the \pgf\ package. 

As a convenience, several commands install a predefined uncovering
behavior.

\begin{command}{\beamertemplatetransparentcovered}
  Makes all covered text quite transparent. 
\end{command}

\begin{command}{\beamertemplatetransparentcoveredmedium}
  Makes all covered text even more transparent. 
\end{command}

\begin{command}{\beamertemplatetransparentcoveredhigh}
  Makes all covered text highly transparent. 
\end{command}

\begin{command}{\beamertemplatetransparentcoveredhigh}
  Makes all covered text extremely transparent, but not totally. 
\end{command}

\begin{command}{\beamertemplatetransparentcovereddynamic}
  Makes all covered text quite transparent, but is a dynamic way. The
  longer it will take till the text is uncovered, the stronger the
  transparency. 
\end{command}

\begin{command}{\beamertemplatetransparentcovereddynamicmedium}
  Like the previous command, only it the ``range'' of dynamics is
  smaller. 
\end{command}

\begin{command}{\beamersetuncovermixins\marg{not yet list}%
    \marg{once more list}}
  The \meta{not yet list} specifies  how to render covered items that
  have not  yet been uncovered. The \meta{once more list} specifies
  how to render covered items that have once more been covered. 
  If you leave one of the specifications empty, the corresponding
  covered items are completely covered, that is, they are invisible.
  \example
\begin{verbatim}
\beamersetuncovermixins
  {\opaqueness<1>{15}\opaqueness<2>{10}\opaqueness<3>{5}\opaqueness<4->{2}}
  {\opaqueness<1->{15}}
\end{verbatim}
  The \meta{not yet list} and the  \meta{once more list} can
  contain any number of |\opaqueness| commands.
\end{command}

\begin{command}{\opaqueness\ssarg{overlay
      specification}\marg{percentage of opaqueness}}
  The \meta{overlay specification} specifies on which slides covered
  text should have which \meta{percentage of opaqueness}. Unlike
  other overlay specifications, this \meta{overlay specification} is a
  ``relative'' overlay specification. For example, the specification
  ``3'' here means ``things that will be uncovered three slides
  ahead,'' respectively ``things that have once more been covered for
  three slides.'' More precisely, if an item is uncovered for more
  than one slide and then covered once more, only the ``first moment
  of uncovering'' is used for the calculation of how long the item has
  been covered once more.

  An opaqueness of 100 is fully opaque and 0 is fully
  transparent. Currently, since real transparency is not yet
  implemented, this command causes all colors to get a mixing of
  \meta{percentage of opaqueness} of the current
  |averagebackgroundcolor|. At some future point this command might
  result in real transparency.

  The alternate \pgf\ extension used inside an opaque area is
  \meta{percentage of opaqueness}|opaque|. In case of nested calls,
  only the innermost opaqueness specification is used. 
  \example
\begin{verbatim}
\beamersetuncovermixins{\opaqueness<1->{15}{\opaqueness<1->{15}}
\pgfdeclareimage{book}{book}
\pgfdeclareimage{book.15opaque}{filenameforbooknearlytransparent}
\end{verbatim}
  Makes everything that is uncovered in two slides only 15 percent
  opaque. 
\end{command}



\subsection{Changing the Colors Used for Different Elements of a Presentation}




%%% Local Variables: 
%%% mode: latex
%%% TeX-master: "beameruserguide"
%%% End: 
