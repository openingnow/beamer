% Copyright 2003, 2004 by Till Tantau <tantau@users.sourceforge.net>.
%
% This program can be redistributed and/or modified under the terms
% of the GNU Public License, version 2.

\section{Color Management}


The color management of the \beamer\ class relies on the packages
|xcolor|, which is a stand-alone extension of the |color| package, 
and on |xxcolor|, which in turn is an extension of
|xcolor| and is part of \pgf. Hopefully, in the future |xxcolor| and
|xcolor| will merge into one package and perhaps they will
someday also merge together with |color|.

Since the |color| package and the |xcolor| package are loaded already
by the \beamer\ class, in order to pass options to these classes, you
need to use the class options |color={|\meta{options for color}|}| or
|xcolor={|\meta{options for xcolor}|}| to pass options to these
classes.

The whole color management will hopefully be much improved in the
future. 

\subsection{Color Themes}

\colorthemeexample{Kranich}

\subsection{Colors of Main Text Elements}

By default, the following colors are used in a presentation:
\begin{itemize}
\item
  Normal text is typeset in |black|.
\item
  All ``structural'' elements, like titles, navigation bars, block
  titles, and so on, are typeset using the color
  |beamerstructure|. By default, this color is bluish. Using one of
  the class options |red|, |blackandwhite|, or |brown|
  changes this. You can also change this color simply be redefining
  the color |beamerstructure|.
\item
  All ``alert'' text is typeset by setting the default color and the
  structure color to 85\% of red. To change this, you can either
  redefine the color |beameralert|, or you can change the whole alert
  template. 
\item
  All examples are typeset using 50\% of green. To change this, you
  must change the example templates.
\end{itemize}

\begin{classoption}{brown}
  Changes the main color of the navigation and title bars
  to a brownish color.
\end{classoption}

\begin{classoption}{red}
  Changes the main color of the navigation and title bars
  to a reddish color.
\end{classoption}

\begin{classoption}{blackandwhite}
  Changes the main color of the navigation and title bars
  to monochrome.
\end{classoption}



\subsection{Average Background Color}

\label{section-average}

In some situations, for example when creating a transparency effect,
it is useful to have access to the current background
color. One can then, for example, mix a color with the background
color to create a ``transparent'' color.

Unfortunately, it is not always clear what exactly the background
color is. If the background is a shading or a picture, different parts
of a slide have different background colors. In these cases, one can
at least try to mix-in an \emph{average} background color, called
|averagebackgroundcolor|. If a shading or picture is not too
colorful, this works fairly well.

To specify the average background color, use the following command:

\begin{command}{\beamersetaveragebackground\marg{color expression}}
  Installs the given color as the average background color. See the
  |xcolor| package for the syntax of color expressions.
  \example |\beamersetaveragebackground{red!10}|
\end{command}

If you use the commands from Section~\ref{section-backgrounds} for
installing a background coloring, the average background color is
computed automatically for you. When you directly use the command
|\usebackgroundtemplate|, you should must set the average
background color afterward.




\subsection{Transparency Effects}
\label{section-transparent}

By default, \emph{covered} items are not shown during a
presentation. Thus if you write |\uncover<2>{Text.}|, the text
is not shown on any but the second slide. On the other slide, the text
is not simply printed using the background color -- it is not shown at
all. This effect is most useful if your background does not have a
uniform color.

Sometimes however, you might prefer that covered items are not
completely covered. Rather, you would like them to be shown already in
a very dim or shaded way. This allows your audience to get a feeling
for what is yet to come, without getting distracted by it. Also, you
might wish text that is covered ``once more'' still to be visible to
some degree.

Ideally, there would be an option to make covered text
``transparent.'' This would mean that when covered text is shown, it
would instead be mixed with the background behind it. Unfortunately,
|pgf| does not support real transparency yet.
Nevertheless, one can come ``quite close'' to transparent text using
the special command
\begin{verbatim}
\beamersetuncovermixins{#1}{#2}
\end{verbatim}
This commands allows you to specify in a quite general way how a
covered item should be rendered. You can even specify different ways
of rendering the item depending on how long it will take before this
item is shown or for how long it has already been covered once
more. The transparency effect will automatically apply to all colors,
\emph{except} for the colors in images and shadings. For images and
shadings there is a workaround, see the documentation of the
\pgf\ package. 

As a convenience, several commands install a predefined uncovering
behavior.

\begin{command}{\beamertemplatetransparentcovered}
  Makes all covered text quite transparent. 
\end{command}

\begin{command}{\beamertemplatetransparentcoveredmedium}
  Makes all covered text even more transparent. 
\end{command}

\begin{command}{\beamertemplatetransparentcoveredhigh}
  Makes all covered text highly transparent. 
\end{command}

\begin{command}{\beamertemplatetransparentcoveredhigh}
  Makes all covered text extremely transparent, but not totally. 
\end{command}

\begin{command}{\beamertemplatetransparentcovereddynamic}
  Makes all covered text quite transparent, but is a dynamic way. The
  longer it will take till the text is uncovered, the stronger the
  transparency. 
\end{command}

\begin{command}{\beamertemplatetransparentcovereddynamicmedium}
  Like the previous command, only it the ``range'' of dynamics is
  smaller. 
\end{command}

\begin{command}{\beamersetuncovermixins\marg{not yet list}%
    \marg{once more list}}
  The \meta{not yet list} specifies  how to render covered items that
  have not  yet been uncovered. The \meta{once more list} specifies
  how to render covered items that have once more been covered. 
  If you leave one of the specifications empty, the corresponding
  covered items are completely covered, that is, they are invisible.
  \example
\begin{verbatim}
\beamersetuncovermixins
  {\opaqueness<1>{15}\opaqueness<2>{10}\opaqueness<3>{5}\opaqueness<4->{2}}
  {\opaqueness<1->{15}}
\end{verbatim}
  The \meta{not yet list} and the  \meta{once more list} can
  contain any number of |\opaqueness| commands.
\end{command}

%\begin{command}{\mixinon\ssarg{overlay specification}\marg{mix-in specification}}
%  The \meta{overlay specification} specifies on which slides the
%  \meta{mix-in specification} should be applied to all colors. Unlike
%  other overlay specifications, this \meta{overlay specification} is a
%  ``relative'' overlay specification. For example, the specification
%  ``3'' here means ``things that will be uncovered three slides
%  ahead,'' respectively ``things that have once more been covered for
%  three slides.'' More precisely, if an item is uncovered for more
%  than one slide and then covered once more, only the ``first moment
%  of uncovering'' is used for the calculation of how long the item has
%  been covered once more.

%  \emph{Mix-in} specifications are a concept introduced by the
%  |xcolor| package. The \meta{mix-in specification} specifies how colors
%  should be altered by adding another color to them. The specification
%  consists of two parts, separated by an exclamation mark. The first
%  part is a number between 0 and 100, where 0 means ``do not mix in the
%  text color at all'' and 100 means ``use only the text color''. The
%  second part is the color that should be mixed in. This second part may
%  be omitted (along with the exclamation mark), in which case ``white''
%  is used as mix-in color. Any color that has been defined using the
%  |\definecolor| command is permissible as a mix-in color.

%  The mix-in specifications is added to the \pgf\ alternate
%  extension for shadings and images (see the \pgf\
%  documentation). Nested uses of mix-in accumulate. 
%  \example
%\begin{verbatim}
%\beamersetuncovermixins{\mixinon<1>{15!blue}{\mixinon<1->{15!white}}
%\pgfdeclareimage{book}{book}
%\pgfdeclareimage{book.!15!averagebackgroundcolor}{filenameforbooknearlyblue}
%\pgfdeclareimage{book.!15!white}{filenameforbooknearlywhite}
%\end{verbatim}
%  For all items that become uncovered on the next slide or that have
%  just been covered on the previous slide (depending on whether this
% command is used as part of the first or second parameter of the command
%  |\beamersetuncovermixins|), use only 15\% of the actual color and
%  85\% of the average background color.
%\end{command}

\begin{command}{\opaqueness\ssarg{overlay
      specification}\marg{percentage of opaqueness}}
  The \meta{overlay specification} specifies on which slides covered
  text should have which \meta{percentage of opaqueness}. Unlike
  other overlay specifications, this \meta{overlay specification} is a
  ``relative'' overlay specification. For example, the specification
  ``3'' here means ``things that will be uncovered three slides
  ahead,'' respectively ``things that have once more been covered for
  three slides.'' More precisely, if an item is uncovered for more
  than one slide and then covered once more, only the ``first moment
  of uncovering'' is used for the calculation of how long the item has
  been covered once more.

  An opaqueness of 100 is fully opaque and 0 is fully
  transparent. Currently, since real transparency is not yet
  implemented, this command causes all colors to get a mixing of
  \meta{percentage of opaqueness} of the current
  |averagebackgroundcolor|. At some future point this command might
  result in real transparency.

  The alternate \pgf\ extension used inside an opaque area is
  \meta{percentage of opaqueness}|opaque|. In case of nested calls,
  only the innermost opaqueness specification is used. 
  \example
\begin{verbatim}
\beamersetuncovermixins{\opaqueness<1->{15}{\opaqueness<1->{15}}
\pgfdeclareimage{book}{book}
\pgfdeclareimage{book.15opaque}{filenameforbooknearlytransparent}
\end{verbatim}
  Makes everything that is uncovered in two slides only 15 percent
  opaque. 
\end{command}




%%% Local Variables: 
%%% mode: latex
%%% TeX-master: "beameruserguide"
%%% End: 
