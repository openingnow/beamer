
% Copyright 2003, 2004 by Till Tantau <tantau@users.sourceforge.net>.
%
% This program can be redistributed and/or modified under the terms
% of the GNU Public License, version 2.


\section{Inner Themes, Outer Themes, and Templates}

This section discusses the inner and outer themes that are
available in \beamer. These themes install certain \emph{templates}
for the different elements of a presentation. The template mechanism is 
explained at the end of the section.

Before we plunge into the details, let us agree on some terminology
for this section. In \beamer, an \emph{element} is part of a
presentation that is potentially typeset in some special way. Examples
of elements are frame titles, the author's name, or the footnote
sign. The appearance of every element is governed by a \emph{template}
for this element. Appropriate templates are installed by inner and
outer themes, where the \emph{inner} themes only install templates for
elements that are typically ``inside the main text,'' while \emph{outer}
themes install templates for elements ``around the main text.'' Thus,
from the templates's point of view, there is no real difference
between inner and outer themes. 



\subsection{Inner Themes}

An inner theme installs templates that dictate how the following
elements are typeset:
\begin{itemize}
\item Title and part pages.
\item Itemize environments.
\item Enumerate environments.
\item Descrition environments.
\item Block environments.
\item Theorem and proof environments.
\item Figures and tables.
\item Footnotes.
\item Bibliography entries.
\end{itemize}


In the following examples, the color themes |seahorse| and |rose| are
used to show where and how background colors are
honoured. Furthermore, background colors have been specified for all
elements the honour them in the default theme. In the default color
theme, all of the large rectangular areas are transparent.

\begin{innerthemeexample}{default}
  The default element theme is quite sober. The only extravagance is
  the fact that a little trianlge is used in |itemize| environments
  instead of the usual dot.

  In some cases the theme will honour background color specifications
  for elements. For example, if you set the background color for block
  titles to green, block titles will have a green background. The
  background specifications are currently honoured for the following
  elements: 
  \begin{itemize}
  \item Title, author, institute, and date fields in the title
    page.
  \item Block environments, both for the title and for the body.
  \end{itemize}
  This list may increase in the future.
\end{innerthemeexample}

\begin{innerthemeexample}{circles}
  In this theme, |itemize| and |enumerate| items start with a small
  circle. Likewise, entries in the table of contents start with
  circles. 
\end{innerthemeexample}

\begin{innerthemeexample}{rectangles}
  In this theme, |itemize| and |enumerate| items and table of contents
  entries  start with small rectangles. 
\end{innerthemeexample}

\begin{innerthemeexample}[\oarg{options}]{rounded}
  In this theme, |itemize| and |enumerate| items and table of contents
  entries start with small balls. If a background is specified for
  blocks, then the corners of the background rectangles will be
  rounded off. The following \meta{options} may be given:

  \begin{itemize}
  \item \declare{|shadow|} adds a shadow to all blocks.
  \end{itemize}
\end{innerthemeexample}




\subsection{Outer Themes}

An outer theme dictates (roughly) the overall layout of frames. It
specifies where any navigational elements should go (like a mini table
of contents or navigational mini frames) and what they should look
like. Typically, an outer theme specifies how the following elements
are rendered: 
\begin{itemize}
\item The head- and footline.
\item The sidebars.
\item The logo.
\item The frame title.  
\end{itemize}

An outer theme will not specify how things like |itemize| environments
should be rendered---that is the job an inner theme.

In the following examples the color theme |seahorse| is
used. Since the default color theme leaves most backgrounds empty,
most of the outer themes look too unstructured with the default
color theme. 


\begin{outerthemeexample}{default}
  The default layout theme is the most sober and minimalistic theme
  around. It will flush left the frame title and it will not install
  any head- or footlines. However, even this theme honours the
  background color specified for the frame title. If a color is
  specified, a bar occupying the whole page width is put behind the
  frame title. A background color the frame subtitle is ignored.
\end{outerthemeexample}

\begin{outerthemeexample}{infolines}
  This theme installs a headline showing the current section and the
  current subsection. It installs a footline showing the author's
  name, the institution, the presentation's title, the current date,
  and a frame count. This theme uses only little space.

  The colors used in the headline and footline are drawn from
  |palette primary|, |palette secondary|, and |primary ternary| (see
  Section~\ref{section-colors} for details on how to change these).
\end{outerthemeexample}

\begin{outerthemeexample}[\oarg{options}]{miniframes}
  This theme installs a headline in which a horizontal navigational
  bar is shown. This bar contains one entry for each section of the
  presentation. Below each section entry, small circles are shown that
  represent the different frames in the section. The frames are
  arranged subsection-wise, that is, there is a line of frames for
  each subsection. If the class  option |compress| is given, the
  frames will instead be arranged in a single row for each
  section. The navigation bars draws its color from
  |section in head/foot|.

  Below the navigation bar, a line is put showing the title of the
  current subsection. The color is drawn from |subsection in head/foot|.

  At the bottom, two lines are put that contain information such as
  the author's name, the institution, or the paper's title. What is
  shown exactly is influenced by the \meta{options} given. The colors
  are drawn from the appropriate \beamer-colors like
  |author in head/foot|.

  At the top and bottom of both the head- and footline and between the
  navigation bar and the subsection name, separation lines are drawn
  \emph{if} the background color of |separation line| is set. This
  separation line will have a height of 3pt. You can get even more
  fine-grained control over the colors of the separation lines by
  setting appropriate colors like |lower separation line head|.

  The following \meta{options} can be given:
  \begin{itemize}
  \item \declare{|footline=empty|} suppressed the footline (default).
  \item \declare{|footline=authorinstitute|} shows the author's name
    and the institute in the footline.
  \item \declare{|footline=authortitle|} shows the author's name
    and the title in the footline.
  \item \declare{|footline=institutetitle|} shows the institute
    and the title in the footline.
  \item \declare{|footline=authorinstitutetitle|} shows the author's
    name, the institute, and the title in the footline.
  \item \declare{|subsection=|\meta{true or false}} shows or supresses
    line showing the subsection in the headline. It is shown by
    default.
  \end{itemize}  
\end{outerthemeexample}

\begin{outerthemeexample}[\oarg{options}]{smoothbars}
  This theme behaves very much like the |miniframes| theme, at least
  with respect to the headline. There, the only difference is the
  smooth transitions are installed between the background colors of
  the navigation bar, the (optional) bar for the subsection name, and
  the background of the frame title. No footline is created. You can
  get the footlines of the |miniframes| theme by first loading the
  theme and then loading the |smoothbars| theme.

  The following \meta{options} can be given:
  \begin{itemize}
  \item \declare{|subsection=|\meta{true or false}} shows or supresses
    line showing the subsection in the headline. It is shown by
    default.
  \end{itemize}  
\end{outerthemeexample}

\begin{outerthemeexample}[\oarg{options}]{sidebar}
  In this layout, a sidebar is shown that contains a small table of
  contents with the current section or subsection is hilighted. The
  frame title is vertically centered in a rectangular area at the top
  that always occupies the same amount of space in all
  frames. Finally, the logo is shown in the ``corner'' resulting from
  the sidebar and the frame title rectangle.

  There are several ways of modifying the layout using the
  \meta{options}. If you set the width of the sidebar to 0pt, it is
  not shown, giving you a layout in which the frame title does not
  ``wobble'' since it always occupies the same amount of space on all
  slides. Conversely, if you set the height of the frame title
  rectangle to 0pt, the rectangular area is not used and the frame
  title is inserted normally (occupying as much space as needed on
  each slide).

  The background color of the sidebar is taken from |sidebar|, the
  background color of the frame title from |frametitle|, and the
  background color of the logo corner from |logo|.

  The colors of the entries in the table of contents are drawn from
  the \beamer-color |section in sidebar| and |section in sidebar current| as well as the
  corresponding \beamer-colors for subsections. If an entry does not
  fit on a single line it is automatically ``linebroken.'' 

  The following \meta{options} maybe given:
  \begin{itemize}
  \item
    \declare{|height=|\meta{dimension}} specifies the height of the
    frame title rectangle. If it is set to 0pt, no frame title
    rectangle is created. Instead, the frame title is inserted
    normally into the frame. The default is 2.5 base line heights of
    the frame title font. Thus, there is about enough space for a
    two-line frame title plus a one-line subtitle.
  \item
    \declare{|hideothersubsections|} causes all subsections except
    those of the current section to be supressed in the table of
    contents. This is useful if you have lot's of subsections.
  \item
    \declare{|hideallsubsections|} causes all subsections to be
    supressed in the table of contents.
  \item
    \declare{|left|} puts the sidebar on the left side. Note that in a
    left-to-right reading culture this is side people look first. Note
    also that this table of contents is usually \emph{not} the most
    important part of the frame, so you do not necessarily want people
    to look at is first. Nevertheless, it is the default.
  \item
    \declare{|right|} puts the sidebar of the right side.
  \item
    \declare{|width=|\meta{dimension}} specifies the width of the
    sidebar. If it is set to 0pt, it is completely supressed. The
    default is 2.5 base line heights of the frame title font.
  \end{itemize}
\end{outerthemeexample}

\begin{outerthemeexample}{split}
  This theme installs a headline in which, on the left, the sections
  of the talk are shown and, on the right, the subsections of the
  current section. If the class option |compress| has been given,
  the sections and subsections will be put in one line; normally there
  is one line per section or subsection.

  The footline shows the author on the left and the talk's title on
  the right.

  The colors are taken from |palette primary| and |palette fourth|.
\end{outerthemeexample}

\begin{outerthemeexample}{shadow}
  This layout theme extends the |split| theme by putting a horizontal
  shading behind the frame title and adding a little ``shadow'' at the
  bottom of the headline.
\end{outerthemeexample}

\begin{outerthemeexample}[\oarg{options}]{tree}
  In this layout, the headline contains three lines that show the
  title of the current talk, the current section in this talk, and the
  current subsection in the section. The colors are drawn from
  |title in head/foot|, |section in head/foot|, and
  |subsection in head/foot|.

  In addition, separation lines of height 3pt are shown above and
  below the three lines \emph{if} the background of |separation line|
  is set. More fine-grained control of the colors of these lines can
  be gained by setting |upper separation line head| and
  |lower separation line head|.

  The following \meta{options} may be given:
  \begin{itemize}
  \item
    \declare{|hooks|} causes little ``hooks'' to be drawn in front of
    the section and subsection entries. These are supposed to increase
    the tree-like appearance. 
  \end{itemize}
\end{outerthemeexample}

\begin{outerthemeexample}{smoothtree}
  This layout is similar to the |tree| layout. The main difference is
  that the background colors change smoothly.
\end{outerthemeexample}





\subsection{Changing the Templates Used for Different Elements of a Presentation}

This section explains how \beamer's template management works.



\subsubsection{Overview of Beamer's Template Management}

If you only wish to modify the appearance of a single or few elements,
you do not need to create a whole new inner or outer theme. Instead,
you can modify the appropriate template. 

A template specifies how an element of a presentation is typeset. For 
example, the |frametitle| template dictates where the frame title is
put, which font is used, and so on.

As the name suggests, you specify a template by writing the exact
\LaTeX\ code you would also use when typesetting a single frame title
by hand. Only, instead of the actual title, you use the command
|\insertframetitle|.

\example
Suppose we would like to have the frame title typeset in
red, centered, and boldface. If we were to typeset a single frame
title by hand, it might be done like this:
\begin{verbatim}
\begin{frame}
  \begin{centering}
    \color{red}
    \textbf{The Title of This Frame.}
    \par
  \end{centering}

  Blah, blah.
\end{frame}
\end{verbatim}

In order to typeset the frame title in this way on all slides, in the
simplest case we can change the frame title template as follows:
\begin{verbatim}
\setbeamertemplate{frametitle}
{
  \begin{centering}
    \color{red}
    \textbf{\insertframetitle}
    \par
  \end{centering}
}
\end{verbatim}

We can then use the following code to get the desired effect:
\begin{verbatim}
\begin{frame}
  \frametitle{The Title of This Frame.}

  Blah, blah.
\end{frame}
\end{verbatim}
When rendering the frame, \beamer\ will use the code of the
frame title template to typeset the frame title and it will replace
every occurrence of |\insertframetitle| by the current frame
title.

We can take this example a step further. It would be nicer if we did
not have to ``hardwire'' the color of the frametitle, but if this
color could be specified independently of the code for the
template. This way, a color theme could change this color. Since this
a problem that is common to most templates, \beamer\ will
automatically setup the \beamer-color |frametitle| when the template
|frametitle| is used. Thus, we can remove the |\color{red}| command if
we set the \beamer-color |frametitle| to red at some point.
\begin{verbatim}
\setbeamercolor{frametitle}{fg=red}
\setbeamertemplate{frametitle}
{
  \begin{centering}
    \textbf{\insertframetitle}
    \par
  \end{centering}
}
\end{verbatim}
Next, we can also make the font ``themable.'' Just like the color, the
\beamer-font |frametitle| is installed before the |frametitle|
template is typeset. Thus, we should rewrite the code as follows:
\begin{verbatim}
\setbeamercolor{frametitle}{fg=red}
\setbeamerfont{frametitle}{series=\bfseries}
\setbeamertemplate{frametitle}
{
  \begin{centering}
    \insertframetitle\par
  \end{centering}
}
\end{verbatim}
Users, themes, or whoever can now easily change the color or font of
the frametitle without having to mess with the code used to typeset
it.







In the following subsections all commands for changing templates are
listed, like the above-mentioned command
|\useframetitletemplate|. Inside these commands, you should use
the |\insertxxxx| commands, which are listed following the template
changing commands. Although the |\insertxxxx| commands are listed
alongside the templates for which they make the most sense, you can
(usually) also use them in all other templates.

\articlenote
In |article| mode, most of the template mechanism is switched off and
has no effect. However, a few templates are also available. If this is
the case, it is specially indicated.
\smallskip

Some of the below subsections start with commands for using
\emph{predefined} templates. Calling one of them will change a template in
a predefined way, making it unnecessary to worry about how exactly one
creates, say, these cute little balls in different sizes. Using them,
you can use, for example, your favorite theme together with a
shading background and a numbered table of contents.

Here are a few hints that might be helpful when you wish to redefine a
template: 
\begin{itemize}
\item
  Usually, you might wish to copy code from an existing template. The
  code often takes care of some things that you may not yet have
  thought about. The file |beamerbasetemplates| might be useful
  starting point.
\item
  When copying code from another template and when inserting this code
  in the preamble of your document (not in another style file), you may
  have to ``switch on'' the at-character (|@|). To do so, add the
  command |\makeatletter| before the |\usexxxtemplate| command and the
  command |\makeatother| afterward.
\item
  Most templates having to do with the frame components (headlines,
  sidebars, etc.)\ can only be changed in the preamble. Other
  templates can be changed during the document.
\item
  The height of the headline and footline templates is calculated
  automatically. This is done by typesetting the templates and then
  ``having a look'' at their heights. This recalculation is done right
  at the beginning of the document, \emph{after} all packages have
  been loaded and even \emph{after} these have executed their
  |\AtBeginDocument| initialization. 
\item
  The left and right margins of the head- and footline templates are
  the same as of the normal text. In order to start the headline and
  footline at the page margin, you must insert a negative
  horizontal skip using |\hskip-\Gm@lmargin|. You may wish to add a
  |\hskip-\Gm@rmargin| at the end to avoid having \TeX\ complain about
  overfull boxes.
\item
  Getting the boxes right inside any template is often a bit of a
  hassle. You may wish to consult the \TeX\ book for the glorious
  details on ``Making Boxes.'' If your headline is simple, you might
  also try putting everything into a |pgfpicture| environment, which
  makes the placement easier.
\end{itemize}

\subsubsection{Using Beamer's Templates}

\subsubsection{Setting Beamer's Templates}



%%% Local Variables: 
%%% mode: latex
%%% TeX-master: "beameruserguide"
%%% End: 
