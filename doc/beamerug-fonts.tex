
% Copyright 2003, 2004 by Till Tantau <tantau@users.sourceforge.net>.
%
% This program can be redistributed and/or modified under the terms
% of the GNU Public License, version 2.


\section{Fonts}

Text and fonts literally surround us constantly. Try to think of the
last time when  there was no text around you within ten
meters. Likely, this has never happened in your life! 
(Whenever you wear clothing, even a swim suit, there is a lot of text
right next to your body.) The history of fonts is nearly as long as
the history of civilization itself. There are tens of thousands of fonts
available these days, some of which are the product of hundreds of
years of optimization.

Choosing the right fonts for a presentation is by no means
trivial and wrong choices will either just ``look bad'' or, worse,
make the audience having trouble reading your slides.
This user's guide cannot replace a good book on typography, but in the
present section you'll find several hints that should help you setup 
fonts for a \beamer\ presentation that look good.

This section is organized as follows. The first subsection 
describes the different attributes a font can have and tells you what
choices there are. The next subsection introduces the predefined font
themes that come with \beamer\ and which make it easy to change the
fonts used in a presentation. The next subsection describes further
special commands for changing some basic attributes of the fonts used in a
presentation. The last subsection explains how you can get a much more
fine-grained control over the fonts used for every individual element
of a presentation. 





\subsection{Font Themes}

\beamer\ comes with a set of font themes. When you use such a thme,
certain fonts are changed as described below. You can use several font
themes in concert. For historical reasons, you cannot change all
aspects of the fonts used using font themes---in same cases special
commands and options are needed, which are described in the next
subsection.

The following font themes only change certain font attributes, they do
not choose special font families (although that would also be possible
and themes doing just that might be added in the future). Currently,
to change the font family, you need to load special packages as
explained in the next subsection.


\begin{fontthemeexample}{default}
  The default font theme installs a sans serif font for all text of
  the presentation. The default theme installs different font sizes
  for things like titles or head- and footlines, but does not use
  boldface or italics for ``hilighting.''  
\end{fontthemeexample}


\begin{fontthemeexample}[\oarg{options}]{structurebold}
  This font theme will cause titles and text in the headlines,
  footlines, and sidebars to be typeset in a bold font.

  The following \meta{options} may be given:
  \begin{itemize}
  \item
    \declare{|onlysmall|}
    will cause only ``small'' text to be typeset in bold. More
    precisely, only the text in the headline, footline, and sidebars
    is changed to be typeset in bold. Large titles are not affected.
  \item
    \declare{|onlylarge|}
    will cause only ``large'' text to be typeset in bold. These are
    the main title, frame titles, and section entries in the table of
    contents.     
  \end{itemize}

  As pointed out in Section~\ref{section-sizes}, you should use this
  theme (possibly with the |onlysmall| option) if your font is not
  scaled down properly or for light-on-dark text.

  The normal themes do not install this theme by default, while the
  old compatibility themes do. Since you can reload the theme once it
  has been loaded, you cannot use this theme with the old
  compatibility themes to set also titles to a bold font. 
\end{fontthemeexample}

\begin{fontthemeexample}[\oarg{options}]{structureitalicserif}
  This theme is similarly as the |structurebold| font theme, but where
  |structurebold| makes text bold, this theme typesets it in italics and
  in the standard serif font. The same \meta{options} as for the
  |structurebold| theme are supported. See
  Section~\ref{section-italics} for the pros and cons 
  of using italics.  
\end{fontthemeexample}

\begin{fontthemeexample}[\oarg{options}]{structuresmallcapsserif}
  Again, this theme does exactly the same as the |structurebold| font theme,
  only this time text is set using small caps and a serif
    font. The same \meta{options} as for the |structurebold| theme are
  supported. See Section~\ref{section-smallcaps} for the pros and cons
  of using small caps.  
\end{fontthemeexample}

\begin{fontthemeexample}[\oarg{options}]{structuresansserif}
  Another theme that works like the |structurebold| theme and accepts the same
  options. Instead of making text set on bold, it causes text to be
  typeset using a sans-serif font. Naturally, using this theme makes
  only sense if you use the |serif| class option since otherwise is
  typeset in sans serif anyway. However, when using the |serif|
  option, it is often a good idea to typeset at least navigation text
  in headlines or sidebars using sans serif font, which are easier to
  read at small sizes.
\end{fontthemeexample}

\begin{fontthemeexample}[\oarg{options}]{structureserif}
  Like |structuresansserif|, except that the text is typeset using a serif
  font instead of sans serif. Using serif text for small text is
  dangerous, but using a serif (possibly italic) title over a sans
  serif text creates an interesting visual effect. Naturally,
  ``interesting typographic effect'' can mean ``terrible typographic
  effect'' if you choose the wrong fonts combinations or sizes. You'll
  need some typographic experience to judge this correctly. If in
  doubt, try asking someone who should now.
\end{fontthemeexample}






\subsection{Font Changes That Cannot Be Made Using Font Themes} 

While most font decisions can be made using font themes, for
historical reasons some changes can only be made using class
options or by loading special packages. These options are explained in
the following. Possibly, these options will be replaced by themes in
the future. 



\subsubsection{Choosing Serif or Sans-Serif Text}

The class options |sans| and |serif| globally decide whether normal
text should be typeset in sans-serif or in serif. By default, |sans|
is selected, so you do not need to specify this.

\begin{classoption}{sans}
  Use a sans-serif font during the presentation. (Default.)
\end{classoption}

\begin{classoption}{serif}
  Use a serif font during the presentation.
\end{classoption}

You can achieve similar effects in other ways, but using these options
is what I recommend since they take care of several remappings
``behind the scenes.'' For example, the |sans| option also cause
sans-serif fonts to be used in mathematical text, which is hard to
achieve otherwise.

By default, if a sans-serif font is used for the main text,
mathematical formulas are also typeset using sans-serif letters. In
most cases, this is visually the most pleasing and easily readable way of
typesetting mathematical formulas. However, in mathematical texts the
font used to render, say, a variable is sometimes used to
differentiate between different meanings of this variable. In such
case, it may be necessary to typeset mathematical text using serif
letters. Also, if you have a lot of mathematical text, the audience
may be quicker to ``parse'' it, if it is typeset the way people
usually read mathematical text: in a serif font.

You can use the two options |mathsans| and |mathserif| to override the
overall sans-serif/serif choice for math text. However, using the option
|mathsans| in a |serif| environment makes little sense in my opinion.

\begin{classoption}{mathsans}
  Override the math font to be a sans-serif font.
\end{classoption}

\begin{classoption}{mathserif}
  Override the math font to be a serif font.
\end{classoption}

The command |\mathrm| will always produce upright (not slanted), serif
text and the command |\mathsf| will always produce upright, sans-serif
text. The command |\mathbf| will produce upright, bold-face,
sans-serif or serif text, depending on whether |mathsans| or
|mathserif| is used.

To produce an upright, sans-serif or serif text, depending on
whether |mathsans| or |mathserif| is used, you can use for instance
the command |\operatorname| from the |amsmath| package. Using this
command instead of |\mathrm| or |\mathsf| directly will  automatically
adjust  upright mathematical text if you switch from sans-serif to
serif or back.


\subsubsection{Choosing a Font Size for Normal Text}

As pointed out in Section~\ref{section-sizes}, measuring the default
font size in points is not really a good idea for
presentations. Nevertheless, \beamer\ does just that, setting the
default font size to 11pt as usual. This may seem ridiculously small, but 
the actual size of each frame is just 128mm times 96mm and the viewer
application enlarges the font. By specifying a default font size
smaller than 11pt you can put more onto each slide, by specifying a
larger font size you can fit on less.

To specify the font size, you can use the following class options:

\begin{classoption}{8pt}
  This is way too small. Requires that the package |extsize|
  is installed.
\end{classoption}

\begin{classoption}{9pt}
  This is also too small. Requires that the package |extsize|
  is installed.
\end{classoption}

\begin{classoption}{10pt}
  If you really need to fit much onto each frame, use this
  option. Works without |extsize|.
\end{classoption}

\begin{classoption}{smaller}
  Same as the |10pt| option.
\end{classoption}

\begin{classoption}{11pt}
  The default font size. You need not specify this option.
\end{classoption}

\begin{classoption}{12pt}
  Makes all fonts a little bigger, which makes the text more
  readable. The downside is that less fits onto each frame.
\end{classoption}

\begin{classoption}{bigger}
  Same as the |12pt| option.
\end{classoption}

\begin{classoption}{14pt}
  Makes all fonts somewhat bigger. Requires |extsize| to be installed.
\end{classoption}

\begin{classoption}{17pt}
  This is about the default size of PowerPoint. Requires |extsize| to
  be installed. 
\end{classoption}

\begin{classoption}{20pt}
  This is really huge. Requires |extsize| to be installed.
\end{classoption}



\subsubsection{Choosing a Font Family}

\label{section-substition}

Independently of the serif/sans-serif choice, you can switch the font
family of normal text globally. To do so, you can use one of the
prepared packages of \LaTeX's font mechanism. For example, to change
to Times/Helvetica, simply add 
\begin{verbatim}
\usepackage{mathptmx}
\usepackage{helvet}
\end{verbatim}
in your preamble. Note that if you do not specify |serif| as a
class option, Helvetica (not Times) will be selected as the text
font.

There may be many other fonts available on your
installation. Typically, at least some of the following packages
should be available: |avant|, |bookman|, |chancery|, |charter|,
|euler|, |helvet|, |mathtime|, |mathptm|, |mathptmx|, |newcent|,
|palatino|, |pifont|, |utopia|.

If you use the |mathtime| package (you have to buy some of the fonts),
you also need to specify the |serif| option.

If you use professional fonts (fonts that you buy and that come with a
complete set of every symbol in all modes), you may need to specify the
class option |professionalfont|. This will tell \beamer\ that it
should not meddle with the fonts you use. The reason is that \beamer\ 
normally replaces certain character glyphs in mathematical text by
more appropriate versions. For example, \beamer\ will normally replace
glyphs such that the italic characters from the main font are used for
variables in mathematical text. If your professional font package
takes care of this already, \beamer's meddling should be switched
off. Note that \beamer's substitution is automatically turned off if
one of the following packages is loaded: |mathtime|, |mathpmnt|,
|lucidabr|, |mtpro|, and |hvmath|. If your favorite professional font
package is not among these, use the |professionalfont| option (and
write me an email, so that the package can be added).

\begin{classoption}{professionalfont}
  Deactivates \beamer's internal font replacements for mathematical
  text. This option should be used if you use a professional font
  package that sets up all mathematical fonts correctly.
\end{classoption}



\subsubsection{Choosing a Font Encodings}
\label{section-font-encoding}

The same font can come in different encodings, which are (very roughly
spoken) the ways the characters of a text are mapped to glyphs (the
actual shape of a particular character in a particular font at a
particular size). In \TeX\ two encodings are often used: the
T1~encoding and the OT1~encoding (old T1~encoding).

Conceptually, the newer T1~encoding is preferable over the old
OT1~encoding. For example, hyphenation of words containing umlauts
(like the famous German word Fr\"aulein) will work only if you use the
T1~encoding. Unfortunately, only the bitmapped version of the Computer
Modern fonts are available in this encoding in a standard
installation. For this reason, using the T1~encoding will produce
\pdf\ files that render very poorly.

Most standard PostScript fonts are available in T1~encoding. For
example, you can use Times in the T1~encoding. The package |lmodern|
makes the standard Computer Modern fonts available in the
T1~encoding. Furthermore, if you use |lmodern| several extra fonts
become available (like a sans-serif boldface math) and extra symbols
(like proper guillemots).

To select the T1 encoding, use |\usepackage[T1]{fontenc}|. Thus, if
you have the |lmodern| fonts installed, you could write
\begin{verbatim}
\usepackage{lmodern}
\usepackage[T1]{fontenc}
\end{verbatim}
to get beautiful outline fonts and correct hyphenation. Note, however,
that some versions of the |lmodern| package do not include correct
glyphs for ligatures like ``fi,'' which may cause
trouble. Double-check whether all ligatures are displayed correctly.





\subsection{Changing the Fonts Used for the Different Elements of a Presentation}




%%% Local Variables: 
%%% mode: latex
%%% TeX-master: "beameruserguide"
%%% End: 
