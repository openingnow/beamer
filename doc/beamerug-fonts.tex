
% Copyright 2003, 2004 by Till Tantau <tantau@users.sourceforge.net>.
%
% This program can be redistributed and/or modified under the terms
% of the GNU Public License, version 2.


\section{Fonts}

Text and fonts literally surround us constantly. Try to think of the
last time when  there was no text around you within ten
meters. Likely, this has never happened in your life! 
(Whenever you wear clothing, even a swim suit, there is a lot of text
right next to your body.) The history of fonts is nearly as long as
the history of civilization itself. There are tens of thousands of fonts
available these days, some of which are the product of hundreds of
years of optimization.

Choosing the right fonts for a presentation is by no means
trivial and wrong choices will either just ``look bad'' or, worse,
make the audience having trouble reading your slides.
This user's guide cannot replace a good book on typography, but in the
present section you'll find several hints that should help you setup 
fonts for a \beamer\ presentation that look good.

This section is organized as follows. The first subsection 
describes the different attributes a font can have and tells you what
choices there are. The next subsection introduces the predefined font
themes that come with \beamer\ and which make it easy to change the
fonts used in a presentation. The next subsection describes further
special commands for changing some basic attributes of the fonts used in a
presentation. The last subsection explains how you can get a much more
fine-grained control over the fonts used for every individual element
of a presentation. 



\subsection{A Review of Font Attributes}

A font has numerous attributes like weight, family, or size. All of
these have an impact on the usability of the font in
presentations. In the following, these attributes are described and
advantages and disadvantages of the different choices are sketched.
A description of how to actually change font attributes globally or
locally is given in the next sections.


\subsubsection{The Font Size}

\label{section-sizes}

Perhaps the most obvious attribute of a font is its size. Fonts are
traditionally measured in ``points.'' How much a point is depends on
whom you ask. \TeX\ thinks a point is the 72.27th part of an inch,
which is 2.54 cm. On the other hand, PostScript and Adobe think a
point is the 72th part of an inch (\TeX\ calls this a big point). The
are differences between American and European points. Once it is
settled how much a point is, claiming that a text is in ``11pt'' means
that the ``height'' of the letters in the font are 11pt. However, this
``height'' stems from the time when letters where still cast in lead
and refers to the the vertical size of the lead letters. It thus does
not need to have any correlation with the actual height of, say, the
letter x or even the letter M. The letter x of an 11pt Times from
Adobe will have a different height than the letter x of an 11pt Times
from UTCs and it will have a totally different height than the letter
x of an 11pt Helvetica from Adobe.

Summing up, the font size has little to do with the actual size of
letters. Rather, these days it is a convention that 10pt or 11pt is
the size a font should be printed for ``normal reading.'' Fonts are
designed so that they can optimally be read at these sizes.

In a presentation the classical font sizes obviously loose their
meaning. Nobody could read a projected text if it were actually
11pt. Instead, the projected letters need to be several centimeters
high. Thus, it does not really make sense to specify ``font sizes''
for presentations in the usual way. Instead, you should try to think
of the number of lines that will fit on a slide if you were to fill
the whole slide with line-by-line text (you are never going to do that
in practice, though). Depending on how far your audience is removed
from the projection on on large the projection is, between 10 and 20
lines should fit on a slide. The less lines, the more readable your
text will usually be.

In \beamer, the default sizes of the fonts are chosen in a way that
makes it difficult to fit ``too much'' onto a slide. Also, it will
ensure that your slides are readable even under bad conditions like a
large room and a small only a small projection area. However, you may
wish to enlarge or shrink the fonts a bit if you know this to be more
appropriate in your presentation environment.

Once the size of the normal text is settled, all other sizes are
usually defined relative to that size. For this reason, \LaTeX\ has 
commands like |\large| or |\small|. The actual size these commands
select depends on the size of normal text.

In a presentation, you will want to use a very small font for text in
headlines, footlines, or sidebars since the text shown there is not
vital and is read at the audience leasure. Naturally, the text should
still be large enough that it actually \emph{can} be read without
binoculars. However, in a normal presentation environment the audience
will still be able to even |\tiny| text from time to time.

However, using small fonts can be tricky. Many PostScript fonts are
just scaled down when used at small sizes. When a font is
used at less than its normal size, the characters should actually be
stroked using a slightly thicker ``pen'' than the one would expect
when just scaling down characters. For this reason, high quality
multiple master fonts or the Computer Modern fonts use a differents
fonts for small characters and for normal characters. However, when
you use a normal Helvetica or Times font, the characters are just
scaled down. A similar problem arises when you use a light font on a
dark background. Even when printed on paper in high resolution,
light-on-dark text tends to be ``overflooded'' by the dark
background. When light-on-dark text is rendered in a presentation this
effect can be much worse, making the text almost impossible to read.

You can counter both nagative effects by using a bold versions for
small text.

In the other direction, you can use larger text for titles. However,
using a larger font does not always have the desired effect. Just
because a frame title is printed in large letters does not
mean that is read first. Indeed, have a look at the cover of your
favorite magazine. Most likely, the magazine's name is the typeset in the
largest font, but you your attention will nevertheless first go to the
topics advertised on the cover. Likewise, in the table of contents you
are likely to first focus on the entries, not on the words ``Table of
Contents.'' Most likely, you would not spot a spelling mistake there
(a friend of mine actually managed to misspell \emph{his own name} on
the cover of his master's thesis and nobody noticed until a year
later). In essence, large text at the top of a page signals
``unimportant since I know what to expect.'' So, instead of using a
very large frame title, also consider using a normal size frame title
that is typeset in bold or in italics. 





\subsubsection{The Font Family}

The other central property of any font is its family. Examples of font
families are Times or Helvetica or Futura. As the name suggests, a lot
of different fonts can belong to the same family. For example, Times
comes in different sizes, there is a bold version of Times, an
italics version, and so on. To confuse matters, font families like
Times are often just called the ``font Times.''

The are two large classes of font families: serif fonts and
sans-serif fonts. A sans-serif font is a font in
which the letters do not have serifs (from French \emph{sans}, which
means ``without''). Serifs are the little hooks at the ending of the
strokes that make up a letter. The font you are currently reading is a
serif font. \textsf{By comparison, this text is in a sans-serif font.}
Sans-serif fonts are (generally considered to be) easier to read
when used in a presentation. In low resolution rendering, serifs
decrease the legibility of a font. However, on projectors with very
high resolution serif text is just as readable as sans-serif text. A
presentation typeset in a serif font creates a more conservative
impression, which might be exactly what you wish to create. 

Most likely, you'll have a lot of different font families preinstalled
on you system. The default font used by \TeX\ (and \beamer) is the
Computer Modern font. It  is the original font family designed by Donald
Knuth himself for the \TeX\ program. It is a mature font that comes
with just about everything you could wish for: extensive mathematical
alphabets, outline PostScript versions, real small caps, real oldstyle
numbers, specially designed small and large letters, and so on.

However, there are reasons using other font families than Computer
Modern: 
\begin{itemize}
\item
  The Computer Modern fonts are a bit boring if you have seen them too
  often. Using another font (but not Times!) can give a fresh look.
\item
  Other fonts, especially Times and Helvetica, are sometime rendered
  better since they seem to have better internal hinting.
\item
  The sans-serif version of Computer Modern is not nearly as
  well-designed as the serif version. Indeed, the sans-serif version
  is, in essence, the serif version with different design parameters,
  not an independent design.
\item
  Computer modern needs much more space than more economic fonts like
  Times (this explains why Times is so popular with people who need
  to sqeeze their great paper into just twelve pages). To be fair,
  Times was specifically designed to be economic.
\end{itemize}

A small selection of alternatives to Computer Modern:
\begin{itemize}
\item
  Helvetica is a very often-used alternative. However, Helvetica also
  tends to look boring (since we see it everywhere) and it has a very
  large x-height (the height of the letter~x in comparison to a letter
  like~M). A large x-height is usually considered good for languages
  (like English) that use uppercase letters seldomly and not-so-good
  for languages (like German) that use uppercase letters a lot. (I
  have never been quite convinced by the argument for this, though.)
  Be warned: the x-height of Helvetica is so different from the
  x-height of Times that mixing the two in a single line looks
  strange.
\item
  Futura is, in my opinion, a beautiful font that is very well-suited
  for presentations. It's thick letters make it robust against
  scaling, inversion, and low contrast. Unfortunately, while it is
  most likely installed on your system somewhere in some for, getting
  \TeX\ to work with it is a complicated process.
\item
  Times is a possible alternative to Computer Modern. Its main
  disadvantage is that it is a serif font, which requires a
  high-resolution projector. Naturally, it also used very often, so we
  all know it very well.
\end{itemize}

Families that you should \emph{not} use for normal text include:
\begin{itemize}
\item
  All monospaced fonts (like Courier).
\item
  Script fonts (which look like handwriting). Their stroke width is
  way too small for a presentation.
\item
  More delicate serif fonts like Stempel and possibly even Garamond.
\item
  Gothic fonts. Only a small fraction of your audience will be able to
  read them fluently. 
\end{itemize}

There is one popular font that is a bit special: Microsoft's Comic
Sans. On the one hand, there is a website lobbying for banning the use
of this font. Indeed, the main trouble with the font is that it is not
particularly well-readable and that math typeset partly using this
font looks terrible. On the other hand, this font \emph{does} create
the impression of a slide ``written by hand,'' which gives the
presentation a natural look. Think twice before using this font, but
do not let yourself be intimidated.

One of the most important rules of typography is that you should use
as little fonts as possible in a text. In particular, typographic
wisdom dictates that you should not use more than two different
families on one page. However, when typesetting
mathematical text, it is often necessary and useful to use different
font families. For example, it used to be common practice to use
Gothic letters do denote vectors. Also, program texts are often
typeset in monospace fonts. If your audience is used to a certain font
family for a certain type of text, use that family, regardless of what
typographic wisdom says.

A common practice in typography is to use a sans serif fonts for
titles and serif fonts for normal text (check your favorite
magazine). You can \emph{also} use two different sans serif fonts or
two differen serif fonts, but you then have to make sure that the
fonts look ``sufficiently different.'' If they look only slightly
different, the page will look ``somehow strange,'' but the audience
will not be able to tell why. For example, do not mix Arial and
Helvetica (they are almost identical) or Computer Modern and
Baskerville (they are quite similar). A combination of Gills Sans and
Helvetica is dangerous but perhaps possible. A combination like Futura
and Optima is certainly ok, at least with respect to the fonts being
very different.




\subsubsection{The Font Shape: Italics and Small Capitals}

\label{section-italics}
\label{section-smallcaps}

\LaTeX\ introduces the concept of the \emph{shape} of a font. The only
really important ones are italic and small caps.
An \emph{italic} font is a font in which the text is slightly slanted
to the right \emph{like this}. Things to know about
italics:
\begin{itemize}
\item
  Italics are commonly used in novels to express emphasis. However,
  especially with sans-serif fonts, italics are typically not ``strong
  enough'' and the emphasis gets lost. Using a different color or bold
  text seems better suited for presentations to create emphasis.
\item
  If you look closely, you'll notice that italic text is not only
  slanted but that different letters are actually used (compare a and
  \emph{a}, for example). However, this is only true for serif text,
  not for sans-serif text. Text that is only slanted without using
  different characters is called  ``slanted'' instead of ``italic.''
  Sometimes, the word ``oblique'' is also used for slanted, but it
  sometimes also used for italics, so it is perhaps best to avoid
  it. Using slanted serif text is very much frowned upon by
  typographers and is considered ``cheap computer typography.''
  However, people who use slanted text in their books include Donald
  Knuth.

  In a presentation, if you go into the trouble of using a serif font
  for some part of it, you should also use italics, not slanted text.
\item
  The different characters used for serif italics have changed much
  less  from the original handwritten letters they are based on than
  normal serif text. For this reason, serif italics creates the
  impression of handwritten text, which may be desirable to give a
  presentation a more ``personal touch'' (although you can't get very
  personal using Times italics, which everyone has seen a thousand
  times). However, it is harder to read than normal text, so do not
  use it for text more than a line long.
\end{itemize}

The second font shape supported by \TeX\ are small capital
letters. Using them can create a conservative, even formal
impression, but some words of caution:

\begin{itemize}
\item
  Small capitals are different from all-uppercase text. A small caps
  text leaves normal uppercase letters unchanged and uses smaller
  versions of the uppercase letters for normal typesetting lowercase
  letters. Thus the word ``German'' is typeset as \textsc{German}
  using small caps, but as \uppercase{German} using all uppercase
  letters.
\item
  Small caps either come as ``faked'' small caps or as ``real''
  small caps. Faked small caps are created by just scaling down
  normal uppercase letters. This leads to letters the look too
  thin. Real small caps are specially designed smaller versions of
  the uppercase letters that have the same stroke width as normal
  text.
\item
  Computer Modern fonts and expert version of PostScript fonts come
  with real small caps (though the small caps of Computer Modern are
  one point size too large for some unfathomable reason---but your
  audience is going to pardon this since it will not notice
  anyway). ``Simple'' PostScript fonts like Helvetica or Times only
  come with faked small caps.
\item
  Text typeset in small caps is harder to read than normal text. The
  reason is that we read by seeing the ``shape'' of words. For
  example, the word ``shape'' is mainly recognized by seing one
  normal letter, one ascending letter, a normal letter, one
  descending letter, and a normal letter. One has much more trouble
  spotting a misspelling like ``shepe''  than ``spape''. Small caps
  destroy the shape of words since \textsc{shape}, \textsc{shepe}
  and \textsc{spape} all have the same shape, thus making it much
  harder to tell them apart. You audience will read small caps more
  slowly than normal text. This is, by the way, why legal
  disclaimers are often written in uppercase letters: not to make
  them appear more important to you, but to make them much harder to
  actually read.
\end{itemize}



\subsubsection{The Font Weight}

The ``weight'' of a font refers to the thickness of the
letters. Usually, font come as regular or as bold fonts. There often
also exist semibold, ultrabold (or black), thin, or ultrathin
versions.

In typography, using a bold font to create emphasis, especially within
normal text, is frowned upon (bold words in the middle of a normal
text are refered to as ``dirt''). For presentations this rule of not
using bold text does not really apply. On a presentation slide there
is usually very little text and there are numerous elements that try
to attract the viewer's attention. Using the traditional italics to
create emphasis will often be overlooked. So, using bold text, seems a
good alternative in a presentation. However, an even better
alternative is using a bright color like red to attract attention.

As pointed out earlier, you should use bold text for small text
unless you use an especially robust font like Futura.






\subsection{Font Themes}

\beamer\ comes with a set of font themes. When you use such a thme,
certain fonts are changed as described below. You can use several font
themes in concert. For historical reasons, you cannot change all
aspects of the fonts used using font themes---in same cases special
commands and options are needed, which are described in the next
subsection.

The followin font themes only change certain font attributes, they do
not choose special font families (although that would also be possible
and themes doing just that might be added in the future). Currently,
to change the font family, you need to load special packages as
explained in the next subsection.



\begin{fontthemeexample}[\oarg{options}]{bold}
  This font theme will cause titles and text in the headlines,
  footlines, and sidebars to be typeset in a bold font.

  The following \meta{options} may be given:
  \begin{itemize}
  \item
    \declare{|onlysmall|}
    will cause only ``small'' text to be typeset in bold. More
    precisely, only the text in the headline, footline, and sidebars
    is changed to be typeset in bold. Large titles are not affected.
  \item
    \declare{|onlylarge|}
    will cause only ``large'' text to be typeset in bold. These are
    the main title, frame titles, and section entries in the table of
    contents.     
  \end{itemize}

  As pointed out in Section~\ref{section-sizes}, you should use this
  theme (possibly with the |onlysmall| option) if your font is not
  scaled down properly or for light-on-dark text.

  The normal themes do not install this theme by default, while the
  old compatibility themes do. Since you can reload the theme once it
  has been loaded, you cannot use this theme with the old
  compatibility themes to set also titles to a bold font. 
\end{fontthemeexample}

\begin{fontthemeexample}[\oarg{options}]{italics}
  This theme does exactly the same as the |bold| font theme, only
  instead of making text bold, text is set in italics. The same
  \meta{options} as for the |bold| theme are supported. See
  Section~\ref{section-italics} for the pros and cons of using
  italics. 
\end{fontthemeexample}

\begin{fontthemeexample}[\oarg{options}]{romanum}
  Again, this theme does exactly the same as the |bold| font theme,
  only this time text is set using small caps. The same
  \meta{options} as for the |bold| theme are supported. See
  Section~\ref{section-smallcaps} for the pros and cons of using
  italics.  
\end{fontthemeexample}

\begin{fontthemeexample}[\oarg{options}]{sansserifstructure}
  Another theme that works like the |bold| theme and accepts the same
  options. Instead of making text set on bold, it causes text to be
  typeset using a sans-serif font. Naturally, using this theme makes
  only sense if you use the |serif| class option since otherwise is
  typeset in sans serif anyway. However, when using the |serif|
  option, it is often a good idea to typeset at least navigation text
  in headlines or sidebars using sans serif font, which are easier to
  read at small sizes.
\end{fontthemeexample}

\begin{fontthemeexample}[\oarg{options}]{serifstructure}
  Like |sansserifstructure|, except that the text is typeset using a serif
  font instead of sans serif. Using serif text for small text is
  dangerous, but using a serif (possibly italic) title over a sans
  serif text creates an interesting visual effect. Naturally,
  ``interesting typographic effect'' can mean ``terrible typographic
  effect'' if you choose the wrong fonts combinations or sizes. You'll
  need some typographic  experience to judge this correctly. If in
  doubt, try asking someone who should now.
\end{fontthemeexample}





\subsection{Special Commands for Changing the Fonts Used in A Presentation}

\LaTeX\ offers numerous commands for changing the font
temporarily. For example, |\bfseries| will switch to boldface,
|\itshape| will switch to an italic font, and so on. However, to
globally change the font attributes of all fonts used in a
presentation, special commands are needed. In the following, these
commands are described.



\subsubsection{Choosing Between Serif Sans-Serif Text}

You must globally decide whether normal text should be typeset in
sans-serif or in serif. To choose this, use either the class option
|sans| or |serif|. By default, |sans| is selected, so you do not
need to specify this.

\begin{classoption}{sans}
  Use a sans-serif font during the presentation. (Default.)
\end{classoption}

\begin{classoption}{serif}
  Use a serif font during the presentation.
\end{classoption}

You can achieve similar effects in other ways, but using these options
is what I recommend since they take care of several remappings
``behind the scenes.'' For example, the |sans| option also cause
sans-serif fonts to be used in mathematical text, which is hard to
achieve otherwise.

By default, if a sans-serif font is used for the main text,
mathematical formulas are also typeset using sans-serif letters. In
most cases, this is visually the most pleasing and easily readable way of
typesetting mathematical formulas. However, in mathematical texts the
font used to render, say, a variable is sometimes used to
differentiate between different meanings of this variable. In such
case, it may be necessary to typeset mathematical text using serif
letters. Also, if you have a lot of mathematical text, the audience
may be quicker to ``parse'' it, if it is typeset the way people
usually read mathematical text: in a serif font.

You can use the two options |mathsans| and |mathserif| to override the
overall sans-serif/serif choice for math text. However, using the option
|mathsans| in a |serif| environment makes little sense in my opinion.

\begin{classoption}{mathsans}
  Override the math font to be a sans-serif font.
\end{classoption}

\begin{classoption}{mathserif}
  Override the math font to be a serif font.
\end{classoption}

The command |\mathrm| will always produce upright (not slanted), serif
text and the command |\mathsf| will always produce upright, sans-serif
text. The command |\mathbf| will produce upright, bold-face,
sans-serif or serif text, depending on whether |mathsans| or
|mathserif| is used.

To produce an upright, sans-serif or serif text, depending on
whether |mathsans| or |mathserif| is used, you can use for instance
the command |\operatorname| from the |amsmath| package. Using this
command instead of |\mathrm| or |\mathsf| directly will  automatically
adjust  upright mathematical text if you switch from sans-serif to
serif or back.




\subsubsection{Choosing a Font Family}

\label{section-substition}

Independently of the serif/sans-serif choice, you can switch the font
family of normal text globally. To do so, you can use one of the
prepared packages of \LaTeX's font mechanism. For example, to change
to Times/Helvetica, simply add 
\begin{verbatim}
\usepackage{mathptmx}
\usepackage{helvet}
\end{verbatim}
in your preamble. Note that if you do not specify |serif| as a
class option, Helvetica (not Times) will be selected as the text
font.

There may be many other fonts available on your
installation. Typically, at least some of the following packages
should be available: |avant|, |bookman|, |chancery|, |charter|,
|euler|, |helvet|, |mathtime|, |mathptm|, |mathptmx|, |newcent|,
|palatino|, |pifont|, |utopia|.

If you use the |mathtime| package (you have to buy some of the fonts),
you also need to specify the |serif| option.

If you use professional fonts (fonts that you buy and that come with a
complete set of every symbol in all modes), you may need to specify the
class option |professionalfont|. This will tell \beamer\ that it
should not meddle with the fonts you use. The reason is that \beamer\ 
normally replaces certain character glyphs in mathematical text by
more appropriate versions. For example, \beamer\ will normally replace
glyphs such that the italic characters from the main font are used for
variables in mathematical text. If your professional font package
takes care of this already, \beamer's meddling should be switched
off. Note that \beamer's substitution is automatically turned off if
one of the following packages is loaded: |mathtime|, |mathpmnt|,
|lucidabr|, |mtpro|, and |hvmath|. If your favorite professional font
package is not among these, use the |professionalfont| option (and
write me an email, so that the package can be added).

\begin{classoption}{professionalfont}
  Deactivates \beamer's internal font replacements for mathematical
  text. This option should be used if you use a professional font
  package that sets up all mathematical fonts correctly.
\end{classoption}





\subsubsection{Choosing a Font Size for Normal Text}

As pointed out in Section~\ref{section-sizes}, measuring the default
font size in points is not really a good idea for
presentations. Nevertheless, \beamer\ does just that, setting the
default font size to 11pt as usual. This may seem ridiculously small, but 
the actual size of each frame is just 128mm times 96mm and the viewer
application enlarges the font. By specifying a default font size
smaller than 11pt you can put more onto each slide, by specifying a
larger font size you can fit on less.

To specify the font size, you can use the following class options:

\begin{classoption}{8pt}
  This is way too small. Requires that the package |extsize|
  is installed.
\end{classoption}

\begin{classoption}{9pt}
  This is also too small. Requires that the package |extsize|
  is installed.
\end{classoption}

\begin{classoption}{10pt}
  If you really need to fit much onto each frame, use this
  option. Works without |extsize|.
\end{classoption}

\begin{classoption}{smaller}
  Same as the |10pt| option.
\end{classoption}

\begin{classoption}{11pt}
  The default font size. You need not specify this option.
\end{classoption}

\begin{classoption}{12pt}
  Makes all fonts a little bigger, which makes the text more
  readable. The downside is that less fits onto each frame.
\end{classoption}

\begin{classoption}{bigger}
  Same as the |12pt| option.
\end{classoption}

\begin{classoption}{14pt}
  Makes all fonts somewhat bigger. Requires |extsize| to be installed.
\end{classoption}

\begin{classoption}{17pt}
  This is about the default size of PowerPoint. Requires |extsize| to
  be installed. 
\end{classoption}

\begin{classoption}{20pt}
  This is really huge. Requires |extsize| to be installed.
\end{classoption}



\subsubsection{Choosing a Font Encodings}
\label{section-font-encoding}

The same font can come in different encodings, which are (very roughly
spoken) the ways the characters of a text are mapped to glyphs (the
actual shape of a particular character in a particular font at a
particular size). In \TeX\ two encodings are often used: the
T1~encoding and the OT1~encoding (old T1~encoding).

Conceptually, the newer T1~encoding is preferable over the old
OT1~encoding. For example, hyphenation of words containing umlauts
(like the famous German word Fr\"aulein) will work only if you use the
T1~encoding. Unfortunately, only the bitmapped version of the Computer
Modern fonts are available in this encoding in a standard
installation. For this reason, using the T1~encoding will produce
\pdf\ files that render very poorly.

Most standard PostScript fonts are available in T1~encoding. For
example, you can use Times in the T1~encoding. The package |lmodern|
makes the standard Computer Modern fonts available in the
T1~encoding. Furthermore, if you use |lmodern| several extra fonts
become available (like a sans-serif boldface math) and extra symbols
(like proper guillemots).

To select the T1 encoding, use |\usepackage[T1]{fontenc}|. Thus, if
you have the |lmodern| fonts installed, you could write
\begin{verbatim}
\usepackage{lmodern}
\usepackage[T1]{fontenc}
\end{verbatim}
to get beautiful outline fonts and correct hyphenation. Note, however,
that some versions of the |lmodern| package do not include correct
glyphs for ligatures like ``fi,'' which may cause
trouble. Double-check whether all ligatures are displayed correctly.





\subsection{Changing the Fonts Used for the Different Elements of a Presentation}




%%% Local Variables: 
%%% mode: latex
%%% TeX-master: "beameruserguide"
%%% End: 
