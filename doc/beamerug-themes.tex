
% Copyright 2003, 2004 by Till Tantau <tantau@users.sourceforge.net>.
%
% This program can be redistributed and/or modified under the terms
% of the GNU Public License, version 2.


\section{Themes}

\emph{Themes} make it easy to change the appearance of a
presentation. The \beamer\ class uses five different kinds of themes:
\begin{description}
\item[Presentation Themes]
  Conceptually, a presentation theme dictates for every single detail
  of a presentation what it looks like. Thus, choosing a particular
  presentation theme will setup for, say, the numbers in enumeration
  what color they have, what color their background has, what font is
  used to render them, whether a circle or ball or rectangle or
  whatever is drawn behind them, and so forth. Thus, when you choose
  a presentation theme, your presentation will look they way someone
  (the creator of the theme) thought that a presentation should look
  like. Presentation themes typically only choose a particular color
  theme, font theme, element theme, and layout theme that go well
  together. 
\item[Color Themes]
  A color theme only dictates which colors are used in a
  presentation. If you have choosen a particular presentation theme
  and then choose a color theme, only the colors of your presentation
  will change. A color theme can specify colors in a very detailed
  way: For example, a color theme can specifically change the colors
  used to render, say, the border of a button, the background of a
  button, and the text on a button.
\item[Font Themes]
  A font theme dictates which fonts or font attributes are used in a
  presentation. As for colors, the font of all text elements used in a
  presentation can be specified independently.
\item[Element Themes]
  An element theme specifies how certain ``elements'' of a
  presentation are typeset. ``Elements'' refers to everything that needs to
  be typeset and that is not part of the headlines, footlines, and
  sidebars. This includes all enumerations, itemize environments,
  block environments, theorem environments, or the table of
  contents. For example, an element theme might
  specify that in an enumeration the number should be typeset without
  a dot and that a small circle should be shown behind it. The element
  theme would \emph{not} specify what color should be used for the
  number or the circle (this is the job of the color theme) nor which font
  should be used (this is the job of the font theme).
\item[Layout Themes]
  A layout theme specifies what the layout of the presentation slides
  should look like. It specifies whether there are head- and
  footlines, what is shown in them, whether there is a sidebar, where
  the logo goes, where the navigation symbols and bars go, and so
  on. It also specifies where the frametitle is put and how it is
  typeset. 
\end{description}

The different themes reside in the four subdirectories |theme|, |color|,
|font|, |elements|, and |layout| of the directory
|beamer/themes|. Internally, a theme is stored as a normal style
file. However, to use a theme, the following special commands should
be used:

\begin{command}{\usetheme\oarg{options}\marg{name}}
  Installs the presentation theme named \meta{name}. Currently, the
  effect of this command is the same as saying |\usepackage| for the
  style file named |beamertheme|\meta{name}|.sty|.
\end{command}


\begin{command}{\usecolortheme\oarg{options}\marg{name}}
  Installs the color theme named \meta{name}. Currently, the effect of this
  command is the same as saying |\usepackage| for the style file named
  |beamercolortheme|\meta{name}|.sty|.
\end{command}

\begin{command}{\usefonttheme\oarg{options}\marg{name}}
  Installs the font theme named \meta{name}. Currently, the effect of this
  command is the same as saying |\usepackage| for the style file named
  |beamerfonttheme|\meta{name}|.sty|.
\end{command}

\begin{command}{\useelementtheme\oarg{options}\marg{name}}
  Installs the element theme named \meta{name}. Currently, the effect of this
  command is the same as saying |\usepackage| for the style file named
  |beamerelementtheme|\meta{name}|.sty|.
\end{command}

\begin{command}{\uselayouttheme\oarg{options}\marg{name}}
  Installs the layout theme named \meta{name}. Currently, the effect of this
  command is the same as saying |\usepackage| for the style file named
  |beamerlayouttheme|\meta{name}|.sty|.
\end{command}

If you do not use any of these commands, a sober \emph{default} theme
is used for all of them. In the following, the presentation, element,
and layout themes that come with the \beamer\ class are described. The
color themes and font themes are discussed in the sections on colors
and fonts, respectively.


\subsection{Guidelines}

\beamer\ comes with a number of different themes. Here are some
guidelines that might help you in choosing a theme for your
presentation: 

\begin{itemize}
\item
  Different themes are appropriate for different occasions. Do not
  become too attached to a favorite theme; choose a 
  theme according to occasion.
\item
  A longer talk is more likely to require navigational hints
  than a short one. When you give a 90 minute lecture to students, you
  should choose a theme that always shows a sidebar with the current
  topic hilighted so that everyone always knows exactly what's the
  current ``status'' of your talk is; when you give a ten-minute
  introductory speech, a table of contents is likely to just seem
  silly.
\item
  First choose a presentation theme that has a layout that is
  appropriate for your talk.
\item
  Next you might wish to change the colors by installing a different
  color theme. This can drastically change the appearance of your
  presentation. A ``colorful'' theme like |Berkeley| will look much
  less flashy sober if you use the color themes |seahorse| and |lily|. 
\item
  You might also wish to change the fonts by installing a different
  font theme.
\end{itemize}





\subsection{Presentation Themes}

A presentation theme dictates for every single detail
of a presentation what it looks like. Normally, having chosen a
particular presentation theme, you do not need to specify anything
else having to do with the appearence of your presentation---the
creator of the theme should have taken of that for you. However, you
still \emph{can} change things afterward either by using a different
color, font, element, or even layout theme; or by changing specific
colors, fonts, or templates directly.

When I started naming the presentation themes, I soon ran out of ideas
on how to call them. Instead of giving them more and more cumbersome
names, I decided to switch to a different naming convention:
Except for two special cases, all presentation themes are named after
cities. These cities happen to be cities in which or near which there
was a conference or workshop that I attended or that a
co-author of mine attended. 


\subsubsection{Presentation Themes Without Navigation Bars}

\begin{themeexample}{default}
  As the name suggests, this theme is installed by default. It is a
  sober no-nonsense theme that makes minimal use of color or font
  variations. This theme is useful for all kinds of talks, except for
  very long talks.
\end{themeexample}


\begin{themeexample}[{\opt{|[headheight=|\meta{head height}|,footheight=|\meta{foot height}|]|}}]{boxes}
  For this theme, you can specify an arbitrary number of templates for
  the boxes in the headline and in the footline. You can add a
  template for another box by using the following commands.
\end{themeexample}

\begin{command}{\addheadboxtemplate%
    \marg{background color command}\marg{box template}}
  Each time this command is invoked, a new box is added to the head
  line, with the first added box being shown on the left. All boxes
  will have the same size.
  \example
\begin{verbatim}
\addheadboxtemplate{\color{black}}{\color{white}\tiny\quad 1. Box}
\addheadboxtemplate{\usebeamercolor[bg]{normal text}}
  {\usebeamercolor[fg]{structure}\tiny\quad 2. Box}
\end{verbatim}
\end{command}

\begin{command}{\addfootboxtemplate%
    \marg{background color command}\marg{box template}}
  \example
\begin{verbatim}
\addfootboxtemplate{\color{black}}{\color{white}\tiny\quad 1. Box}
\addheadboxtemplate{\usebeamercolor[bg]{normal text}}
  {\usebeamercolor[fg]{structure}\tiny\quad 2. Box}
\end{verbatim}
\end{command}


\begin{themeexample}[\oarg{options}]{Madrid}
  A theme giving much information on little space. The present form
  was slightly adapted from a theme contributed by Manuel Carro.
  The following \meta{options} may be given:
  \begin{itemize}
  \item \declare{|secheader|} causes a headline to be inserted showing
    the current section and subsection. By default, this
    headline is not shown.
  \end{itemize}
\end{themeexample}


\begin{themeexample}{Pittsburgh}
  A sober theme. The right-flushed frame titles creates an interesting 
  ``tension'' inside each frame. 
\end{themeexample}


\begin{themeexample}[\oarg{options}]{Rochester}
  A dominant theme without any navigational elements. It can be made less
  dominant by using a different color theme.

  The following \meta{options} may be given:
  \begin{itemize}
  \item \declare{|height=|\meta{dimension}} sets the height of the
    frame title bar.
  \end{itemize}
\end{themeexample}




\subsubsection{Presentation Themes with a Tree-Like Navigation Bar}

\begin{themeexample}{Antibes}
  A dominant theme with a tree-like navigation at the top. The 
  rectangular elements mirror the rectangular navigation at the
  top. The theme can be made less dominant by using a different color
  theme. 
\end{themeexample}

\begin{themeexample}{JuanLesPins}
  A variation on the |Antibes| theme that has a much ``smoother''
  appearence. It can be made less dominant by chosing a different
  color theme.
\end{themeexample}


\begin{themeexample}{Montpellier}
  A sober theme giving basic navigational hints. The headline can be
  made more dominant by using a different color theme.
\end{themeexample}



\subsubsection{Presentation Themes with a Table of Contents Sidebar}

\begin{themeexample}[\oarg{options}]{Berkeley}
  A dominant theme. If the navigation bar is on the left, it dominates
  since it is seen first. The height of the frame title is fixed to
  two and a half lines, thus you should be careful with overly long
  titles. A logo will be put in the corner area. Rectangular areas
  dominate the layout. The theme can be made less dominant by using a
  different color theme.

  By default, the current entry of the table of contents in the
  sidebar will be hilighted by using a more vibrant color. A good
  alternative is to hilight the current entry by using a different
  color for the background of the current point. The color theme
  |sidebartab| installs the appropriate colors, so you just have to
  say
\begin{verbatim}
\usecolorhteme{sidebartab}
\end{verbatim}
  This color theme works with all themes that show a table of contents
  in the sidebar.

  This theme is useful for long talks like lectures that require a
  table of contents to be visible all the time.

  The following \meta{options} may be given:
  \begin{itemize}
  \item \declare{|hideallsubsections|} causes only sections to be
    shown in the sidebar. This is useful, if you need to save
    space.
  \item \declare{|hideothersubsections|} causes only the subsections
    of the current section to be shown. This is useful, if you need to
    save  space.      
  \item \declare{|left|} puts the sidebar on the left (default).
  \item \declare{|right|} puts the sidebar on the right.
  \item \declare{|width=|\meta{dimension}} sets the width of the
    sidebar. If set to zero, no sidebar is created.
  \end{itemize}
\end{themeexample}

\begin{themeexample}[\oarg{options}]{PaloAlto}
  A variation in the |Berkeley| theme with less dominance of
  rectangular areas. The same \meta{options} as for the |Berkeley|
  theme can be given. 
\end{themeexample}

\begin{themeexample}[\oarg{options}]{Goettingen}
  A relatively sober theme useful for a longer talk that demands a
  sidebar with a full table of contents.  The same \meta{options} as
  for the |Berkeley| theme can be given. 
\end{themeexample}

\begin{themeexample}[\oarg{options}]{Marburg}
  A very dominat variation of the |Goettingen| theme. The same
  \meta{options} may be given.
\end{themeexample}

\begin{themeexample}[\oarg{options}]{Hannover}
  In this theme, the sidebar on the left is balanced by
  right-flushed frame titles.
    
  The following \meta{options} may be given:
  \begin{itemize}
  \item \declare{|hideallsubsections|} causes only sections to be
    shown in the sidebar. This is useful, if you need to save
    space.
  \item \declare{|hideothersubsections|} causes only the subsections
    of the current section to be shown. This is useful, if you need to
    save  space.      
  \item \declare{|width=|\meta{dimension}} sets the width of the
    sidebar.
  \end{itemize}
\end{themeexample}




\subsubsection{Presentation Themes with a Mini Frame Navigation}

\begin{themeexample}[\oarg{options}]{Berlin}
  A dominant theme with strong colors and dominating rectangular
  areas. The head- and footlines give lot's of information and leave
  little space for the actual slide contents. This theme is useful for
  conferences where the audience is not likely to know the title of
  the talk or who is presenting it.  The theme can be made less
  dominant by using a different color theme.
  
  The following \meta{options} may be given:
  \begin{itemize}
  \item \declare{|compress|} causes the mini frames in the headline to
    use only a single line. This is useful for saving space.
  \end{itemize}
\end{themeexample}

\begin{themeexample}[\oarg{options}]{Ilmenau}
  A variation on the |Berlin| theme. The same \meta{options} may be
  given.  
\end{themeexample}

\begin{themeexample}{Dresden}
  A variation on the |Berlin| theme with a strong separtion into
  navigational stuff at the top/bottom and a sober main text. The same
  \meta{options} may be given.  
\end{themeexample}


\begin{themeexample}{Darmstadt}
  A theme with a strong separation into a navigational upper part and
  an informational main part. By using a different color theme, this
  separation can be lessened. 
\end{themeexample}

\begin{themeexample}{Frankfurt}
  A variaton on the |Darmstadt| theme that is slightly less cluttered
  by leaving out the subsection information.
\end{themeexample}

\begin{themeexample}{Singapore}
  A not-too-sober theme with navigation that does not dominate.
\end{themeexample}

\begin{themeexample}{Szeged}
  A sober theme with a strong dominance of horizontal lines. 
\end{themeexample}




\subsubsection{Presentation Themes with Section and Subsection Tables}

\begin{themeexample}{Copenhagen}
  A not-quite-too-dominant theme. This theme gives compressed
  information about the current section and subsection at the top and
  about the title and the author at the bottom. No shadows are used,
  giving the presentation a ``flat'' look. The theme can be made less
  dominant by using a different color theme.
\end{themeexample}


\begin{themeexample}{Luebeck}
  A variation on the |Copenhagen| theme.
\end{themeexample}

\begin{themeexample}{Malmoe}
  A more sober variation of the |Copenhagen| theme.
\end{themeexample}


\begin{themeexample}{Warsaw}
  A dominant variation of the |Copenhagen| theme.
\end{themeexample}









\subsection{Element Themes}

An element theme installs templates that dictate how the following
``elements'' are typeset:
\begin{itemize}
\item Title and part pages.
\item Itemize environments.
\item Enumerate environments.
\item Descrition environments.
\item Block environments.
\item Theorem and proof environments.
\item Figures and tables.
\item Footnotes.
\item Bibliography entries.
\end{itemize}


In the following examples, the color themes |seahorse| and |rose| are
used to show where and how background colors are
honoured. Furthermore, background colors have been specified for all
elements the honour them in the default theme. In the default color
theme, all of the large rectangular areas are transparent.

\begin{elementthemeexample}{default}
  The default element theme is quite sober. The only extravagance is
  the fact that a little trianlge is used in |itemize| environments
  instead of the usual dot.

  In some cases the theme will honour background color specifications
  for elements. For example, if you set the background color for block
  titles to green, block titles will have a green background. The
  background specifications are currently honoured for the following
  elements: 
  \begin{itemize}
  \item Title, author, institute, and date fields in the title
    page.
  \item Block environments, both for the title and for the body.
  \end{itemize}
  This list may increase in the future.
\end{elementthemeexample}

\begin{elementthemeexample}{circles}
  In this theme, |itemize| and |enumerate| items start with a small
  circle. Likewise, entries in the table of contents start with
  circles. 
\end{elementthemeexample}

\begin{elementthemeexample}{rectangles}
  In this theme, |itemize| and |enumerate| items and table of contents
  entries  start with small rectangles. 
\end{elementthemeexample}

\begin{elementthemeexample}[\oarg{options}]{rounded}
  In this theme, |itemize| and |enumerate| items and table of contents
  entries start with small balls. If a background is specified for
  blocks, then the corners of the background rectangles will be
  rounded off. The following \meta{options} may be given:

  \begin{itemize}
  \item \declare{|shadow|} adds a shadow to all blocks.
  \end{itemize}
\end{elementthemeexample}




\subsection{Layout Themes}

A layout theme dictates (roughly) the overall layout of frames. It
specifies where any navigational elements should go (like a mini table
of contents or navigational mini frames) and what they should look
like. Typically, a layout theme specifies how the following parts of a
frame are rendered:
\begin{itemize}
\item The head- and footline.
\item The sidebars.
\item The logo.
\item The frame title.  
\end{itemize}

A layout theme will not specify how things like |itemize| environments
should be rendered---that is the job an element theme.

In the following examples the color theme |seahorse| is
used. Since the default color theme leaves most backgrounds empty,
most of the layout themes will look too unstructured with the default
color theme. 


\begin{layoutthemeexample}{default}
  The default layout theme is the most sober and minimalistic theme
  around. It will flush left the frame title and it will not install
  any head- or footlines. However, even this theme honours the
  background color specified for the frame title. If a color is
  specified, a bar occupying the whole page width is put behind the
  frame title. A background color the frame subtitle is ignored.
\end{layoutthemeexample}

\begin{layoutthemeexample}{infolines}
  This theme installs a headline showing the current section and the
  current subsection. It installs a footline showing the author's
  name, the institution, the presentation's title, the current date,
  and a frame count. This theme uses only little space.

  The colors used in the headline and footline are drawn from
  |palette primary|, |palette secondary|, and |primary ternary| (see
  Section~\ref{section-colors} for details on how to change these).
\end{layoutthemeexample}

\begin{layoutthemeexample}[\oarg{options}]{miniframes}
  This theme installs a headline in which a horizontal navigational
  bar is shown. This bar contains one entry for each section of the
  presentation. Below each section entry, small circles are shown that
  represent the different frames in the section. The frames are
  arranged subsection-wise, that is, there is a line of frames for
  each subsection. If the class  option |compress| is given, the
  frames will instead be arranged in a single row for each
  section. The navigation bars draws its color from
  |section in head/foot|.

  Below the navigation bar, a line is put showing the title of the
  current subsection. The color is drawn from |subsection in head/foot|.

  At the bottom, two lines are put that contain information such as
  the author's name, the institution, or the paper's title. What is
  shown exactly is influenced by the \meta{options} given. The colors
  are drawn from the appropriate \beamer-colors like
  |author in head/foot|.

  At the top and bottom of both the head- and footline and between the
  navigation bar and the subsection name, separation lines are drawn
  \emph{if} the background color of |separation line| is set. This
  separation line will have a height of 3pt. You can get even more
  fine-grained control over the colors of the separation lines by
  setting appropriate colors like |lower separation line head|.

  The following \meta{options} can be given:
  \begin{itemize}
  \item \declare{|footline=empty|} suppressed the footline (default).
  \item \declare{|footline=authorinstitute|} shows the author's name
    and the institute in the footline.
  \item \declare{|footline=authortitle|} shows the author's name
    and the title in the footline.
  \item \declare{|footline=institutetitle|} shows the institute
    and the title in the footline.
  \item \declare{|footline=authorinstitutetitle|} shows the author's
    name, the institute, and the title in the footline.
  \item \declare{|subsection=|\meta{true or false}} shows or supresses
    line showing the subsection in the headline. It is shown by
    default.
  \end{itemize}  
\end{layoutthemeexample}

\begin{layoutthemeexample}[\oarg{options}]{smoothbars}
  This theme behaves very much like the |miniframes| theme, at least
  with respect to the headline. There, the only difference is the
  smooth transitions are installed between the background colors of
  the navigation bar, the (optional) bar for the subsection name, and
  the background of the frame title. No footline is created. You can
  get the footlines of the |miniframes| theme by first loading the
  theme and then loading the |smoothbars| theme.

  The following \meta{options} can be given:
  \begin{itemize}
  \item \declare{|subsection=|\meta{true or false}} shows or supresses
    line showing the subsection in the headline. It is shown by
    default.
  \end{itemize}  
\end{layoutthemeexample}

\begin{layoutthemeexample}[\oarg{options}]{sidebar}
  In this layout, a sidebar is shown that contains a small table of
  contents with the current section or subsection is hilighted. The
  frame title is vertically centered in a rectangular area at the top
  that always occupies the same amount of space in all
  frames. Finally, the logo is shown in the ``corner'' resulting from
  the sidebar and the frame title rectangle.

  There are several ways of modifying the layout using the
  \meta{options}. If you set the width of the sidebar to 0pt, it is
  not shown, giving you a layout in which the frame title does not
  ``wobble'' since it always occupies the same amount of space on all
  slides. Conversely, if you set the height of the frame title
  rectangle to 0pt, the rectangular area is not used and the frame
  title is inserted normally (occupying as much space as needed on
  each slide).

  The background color of the sidebar is taken from |sidebar|, the
  background color of the frame title from |frametitle|, and the
  background color of the logo corner from |logo|.

  The colors of the entries in the table of contents are drawn from
  the \beamer-color |section in sidebar| and |section in sidebar current| as well as the
  corresponding \beamer-colors for subsections. If an entry does not
  fit on a single line it is automatically ``linebroken.'' 

  The following \meta{options} maybe given:
  \begin{itemize}
  \item
    \declare{|height=|\meta{dimension}} specifies the height of the
    frame title rectangle. If it is set to 0pt, no frame title
    rectangle is created. Instead, the frame title is inserted
    normally into the frame. The default is 2.5 base line heights of
    the frame title font. Thus, there is about enough space for a
    two-line frame title plus a one-line subtitle.
  \item
    \declare{|hideothersubsections|} causes all subsections except
    those of the current section to be supressed in the table of
    contents. This is useful if you have lot's of subsections.
  \item
    \declare{|hideallsubsections|} causes all subsections to be
    supressed in the table of contents.
  \item
    \declare{|left|} puts the sidebar on the left side. Note that in a
    left-to-right reading culture this is side people look first. Note
    also that this table of contents is usually \emph{not} the most
    important part of the frame, so you do not necessarily want people
    to look at is first. Nevertheless, it is the default.
  \item
    \declare{|right|} puts the sidebar of the right side.
  \item
    \declare{|width=|\meta{dimension}} specifies the width of the
    sidebar. If it is set to 0pt, it is completely supressed. The
    default is 2.5 base line heights of the frame title font.
  \end{itemize}
\end{layoutthemeexample}

\begin{layoutthemeexample}{split}
  This theme installs a headline in which, on the left, the sections
  of the talk are shown and, on the right, the subsections of the
  current section. If the class option |compress| has been given,
  the sections and subsections will be put in one line; normally there
  is one line per section or subsection.

  The footline shows the author on the left and the talk's title on
  the right.

  The colors are taken from |palette primary| and |palette fourth|.
\end{layoutthemeexample}

\begin{layoutthemeexample}{shadow}
  This layout theme extends the |split| theme by putting a horizontal
  shading behind the frame title and adding a little ``shadow'' at the
  bottom of the headline.
\end{layoutthemeexample}

\begin{layoutthemeexample}[\oarg{options}]{tree}
  In this layout, the headline contains three lines that show the
  title of the current talk, the current section in this talk, and the
  current subsection in the section. The colors are drawn from
  |title in head/foot|, |section in head/foot|, and
  |subsection in head/foot|.

  In addition, separation lines of height 3pt are shown above and
  below the three lines \emph{if} the background of |separation line|
  is set. More fine-grained control of the colors of these lines can
  be gained by setting |upper separation line head| and
  |lower separation line head|.

  The following \meta{options} may be given:
  \begin{itemize}
  \item
    \declare{|hooks|} causes little ``hooks'' to be drawn in front of
    the section and subsection entries. These are supposed to increase
    the tree-like appearance. 
  \end{itemize}
\end{layoutthemeexample}

\begin{layoutthemeexample}{smoothtree}
  This layout is similar to the |tree| layout. The main difference is
  that the background colors change smoothly.
\end{layoutthemeexample}


%%% Local Variables: 
%%% mode: latex
%%% TeX-master: "beameruserguide"
%%% End: 
