\documentclass{beamer}

% Copyright 2004 by Till Tantau <tantau@users.sourceforge.net>.
%
% This file can be redistributed and/or modified under
% the terms of the GNU Public License, version 2.

\ifx\themename\undefined
  \def\themename{default}
\fi

\beamertemplatesolidbackgroundcolor{black!5}

\usetheme{\themename}

\beamertemplatetransparentcovered

\usepackage{times}
\usepackage[T1]{fontenc}

\title{There Is No Largest Prime Number}
\subtitle{With an introduction to a new proof technique}

\author{Euklid of Alexandria}
\institute{Department of Mathematics\\ University of Alexandria}
\date{27th International Symposium on Prime Numbers, --280}

\begin{document}

\begin{frame}
  \titlepage
%\end{frame}

%\begin{frame}
%  \frametitle{Outline}
  \tableofcontents
\end{frame}

\section{Results}
\subsection{Proof of the Main Theorem}

\begin{frame}<1>
  \frametitle{Proof That There Is No Largest Prime Number}
  \framesubtitle{A proof using \textit{reductio ad absurdum}.}

  \begin{theorem}
    There is no largest prime number.
  \end{theorem}
  \begin{proof}
    \begin{enumerate}
    \item<1-> Suppose $p$ were the largest prime number.
    \item<2-> Let $q := 1 + \prod_{i=1}^p i = 1+p!$.
    \item<3-> Then $q$ is not divisible by any $p' \in \{1,\dots,p\}$.
    \item<1-> Thus $q>p$ is also prime.\qedhere
    \end{enumerate}      
  \end{proof}
\end{frame}

\end{document}


