

% Copyright 2003, 2004 by Till Tantau <tantau@users.sourceforge.net>.
%
% This program can be redistributed and/or modified under the terms
% of the GNU Public License, version 2.

\section{Introduction}

\beamer\ is a \LaTeX\ class for creating presentations that are 
held using a projector, but it can also be used to create transparency
slides. Preparing presentations with \beamer\ is different from
preparing them with \textsc{wysiwyg} programs like OpenOffice's Impress, 
Apple's Keynotes, or KOffice's KPresenter. A \beamer\ presentation is
created like any other \LaTeX\ document: It has a preamble and a body,
the body contains |\section|s and |\subsection|s, the different
slides (called \emph{frames} in \beamer) are put in environments, they
are structured using |itemize| and |enumerate| environments, and so
on. The obvious disadvantage of this approach is that you have to know
\LaTeX\ in order to use \beamer. The advantage is that if you know
\LaTeX, you can use your knowledge of \LaTeX\ also when creating a
presentation, not only when writing papers.


\subsection{Main Features}

The list of features supported by \beamer\ is quite long
(unfortunately, so is presumably the list of bugs supported by
\beamer). The most important features, in my opinion, are:
\begin{itemize}
\item
  You can use \beamer\ both with |pdflatex| and |latex|+|dvips|.
\item
  The standard commands of \LaTeX\ still work. A |\tableofcontents| will
  still create a table of contents, |\section| is still used to create
  structure, and |itemize| still creates a list.
\item
  You can easily create overlays and dynamic effects.
\item
  Themes allow you to change the appearance of your presentation to
  suit you purposes.
\item
  The themes are designed to be usable in practice, they are not just
  for show. You will not find such nonsense as a green body text on
  a picture of a green meadow.
\item
  The layout, the colors, and the fonts used in a presentation can
  easily be changed globally, but you still also have control over
  the most minute detail.
\item
  A special style file allows you to use the \LaTeX-source of a
  presentation directly in other \LaTeX\ classes like |article| or
  |book|. This makes it easy to create presentations out of lecture
  notes or lecture notes out of presentations.
\item
  The final output is typically a \textsc{pdf}-file. Viewer
  applications for this format exist for virtually every
  platform. When bringing your presentation to a conference on a
  memory stick, you do not have to worry about which version of the
  presentation program might be installed there. Also, your
  presentation is going to look exactly the way it looked on your
  computer. 
\end{itemize}


\subsection{History}

I created \beamer\ mainly in my spare time. Many other people have
helped by writing me emails containing suggestions for improvement or
corrections or patches or whole new themes (by now I have gotten over a
thousand emails concerning \beamer). Indeed, most of the
development was only initiated by feature requests and bug
reports. Without this feedback, \beamer\ would still be what it was
originally intended to be: a small private collection of macros that
make using the |seminar| class easier. I created the first version of
\beamer\ for my PhD defense presentation in February 2003. Month
later, I put the package on \textsc{ctan} at the request of some
colleagues. After that, things somehow got out of hand.







\subsection{How to Read this User's Guide}

You should start with the first part. If you have not yet installed
the package, please read Section~\ref{section-installation} first. If 
you are new to \beamer, you should next read the tutorial in
Section~\ref{section-tutorial}. When you set down to create your first
real presentation using \beamer, read Section~\ref{section-workflow}
where the technical details of a possible workflow are
discussed. If you are still new to creating presentations in general, you
might find Section~\ref{section-guidelines} helpful, where many
guidelines are given on what to do and what not to do. Finally, you
should browse through Section~\ref{section-solutions}, where you will
find ready-to-use solution templates for creating talks, possibly even
in the language you intend to use.

The second part of this user's guide goes into the details of all the
commands defined in \beamer, but it also addresses  other technical
issues having to do with creating presentations (like how to include
graphics or animations).

The third part explains how you can change the appearance of your
presentation easily either using themes or by specifying colors or
fonts for specific elements of a presentation (like, say, the font
used for the numbers in an enumerate). 

The last part contains ``howtos,'' which are explanations of how to
get certain things done using \beamer.

\medskip
\noindent
This user's guide contains descriptions of all ``public''
commands provided by the |beamer| class. In each description, the
described command, environment, or option is printed in red. Text
shown in green is optional and can be left out.

\beamernote
As with this paragraph, you will sometimes find the word
\textsc{presentation} in blue next to some paragraph. This means that
the paragraph applies only when you ``normal typeset your presentation
using \LaTeX.''

\articlenote
Opposed to this, a paragraph with \textsc{article} next to it
describes some behaviour that is special for the |article| mode. This
special mode is used to create lecture notes out of a presentation
(the two can coexist in one file).

\lyxnote
A paragraph with \textsc{lyx} next to it describes behaviour that is
special when you use \LyX\ to prepare your presentation.

\appearancenote
You will find \textsc{appearance} next to special explanations of how
you can change the appearance of some aspect of your
presentation. These explanations presume that you have read
Section~\ref{section-customization}.



%%% Local Variables: 
%%% mode: latex
%%% TeX-master: "beameruserguide"
%%% End: 
