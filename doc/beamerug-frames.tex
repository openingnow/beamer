% Copyright 2003, 2004 by Till Tantau <tantau@users.sourceforge.net>.
%
% This program can be redistributed and/or modified under the terms
% of the GNU Public License, version 2.

\section{Creating Frames}

\label{section-frames}

\subsection{The Frame Environment}

A presentation consists of a series of frames. Each frame consists of
a series of slides. You create a frame using the command
|\frame| or the environment |frame|, which do the same. The command
takes one parameter, namely the contents of the frame. All of the
text that is not tagged by overlay specifications is shown on all
slides of the frame. (Overlay specifications is explain
in more detail in later sections. For the moment, let's just say
that an overlay specification is a list of number or number ranges in
pointed brackets that is put after certain commands as in
|\uncover<1,2>{Text}|.) If a frame contains commands that have an
overlay specification, the frame will contain multiple slides;
otherwise it contains only one slide.

\begin{environment}{{frame}\sarg{overlay specification}%
    \opt{|[<|\meta{default overlay specification}|>]|}\oarg{options}}
  The \meta{overlay specification} dictates which slides of a frame are
  to be shown. If left out, the number is calculated automatically.
  The \meta{environment contents} can be normal \LaTeX\ text, but may not contain
  |\verb| commands or |verbatim| environments, unless the
  |containsverbatim| options is given, see also
  Section~\ref{section-verbatim}. 
 
  The normal \LaTeX\ command |\frame| is available \emph{inside}
  frames with its usual meaning. Both outside and inside frames it is
  always available as {\color{red!75!black}|\framelatex|}. 
 
  \example
\begin{verbatim}
\begin{frame}
  \frametitle{A title}
  Some content.
\end{frame}
\end{verbatim}
  
  \example
\begin{verbatim}
\begin{frame}<beamer>  % frame is only shown in beamer mode
  \frametitle{Outline}
  \tabelofcontent[current]
\end{frame}
\end{verbatim}

  Normally, the complete \meta{environment contents} is put on a slide. If
  the text does not fit on a slide, being too high, it will be
  squeezed as much as possible, a warning will be issued, and the text
  just extends unpleasantly over the bottom. You can use the option
  |allowframebreaks| to cause the \meta{frame text} to be split among several
  slides, though you cannot use overlays then. See the explanation of
  the |allowframebreaks| option for details. 
  
  The \meta{default overlay specification} is an optional argument
  that is ``detected'' according to the following rule: If the first
  optional argument in square brackets starts with a |<|, then this
  argument is a \meta{default overlay specification}, otherwise it is
  a normal \meta{options} argument. Thus |\begin{frame}[<+->][plain]| would
  be legal, but also |\begin{frame}[plain]|.

  The effect of the \meta{default overlay specification} is the
  following: Every command or environment \emph{inside the frame} that
  accepts an action specification, see
  Section~\ref{section-action-specifications}, (this includes the
  |\item| command, the |actionenv| environment, |\action|, and all
  block environments) and that is not followed by 
  an overlay specification gets the \meta{default overlay
    specification} as its specification. By providing an incremental
  specification like |<+->|, see Section~\ref{section-incremental},
  this will essentially cause all blocks and all enumerations to be
  uncovered piece-wise (blocks internally employ action
  specifications).
  
  \example In this frame, the theorem is shown from the first slide
  on, the proof from the second slide on, with the first two itemize
  points shown one after the other; the last itemize point is shown
  together with the first one. In total, this frame will contain four
  slides.
\begin{verbatim}
\begin{frame}[<+->]
  \begin{theorem}
    $A = B$.
  \end{theorem}
  \begin{proof}
    \begin{itemize}
    \item Clearly, $A = C$.
    \item As shown earlier,  $C = B$.
    \item<3-> Thus $A = B$.
    \end{itemize}
  \end{proof}
\end{frame}
\end{verbatim}
 
  The following \meta{options} may be given:
  \begin{itemize}
  \item
    \declare{|allowdisplaybreaks|}\opt{|=|\meta{break
    desirability}} causes the AMS\TeX\ command
    |\allowdisplaybreaks|\penalty0|[|\meta{break desirability}|]| to be issued
    for the current frame. The \meta{break desirability} can be a
    value between 0 (meaning formulas may never be broken) and 4 (the
    default, meaning that formulas can be broken anywhere without any
    penalty). The option is just a convenience and makes sense only
    together with the |allowsframebreaks| option.  
  \item
    \declare{|allowframebreaks|}\opt{|=|\meta{fraction}}. When this option is
    given, the frame will be automatically broken up into several
    frames if the text does not fit on a single slide. In detail, when this
    option is given, the following things happen:
    \begin{enumerate}
    \item
      Overlays are not supported.
    \item
      If your frame contains verbatim text, you can and must specify the
      option |containsverbatim|. Since overlays are not supported
      anyway, specifying this option never hurts.
    \item
      Any notes for the frame created using the |\note| command will
      be inserted after the first page of the frame.
    \item
      Any footnotes for the frame will be inserted on the last page of
      the frame.
    \item
      If there is a frame title, each of the pages will have this
      frame title, with a special note added indicating which page of
      the frame that page is. By default, this special note is a
      Roman number. However, this can be changed using the following
      template.
      \begin{element}{frametitle continuation}\yes\yes\yes
        The text of this template is inserted at the end of every
        title of a frame with the |allowframebreaks| option set.
        \begin{templateoptions}
          \itemoption{default}{}
          Installs a Roman number as the template. The number
          indicates the current page of the frame.
          
          \itemoption{roman}{}
          Alias for the default.
          
          \itemoption{from second}{\oarg{text}}
          Installs a template that inserts \meta{text} from the second
          page of a frame on. By default, the text inserted is
          |\insertcontinuationtext|,  which  in turn is |(cont.)| by
          default. 
        \end{templateoptions}
        The following inserts are available:
        \begin{templateinserts}
          \iteminsert{\insertcontinuationcount}
          inserts the current page of the frame as an arabic number.
          \iteminsert{\insertcontinuationcountroman}
          inserts the current page of the frame as an (uppercase)
          Roman number.
          \iteminsert{\insertcontinuationtext}
          just inserts the text |(cont.)| or, possibly, a translation
          thereof (like |(Forts.)| in German). 
        \end{templateinserts}
      \end{element}
    \end{enumerate}
    If a frame needs to be broken into several pages, the material on
    all but the last page fills only 95\% of each page by
    default. Thus, there will be some space left at the top and/or
    bottom, depending on the vertical placement option for the
    frame. This yields a better visual result than a 100\% filling,
    which typically looks crowded. However, you can change this
    percentage using the optional argument \meta{fraction}, where 1
    means 100\% and 0.5 means 50\%. This percentage includes
    the frame title. Thus, in order to split a frame ``roughly in
    half,'' you should give 0.6 as \meta{fraction}.

    Most of the fine details of normal \TeX\ page breaking also apply
    to this option. For example, when you wish equations to be broken
    automatically, be sure to use the |\allowdisplaybreaks|
    command. You can insert |\break|, |\nobreak|, and |\penalty|
    commands to control where breaks should occur. The commands 
    |\pagebreak| and |\nopagebreak| also work, including their
    options. Since you typically do not want page breaks for the frame
    to apply also to the |article| mode, you can add a mode
    specification like |<presentation>| to make these commands apply
    only to the presentation modes. The command
    \declare{|\string\framebreak|} is a shorthand for
    |\pagebreak<presentation>| and \declare{|\string\noframebreak|} is
    a shorthand for |\nopagebreak<presentation>|.

    The use of this  option \emph{evil}. In a (good) presentation you
    prepare each slide carefully and think twice before putting something
    on a certain slide rather than on some different slide. Using the
    |allowframebreaks| option invites the creation of horrible, endless
    presentations that resemble more a ``paper projected on the wall''
    than a presentation. Nevertheless, the option does have its
    uses. Most noticeably, it can be convenient for automatically
    splitting bibliographies or long equations.

    \example
\begin{verbatim}
\begin{frame}[allowframebreaks]
  \frametitle{References}

  \begin{thebibliography}{XX}

  \bibitem...
  \bibitem...
    ...
  \bibitem...
  \end{thebibliography}
\end{frame}
\end{verbatim}
    \example
\begin{verbatim}
\begin{frame}[allowframebreaks,allowdisplaybreaks]
  \frametitle{A Long Equation}

  \begin{align}
    \zeta(2) &= 1 + 1/4 + 1/9 + \cdots \\
    &= ... \\
    ...
    &= \pi^2/6.
  \end{align}
\end{frame}
\end{verbatim}
  \item
    \declare{|b|}, \declare{|c|}, \declare{|t|} will cause the frame
    to be vertically aligned at the bottom/center/top. This overrides
    the global placement policy, which is governed by the class
    options |t| and |c|.
  \item
    \declare{|containsverbatim|} tells \beamer\ that the frame
    contains verbatim commands. In this case, only one slide of
    the frame is typeset (unless all slides are suppressed by the
    \meta{overlay specification}). If you wish to use verbatim text in
    a frame with several slides, a more roundabout approach is
    necessary, see Section~\ref{section-verbatim}. This option cannot
    be used together with the |label| option.
    
  \item
    \declare{|label=|\meta{name}} causes the frame's contents to
    be stored under the name \meta{name} for later resumption using
    the command |\againframe|. If this option is given, you cannot
    directly include verbatim text in the frame. The frame is still rendered
    normally. See also |\againframe|.

    Furthermore, on each slide of the frame a label with the name
    \meta{name}|<|\meta{slide number}|>| is created. On the
    \emph{first} slide, furthermore, a label with the name \meta{name}
    is created (so the labels \meta{name} and \meta{name}|<1>| point
    to the same slide). Note that labels in general, and these labels
    in particular, can be used as targets for hyperlinks.
  \item
    \declare{|plain|} causes  the headlines, footlines,
    and sidebars to be suppressed. This is useful for creating single
    frames with different head- and footlines or for creating frames
    showing big pictures that completely fill the frame.

  \example A frame with a picture completely filling the frame:  
\begin{verbatim}
\begin{frame}[plain]
  \begin{centering}%
    \pgfimage[height=\paperheight]{somebigimagefile}%
    \par%
  \end{centering}%
\end{frame}
\end{verbatim}
  
  \example A title page, in which the head- and footlines are replaced
  by two graphics.
\begin{verbatim}
\setbeamertemplate{title page}
{
  \pgfuseimage{toptitle}
  \vskip0pt plus 1filll

  \begin{centering}
    {\usebeamerfont{title}\usebeamercolor[fg]{title}\inserttitle}
    
    \insertdate
  \end{centering}

  \vskip0pt plus 1filll
  \pgfuseimage{bottomtitle}
}
\begin{frame}[plain]
  \titlepage
\end{frame}
\end{verbatim}
  \item
    \declare{|shrink|}\opt{|=|\meta{minimum shrink percentage}}. This
    option will cause the text of the frame to be shrunken if it is
    too large to fit on the frame. \beamer\ will first normally
    typeset the whole frame. Then it has a look at vertical size of
    the frame text (excluding the frame title). If this vertical size
    is larger than  the text height minus the frame title height,
    \beamer\ computes a shrink factor and scales down the frame text
    by this factor such that the frame text then fills the frame
    completely. Using this option will automatically cause the
    |squeeze| option to be used, also.

    Since the shrinking takes place only after everything has been
    typeset, shrunken frame text will not fill the frame completely
    horizontally. For this reason, you can specify a \meta{minimum
    shrink percentage} like |20|. If this percentage is specified, the
    frame will be shrunk by \emph{at least} by this percentage. Since
    \beamer\ knows this, it can increase the horizontal width
    proportionally such that the shrunken text once more fills the
    entire frame. If, however, the percentage is not enough, the text
    will be shrunken as needed and you will be punished with a warning
    message.

    The best way to use this option is to identify frames that are
    overly full, but in which all text absolutely has to be fit on a
    single frame. Then start specifying first |shrink=5|, then
    |shrink=10|, and so on, until no warning is issued any more (or
    just ignore the warning when things look satisfactory).

    Using this option is \emph{very evil}. It will
    result in changes of the font size from slide to slide, which is a
    typographic nightmare. Its usage can \emph{always} be avoided by
    restructuring and simplifying frames, which will result in a
    better presentation.

    \example
\begin{verbatim}
\begin{frame}[shrink=5]
  Some evil endless slide that is 5\% too large.
\end{frame}
\end{verbatim}
  \item
    \declare{|squeeze|} causes all vertical spaces in the text to be
    squeezed together as much as possible. Currently, this just causes
    the vertical space in enumerations or itemizations to be reduced
    to zero.

    Using this option is not good, but also not evil.   
  \end{itemize}

  \lyxnote
  Use the style ``BeginFrame'' to start a frame and the style
  ``EndFrame'' to end it. A frame is automatically ended by the start
  of a new frame and by the start of a new section or section (but
  not by the end of the document!).

  \lyxnote
  You can pass options and an overlay specification to a frame by
  giving these in \TeX-mode as the first thing in the frame
  title. (Some magic is performed to extract them in \LyX-mode from
  there.)

  \lyxnote
  The style ``BeginPlainFrame'' is included as a convenience. It
  passes the |plain| option to the frame. To pass further options to a
  plain frame, you should use the normal ``BeginFrame'' style and
  specify all options (including |plain|).

  \lyxnote
  In \LyX, you can insert verbatim text directly even in overlayed
  frames. The reason is that \LyX\ uses a different internal mechanism
  for typesetting verbatim text, that is easier to handle for \beamer.

  \articlenote
  In |article| mode, the |frame| environment does not create any visual
  reference to the original frame (no frame is drawn). Rather, the
  frame text is inserted into the normal text. To change this, you can
  modify the templates |frame begin| and |frame end|, see below. To
  suppress a frame in |article| mode, you can, for example, specify
  |<presentation>| as overlay specification.

  \begin{element}{frame begin}\yes\no\no
    The text of this template is inserted at the beginning of each
    frame in |article| mode (and only there). You can use it, say, to
    start a |minipage| environment at the beginning of a frame or to
    insert a horizontal bar or whatever.
  \end{element}

  \begin{element}{frame end}\yes\no\no
    The text of this template is inserted at the end of each
    frame in |article| mode.
  \end{element}
\end{environment}

\begin{command}{\frame\sarg{overlay specification}%
    \opt{|[<|\meta{default overlay
        specification}|>]|}\oarg{options}\marg{contents}}
  This command does the same as putting the \meta{contents} in a
  |frame| environment called.

  \example The following two frame will be identical:
\begin{verbatim}
\frame{Hi!}

\begin{frame}
  Hi!
\end{frame}
\end{verbatim}
\end{command}

Internally, the |\frame| command is what is actually executed. The
|frame| environment just collects its environment contents and then
calls the |\frame| command; except if the option |containsverbatim|
is specified, in which case the contents is not collected, but
control is nevertheless passed on the |\frame| (the internals are a
bit obscure). You \emph{can} use the |frame| environment inside other
environments like this:
\begin{verbatim}
\newenvironment{myframe}[1]{\begin{frame}\frametitle{#1}}{\end{frame}}
\end{verbatim}
However, the actual mechanics are somewhat fragile since the
``collecting'' of the frame contents is not easy, so
do not attempt anything too fancy.

If, for whatever reason, the |frame| environment has a problem with
some contents, it \emph{might} help to try using the |\frame|
command instead.

For compatibility with earlier versions, you can also give an overlay
specification in square brackets. If the sole argument to the |\frame|
command is an argument in square brackets, the \beamer\ class will try
to check whether this argument ``looks like'' an overlay
specification. If so, it is assumed to be an overlay specification.



\subsection{Components of a Frame}

Each frame consists of several components:
\begin{enumerate}\itemsep=0pt\parskip=0pt
\item a headline and a footline,
\item a left and a right sidebar,
\item navigation bars,
\item navigation symbols,
\item a logo,
\item a frame title, 
\item a background, and
\item some frame contents.
\end{enumerate}

A frame need not have all of these components. Usually, the first
three components are automatically setup by the theme you are using. 


\subsubsection{The Headline and Footline}

The headline of a frame is the area at the top of the frame. If it is
not empty, it should show some information that helps the audience
orientate itself during your talk. Likewise, the footline is the area
at the bottom of the frame.

\beamer\ does not use the standard \LaTeX\ mechanisms for typesetting
the headline and the footline. Instead, the special |headline| and
|footline| templates are used to typeset them.

The size of the headline and the footline is determined as follows:
Their width is always the paper width. Their height is determined by
tentatively typesetting the headline and the footline right after the
|\begin{document}| command. The head of the headline and the footline
at that point is ``frozen'' and will be used throughout the whole
document, even if the headline and footline vary in height later on
(which they should not).

The appearance of the headline and footline is determined by the
following templates:

\begin{element}{headline}\yes\yes\yes
  This template is used to typeset the headline. The \beamer-color and
  -font |headline| are installed at the beginning. The background of
  the \beamer-color is not used by default, that is, no background
  rectangle is drawn behind the headline and footline (this may change
  in the future with the introduction of a headline and a footline
  canvas).

  The width of the headline is the whole paper width. The height is
  determined automatically as described above. The headline is typeset
  in vertical mode with interline skip turned off and the paragraph
  skip set to zero.

  Inside this template, the |\\| command is changed such that it
  inserts a comma instead.

  \example
\begin{verbatim}
\setbeamertemplate{headline}
{%
  \begin{beamercolorbox}{section in head/foot}
    \vskip2pt\insertnavigation{\paperwidth}\vskip2pt
  \end{beamercolorbox}%
}
\end{verbatim}

  \begin{templateoptions}
    \itemoption{default}{}
    The default is just an empty headline. To get the default headline
    of earlier versions of the \beamer\ class, use the |compatibility|
    theme.
    \itemoption{infolines theme}{}
    This option becomes available (and is used) if the |infolines|
    outer theme is loaded. The headline shows current section and subsection.
    \itemoption{miniframes theme}{}
    This option becomes available (and is used) if the |miniframes|
    outer theme is loaded. The headline shows the sections with small
    clickable mini frames below them.    
    \itemoption{sidebar theme}{}
    This option becomes available (and is used) if the |sidebar|
    outer theme is loaded and if the head height (and option of the
    |sidebar| theme) is not zero. In this case, the headline is an
    empty bar of the background color |frametitle| with the logo to
    the left or right of this bar. 
    \itemoption{smoothtree theme}{}
    This option becomes available (and is used) if the |smoothtree|
    outer theme is loaded. A ``smoothed'' navigation
    tree is shown in the headline.
    \itemoption{smoothbars theme}{}
    This option becomes available (and is used) if the |smoothbars|
    outer theme is loaded. A ``smoothed'' version of the
    |miniframes| headline is shown.
    \itemoption{tree}{}
    This option becomes available (and is used) if the |tree|
    outer theme is loaded. A navigational tree is shown in the headline.
    \itemoption{split}{}
    This option becomes available (and is used) if the |split|
    outer theme is loaded. The headline is split into a left part
    showing the sections and a right part showing the subsections.
    \itemoption{text line}{\marg{text}}
    The headline is typeset more or less as if it were a normal text
    line with the \meta{text} as contents. The left and right margin
    are setup such that they are the same as the margins of normal
    text. The \meta{text} is typeset inside an |\hbox|, while the
    headline is normally typeset in vertical mode.
  \end{templateoptions}

  Inside the template numerous inserts can be used:
  \begin{itemize}
    \iteminsert{\insertnavigation\marg{width}}
    Inserts a horizontal navigation bar of the given \meta{width} into a
    template. The bar lists the sections and below them mini frames for
    each frame in that section.

    \iteminsert{\insertpagenumber}
    Inserts the current page number into a template.

    \iteminsert{\insertsection}
    Inserts the current section into a template.

    \iteminsert{\insertsectionnavigation}\marg{width}
    Inserts a vertical navigation bar containing all sections, with the
    current section hilighted.

    \iteminsert{\insertsectionnavigationhorizontal}\marg{width}\marg{left insert}\marg{right insert}
    Inserts a horizontal navigation bar containing all sections, with
    the current section hilighted. The \meta{left insert} will be
    inserted to the left of the sections, the \marg{right insert} to the
    right. By inserting a triple fill (a
    |filll|) you can flush the bar to the left or right.
    \example
\begin{verbatim}
\insertsectionnavigationhorizontal{.5\textwidth}{\hskip0pt plus1filll}{}
\end{verbatim}

    \iteminsert{\insertshortauthor}\oarg{options}
    Inserts the short version of the author into a template. The text
    will be printed in one long line, line breaks introduced using the
    |\\| command are suppressed.  The
    following \meta{options} may be given:
    \begin{itemize}
    \item
      \declare{|width=|\meta{width}}
      causes the text to be put into a multi-line minipage of the given
      size. Line breaks are still suppressed by default.
    \item
      \declare{|center|}
      centers the text inside the minipage created using the |width|
      option, rather than having it left aligned.
    \item
      \declare{|respectlinebreaks|}
      causes line breaks introduced by the |\\| command to be honored.    
    \end{itemize}

    \example |\insertauthor[width={3cm},center,respectlinebreaks]|

    \iteminsert{\insertshortdate}\oarg{options}
    Inserts the short version of the date into a template. The same
    options as for |\insertshortauthor| may be given. 

    \iteminsert{\insertshortinstitute}\oarg{options}
    Inserts the short version of the institute into a template. The same
    options as for |\insertshortauthor| may be given. 

    \iteminsert{\insertshortpart}\oarg{options}
    Inserts the short version of the part name into a template. The same
    options as for |\insertshortauthor| may be given. 

    \iteminsert{\insertshorttitle}\oarg{options}
    Inserts the short version of the document title into a template. Same
    options as for |\insertshortauthor| may be given. 

    \iteminsert{\insertshortsubtitle}\oarg{options}
    Inserts the short version of the document subtitle. Same
    options as for |\insertshortauthor| may be given. 

    \iteminsert{\insertsubsection}
    Inserts the current subsection into a template.

    \iteminsert{\insertsubsectionnavigation}\marg{width}
    Inserts a vertical navigation bar containing all subsections of the
    current section, with the current subsection hilighted.

    \iteminsert{\insertsubsectionnavigationhorizontal}\marg{width}\marg{left
      insert}\marg{right insert}
    
    See |\insertsectionnavigationhorizontal|.

    \iteminsert{\insertverticalnavigation}\marg{width}
    Inserts a vertical navigation bar of the given \meta{width} into a
    template. The bar shows a little table of contents. The individual
    lines are typeset using the templates
    |section in head/foot| and |subsection in head/foot|.

    \iteminsert{\insertframenumber}
    Inserts the number of the current frame (not slide) into a template.

    \iteminsert{\inserttotalframenumber}
    Inserts the total number of the frames (not slides) into a
    template. The number is only correct on the second run of \TeX\ on
    your document.

    \iteminsert{\insertframestartpage}
    Inserts the page number of the first page of the current frame.

    \iteminsert{\insertframeendpage}
    Inserts the page number of the last page of the current frame.

    \iteminsert{\insertsubsectionstartpage}
    Inserts the page number of the first page of the current subsection.

    \iteminsert{\insertsubsectionendpage}
    Inserts the page number of the last page of the current subsection.

    \iteminsert{\insertsectionstartpage}
    Inserts the page number of the first page of the current section.

    \iteminsert{\insertsectionendpage}
    Inserts the page number of the last page of the current section.

    \iteminsert{\insertpartstartpage}
    Inserts the page number of the first page of the current part.

    \iteminsert{\insertpartendpage}
    Inserts the page number of the last page of the current part.

    \iteminsert{\insertpresentationstartpage}
    Inserts the page number of the first page of the presentation.

    \iteminsert{\insertpresentationendpage}
    Inserts the page number of the last page of the presentation
    (excluding the appendix).


    \iteminsert{\insertappendixstartpage}
    Inserts the page number of the first page of the appendix. If there
    is no appendix, this number is the last page of the document.

    \iteminsert{\insertappendixendpage}
    Inserts the page number of the last page of the appendix. If there
    is no appendix, this number is the last page of the document.

    \iteminsert{\insertdocumentstartpage}
    Inserts 1.

    \iteminsert{\insertdocumentendpage}
    Inserts the page number of the last page of the document (including
    the appendix).
  \end{itemize}
\end{element}

\begin{element}{footline}\yes\yes\yes
  This template behaves exactly the same way as the headline. Note
  that, sometimes quite annoyingly, \beamer\ currently adds a space of
  4pt between the bottom of the frame's text and the top of the
  footline.  

  \begin{templateoptions}
    \itemoption{default}{}
    The default is an empty footline. Note that the navigational
    symbols are \emph{not} part of the footline by default. Rather,
    they are part of an (invisible) right sidebar.
    \itemoption{infolines theme}{}
    This option becomes available (and is used) if the |infolines|
    outer theme is loaded. The footline shows things like the author's
    name and the title of the talk.
    \itemoption{miniframes theme}{}
    This option becomes available (and is used) if the |miniframes|
    outer theme is loaded. Depending on the exact options that are
    used when the |miniframes| theme is loaded, different things can
    be shown in the footline.
    \itemoption{page number}{}
    Shows the current page number in the footline.
    \itemoption{frame number}{}
    Shows the current frame number in the footline.
    \itemoption{split}{}
    This option becomes available (and is used) if the |split|
    outer theme is loaded. The footline (just like the headline) is
    split into a left part showing the author's name and a right part
    showing the talk's title.
    \itemoption{text line}{\marg{text}}
    The footline is typeset more or less as if it were a normal text
    line with the \meta{text} as contents. The left and right margin
    are setup such that they are the same as the margins of normal
    text. The \meta{text} is typeset inside an |\hbox|, while the
    headline is normally typeset in vertical mode. Using the |\strut|
    command somewhere in such a line might be a good idea.
  \end{templateoptions}

  The same inserts as for headlines can be used.

  \begin{element}{page number in head/foot}\no\yes\yes
    These \beamer-color and -font are used to typeset the page number
    or frame number in the footline.
  \end{element}
\end{element}




\subsubsection{The Sidebars}


Sidebars are vertical areas that stretch from the lower end of the
headline to the top of the footline. There can be a sidebar at the
left and another one at the right (or even both). Sidebars can show a
table of contents, but they could also be added for purely aesthetic
reasons.

When you install a sidebar template, you must explicitly specify the
horizontal size of the sidebar using the command |\setbeamesize| with
the option |sidebar left width| or |sidebar right width|. The vertical
size is determined automatically. Each sidebar has its own background
canvas, which can be setup using the sidebar canvas templates. 

Adding a sidebar of a certain size, say 1cm, will make the main text
1cm narrower. The distance between the inner side of a side
bar and the outer side of the text, as specified by
the command |\setbeamersize| with the option |text margin left| and
its counterpart for the right margin, is not changed when a sidebar is
installed. 

Internally, the sidebars are typeset by showing them as part of the
headline. The \beamer\ class keeps track of six dimensions, three 
for each side: the variables |\beamer@leftsidebar| and
|\beamer@rightsidebar| store the (horizontal) sizes of the side
bars, the variables |\beamer@leftmargin| and
|\beamer@rightmargin| store the distance between sidebar and
text, and the macros |\Gm@lmargin| and  |\Gm@rmargin| store
the distance from the edge of the paper to the edge of the text. Thus
the sum |\beamer@leftsidebar| and |\beamer@leftmargin| is
exactly  |\Gm@lmargin|. Thus, if you wish to put some text right
next to the left sidebar, you might write
|\hskip-\beamer@leftmargin| to get there.

\begin{element}{sidebar left}\yes\yes\yes
  \colorfontparents{sidebar}
  The template is used to typeset the left sidebar. As mentioned
  above, the size of the left sidebar is set using the command
\begin{verbatim}
\setbeamersize{sidebar widt left=2cm}
\end{verbatim}
  \beamer\ will not clip sidebars automatically if they are too
  large.

  When the sidebar is typeset, it is put inside a |\vbox|. You should
  currently setup things like the |\hsize| or the |\parskip|
  yourself. 

  \begin{templateoptions}
    \itemoption{default}{}
    installs an empty template.
    \itemoption{sidebar theme}{}
    This option is available, if the outer theme |sidebar| is loaded
    with the |left| option. In this case, this options is selected
    automatically. It shows a mini table of contents in the sidebar.
  \end{templateoptions}
\end{element}

\begin{element}{sidebar right}\yes\yes\yes
  \colorfontparents{sidebar}
  This template works the same way as the template for the left.
  
  \begin{templateoptions}
    \itemoption{default}{}
    The default right sidebar has zero width. Nevertheless, it shows
    navigational symbols and, if installed, a logo at the bottom of
    the sidebar, protruding to the left into the text.
    \itemoption{sidebar theme}{}
    This option is available, if the outer theme |sidebar| is loaded
    with the |left| option. In this case, this options is selected
    automatically. It shows a mini table of contents in the sidebar.
  \end{templateoptions}
\end{element}

\begin{element}{sidebar canvas left}\yes\no\no
  Like the overall background canvas, this canvas is drawn behind the
  actual text of the sidebar. This template should normally insert a
  rectangle of the size of the sidebar, though a too large height will
  not lead to an error or warning. When this template is called, the
  \beamer-color |sidebar left| will have been installed.

  \begin{templateoptions}
  \item{default}{}
    uses a large rectangle colored with |sidebar.bg| as the sidebar
    canvas. However, if the background of |sidebar| is empty, nothing
    is drawn and the canvas is ``transparent.''
    
    \itemoption{vertical shading}{\oarg{color options}}
    installs a vertically shaded background. The following
    \meta{color options} may be given:
    \begin{itemize}
    \item \declare{|top=|\meta{color}} specifies the color at the
      top of the sidebar. By default, 25\% of the foreground of the
      \beamer-color |palette primary| is used.
    \item \declare{|bottom=|\meta{color}} specifies the color at the
      bottom of the sidebar (more precisely, at a distance of the page
      height below the top of the sidebar). By default, the background of
      |normal text| at the moment of invocation of this command is
      used. 
    \item \declare{|middle=|\meta{color}} specifies the color
      for the middle of the sidebar. Thus, if this option is given, the
      shading changes from the bottom color to this color and then
      to the top color.
    \item \declare{|midpoint=|\meta{factor}} specifies at which
      point of the page the middle color is used. A factor of |0| is
      the bottom of the page, a factor of |1| is the top. The
      default, which is |0.5|, is in the middle.
    \end{itemize}
    Note that you must give ``real'' \LaTeX\ colors here. This often
    makes it necessary to invoke the command |\usebeamercolor| before
    this command can be used.

    Also note, that the width of the sidebar should be setup before
    this option is used.

    \example A stylish, but not very useful shading:
\begin{verbatim}
{\usebeamercolor{palette primary}}
\setbeamertemplate{sidebar canvas}[vertical shading]
[top=palette primary.bg,middle=white,bottom=palette primary.bg]
\end{verbatim}
    
    \itemoption{horizontal shading}{\oarg{color options}}
    installs a horizontally shaded background. The following
    \meta{color options} may be given:
    \begin{itemize}
    \item \declare{|left=|\meta{color}} specifies the color at the
      left of the sidebar.
    \item \declare{|right=|\meta{color}} specifies the color at the
      right of the sidebar.
    \item \declare{|middle=|\meta{color}} specifies the color
      in the middle of the sidebar.
    \item \declare{|midpoint=|\meta{factor}} specifies at which
      point of the sidebar the middle color is used. A factor of |0| is
      the left of the sidebar, a factor of |1| is the right. The
      default, which is |0.5|, is in the middle.
    \end{itemize}

    \example Adds two ``pillars''
\begin{verbatim}
\setbeamersize{sidebar width left=0.5cm,sidebar width right=0.5cm}

{\usebeamercolor{sidebar}}

\setbeamertemplate{sidebar canvas left}[horizontal shading]
[left=white,middle=sidebar.bg,right=white]
\setbeamertemplate{sidebar canvas right}[horizontal shading]
[left=white,middle=sidebar.bg,right=white]
\end{verbatim}
  \end{templateoptions}  
\end{element}

\begin{element}{sidebar canvas right}\yes\no\no
  Works exactly as for the left side.
\end{element}



\subsubsection{Navigation Bars}
\label{section-navigation-bars}

Many themes install a headline or a sidebar that shows a
\emph{navigation bar}. Although these navigation bars take up quite
a bit of space, they are often useful for two reasons: 

\begin{itemize}
\item
  They provide the audience with a visual feedback of how much of your
  talk you have covered and what is yet to come. Without such
  feedback, an audience will often puzzle whether something you are
  currently introducing will be explained in more detail later on or
  not.
\item
  You can click on all parts of the navigation bar. This will directly
  ``jump'' you to the part you have clicked on. This is particularly
  useful to skip certain parts of your talk and during a ``question
  session,'' when you wish to jump back to a particular frame someone
  has asked about.
\end{itemize}

Some navigation bars can be ``compressed'' using the following option:

\begin{classoption}{compress}
  Tries to make all navigation bars as small as possible. For example,
  all small frame representations in the navigation bars for a single
  section are shown alongside each other. Normally, the representations
  for different subsections are shown in different lines. Furthermore,
  section and subsection navigations are compressed into one line.
\end{classoption}



Some themes use the |\insertnavigation| to insert a navigation bar
into the headline. Inside this bar, small icons are shown (called
``mini frames'') that represent the frames of a presentation. 
When you click on such an icon, the
following happens: 
\begin{itemize}
\item
  If you click on (the icon of) any frame other than the current frame, the
  presentation will jump to the first slide of the frame you clicked
  on.
\item
  If you click on the current frame and you are not on the last slide
  of this frame, you will jump to the last slide of the frame.
\item
  If you click on the current frame and you are on the last slide, you
  will jump to the first slide of the frame.
\end{itemize}
By the above rules you can:
\begin{itemize}
\item
  Jump to the beginning of a frame from somewhere else by clicking on
  it once.  
\item
  Jump to the end of a frame from somewhere else by clicking on it
  twice.
\item
  Skip the rest of the current frame by clicking on it once.
\end{itemize}

I also tried making a jump to an already-visited frame jump
automatically to the last slide of this frame. However, this turned
out to be more confusing than helpful. With the current implementation
a double-click always brings you to the end of a slide, regardless
from where you ``come.''

\begin{element}{mini frames}\semiyes\no\no
  This parent template has the children |mini frame| and
  |mini frame in current subsection|.

  \example |\setbeamertemplate{mini frames}[box]|

  \begin{templateoptions}
    \itemoption{default}{}
    shows small circles as mini frames.
    \itemoption{box}{}
    shows small rectangles as mini frames.
    \itemoption{tick}{}
    shows small vertical bars as mini frames.
  \end{templateoptions}
\end{element}

\begin{element}{mini frame}\yes\yes\yes
  The template is used to render the mini frame of the current frame
  in a navigation bar.

  The width of the template is ignored. Instead, when multiple mini
  frames are shown, their position is calculated based on the
  \beamer-sizes |mini frame size| and |mini frame offset|. See the
  command |\setbeamersize| for a description of how to change them.
\end{element}

\begin{element}{mini frame in current subsection}\yes\no\no
  This template is used to render the mini frame of frames in the
  current subsection that are not the current frame. The
  \beamer-color/-font |mini frame| is installed prior to the usage of
  this template is invoked.
\end{element}

\begin{element}{mini frame in other subsection}\yes\no\no
  This template is used to render mini frames of frame from
  subsections other than the current one.
  \begin{templateoptions}
    \itemoption{default}{\oarg{percentage}}
    By default, this template shows |mini frame in current subsection|, except that the color is first
    changed to |fg!|\meta{percentage}|!bg|. The default
    \meta{percentage} is 50\%.

    \example To get an extremely ``shaded'' rendering of the frames
    outside the current subsection you can use the following:
\begin{verbatim}
\setbeamertemplate{mini frame in other subsection}[default][20]
\end{verbatim}

    \example To render all mini frames other than the current one in
    the same way, use
\begin{verbatim}
\setbeamertemplate{mini frame in other subsection}[default][100]
\end{verbatim}
  \end{templateoptions}
\end{element}



Some themes show sections and/or subsections in the navigation bars. 
By clicking on a section or subsection in the navigation bar, you will
jump to that section. Clicking on a section is particularly useful if
the section starts with a |\tableofcontents[currentsection]|, since you
can use it to jump to the different subsections.

\begin{element}{section in head/foot}\yes\yes\yes
  This template is used to render a section entry if it occurs in the
  headline or the footline. The background of the \beamer-color is
  typically used as the background of the whole ``area'' where section
  entries are shown in the headline. You cannot usually use this
  template yourself since the insert |\insertsectionhead| is setup
  correctly only when a list of sections is being typeset in the
  headline. 

  The default template just inserts the section name. The following
  inserts are useful for this template:
  \begin{itemize}
    \iteminsert{\insertsectionhead}
    inserts the name of the section that is to be typeset in a
    navigation bar. 

    \iteminsert{\insertsectionheadnumber}
    inserts the number of the section that is to be typeset in a
    navigation bar. 

    \iteminsert{\insertpartheadnumber}
    inserts the number of the part of the current section or subsection
    that is to be typeset in a navigation bar. 
  \end{itemize}
\end{element}

\begin{element}{section in head/foot shaded}\yes\no\no
  This template is used instead of |section in head/foot| for
  typesetting sections that are currently shaded. Such shading is
  usually applied to all sections but the current one.

  Note that this template does \emph{not} have its own color and
  font. When this template is called, the \beamer-font and color
  |section in head/foot| will have been setup. Then, at the start of
  the template, you will typically change the current color or start a
  |colormixin| environment.

  \begin{templateoptions}
    \itemoption{default}{\oarg{percentage}}
    The default template changes the current color to
    |fg!|\meta{percentage}|!bg|. This causes the current color to
    become ``washed out'' or ``shaded.'' The default percentage is
    |50|.

    \example You can use the following command to make the shaded
    entries very ``light'':
\begin{verbatim}
\setbeamertemplate{section in head/foot shaded}[default][20]
\end{verbatim}
  \end{templateoptions}
\end{element}



\begin{element}{section in sidebar}\yes\yes\yes
  This template is used to render a section entry if it occurs in the
  sidebar, typically as part of a mini table of contents shown
  there. The background of the \beamer-color is used as background for
  the entry. Just like |section in head/foot|, you cannot usually use
  this template yourself and you should also use |\insertsectionhead|
  to insert the name of the section that is to be typeset.

  For once, no default is installed for this template.

  \begin{templateoptions}
    \itemoption{sidebar theme}{}
    This template, which is only available if the |sidebar| outer
    theme is loaded, inserts a bar with the \beamer-color's
    foreground and background that shows the section  name. The width of
    the bar is the same as the width of the whole sidebar.
  \end{templateoptions}

  The same inserts as for |section in head/foot| can be used.
\end{element}

\begin{element}{section in sidebar shaded}\yes\yes\no
  This template is used instead of |section in sidebar| for
  typesetting sections that are currently shaded. Such shading is
  usually applied to all sections but the current one.

  Differently from |section in head/foot shaded|, this template
  \emph{has} its own \beamer-color.

  \begin{templateoptions}
    \itemoption{sidebar theme}{}
    Does the same as for the nonshaded version, except that a
    different \beamer-color is used.
  \end{templateoptions}
\end{element}


\begin{element}{subsection in head/foot}\yes\yes\yes
  This template behaves exactly like |section in head/foot|, only for
  subsections. 
  \begin{itemize}
    \iteminsert{\insertsubsectionhead}
    works like |\insertsectionhead|.

    \iteminsert{\insertsubsectionheadnumber}
    works like |\insertsectionheadnumber|.
  \end{itemize}
\end{element}

\begin{element}{subsection in head/foot shaded}\yes\no\no
  This template behaves exactly like |section in head/foot shaded|,
  only for subsections. 
  \begin{templateoptions}
    \itemoption{default}{\oarg{percentage}}
    works like the corresponding option for sections.

    \example
\begin{verbatim}
\setbeamertemplate{section in head/foot shaded}[default][20]
\setbeamertemplate{subsection in head/foot shaded}[default][20]
\end{verbatim}
  \end{templateoptions}
\end{element}

\begin{element}{subsection in sidebar}\yes\yes\yes
  This template behaves exactly like |section in sidebar|, only for
  subsections. 
\end{element}

\begin{element}{subsection in head/foot shaded}\yes\no\no
  This template behaves exactly like |section in sidebar shaded|,
  only for subsections. 
\end{element}

By clicking on the document title in a navigation bar (not all themes
show it), you will jump to the first slide of your presentation
(usually the title page) \emph{except} if you are already at the first
slide. On the first slide, clicking on the document title will jump to
the end of the presentation, if there is one. Thus by \emph{double}
clicking the document title in a navigation bar, you can jump to the end.



\subsubsection{The Navigation Symbols}
\label{section-navigation-symbols}


Navigation symbols are small icons that are shown on every slide
by default. The following symbols are shown: 
\begin{enumerate}
\item
  A slide icon, which is depicted as  a single rectangle. To the left and
  right of this symbol, a left and right arrow are shown.
\item
  A frame icon, which is depicted as three slide icons ``stacked on top of
  each other''. This symbol is framed by arrows.
\item
  A subsection icon, which is depicted as a highlighted subsection
  entry in a table of contents. This  symbols is framed by arrows.
\item
  A section icon, which is depicted as a highlighted section entry
  (together with all subsections) in a table of contents. This symbol
  is framed by arrows.
\item
  A presentation icon, which is depicted as a completely highlighted
  table of contents.
\item
  An appendix icon, which is depicted as a completely highlighted
  table of contents consisting of only one section. (This icon is only
  shown if there is an appendix.)
\item
  Back and forward icons, depicted as circular arrows.
\item
  A ``search'' or ``find'' icon, depicted as a detective's
  magnifying glass.
\end{enumerate}

Clicking on the left arrow next to an icon always jumps to (the
last slide of) the previous slide, frame, subsection, or
section. Clicking on the right arrow next to an icon always jump to
(the first slide of) the next slide, frame, subsection, or section. 

Clicking \emph{on} any of these icons has different effects:
\begin{enumerate}
\item
  If supported by the viewer application, clicking on a slide icon
  pops up a window that allows you to enter a slide number to which
  you wish to jump.
\item
  Clicking on the left side of a frame icon will jump to the first
  slide of the frame, clicking on the right side will jump to the last
  slide of the frame (this can be useful for skipping overlays).
\item
  Clicking on the left side of a subsection icon will jump to the
  first slide of the subsection, clicking on the right side will jump
  to the last slide of the subsection.
\item
  Clicking on the left side of a section icon will jump to the
  first slide of the section, clicking on the right side will jump
  to the last slide of the section.
\item
  Clicking on the left side of the presentation icon will jump to the
  first slide, clicking on the right side will jump to the last slide
  of the presentation. However, this does \emph{not} include the
  appendix. 
\item
  Clicking on the left side of the appendix icon will jump to the
  first slide of the appendix, clicking on the right side will jump to
  the last slide of the appendix.
\item
  If supported by the viewer application, clicking on the back and
  forward symbols jumps to the previously visited slides.
\item
  If supported by the viewer application, clicking on the search icon
  pops up a window that allows you to enter a search string. If found,
  the viewer application will jump to this string.
\end{enumerate}

You can reduce the number of icons that are shown or their layout by
adjusting the |navigation symbols| template.

\begin{element}{navigation symbols}\yes\yes\yes
  This template is invoked in ``three-star-mode'' by themes
  at the place where the navigation symbols should be
  shown. ``Three-star-mode'' means that the command
  |\usebeamertemplate***| is used.

  Note that, although it may \emph{look} like that the symbols are part of
  the footline, they are more often part of an invisible right
  sidebar.

  \begin{templateoptions}
    \itemoption{default}{}
    Organizes the navigation symbols horizontally.
    \itemoption{horizontal}{}
    This is an alias for the default.
    \itemoption{vertical}{}
    Organizes the navigation symbols vertically.
    \itemoption{only frame symbol}{}
    Shows only the navigational symbol for navigating frames.
  \end{templateoptions}

  \example The following command suppresses all navigation symbols:
\begin{verbatim}
\setbeamertemplate{navigation symbols}{}
\end{verbatim}

  Inside this template, the following inserts are useful:
  \begin{itemize}
    \iteminsert{\insertslidenavigationsymbol}
    Inserts the slide navigation symbols, that is, the slide symbols
    (a rectangle) together with arrows to the left and right that are
    hyperlinked.

    \iteminsert{\insertframenavigationsymbol}
    Inserts the frame navigation symbol.

    \iteminsert{\insertsubsectionnavigationsymbol}
    Inserts the subsection navigation symbol.

    \iteminsert{\insertsectionnavigationsymbol}
    Inserts the section navigation symbol.

    \iteminsert{\insertdocnavigationsymbol}
    Inserts the presentation navigation symbol and (if necessary) the
    appendix navigation symbol.

    \iteminsert{\insertbackfindforwardnavigationsymbol}
    Inserts a back, a find, and a forward navigation symbol.
  \end{itemize}
\end{element}






\subsubsection{The Logo}

To install a logo, use the following command:

\begin{command}{\logo\marg{logo text}}
  The \meta{logo text} is usually a command for including a graphic,
  but it can be any text. The position where the logo is inserted is
  determined by the current theme, you cannot (currently) specify this
  position directly.
  
  \example
\begin{verbatim}
\pgfdeclareimage[height=0.5cm]{logo}{tu-logo}
\logo{\pgfuseimage{logo}}
\end{verbatim}

  \example
\begin{verbatim}
\logo{\includegraphics[height=0.5cm]{logo.pdf}}
\end{verbatim}

  Currently, the effect of this command is just to setup the |logo|
  template. However, a more sophisticated effect might be implemented
  in the future.
  
  \articlenote This command has no effect.

  \begin{element}{logo}\yes\yes\yes
    This template is used to render the logo.
  \end{element}

  The following insert can be used to insert a logo somewhere:
  \begin{itemize}
    \iteminsert{\insertlogo}
    inserts the logo at the current position. This command has the
    same effect as |\usebeamertemplate*{logo}|.
  \end{itemize}
\end{command}


\subsubsection{The Frame Title}

The frame title is shown prominently at the top of the frame and can
be specified with the following command:

\begin{command}{\frametitle\sarg{overlay specification}\oarg{short
  frame title}\marg{frame title text}} 
  You should end the \meta{frame title text} with a period, if the title is a
  proper sentence. Otherwise, there should not be a period. The
  \meta{short frame title} is normally not shown, but its available
  via the |\insertshortframetitle| command. The \meta{overlay
  specification} is mostly useful for suppressing the frame title in
  |article| mode.

  \example
\begin{verbatim}
\begin{frame}
  \frametitle{A Frame Title is Important.}
  \framesubtitle{Subtitles are not so important.}

  Frame contents.
\end{frame}
\end{verbatim}

  If you are using the |allowframebreaks| option with the current frame,
  a continuation text (like ``(cont.)'' or something similar,
  depending on the template |frametitle continuation|) is
  automatically added to the \meta{frame title text} at the end,
  separated with a space.

  \beamernote
  The frame title is not typeset immediately when the command
  |\frametitle| is encountered. Rather, the argument of the command is
  stored internally and the frame title is only typeset when the
  complete frame has been read. This gives you access to both the
  \meta{frame title text} and to the \meta{subframe title text} that
  is possibly introduced using the |\framesubtitle| command.

  \articlenote
  By default, this command creates a new paragraph in |article| mode,
  entitled \meta{frame title text}. Using the \meta{overlay
    specification} makes it easy to suppress the a frame title once in
  a while. If you generally wish to suppress \emph{all} frame
  titles in |article| mode, say |\setbeamertemplate<article>{frametitle}{}|.

  \lyxnote
  The frame title is the text that follows on the line of the
  ``BeginFrame'' style.

  \begin{element}{frametitle}\yes\yes\yes
    \colorfontparents{structure}
    
    When the frame title and subtitle are to be typeset, this template
    is invoked with the \beamer-color and -font |frametitle| set. This
    template is \emph{not} invoked when the commands |\frametitle| or
    |\framesubtitle| are called. Rather, it is invoked when the whole
    frame has been completely read. Till then, the frame title and frame
    subtitle text are stored in a special place. This way, when the
    template is invoked, both inserts are setup correctly. The resulting
    \TeX-box is then magically put back to the top of the frame.

    \begin{templateoptions}
      \itemoption{default}{\oarg{alignment}}
      The frame is typeset using the \beamer-color |frametitle| and the
      \beamer-font |frametitle|. The subtitle is put below using the
      color and font |framesubtitle|. If the color |frametitle| has a
      background, a background bar stretching the whole frame width is put
      behind the title. A background color of the subtitle is ignored. The
      \meta{alignment} is passed on to the |beamercolorbox|
      environment. In particular, useful options are |left|, |center|, and
      |right|. As a special case, the |right| option causes the left
      border of the frame title to be somewhat larger than normal so that
      the frame title is more in the middle of the frame.
      
      \itemoption{shadow theme}{}
      This option is available if the outer theme |shadow| is
      loaded. It draws the frame title on top of a horizontal shading
      between the background colors of |frametitle| and
      |frametitle right|. A subtitle is, if present, also put on this
      bar. Below the bar, a ``shadow'' is drawn.
      
      \itemoption{sidebar theme}{}
      This option is available if the outer theme |sidebar| is loaded
      and if the headline height is not set to 0pt (which can be done
      using and option of the |sidebar| theme). With this option, the
      frame title is put inside a rectangular area that is part of the
      headline (some ``negative space'' is used to raise the frame title
      into this area). The background of the color |frametitle| is not
      used, this is the job of the headline template in this case.

      \itemoption{smoothbars theme}{}
      This option is available if the outer theme |smoothbars| is
      loaded. It typesets the frametitle on a colored bar with the
      background color of |frametitle|. The top and bottom of the bar
      smoothly blend over to backgrounds above and below.

      \itemoption{smoothtree theme}{}
      Like |smoothbars theme|, only for the |smoothtree| theme.
    \end{templateoptions}

    The following commands are useful for this template:
    \begin{templateinserts}
      \iteminsert{\insertframetitle} yields the frame title.
      \iteminsert{\insertframesubtitle} yields the frame subtitle.
    \end{templateinserts}
  \end{element}
\end{command}


\begin{command}{\framesubtitle\sarg{overlay specification}\marg{frame
      subtitle text}}
  If present, a subtitle will be shown in a smaller font below the
  main title. Like the |\frametitle| command, this command can be
  given anywhere in the frame, since the frame title is actually
  typeset only when everything else has already been typeset.
  
  \example
\begin{verbatim}
\begin{frame}
  \frametitle<presentation>{Frame Title Should Be in Uppercase.}
  \framesubtitle{Subtitles can be in lowercase if they are full sentences.}

  Frame contents.
\end{frame}
\end{verbatim}

  \articlenote
  By default, the subtitle is not shown in any way in |article| mode.

  \begin{element}{framesubtitle}\no\yes\yes
    \colorfontparents{frametitle}
    This element provides a color and a font for the subtitle, but no
    template. It is the job of the |frametitle| template to also
    typeset the subtitle.
  \end{element}
\end{command}

Be default, all material for a slide is vertically centered. You can
change this using the following class options:

\begin{classoption}{t}
  Place text of slides at the (vertical) top of the slides. This
  corresponds to a vertical ``flush.'' You can override this for
  individual frames using the |c| or |b| option.
\end{classoption}

\begin{classoption}{c}
  Place text of slides at the (vertical) center of the slides. This is
  the default. You can override this for
  individual frames using the |t| or |b| option.
\end{classoption}


\subsubsection{The Background}
\label{section-canvas}
\label{section-background}

Each frame has a \emph{background}, which---as the name suggests---is
``behind everything.'' The background is a surprisingly complex
object: in \beamer, it consists of a \emph{background canvas} and the
\emph{main background}.

The background canvas can be imagined as a large area on which
everything (the main background and everything else) is painted on. By
default, this canvas is a big rectangle filling the whole frame whose
color is the background of the \beamer-color
|background canvas|. Since this color inherits from |normal text|, by
changing the background color of the normal text, you can change this
color of the canvas.

\example The following command changes the background color to a light
red.
\begin{verbatim}
\setbeamercolor{normal text}{bg=red!20} 
\end{verbatim}

The canvas need not be monochrome. Instead, you can install a shading
or even make it transparent. Making it transparent is a good idea if
you wish to include your slides in some other document.

\example The following command makes the background canvas transparent:
\begin{verbatim}
\setbeamercolor{background canvas}{bg=} 
\end{verbatim}

\begin{element}{background canvas}\yes\yes\yes
  \colorparents{normal text}
  The template is inserted ``behind everything.'' The template should
  typically be some \TeX\ commands that produce a rectangle of height
  |\paperheight| and width |\paperwidth|.

  \begin{templateoptions}
    \itemoption{default}{}
    installs a large rectangle with the background color. If the
    background is empty, the canvas is ``transparent.'' Since
    |background canvas| inherits from |normal text|, you can change the 
    background of the \beamer-color |normal text| to change the
    color of the default canvas. However, to make the canvas
    transparent, only set the background of the canvas empty; leave
    the background of normal text white.
    
    \itemoption{vertical shading}{\oarg{color options}}
    installs a vertically shaded background. \emph{Use with care:
      Background shadings are often distracting!} The following
    \meta{color options} may be given:
    \begin{itemize}
    \item \declare{|top=|\meta{color}} specifies the color at the
      top of the page. By default, 25\% of the foreground of the
      \beamer-color |palette primary| is used.
    \item \declare{|bottom=|\meta{color}} specifies the color at the
      bottom of the page. By default, the background of
      |normal text| at the moment of invocation of this command is
      used. 
    \item \declare{|middle=|\meta{color}} specifies the color
      for the middle of the page. Thus, if this option is given, the
      shading changes from the bottom color to this color and then
      to the top color.
    \item \declare{|midpoint=|\meta{factor}} specifies at which
      point of the page the middle color is used. A factor of |0| is
      the bottom of the page, a factor of |1| is the top. The
      default, which is |0.5| is in the middle.
    \end{itemize}
  \end{templateoptions}  
\end{element}

The main background is drawn on top of the background canvas. It can
be used to add, say, a grid to every frame or a big background picture
or whatever.

\begin{element}{background}\yes\yes\yes
  \colorparents{background canvas}
  The template is inserted ``behind everything, but on top of the
  background canvas.'' Use it for pictures or grids or anything that
  does not necessarily fill the whole background. When this template
  is typeset, the \beamer-color and -font |background| will be
  setup.

  \begin{templateoptions}
    \itemoption{default}{} is empty.
    
    \itemoption{grid}{\oarg{grid options}}
    places a grid on the background. The following
    \meta{grid options} may be given:
    \begin{itemize}
    \item \declare{|step=|\meta{dimension}} specifies the distance
      between grid lines. The default is 0.5cm.
    \item \declare{|color=|\meta{color}} specifies the color of the
      grid lines. The default is 10\% foreground.
    \end{itemize}
  \end{templateoptions}  
\end{element}




\subsection{Margin Sizes}

The ``paper size'' of a \beamer\ presentation is fixed to 128mm times
96mm. The aspect ratio of this size is 4:3, which is exactly what most
beamers offer these days. It is the job of the
presentation program (like |acroread| or |xpdf|) to display the slides
at full screen size. The main advantage of using a small ``paper size''
is that you can use all your normal fonts at their natural sizes. In
particular, inserting a graphic with 11pt labels will result in
reasonably sized labels during the presentation.

You should refrain from changing the ``paper size.'' However, you
\emph{can} change the size of the left and right margins, which
default to 1cm. To change them, you should use the following 
command:

\begin{command}{\setbeamersize\marg{options}}
  The following \meta{options} can be given:
  \begin{itemize}
  \item
    \declare{|text margin left=|\meta{\TeX\ dimension}} sets a new left
    margin. This excludes the left sidebar. Thus, it is the distance
    between the right edge of the left sidebar and the left edge of
    the text.
  \item
    \declare{|text margin right=|\meta{\TeX\ dimension}} sets a new right
    margin.
  \item
    \declare{|sidebar width left=|\meta{\TeX\ dimension}} sets the
    size of the left sidebar. Currently, this command should be given
    \emph{before} a shading is installed for the sidebar canvas.
  \item
    \declare{|sidebar width right=|\meta{\TeX\ dimension}} sets the
    size of the right sidebar.
  \item
    \declare{|description width=|\meta{\TeX\ dimension}} sets the
    default width of description labels, see
    Section~\ref{section-descriptions}. 
  \item
    \declare{|description width of=|\meta{text}} sets the
    default width of description labels to the width of the
    \meta{text}, see  Section~\ref{section-descriptions}.
  \item
    \declare{|mini frame size=|\meta{\TeX\ dimension}} sets the size
    of mini frames in a navigation bar. When two mini frame icons are
    shown alongside each other, their left end points are \meta{\TeX\
      dimension} far apart.
  \item
    \declare{|mini frame offset=|\meta{\TeX\ dimension}} set an
    additional vertical offset that is added to the mini frame size
    when arranging mini frames vertically.
  \end{itemize}

  \articlenote
  This command has no effect in |article| mode.
\end{command}



\subsection{Restricting the Slides of a Frame}
\label{section-restriction}

The number of slides in a frame is automatically
calculated. If the largest number mentioned in any
overlay specification inside the frame is 4, four slides are
introduced (despite the fact that a specification like |<4->|
might suggest that more than four slides would be possible).

You can also specify the number of slides in the frame ``by hand.'' To
do so, you pass an overlay specification the |\frame| command. The
frame will contain only the slides specified in this
argument. Consider the following example. 

\begin{verbatim}
\begin{frame}<1-2,4->
  This is slide number \only<1>{1}\only<2>{2}\only<3>{3}%
  \only<4>{4}\only<5>{5}.
\end{frame}
\end{verbatim}
This command will create a frame containing four slides. The first
will contain the text ``This is slide number~1,'' the second ``This is
slide number~2,'' the third ``This is slide number~4,'' and the fourth
``This is slide number~5.''

A useful specification is just |<0>|, which causes the frame to
have to no slides at all. For example, |\begin{frame}<handout:0>| causes
the frame to be suppressed in the handout version, but to be shown
normally in all other versions. Another useful specification is
|<beamer>|, which causes the frame to be shown normally in |beamer|
mode, but to be suppressed in all other versions.


\subsection{Verbatim Commands and Listings inside Frames}
\label{section-verbatim}

The |\verb| command, the |verbatim| environment, the
|lstlisting| environment, and related environments that allow
you to typeset arbitrary text work only in
frames that contain a single slide or that are suppressed
altogether. Furthermore, you must explicitly specify that the frame
contains verbatim text using the |containsverbatim| commands:
\begin{verbatim}
\begin{frame}[containsverbatim]
  \frametitle{Our Search Procedure}

\begin{verbatim}
  int find(int* a, int n, int x)
  {
    for (int i = 0; i<n; i++)
      if (a[i] == x)
        return i;
  }
\end{verbatim}
\unskip{\MacroFont|\end{verbatim}|}
                                %\begin{verbatim}
\begin{verbatim}
\end{frame}
\end{verbatim}

You may \emph{not} use the |label=|\meta{label name} option if you
have a verbatim text on a slide.

If you need to use verbatim commands in frames that contain several
slides or on a frame that uses the |label| option, you must
\emph{declare} your verbatim texts before the frame starts. This is
done using two special commands:


\begin{command}{\defverb\marg{command name}\opt{|*|}%
    \meta{delimiter symbol}\meta{verbatim text}\meta{delimiter symbol}}
  Declares a verbatim text for later use. The declaration should be
  done outside the frame. Once declared, the text can be used
  in overlays like normal text. The one-line \meta{verbatim text} must
  be delimited by a special \meta{delimiter symbol} (works like the
  |\verb| command). Adding a star makes spaces visible.

\example
\begin{verbatim}
\defverb\mytext!int main (void) { ...!
\defverb\mytextspaces*!int  main  (void ){  ...!

\begin{frame}
  \begin{itemize}
  \item<1-> In C you need a main function.
  \item<2-> It is declare like this: \mytext
  \item<3-> Spaces are not important: \mytextspaces
  \end{itemize}
\end{frame}
\end{verbatim}
\end{command}


\begin{command}{\defverbatim\oarg{options}\marg{command name}\marg{text}}
  The \meta{text} may contain a |verbatim|,  |verbatim*|,
  |lstlisting|, or a related environment. The command \marg{command
    name} can be used later inside frames. The declaration
  should be done outside the frame. Once declared, the text can be
  used in overlays like normal text.

  The following \meta{options} may be given:
  \begin{itemize}
  \item
    \declare{|colored|} declares that the verbatim text will have
    its ``own'' colors. Normally, the verbatim text is typeset using
    the current color, which allows you to use commands like |\alert|
    to make verbatim text red on certain slides. However, if the
    verbatim text has, say, a special background color or different
    parts of it a colored differently (the |lstlisting| environment
    does this), then you do \emph{not} want the verbatim text to
    inherit its color from the ``outside.'' In this case, you should
    give the |colored| option.
  \item
    \declare{|width=|\meta{dimension}} sets the width of the verbatim
    box. The default is the text width at the moment when the
    |\defverbatim| command is used; but this will be too large if
    the box is used inside a |columns| environment later on. In such a
    case you can use this option to specify the width the box will
    later need to have.
  \end{itemize}  

  \example
\begin{verbatim}
\defverbatim\algorithmA{
\begin{verbatim}
int main (void)
{
  cout << "Hello world." << endl;
  return 0;
}
\end{verbatim}
\unskip{\MacroFont|\end{verbatim}|}
\begin{verbatim}
}

\defverbatim[colored]\algorithmB{
\begin{lstlisting}[language={C++},backgroundcolor=\color{yellow}]
int main (void)
{
  cout << "Hello world." << endl;
  return 0;
}
\end{lstlisting}
}

\begin{frame}
  Our algorithm:
  \alert<1>{\algorithmA}
  \uncover<2>{Note the return value.}
\end{frame}

\begin{frame}
  Same algorithm typeset using the lstlisting environment:
  \algorithmB
\end{frame}
\end{verbatim}
%\begin{verbatim}
\end{command}










%%% Local Variables: 
%%% mode: latex
%%% TeX-master: "beameruserguide"
%%% End: 
