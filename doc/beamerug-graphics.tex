
% Copyright 2003, 2004 by Till Tantau <tantau@users.sourceforge.net>.
%
% This program can be redistributed and/or modified under the terms
% of the GNU Public License, version 2.

\section{Graphics, Animations, Sounds, and Slide Transitions}

\subsection{Guidelines}

As always, I start this section with some guidelines on the usage
of graphics and animations that I stick to. The technical issues are
addressed in later subsections.


\subsubsection{Graphics}

\begin{itemize}
\item
  Graphics often convey concepts or ideas much more efficiently than
  text: A picture can say more than a thousand words. (Although,
  sometimes a word can say more than a thousand pictures.)
\item
  Put (at least) one graphic on each slide, whenever
  possible. Visualizations help an audience enormously.
\item
  Usually, place graphics to the left of the text. (Use the
  |columns| environment.) In a left-to-right reading culture, we look
  at the left first.
\item
  Graphics should have the same typographic parameters as the
  text: Use the same fonts (at the same size) in graphics as in the
  main text. A small dot in a graphic should have exactly the same 
  size as a small dot in a text. The line width should be the same as
  the stroke width used in creating the glyphs of the font. For
  example, an 11pt non-bold Computer Modern font has a stroke width of
  0.4pt.
\item
  While bitmap graphics, like photos, can be much more colorful than the
  rest of the text, vector graphics should follow the same ``color
  logic'' as the main text (like black~= normal lines, red~= hilighted
  parts, green~= examples, blue~= structure).
\item
  Like text, you should explain everything that is shown on a
  graphic. Unexplained details make the audience puzzle whether this
  was something important that they have missed. Be careful when
  importing graphics from a paper or some other source. They usually
  have much more detail than you will be able to explain and should be
  radically simplified.
\item
  Sometimes the complexity of a graphic is intensional and you
  are willing to spend much time explaining the graphic in great
  detail. In this case, you will often run into the problem that fine
  details of the graphic are hard to discern for the audience. In this
  case you should use a command like |\framezoom| to create
  anticipated zoomings of interesting parts of the graphic, see
  Section~\ref{section-zooming}. 
\end{itemize}




\subsubsection{Animations and Special Effects}

\begin{itemize}
\item
  Use animations to explain the dynamics of systems, algorithms, etc.
\item
  Do \emph{not} use animations just to attract the attention of your
  audience. This often distracts attention away from the main topic of the
  slide. No matter how cute a rotating,
  flying theorem seems to look and no matter how badly you feel your
  audience needs some action to keep it happy, most people in the
  audience will typically feel you are making fun of them. 
\item
  Do \emph{not} use distracting special effects like ``dissolving''
  slides unless you have a very good reason for using them. If you use
  them, use them sparsely. 
  They \emph{can} be useful in some situations: For example, you might
  show a   young boy on a slide and might wish to dissolve this slide
  into   slide showing a grown man instead. In this case, the
  dissolving  gives the audience visual feedback that the young boy
  ``slowly becomes'' the man. 
\end{itemize}





\subsection{Graphics}
\label{section-graphics}

In the following, the advantages and disadvantages of different
possible ways of creating graphics for \beamer\ presentations are
discussed. Much of the information presented in the following is not
really special to \beamer, but applies to any other document class as well.


\subsubsection{Including External Graphic Files Versus Inlines Graphics}

There are two principal ways of creating \TeX-documents that include
graphics: Either the graphic resides in an external file that is
\emph{included} or the graphic is \emph{inlined}, which means that
\TeX-file contains a bunch commands like ``draw a red line from
here to there.'' In the following, the advantages and disadvantages of
these two approaches are discussed.

You can use an external program, like |xfig| or the Gimp, to create a
graphic. These programs have an option to \emph{export} graphic files
in a format that can then be inserted into the presentation.

The main advantage is:
\begin{itemize}
\item
  You can use a powerful program to create a high-quality graphic.
\end{itemize}

The main disadvantages are:
\begin{itemize}
\item
  You  have to worry about many files. Typically there are at least
  two for each presentation, namely the program's graphic data file and the
  exported graphic file in a format that can be read by \TeX.
\item
  Changing the graphic using the program does not automatically change
  the graphic in the presentation. Rather, you must reexport the
  graphic and rerun \LaTeX.
\item
  It may be difficult to get the line width, fonts, and font sizes
  right.
\item
  Creating formulas as part of graphics is often difficult or
  impossible.
\end{itemize}

You can use all the standard \LaTeX\ commands for inserting graphics,
like |\includegraphics| (be sure to include the package
|graphics| or |graphicx|). Also, the |pgf| package offers commands for
including graphics. Either will work fine in most situations, so
choose whichever you like. Like |\pgfdeclareimage|,
|\includegraphics| also includes an image only once in a |.pdf| file,
even if it used several times (as a matter of fact, the |graphics|
package is even a bit smarter about this than |pgf|). However,
currently only |pgf| offers the ability to include images that are
partly transparent.

At the end of this section you will find notes on how to include
specific graphic formats like |.eps| or |.jpg|.

\lyxnote
You can use the usual ``Insert Graphic'' command to insert a graphic.

The commands |\includegraphics|, |\pgfuseimage|, and |\pgfimage| are
overlay-specification-aware in \beamer. If the overlay specification
does not apply, the command has no effect. This is useful for creating
a simple animation where each picture of the animation resides in a
different file:

\begin{verbatim}
\begin{frame}
  \includegraphics<1>[height=2cm]{step1.pdf}%
  \includegraphics<2>[height=2cm]{step2.pdf}%
  \includegraphics<3>[height=2cm]{step3.pdf}%
\end{frame}
\end{verbatim}


A different way of creating graphics is to insert
graphic drawing commands directly into your \LaTeX\ file. There are numerous
packages that help you do this. They have various degrees of
sophistication. Inlining graphics suffers from none of the
disadvantages mentioned above for including external graphic files,
but the main disadvantage is that it is often hard to use
these packages. In some sense, you ``program'' your graphics, which 
requires a bit of practice.

When choosing a graphic package, there are a few things to keep in
mind:
\begin{itemize}
\item
  Many packages produce poor quality graphics. This is especially true
  of the standard |picture| environment of \LaTeX.
\item
  Powerful packages that produce high-quality graphics often do not
  work together with |pdflatex|.
\item
  The most powerful and easiest-to-use package around, namely
  |pstricks|, does not work together with |pdflatex| and
  this is a fundamental problem. Due to the fundamental differences
  between \textsc{pdf} and PostScript, it is not possible to write a
  ``|pdflatex| back-end for |pstricks|.''
\end{itemize}

A solution to the above problem (though not necessarily the best) is
to use the \textsc{pgf} package. It produces high-quality graphics and
works together with |pdflatex|, but also with normal
|latex|. It is not as powerful as |pstricks| (as pointed
out above, this is because of rather fundamental reasons) and not as
easy to use, but it should be sufficient in most cases.

\lyxnote
Inlined graphics must currently by inserted in a large \TeX-mode
box. This is not very convenient.


\subsubsection{Including Graphic Files Ending \texttt{.eps} or \texttt{.ps}}

External graphic files ending with the extension |.eps| (Encapsulated
PostScript) or |.ps| (PostScript) can be included if you use |latex|
and |dvips|, but \emph{not} when using |pdflatex|. This is true
both for the normal |graphics| package and for |pgf|. When using
|pgf|, do \emph{not} add the extension |.eps|. When using
|graphics|, do add the extension.

If you have a |.eps| graphic and wish to use |pdflatex|, you can use
the program |ps2pdf| to convert the graphic to a |.pdf| file. Note,
however, that it is often a better idea to directly generate a |.pdf|
if the program that produced the |.eps| supports this.


\subsubsection{Including Graphic Files Ending \texttt{.pdf},
  \texttt{.jpg}, \texttt{.jpeg} or \texttt{.png}}

The four formats |.pdf|, |.jpg|, |.jpeg|, and |.png| can only be
included by |pdflatex|. As before, do not add these extension when
using |pgf|, but do add them when using |graphics|. If your graphic
file has any of these formats and you wish/must you |latex| and
|dvips|, you  have to convert your graphic to |.eps| first.


\subsubsection{Including Graphic Files Ending \texttt{.mps}}

A graphic file ending |.mps| (MetaPost PostScript) is a special kind
of Encapsulated PostScript file. Files in this format are produced by
the MetaPost program. As you know, \TeX\ is a program that converts
simple plain text into beautifully typeset documents. The MetaPost
program is similar, only it converts simple plain text into beautiful
graphics.

The MetaPost program converts a plain text file ending |.mp|
into an |.mps| file (although for some unfathomable reason the
extension is not added). The |.mp| file 
must contain text written in the MetaFont programming language. Since
|.mps| files are actually also |.eps| files, you can use the normal 
|\includegraphics| command to include them.

However, as a special bonus, you can \emph{also}
include such an file when using |pdflatex|. Normally, |pdflatex| cannot
handle |.eps| files, but the |.mps| files produced by MetaPost have
such a simple and special structure that this possible. The |graphics|
package implements some filters to convert such PostScript output to
\pdf\ on-the-fly. For this to work, the file should end |.mps| instead
of |.eps|. The following command can be used to make the |graphics|
package just \emph{assume} the extension |.mps| for any file it knows
nothing about (like files ending with |.1|, which is what MetaPost
loves to produce): 
\begin{verbatim}
\DeclareGraphicsRule{*}{mps}{*}{}
\end{verbatim}

This special feature currently  only works with the |graphics|
package, not with |pgf|.



\subsubsection{Including Graphic Files Ending \texttt{.mmp}}

The format |.mmp| (Multi-MetaPost) is actually not format that can be
included directly in a \TeX-file. Rather, like a |.mp| file, it first
has to be converted using the MetaPost program. The crucial difference
between |.mp| and |.mmp| is that in the latter multiple graphics
can reside in a single |.mmp| file (actually, mulitple graphics can
also reside in a |.mp| file, but by convention such a file is called
|.mmp|). When running MetaPost on a |.mmp| file, it will create not a
single encapsulated PostScript file, but several, ending |.0|, |.1|,
|.2|, and so on. The idea is that |.0| might contain a main graphics
and the following pictures contain overlay material that should be
incrementally added to this graphic. 

To include the series of resulting files, you can use the command
|\multiinclude| from the |mpmulti| or from the |xmpmulti| package. How
this program works is explained in Section~\ref{section-mpmulti}.




\subsection{Animations}

\subsubsection{Including External Animation Files}

\label{section-multimedia}

If you have created an animation using some external
program (like a renderer), you can use the capabilities of the
presentation program (like the Acrobat Reader) to show the
animation. Unfortunately, currently there is no portable way of doing
this and even the Acrobat Reader does not support this feature on all
platforms.

To include an animation in a presentation, you can use, for example,
the package |multimedia.sty| which is part of the \beamer\
package. You have to include this package explicitly. Despite being
distributed as part of the \beamer\ distribution, this package is
perfectly self-sufficient and can be used independently of \beamer.

\begin{package}{{multimedia}}
  A stand-alone package that implements several commands for including
  external animation and sound files in a \pdf\ document. The package
  can be used together with both |dvips| plus |ps2pdf| and |pdflatex|,
  though the special sound support is available only in |pdflatex|.

  When including this package, you must also include the |hyperref|
  package. Since you will typically want to include |hyperref| only at
  the very end of the preamble, |multimedia| will not include
  |hyperref| itself. However, |multimedia| can be included both before
  and after |hyperref|. Since \beamer\ includes |hyperref|
  automatically, you need not worry about this when creating a
  presentation using \beamer.
\end{package}

For including an animation in a \pdf\ file, you can use the command
|\movie|, which is explained below. Depending on the used options,
this command will either setup the \pdf\ file such that the viewer
application (like the Acrobat Reader) itself will try to play the
movie or that an external program will be called. The latter approach,
though much less flexible, must be taken if the viewer application is
unable to display the movie itself.

\begin{command}{\movie\oarg{options}\marg{poster text}\marg{movie
  filename}} 
  This command will insert the movie with the filename \meta{movie
  filename} into the \pdf\ file. The movie file must reside at some
  place where the viewer application will be able to find it,
  which is typically only the directory in which the final \pdf\ file
  resides. The movie file will \emph{not} be embedded into the \pdf\
  file in the sense that the actual movie data is part of the
  |main.pdf| file. The movie file must hence be copied and passed
  along with the \pdf\ file. (Nevertheless, one often says that the
  movie is ``embedded'' in the document, but that just means that one can
  click on the movie when viewing the document and the movie will
  start to play.) 

  The movie will use a rectangular area whose size is determined
  either by the |width=| and |height=| options or by the size of the
  \meta{poster text}. The \meta{poster text} can be any \TeX\ text;
  for example, it might be a |\pgfuseimage| command or an
  |\includegraphics| command or a |pgfpicture| environment or just
  plain text. The \meta{poster text} is typeset in a box, the box is
  inserted into the normal text, and the movie rectangle is put
  exactly over this box. Thus, if the \meta{poster text} is an image
  from the movie, this image will be shown until the movie is started,
  when it will be exactly replaced by the movie itself. However, there
  is also a different, sometimes better, way of creating a poster image,
  namely by using the |poster| option as explained later on.

  The aspect ratio of the movie will \emph{not} be corrected
  automatically if the dimension of the \meta{poster text} box does
  not have the same aspect ratio. Most movies have an aspect ratio of
  4:3 or 16:9.

  Despite the name, a movie may consist only of sound with no
  images. In this case, the \meta{posert text} might be a symbol
  representing the sound. There is also a different, dedicated
  command for including sounds in a \pdf\ file, see the |\sound|
  command in Section~\ref{section-sound}. 

  Unless further options are given, the movie will start only when the
  user clicks on it. Whether the viewer application can actually
  display the movie depends on the application and the version. For
  example, the Acrobat Reader up to version~5 does not seem to be able
  to display any movies or sounds on Linux. On the other hand, the
  Acrobat Reader Version~6 on MacOS is able to display anything that
  QuickTime can display, which is just about everything. Embedding
  movies in a \pdf\ document is provided for by the \pdf\ standard and
  is not a pecularity of the Acrobat Reader. In particular, one might
  expect other viewers like |xpdf| to support embedded movies in the
  future.

  \example
  |\movie{\pgfuseimage{myposterimage}}{mymovie.avi}|

  \example
  |\movie[width=3cm,height=2cm,poster]{}{mymovie.mpg}|

  If your viewer application is not able to render your movie, but
  some external application is, you must use the |externalviewer|
  option. This will ask the viewer application to launch an
  application for showing the movie instead of displaying it
  itself. Since this application is started in a new window, this not
  nearly as nice as having the movie be displayed directly by the
  viewer (unless you use evil trickery to suppress the frame of the
  viewer application). Which application is choosen is left to the
  discreetion of the viewer application, which tries to make its
  choice according to the extension of the \meta{movie filename} and
  according to some mapping table for mapping extensions 
  to viewer applications. How this mapping table can be modified
  depends on the viewer application, please see the release notes of
  your viewer.
  
  The following \meta{options} may be given:
  \begin{itemize}
  \item
    \declare{|autostart|}. Causes the movie to start playing immediately
    when the page is shown. At most one movie can be started in this
    way. The viewer application will typically be able to show only at
    most one movie at the same time anyway. When the page is no longer
    shown, the movie immediately stops. This can be a problem if you
    use the |\movie| command to include a sound that should be played
    on after the page has been closed. In this case, the |\sound|
    command must be used.
  \item
    \declare{|borderwidth=|}\meta{\TeX\ dimension}. Causes a border of
    thickness \meta{\TeX\ dimension} to be drawn around the
    movie. Some versions of the Acrobat Reader seem to have a bug and
    do not display this border if is smaller than 0.5bp (about
    0.51pt).
  \item
    \declare{|depth=|}\meta{\TeX\ dimension}. Overrides the depth of the
    \meta{poster text} box and sets it to the given dimension.
  \item
    \declare{|duration=|}\meta{time}|s|. Specifies in seconds how long
    the movie should be shown. The \meta{time} may be a fractional
    value and must be followed by the letter |s|. For example,
    |duration=1.5s| will show the movie for one and a half seconds. In
    conjunction with the |start| option, you can ``cut out'' a part of
    a movie for display.
  \item
    \declare{|externalviewer|}. As explained above, this causes an
    external application to be launched for displaying the movie in a
    separate window. Most options, like |duration| or |loop|, have no
    effect since they are not passed along to the viewer application.
  \item
    \declare{|height=|}\meta{\TeX\ dimension}. Overrides the height of the
    \meta{poster text} box and sets it to the given dimension.
  \item
    \declare{|label=|}\meta{movie label}. Assigns a label to the movie
    such that it can later be referenced by the command
    |\hyperlinkmovie|, which can be used to stop the movie or to show
    a different part of it. The \meta{movie label} is not a normal
    label. It should not be too fancy, since it is inserted literally
    into the \pdf\ code. In particular, it should not contain closing
    parantheses. 
  \item
    \declare{|loop|}. Causes the movie to start again when the end has
    been reached. Normally, the movie just stops at the end.
  \item
    \declare{|once|}. Causes the movie to just stop at the end. This is
    the default.
  \item
    \declare{|palindrome|}. Causes the movie to start playing backwards
    when the end has been reached, and to start playing forward once
    more when the beginning is reached, and so on.
  \item
    \declare{|poster|}. Asks the viewer application to show the first
    image of the movie when the movie is not playing. Normally,
    nothing is shown when the movie is not playing (and thus the box
    containing the \meta{poster text} is shown). For a movie that does
    not have any images (but sound) or for movies with an
    uninformative first image this option is not so useful.
  \item
    \declare{|repeat|} is the same as |loop|.
  \item
    \declare{|showcontrols=|}\meta{true or false}. Causes a control bar
    to be displayed below the movie while it is playing. Instead of
    |showcontrols=true| you can also just say |showcontrols|. By
    default, no control bar is shown.
  \item
    \declare{|start=|}\meta{time}|s|. Causes the first
    \meta{time} seconds of the movie to be skipped. For example,
    |start=10s,duration=5s| will show seconds 10 to 15 of the movie,
    when you play the movie.
  \item
    \declare{|width=|}\meta{\TeX dimension} works like the |height|
    option, only for the width of the poster box.
  \end{itemize}

  \example The following example creates a ``background sound'' for
  the slide.
\begin{verbatim}
\movie[autostart]{}{test.wav}
\end{verbatim}

  \example A movie with two extra buttons for showing different parts
  of the movie.
\begin{verbatim}
\movie[label=cells,width=4cm,height=3cm,poster,showcontrols,duration=5s]{}{cells.avi}

\hyperlinkmovie[start=5s,duration=7s]{cells}{\beamerbutton{Show the middle stage}}

\hyperlinkmovie[start=12s,duration=5s]{cells}{\beamerbutton{Show the late stage}}
\end{verbatim}
\end{command}

A movie can serve as the destination of a special kind of hyperlink,
namely a hyperlink introduced using the following command:

\begin{command}{\hyperlinkmovie\oarg{options}\marg{movie
      label}\marg{text}}
  Causes the \meta{text} to become a movie hyperlink. When you click
  on the \meta{text}, the movie with the label \meta{movie label} will
  start to play (or stop or pause or resume, depending on the
  \meta{options}). The movie must be on the same page as the
  hyperlink.

  The following \meta{options} may be given, many of which are the
  same as for the |\movie| command; if a different option is given for
  the link than for the movie itself, the option for the link takes precedence:
  \begin{itemize}
  \item
    \declare{|duration=|}\meta{time}|s|. As for |\movie|, this causes
    the movie to be played only for the given number of seconds.
  \item
    \declare{|loop|} and \declare{|repeat|}.  As for |\movie|, this causes
    the movie to loop.
  \item
    \declare{|once|}.  As for |\movie|, this causes
    the movie to played only once.
  \item
    \declare{|palindrome|}.  As for |\movie|, this causes
    the movie to be played forth and back.
  \item
    \declare{|pause|}. Causes the playback of the movie to be paused, if
    the movie was currently playing. If not, nothing happens.
  \item
    \declare{|play|}. Causes the movie to be played from whatever start
    position is specified. If the movie is already playing, it will be
    stopped and restarted at the starting position. This is the default.
  \item
    \declare{|resume|}. Resumes playback of the movie, if it has
    previously been paused. If has not been paused, but not started or
    is already playing, nothing happens.
  \item
    \declare{|showcontrols=|}\meta{true or false}. As for |\movie|, this
    causes a control bar to be shown or not shown during playback.
  \item
    \declare{|start=|}\meta{time}|s|. As for |\movie|, this causes
    the given number of seconds to be skipped at the beginning of the
    movie if |play| is used to start the movie.
  \item
    \declare{|stop|}. Causes the playback of the movie to be stopped.
  \end{itemize}
\end{command}


\subsubsection{Animations Created by Showing Slides in Rapid Succession}

You can create an animation in a portable way by using the
overlay commands of the \beamer\ package to create a series of slides
that, when shown in rapid succession, present an animation. This is a
flexible approach, but such animations will typically be rather static
since it will take some time to advance from one slide to the
next. This approach is mostly useful for animations where you want
to explain each ``picture'' of the animation.
When you advance slides ``by hand,'' that is, by pressing a forward
button, it typically takes at least a second for the next slide to
show.

More ``lively'' animations can be created by relying on a capability
of the viewer program. Some programs support
showing slides only for a certain number of seconds during a
presentation (for the Acrobat Reader this works only in full-screen
mode). By setting the number of seconds to zero, you can create a
rapid succession of slides.

To facilitate the creation of animations using this feature, the
following commands can be used: |\animate| and |\animatevalue|.

\begin{command}{\animate\ssarg{overlay specification}}
  The slides specified by \meta{overlay specification} will be shown
  only as shortly as  possible.
\example
\begin{verbatim}
\begin{frame}
  \frametitle{A Five Slide Animation}
  \animate<2-4>

  The first slide is shown normally. When the second slide is shown
  (presumably after pressing a forward key), the second, third, and
  fourth slides ``flash by.'' At the end, the content of the fifth
  slide is shown.

  ... code for creating an animation with five slides ...
\end{frame}
\end{verbatim}

  \articlenote
  This command is ignored in |article| mode.
\end{command}

\begin{command}{\animatevalue|<|\meta{start slide}|-|\meta{end slide}|>|%
    \marg{name}\marg{start value}\marg{end value}}
  The \meta{name} must be the name of a counter or a dimension.
  It will be varied between two values. For the slides in the
  specified range, the counter or dimension is set to an interpolated
  value that depends on the current slide number. On slides before the
  \meta{start slide}, the counter or dimension is set to \meta{start
    value}; on the slides after the \meta{end slide} it is set to
  \meta{end value}.
  \example
\begin{verbatim}
\newcount\opaqueness
\begin{frame}
  \animate<2-10>
  \animatevalue<1-10>{\opaqueness}{100}{0}
  \begin{colormixin}{\the\opaqueness!averagebackgroundcolor}
    \frametitle{Fadeout Frame}

    This text (and all other frame content) will fade out when the
    second slide is shown. This even works with
    {\color{green!90!black}colored} \alert{text}.
  \end{colormixin}
\end{frame}

\newcount\opaqueness
\newdimen\offset
\begin{frame}
  \frametitle{Flying Theorems (You Really Shouldn't!)}

  \animate<2-14>

  \animatevalue<1-15>{\opaqueness}{100}{0}
  \animatevalue<1-15>{\offset}{0cm}{-5cm}
  \begin{colormixin}{\the\opaqueness!averagebackgroundcolor}
  \hskip\offset
    \begin{minipage}{\textwidth}
      \begin{theorem}
        This theorem flies out.
      \end{theorem}
    \end{minipage}
  \end{colormixin}

  \animatevalue<1-15>{\opaqueness}{0}{100}
  \animatevalue<1-15>{\offset}{-5cm}{0cm}
  \begin{colormixin}{\the\opaqueness!averagebackgroundcolor}
  \hskip\offset
    \begin{minipage}{\textwidth}
      \begin{theorem}
        This theorem flies in.
      \end{theorem}
    \end{minipage}
  \end{colormixin}
\end{frame}
\end{verbatim}

  \articlenote
  This command is ignored in |article| mode.
\end{command}

If your animation ``graphics'' reside in indivual external graphic
files, you might also consider using the |\multiinclude| command,
which is explained in Section~\ref{section-mpmulti}, together with
|\animate|. For example, you might create an animation like this,
assuming you have created graphic files named |animation.1| through to
|animation.10|:

\begin{verbatim}
\begin{frame}
  \animate<2-9>
  \multiinclude[start=1]{animation}
\end{frame}
\end{verbatim}




\subsection{Sounds}
\label{section-sound}
You can include sounds in a presentation. Such sound can be played
when you open a slide or when a certain button is clicked. The
commands for including sounds are defined in the package |multimedia|,
which is introduced in Section~\ref{section-multimedia}.

As was already pointed out in Section~\ref{section-multimedia}, a
sound can be included in a \pdf\ presentation by treating it as a
movie and using the |\movie| command. While this is perfectly
sufficient in most cases, there are two cases where this approach is
not satisfactory:
\begin{enumerate}
\item
  When a page is closed, any playing movie is immediately
  stopped. Thus, you cannot use the |\movie| command to create sounds
  that persist for a longer time.
\item
  You cannot play two movies at the same time.
\end{enumerate}

The \pdf\ specification introduces special sound objects, which are
treated quite differently from movie objects. You can create a sound
object using the command |\sound|, which is somewhat similar to
|\movie|. There also exists a |\hyperlinksound| command, which is
similar to |\hyperlinkmovie|. While it is conceptually better to use
|\sound| for sounds, there are a number of this to consider before
using it:
\begin{itemize}
\item
  Several sounds \emph{can} be played at the same time. In particular,
  it is possible to play a general sound in parallel to a (hopefully
  silent) movie.
\item
  A sound playback \emph{can} persist after the current page is closed
  (though it need not).
\item
  The data of a sound file \emph{can} be completely embedded in a
  \pdf\ file, obliberating the need to ``carry around'' other files.
\item
  The sound objects do \emph{not} work together with |dvips| and
  |ps2pdf|. They only work with |pdflatex|.
\item
  There is much less control over what part of a sound should be
  played. In particular, no control bar is shown and you can
  specify neither the start time nor the duration.
\item
  A bug in some versions of the Acrobat Reader makes it necessary to
  provide very exact details on the encoding of the sound file. You
  have to provide the sampling rate, the number of channels (mono or
  stereo), the number of bits per sample, and the sample encoding
  method (raw, signed, Alaw or $\mu$law). If you do not know this data
  or provide it incorrectly, the sound will be played incorrectly.
\item
  It seems  that you can only include uncompressed sound data,
  which can easily become huge. This is not required by the
  specification, but I have been unable to make the Acrobat Reader
  play any compressed data. Data formats that \emph{do} work are
  |.aif| and |.au|.
\end{itemize}


\begin{command}{\sound\oarg{options}\marg{sound poster text}\marg{sound
      filename}}
  This command will insert the sound with the filename \meta{sound
    filename} into the \pdf\ file. As for |\movie|, the file must be
  accesible when the sound is to be played. Unlike |\movie|, you can
  however use the option |inlinesound| to actually embed the sound
  data in the \pdf\ file. 

  Also as for a movie, the \meta{sound poster text} will be be put in
  a box that, when clicked on, will start playing the movie. However,
  you might also leave this box empty and only use the |autostart|
  option. Once playback of a sound has started, it can only be stopped
  by starting the playback of a different sound or by use of the
  |\hyperlinkmute| command. 
  
  The supported sound formats depend on the viewer application. My
  version of the Acrobat Reader supports |.aif| and |.au|. I also need
  to specify information like the sampling rate, even though this
  information could be extracted from the sound file and even though
  the \pdf\ standard specifies that the viewer application should do
  so. In this regard, my version of the Acrobat Reader seems to be
  non-standard-conforming.

  This command only works together with |pdflatex|. If you use
  |dvips|, the poster is still shown, but clicking it has no effect
  and no sound is embedded in any way.

  \example
  |\sound[autostart,samplingrate=22050]{}{applause.au}|

  The following \meta{options} may be given:
  \begin{itemize}
  \item
    \declare{|autostart|}. Causes the sound to start playing immediately
    when the page is shown.
  \item
    \declare{|automute|}. Causes all sounds to be muted when the current
    page is left.
  \item
    \declare{|bitspersample=|}\meta{8 or 16}. Specifies the number of
    bits per sample in the sound file. If this number is 16, this
    option need not be specified.
  \item
    \declare{|channels=|}\meta{1 or 2}. Specifies whether the sound is
    mono or stereo. If the sound is mono, this option need not be specified.
  \item
    \declare{|depth=|}\meta{\TeX\ dimension}. Overrides the depth of the
    \meta{sound poster text} box and sets it to the given dimension.
  \item
    \declare{|encoding=|}\meta{method}. Specifies the encoding method,
    which may be |Raw|, |Signed|, |muLaw|, or |ALaw|. If the method is
    |muLaw|, this option need not be specified.
  \item
    \declare{|height=|}\meta{\TeX\ dimension}. Overrides the height of the
    \meta{sound poster text} box and sets it to the given dimension.
  \item
    \declare{|inlinesound|} causes the sound data to be stored directly
    in the \pdf-file.
  \item
    \declare{|label=|}\meta{sound label}. Assigns a label to the sound
    such that it can later be referenced by the command
    |\hyperlinksound|, which can be used to start a sound. The
    \meta{sound label} is not a normal label. 
  \item
    \declare{|loop|} or \declare{|repeat|}. Causes the sound to start
    again when the end has been reached.
  \item
    \declare{|mixsound=|}\meta{true or false}. If set to |true|, the
    sound is played in addition to any sound that is already
    playing. If set to |false| all other sounds (though not sound from
    movies) are stopped before the the sound is played. The default is
    |false|. 
  \item
    \declare{|samplingrate=|}\meta{number}. Specifies the number of
    samples per second in the sound file. If this number is 44100,
    this option need not be specified.
  \item
    \declare{|width=|}\meta{\TeX\ dimension} works like the |height|
    option, only for the width of the poster box.
  \end{itemize}

  \example The following example creates a ``background sound'' for
  the slide, assuming that |applause.au| is encoded correctly (44100
  samples per second, mono, $\mu$law encoded, 16 bits per sample).
\begin{verbatim}
\sound[autostart]{}{applause.au}
\end{verbatim}
\end{command}

Just like movies, sounds can also serve as  destinations of special
sound hyperlinks. 

\begin{command}{\hyperlinksound\oarg{options}\marg{sound
      label}\marg{text}}
  Causes the \meta{text} to become a sound hyperlink. When you click
  on the \meta{text}, the sound with the label \meta{sound label} will
  start to play.

  The following \meta{options} may be given:
  \begin{itemize}
  \item
    \declare{|loop|} or \declare{|repeat|}. Causes the sound to start
    again when the end has been reached.
  \item
    \declare{|mixsound=|}\meta{true or false}. If set to |true|, the
    sound is played in addition to any sound that is already
    playing. If set to |false| all other sounds (though not sound from
    movies) is stopped before the the sound is played. The default is
    |false|. 
  \end{itemize}
\end{command}

Since there is no direct way of stopping the playback of a sound, the
following command is useful:


\begin{command}{\hyperlinkmute\marg{text}}
  Causes the \meta{text} to become a hyperlink that, when clicked,
  stops the playback of all sounds.
\end{command}




\subsection{Slide Transitions}

\textsc{pdf} in general, and the Acrobat Reader in particular, offer a
standardized way of defining \emph{slide transitions}. Such a
transition is a visual effect that is used to show the slide. For
example, instead of just showing the slide immediately, whatever was
shown before might slowly ``dissolve'' and be replaced by the slide's
content.

There are a number of commands that can be used to specify what effect
should be used when the current slide is presented. Consider the
following example:

\begin{verbatim}
\frame{\pgfuseimage{youngboy}}
\frame{
  \transdissolve
  \pgfuseimage{man}
}
\end{verbatim}
The command |\transdissolve| causes the slide of the
second frame to be shown in a ``dissolved way.'' Note that the
dissolving is a property of the second frame, not of the first one. We
could have placed the command anywhere on the frame.

The transition commands are overlay-specification-aware. We could
collapse the two frames into one frame like this:
\begin{verbatim}
\begin{frame}
  \only<1>{\pgfuseimage{youngboy}}
  \only<2>{\pgfuseimage{man}}
  \transdissolve<2>
\end{frame}
\end{verbatim}
This states that on the first slide the young boy should be shown, on
the second slide the old man should be shown, and when the second
slide is shown, it should be  shown in a ``dissolved way.''

In the following, the different commands for creating transitional
effects are listed. All of them take an optional argument that may
contain a list of \meta{key}|=|\meta{value} pairs. The following
options are possible:

\begin{itemize}
\item
  |duration=|\meta{seconds}. Specifies the number of \meta{seconds}
  the transition effect needs. Default is one second, but often a
  shorter one (like 0.2 seconds) is more appropriate. Viewer
  applications, especially Acrobat, may interpret this option in
  slightly strange ways.
\item
  |direction=|\meta{degree}. For ``directed'' effects, this option
  specifies the effect's direction. Allowed values are |0|, |90|,
  |180|, |270|, and for the glitter effect also |315|.
\end{itemize}

\articlenote
All of these commands are ignored in |article| mode.

\lyxnote
You must insert these commands using \TeX-mode.

\begin{command}{\transblindshorizontal\sarg{overlay specification}\oarg{options}}
  Show the slide as if horizontal blinds where pulled away.
  \example|\transblindshorizontal|
\end{command}
  
\begin{command}{\transblindsvertical\sarg{overlay specification}\oarg{options}}
  Show the slide as if vertical blinds where pulled away.
  \example|\transblindsvertical<2,3>|
\end{command}
  
\begin{command}{\transboxin\sarg{overlay specification}\oarg{options}}
  Show the slide by moving to the center from all four sides.
  \example|\transboxin<1>|
\end{command}
  
\begin{command}{\transboxout\sarg{overlay specification}\oarg{options}}
  Show the slide by showing more and more of a rectangular area that
  is centered on the slide center.
  \example|\transboxout|
\end{command}
 
\begin{command}{\transdissolve\sarg{overlay specification}\oarg{options}}
  Show the slide by slowly dissolving what was shown before.
  \example|\transdissolve[duration=0.2]|
\end{command}
  
\begin{command}{\transglitter\sarg{overlay specification}\oarg{options}}
  Show the slide with a glitter effect that sweeps in the specified
  direction.
  \example|\transglitter<2-3>[direction=90]|
\end{command}
  
\begin{command}{\transsplitverticalin\sarg{overlay specification}\oarg{options}}
  Show the slide by sweeping two vertical lines from the sides inward.
  \example|\transsplitverticalin|
\end{command}
  
\begin{command}{\transsplitverticalout\sarg{overlay specification}\oarg{options}}
  Show the slide by sweeping two vertical lines from the center outward.
  \example|\transsplitverticalout|
\end{command}
  
\begin{command}{\transsplithorizontalin\sarg{overlay specification}\oarg{options}}
  Show the slide by sweeping two horizontal lines from the sides inward.
  \example|\transsplithorizontalin|
\end{command}
  
\begin{command}{\transsplithorizontalout\sarg{overlay specification}\oarg{options}}
  Show the slide by sweeping two horizontal lines from the center outward.
  \example|\transsplithorizontalout|
\end{command}
 
\begin{command}{\transwipe\sarg{overlay specification}\oarg{options}}
  Show the slide by sweeping a single line in the specified direction,
  thereby ``wiping out'' the previous contents.
  \example|\transwipe[direction=90]|
\end{command}


You can also specify how \emph{long} a given slide should be shown,
using the following overlay-specification-aware command:

\begin{command}{\transduration\sarg{overlay specification}\marg{number of seconds}}
  In full screen mode, show the slide for \meta{number of seconds}.
  In zero is specified, the slide is shown as short as possible. This
  can be used to create interesting pseudo-animations.
  \example|\transduration<2>{1}|
\end{command}




%%% Local Variables: 
%%% mode: latex
%%% TeX-master: "beameruserguide"
%%% End: 
