
% Copyright 2003, 2004 by Till Tantau <tantau@users.sourceforge.net>.
%
% This program can be redistributed and/or modified under the terms
% of the GNU Public License, version 2.


\section{Templates}

\label{section-customization}




\subsection{Sidebars}

\label{section-sidebar-templates}

In the following, only the commands for the left sidebars are
listed. Each of these commands also exists for the right sidebar,
with ``left'' replaced by ``right'' everywhere.


\begin{command}{\useleftsidebartemplate\marg{horizontal size}\marg{template}}
  When the sidebar is typeset, the \meta{template} is invoked inside a
  |\vbox| of the height of the sidebar. Thus, the below example
  will produce a sidebar of half a centimeter width, in which the word
  ``top'' is printed just below the headline and ``bottom'' is printed
  just above the footline.
  \example
\begin{verbatim}
\useleftsidebartemplate{1cm}{
  top
  \vfill
  bottom
}
\end{verbatim}
\end{command}

\begin{command}{\useleftsidebarbackgroundtemplate\marg{template}}
  The template is shown behind whatever is shown in the left side
  bar. 
  \example
\begin{verbatim}
\useleftsidebarbackgroundtemplate
  {\color{red}\vrule height\paperheight width\beamer@leftsidebar}
\end{verbatim}
\end{command}


\begin{command}{\useleftsidebarcolortemplate\marg{color expression}}
  Uses the given color as background for the sidebar.
  \example
\begin{verbatim}
\useleftsidebarcolortemplate{\color{red}}
\useleftsidebarcolortemplate{\color[rgb]{1,0,0.5}}
\end{verbatim}
\end{command}

\begin{command}{\useleftsidebarverticalshadingtemplate\marg{bottom
      color expression}\marg{top color expression}}
  Installs a smooth vertical transition between the given colors as
  background for the sidebar.
  \example
\begin{verbatim}
\useleftsidebarverticalshadingtemplate{white}{red}
\end{verbatim}
\end{command}


\begin{command}{\useleftsidebarhorizontalshadingtemplate\marg{left end
      color expression}\marg{right end color expression}}
  Installs a smooth horizontal transition between the given colors as
  background for the sidebar.
  \example
\begin{verbatim}
\useleftsidebarhorizontalshadingtemplate{white}{red}
\end{verbatim}
\end{command}


\begin{command}{\usesectionsidetemplate\marg{current section
      template}\marg{other section template}}
  Both parameters should be |\hbox|es. The templates are used to
  typeset a section name inside a side navigation bar.
  \example
\begin{verbatim}
\usesectionsidetemplate
{\setbox\tempbox=\hbox{\color{black}\tiny{\kern3pt\insertsectionhead}}%
  \ht\tempbox=8pt%
  \dp\tempbox=2pt%
  \wd\tempbox=\beamer@sidebarwidth%
  \box\tempbox}
{\setbox\tempbox=\hbox{\color{structure!75}\tiny{\kern3pt\insertsectionhead}}%
  \ht\tempbox=8pt%
  \dp\tempbox=2pt%
  \wd\tempbox=\beamer@sidebarwidth%
  \box\tempbox}
\end{verbatim}
\end{command}



\begin{command}{\usesubsectionsidetemplate\marg{current subsection
      template}\marg{other subsection template}}
  See |\usesectionsidetemplate|.
  \example
\begin{verbatim}
\usesectionsidetemplate
{\setbox\tempbox=\hbox{\color{black}\tiny{\kern3pt\insertsectionhead}}%
  \ht\tempbox=8pt%
  \dp\tempbox=2pt%
  \wd\tempbox=\beamer@sidebarwidth%
  \box\tempbox}
{\setbox\tempbox=\hbox{\color{structure!75}\tiny{\kern3pt\insertsectionhead}}%
  \ht\tempbox=8pt%
  \dp\tempbox=2pt%
  \wd\tempbox=\beamer@sidebarwidth%
  \box\tempbox}
\end{verbatim}
\end{command}








\subsection{Navigation Bars}

\paragraph{Predefined Templates}\ 

\begin{command}{\beamertemplatecircleminiframe}
  Changes the symbols in a navigation bar used to represent
  a frame to a small circle.
\end{command}

\begin{command}{\beamertemplatecircleminiframeinverted}
  Changes the symbols in a navigation bar used to represent
  a frame to a small circle, but with the colors inverted. Use this if
  the navigation bar is shown on a dark background.
\end{command}

\begin{command}{\beamertemplatesphereminiframe}
  Changes the symbols in a navigation bar used to represent
  a frame to a small sphere.
\end{command}

\begin{command}{\beamertemplatesphereminiframeinverted}
  Changes the symbols in a navigation bar used to represent
  a frame to a small sphere, but with the colors inverted. Use this if
  the navigation bar is shown on a |structure| background.
\end{command}

\begin{command}{\beamertemplateboxminiframe}
  Changes the symbols in a navigation bar used to represent
  a frame to a small box.
\end{command}

\begin{command}{\beamertemplateticksminiframe}
  Changes the symbols in a navigation bar used to represent
  a frame to a small vertical bar of varying length.
\end{command}


\paragraph{Template Installation Commands}\ 

\begin{command}{\usesectionheadtemplate\marg{current section
      template}\marg{other section template}}
  The templates are used to render the section names in a navigation
  bar. 
  \example
\begin{verbatim}
\usesectionheadtemplate
  {\hfill\color{white}\tiny\textbf{\insertsectionheadnumber.\ \
    \insertsectionhead}}
  {\hfill\color{white!50!black}\tiny\textbf{\insertsectionheadnumber.\ \ 
    \insertsectionhead}}
\end{verbatim}
\end{command}
  

\begin{command}{\usesubsectionheadtemplate\marg{current subsection
      template}\marg{other subsection template}}
  See |\usesectionheadtemplate|.
  \example
\begin{verbatim}
\usesubsectionheadtemplate{\color{white}%
  \tiny\textbf{\insertsectionheadnumber.\insertsubsectionheadnumber\ \
  \insertsubsectionhead}}%
  {\color{white!50!beamerstructure}%
  \tiny\textbf{\insertsectionheadnumber.\insertsubsectionheadnumber\ \ 
  \insertsubsectionhead}}
\end{verbatim}
\end{command}

\begin{command}{\useminislidetemplate%
    \marg{template current frame icon}%
    \marg{template current subsection frame icon}\\%
    \marg{template other frame icon}%
    \marg{horizontal offset}%
    \marg{vertical offset}}
  The templates are used to draw frame icons in navigation bars. The
  offsets describe the offset between icons.
  \example
\begin{verbatim}
\useminislidetemplate
  {
    \color{structure}%
    \hskip-0.4pt\vrule height\boxsize width1.2pt%
  }  
  {%
    \color{structure}%
    \vrule height\boxsize width0.4pt%
  }
  {%
    \color{structure!50}%
    \vrule height\boxsize width0.4pt%
  }
  {.1cm}
  {.05cm}
\end{verbatim}
\end{command}


\paragraph{Inserts}\


\begin{command}{\insertsectionhead}
  Inserts the text of the section that is to be typeset in a
  navigation bar.
\end{command}

\begin{command}{\insertsubsectionhead}
  Inserts the text of the subsection that is to be typeset in a
  navigation bar. 
\end{command}

\begin{command}{\insertsectionheadnumber}
  Inserts the number of the section that is to be typeset in a
  navigation bar. 
\end{command}

\begin{command}{\insertsubsectionheadnumber}
  Inserts the number of the subsection that is to be typeset in a
  navigation bar. 
\end{command}

\begin{command}{\insertpartheadnumber}
  Inserts the number of the part of the current section of subsection
  that is to be typeset in a navigation bar. 
\end{command}





\subsection{Navigation Symbols}
\label{section-navigation-symbols-template}

\paragraph{Predefined Templates}\ 

\begin{command}{\beamertemplatenavigationsymbolsempty}
  Suppresses all navigation symbols.
\end{command}

\begin{command}{\beamertemplatenavigationsymbolsframe}
  Shows only the frame symbol as navigation symbol.
\end{command}

\begin{command}{\beamertemplatenavigationsymbolsvertical}
  Organizes the navigation symbols vertically.
\end{command}

\begin{command}{\beamertemplatenavigationsymbolshorizontal}
  Organizes the navigation symbols horizontally.
\end{command}



\paragraph{Template Installation Commands}\ 

\begin{command}{\usenavigationsymbolstemplate\marg{symbols template}}
  Installs a new symbols template. This template is invoked by themes
  at the place where the navigation symbols should be shown.
  \example
\begin{verbatim}
\usenavigationsymbolstemplate{\vbox{%
  \hbox{\insertslidenavigationsymbol}
  \hbox{\insertframenavigationsymbol}
  \hbox{\insertsubsectionnavigationsymbol}
  \hbox{\insertsectionnavigationsymbol}
  \hbox{\insertdocnavigationsymbol}
  \hbox{\insertbackfindforwardnavigationsymbol}}}
\end{verbatim}
\end{command}


\paragraph{Inserts for these Templates}\ 

The following inserts are useful for the navigation symbols template:

\begin{command}{\insertslidenavigationsymbol}
  Inserts the slide navigation symbol, see
  Section~\ref{section-navigation-symbols}.
\end{command}

\begin{command}{\insertframenavigationsymbol}
  Inserts the frame navigation symbol, see
  Section~\ref{section-navigation-symbols}.
\end{command}

\begin{command}{\insertsubsectionnavigationsymbol}
  Inserts the subsection navigation symbol, see
  Section~\ref{section-navigation-symbols}.
\end{command}

\begin{command}{\insertsectionnavigationsymbol}
  Inserts the section navigation symbol, see
  Section~\ref{section-navigation-symbols}.
\end{command}

\begin{command}{\insertdocnavigationsymbol}
  Inserts the presentation navigation symbol and (if necessary) the
  appendix navigation symbol, see
  Section~\ref{section-navigation-symbols}.
\end{command}

\begin{command}{\insertbackfindforwardnavigationsymbol}
  Inserts a back, a find, and a forward navigation symbol, see
  Section~\ref{section-navigation-symbols}.
\end{command}





\subsection{Theorem Environments}

\label{section-theorems-templates}

\paragraph{Predefined Templates}\

\begin{command}{\beamertemplatetheoremssimple}
  Causes the theorem head and text to be directly passed to the
  |block| or |exampleblock| environment. All font specifications for
  theorems are ignored. 
\end{command}

\begin{command}{\beamertemplatetheoremsunnumbered}
  Causes theorems to be typeset as follows: The font specification for
  the body is honored, the font specification for the head is
  ignored. No theorem number is printed. This is the default.
\end{command}

\begin{command}{\beamertemplatetheoremsnumbered}
  Like |\beamertemplatetheoremsunnumbered|, except that the theorem
  number is printed for environments that are numbered.
\end{command}

\begin{command}{\beamertemplatetheoremsamslike}
  This causes theorems to be put in a |block| or |exampleblock|, but
  to be otherwise typeset as is normally done in |amsthm|. Thus the
  head font and body font depend on the setting for the theorem to be
  typeset and theorems are numbered. 
\end{command}



\paragraph{Template Installation Commands}\

\begin{command}{\usetheoremtemplate\marg{block beginning
      template}\marg{block end template}}
  \beamernote
  Whenever an environment declared using the command |\newtheorem| is
  to be typeset, the \meta{block beginning template} is inserted at
  the beginning and the \meta{block end template} at the end. If there
  is a overlay specification when an environment like |theorem| is
  used, this overlay specification will directly follow the
  \meta{block beginning template} upon invocation. This is even true
  if there was an optional argument to the |theorem| environment. This
  optional argument is available via the insert |\inserttheoremaddition|.

  Numerous inserts are available in this template, see below.  

  Before the template starts, the font is set to the body font
  prescribed by the environment to be typeset.
  
  \example The following typesets theorems like |amsthm|:
\begin{verbatim}
\usetheoremtemplate{\begin{\inserttheoremblockenv}
  {%
    \inserttheoremheadfont
    \inserttheoremname
    \inserttheoremnumber
    \ifx\inserttheoremaddition\@empty\else\ (\inserttheoremaddition)\fi%
    \inserttheorempunctuation
  }%
}{\end{\inserttheoremblockenv}}
\end{verbatim}

  \example In the following example, all font ``suggestions'' for the
  environment are suppressed or ignored; and the theorem number is
  suppressed.
\begin{verbatim}
\usetheoremtemplate{%
  \normalfont% ignore body font
  \begin{\inserttheoremblockenv}
  {%
    \inserttheoremname
    \ifx\inserttheoremaddition\@empty\else\ (\inserttheoremaddition)\fi%
  }%
}{\end{\inserttheoremblockenv}}
\end{verbatim}
  
  \articlenote
  This command is not available in |article| mode.
\end{command}


\paragraph{Inserts for these Templates}\

\begin{command}{\inserttheoremblockenv}
  This will normally expand to |block|, but if a theorem that has
  theorem style |example| is typeset, it will expand to
  |exampleblock|. Thus you can use this insert to decide which
  environment should be used when typesetting the theorem.
\end{command}

\begin{command}{\inserttheoremheadfont}
  This will expand to a font chainging command that switches to the
  font to be used in the head of the theorem. By not inserting it, you
  can ignore the head font.
\end{command}

\begin{command}{\inserttheoremname}
  This will expand to the name of the environment to be typeset (like
  ``Theorem'' or ``Corollary''). 
\end{command}


\begin{command}{\inserttheoremnumber}
  This will expand to the number of the current theorem preceeded by a
  space or to nothing, if the current theorem does not have a number.
\end{command}

\begin{command}{\inserttheoremaddition}
  This will expand to the optional argument given to the environment
  or will be empty, if there was no optional argument.
\end{command}

\begin{command}{\inserttheorempunctuation}
  This will expand to the punctuation character for the current
  environment. This is usually a period.
\end{command}




%%% Local Variables: 
%%% mode: latex
%%% TeX-master: "beameruserguide"
%%% End: 
