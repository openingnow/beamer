
% Copyright 2003, 2004 by Till Tantau <tantau@users.sourceforge.net>.
%
% This program can be redistributed and/or modified under the terms
% of the GNU Public License, version 2.


\section{Templates}

\label{section-customization}




\subsection{Sidebars}

\label{section-sidebar-templates}

In the following, only the commands for the left sidebars are
listed. Each of these commands also exists for the right sidebar,
with ``left'' replaced by ``right'' everywhere.


\begin{command}{\useleftsidebartemplate\marg{horizontal size}\marg{template}}
  When the sidebar is typeset, the \meta{template} is invoked inside a
  |\vbox| of the height of the sidebar. Thus, the below example
  will produce a sidebar of half a centimeter width, in which the word
  ``top'' is printed just below the headline and ``bottom'' is printed
  just above the footline.
  \example
\begin{verbatim}
\useleftsidebartemplate{1cm}{
  top
  \vfill
  bottom
}
\end{verbatim}
\end{command}

\begin{command}{\useleftsidebarbackgroundtemplate\marg{template}}
  The template is shown behind whatever is shown in the left side
  bar. 
  \example
\begin{verbatim}
\useleftsidebarbackgroundtemplate
  {\color{red}\vrule height\paperheight width\beamer@leftsidebar}
\end{verbatim}
\end{command}


\begin{command}{\useleftsidebarcolortemplate\marg{color expression}}
  Uses the given color as background for the sidebar.
  \example
\begin{verbatim}
\useleftsidebarcolortemplate{\color{red}}
\useleftsidebarcolortemplate{\color[rgb]{1,0,0.5}}
\end{verbatim}
\end{command}

\begin{command}{\useleftsidebarverticalshadingtemplate\marg{bottom
      color expression}\marg{top color expression}}
  Installs a smooth vertical transition between the given colors as
  background for the sidebar.
  \example
\begin{verbatim}
\useleftsidebarverticalshadingtemplate{white}{red}
\end{verbatim}
\end{command}


\begin{command}{\useleftsidebarhorizontalshadingtemplate\marg{left end
      color expression}\marg{right end color expression}}
  Installs a smooth horizontal transition between the given colors as
  background for the sidebar.
  \example
\begin{verbatim}
\useleftsidebarhorizontalshadingtemplate{white}{red}
\end{verbatim}
\end{command}


\begin{command}{\usesectionsidetemplate\marg{current section
      template}\marg{other section template}}
  Both parameters should be |\hbox|es. The templates are used to
  typeset a section name inside a side navigation bar.
  \example
\begin{verbatim}
\usesectionsidetemplate
{\setbox\tempbox=\hbox{\color{black}\tiny{\kern3pt\insertsectionhead}}%
  \ht\tempbox=8pt%
  \dp\tempbox=2pt%
  \wd\tempbox=\beamer@sidebarwidth%
  \box\tempbox}
{\setbox\tempbox=\hbox{\color{structure!75}\tiny{\kern3pt\insertsectionhead}}%
  \ht\tempbox=8pt%
  \dp\tempbox=2pt%
  \wd\tempbox=\beamer@sidebarwidth%
  \box\tempbox}
\end{verbatim}
\end{command}



\begin{command}{\usesubsectionsidetemplate\marg{current subsection
      template}\marg{other subsection template}}
  See |\usesectionsidetemplate|.
  \example
\begin{verbatim}
\usesectionsidetemplate
{\setbox\tempbox=\hbox{\color{black}\tiny{\kern3pt\insertsectionhead}}%
  \ht\tempbox=8pt%
  \dp\tempbox=2pt%
  \wd\tempbox=\beamer@sidebarwidth%
  \box\tempbox}
{\setbox\tempbox=\hbox{\color{structure!75}\tiny{\kern3pt\insertsectionhead}}%
  \ht\tempbox=8pt%
  \dp\tempbox=2pt%
  \wd\tempbox=\beamer@sidebarwidth%
  \box\tempbox}
\end{verbatim}
\end{command}














%%% Local Variables: 
%%% mode: latex
%%% TeX-master: "beameruserguide"
%%% End: 
