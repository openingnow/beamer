
% Copyright 2003, 2004 by Till Tantau <tantau@users.sourceforge.net>.
%
% This program can be redistributed and/or modified under the terms
% of the GNU Public License, version 2.


\section{Templates}

\label{section-customization}


%% \subsection{Themes}

%% Just like \LaTeX\ in general, the \beamer\ class tries to separate the
%% contents of a text from the way it is typeset (displayed). There are two ways in
%% which you can change how a presentation is typeset: you can specify a
%% different theme and you can specify different templates. A theme is
%% a predefined collection of templates.

%% There exist a number of different predefined themes that can be used
%% together with the \beamer\ class. Feel free to add further themes.
%% Themes are used by including an appropriate \LaTeX\ style file, using
%% the standard |\usepackage| command.


%% \begin{smallpackage}{{beamerthemebars}}
%%   \example

%%   \pgfuseimage{themebars}\quad\pgfuseimage{themebars2}
%% \end{smallpackage}


%% \begin{package}{{beamerthemeboxes}\opt{|[headheight=|\meta{head height}|,footheight=|\meta{foot height}|]|}}
%%   \example

%%   \pgfuseimage{themeboxes}\quad\pgfuseimage{themeboxes2}

%%   \example
%% \begin{verbatim}
%% \usepackage[headheight=12pt,footheight=12pt]{beamerthemeboxes}
%% \end{verbatim}

%%   For this theme, you can specify an arbitrary number of templates for
%%   the boxes in the headline and in the footline. You can add a
%%   template for another box by using the following commands.
%% \end{package}

%% \begin{command}{\addheadboxtemplate%
%%     \marg{background color command}\marg{box template}}
%%   Each time this command is invoked, a new box is added to the head
%%   line, with the first added box being shown on the left. All boxes
%%   will have the same size.
%%   \example
%% \begin{verbatim}
%% \addheadboxtemplate{\color{black}}{\color{white}\tiny\quad 1. Box}
%% \addheadboxtemplate{\color{structure}}{\color{white}\tiny\quad 2. Box}
%% \addheadboxtemplate{\color{structure!50}}{\color{white}\tiny\quad 3. Box}
%% \end{verbatim}
%% \end{command}

%% \begin{command}{\addfootboxtemplate%
%%     \marg{background color command}\marg{box template}}
%%   \example
%% \begin{verbatim}
%% \addfootboxtemplate{\color{black}}{\color{white}\tiny\quad 1. Box}
%% \addfootboxtemplate{\color{structure}}{\color{white}\tiny\quad 2. Box}
%% \end{verbatim}
%% \end{command}


%% \begin{smallpackage}{{beamerthemeclassic}}
%%   \example

%%   \pgfuseimage{themeclassic}\quad\pgfuseimage{themeclassic2}
%% \end{smallpackage}


%% \begin{smallpackage}{{beamerthemelined}}
%%   \example

%%   \pgfuseimage{themelined}\quad\pgfuseimage{themelined2}
%% \end{smallpackage}


%% \begin{smallpackage}{{beamerthemeplain}}
%%   \example

%%   \pgfuseimage{themeplain}\quad\pgfuseimage{themeplain2}
%% \end{smallpackage}


%% \begin{package}{{beamerthemesidebar}\opt{|[width=|\meta{sidebar
%%         width}|,dark,tab]|}}
%%   The option |width| sets the width of the sidebar to \meta{sidebar
%%     width}. The option |dark| makes the sidebar and the whole theme
%%   darked. The option |tab| causes the current section or subsection to
%%   be hilighted by changing the background behind the entry, rather
%%   than hilighting the entry itself.
  
%%   \example |\usepackage{beamerthemesidebar}|

%%   \pgfuseimage{themesidebar}\quad\pgfuseimage{themesidebar2}

%%   \example |\usepackage[tab]{beamerthemesidebar}|

%%   \pgfuseimage{themesidebartab}\quad\pgfuseimage{themesidebartab2}

%%   \example |\usepackage[dark]{beamerthemesidebar}|

%%   \pgfuseimage{themesidebardark}\quad\pgfuseimage{themesidebardark2}

%%   \example |\usepackage[dark,tab]{beamerthemesidebar}|

%%   \pgfuseimage{themesidebardarktab}\quad\pgfuseimage{themesidebardarktab2}
%% \end{package}


%% \begin{smallpackage}{{beamerthemeshadow}}
%%   \example

%%   \pgfuseimage{themeshadow}\quad\pgfuseimage{themeshadow2}
%% \end{smallpackage}

%% \begin{smallpackage}{{beamerthemesplit}}
%%   \example

%%   \pgfuseimage{themesplit}\quad\pgfuseimage{themesplit2}
%% \end{smallpackage}

%% \begin{smallpackage}{{beamerthemetree}}
%%   \example

%%   \pgfuseimage{themetree}\quad\pgfuseimage{themetree2}
%% \end{smallpackage}


%% \begin{smallpackage}{{beamerthemetree}\declare{|[bar]|}}
%%   \example

%%   \pgfuseimage{themetreebars}\quad\pgfuseimage{themetreebars2}
%% \end{smallpackage}



%\subsection{Templates}
\label{section-templates}

%\subsection{Introduction to Templates}

\subsection{Introduction to Templates}



\subsection{Default Templates, Fonts, and Colors}

\label{section-color-listing}

In this section all templates, colors, and fonts that are used by the
default \beamer\ themes are listed in alphabetical order. 

\begin{itemize}
  \templatefontcolor{alerted text}\no\yes\yes
  The color/font is used when text is typeset inside an |alertenv| or
  when the |\alert| command is used. The background is currently
  ignored.

  \templatefontcolor{background}\yes\yes\yes
  \colorparents{background canvas}
  Used to typeset the background. It is empty be default. See
  Section~\ref{section-background} for details. 

  \templatefontcolor{background canvas}\yes\yes\yes
  \colorparents{normal text}
  Used to render the background canvas. By default, the background
  of the color is used to fill a large rectangle. See
  Section~\ref{section-canvas} for details.  

  \colornote
  If the background is empty, no canvas is drawn by default.
  
  \templatefontcolor{example text}\no\yes\yes
  The color/font is used when text is typeset inside an |example|
  block. 
  
  \templatefontcolor{local structure}\no\yes\no
  \colornote
  This color should be used to typeset structural elements that change
  their color according to the ``local environment.'' For example, the
  color of an item ``button'' in an |itemize| environment changes its
  color according to circumstances. If it is used inside an example
  block, it should have the |example text| color; if it is currently
  ``alerted'' it should have the |alerted text| color. This color
  will setup by certain environments to have the color that should be
  used to typset things like item buttons. Since the color used for
  items, |item|, inherits from this color by default, items
  automatically change their color according to the current
  situation.

  If you write your own environment in which the item buttons are
  similar structural elements should have a different color, you
  should change the color |local structure| inside these
  environments. 

  \templatefontcolor{math text}\no\yes\no
  This color is the parent of |math text inlined| and
  |math text displayed|. It is empty by default. See those colors for
  details. 

  \templatefontcolor{math text inlined}\no\yes\no
  \colorparents{math text}
  \colornote
  If the foreground of this color is set, inlined mathematical text is
  typeset using this color. This is done via some |\everymath| hackery
  and may not work in all cases. If not, you'll have to try to find a
  way around the problem. The background is currently
  ignored. 

  \templatefontcolor{math text displayed}\no\yes\no
  \colorparents{math text}
  \colornote
  Like |math text inlined|, only for so-called ``displayed''
  mathematical text. This is mathematical text between |\[| and |\]| or
  between |$$| and |$$| or inside environments like |equation| or
  |align|. The setup of this color is somewhat fragile, use at your
  own risk. The background is currently
  ignored.   

  \templatefontcolor{normal text}\no\yes\yes
  \colornote
  The color is used for normal text. At the beginning of
  the document the foreground color is installed as
  |\normalcolor|. The background of this color is used by the
  default background canvas for the background of the
  presentation, see Section~\ref{section-canvas}. The background is
  also the default value of the normal color |bg|. 

  Since the color is the ``root'' of all other \beamer-colors, both a
  foreground and a background must be installed. In particular, to get
  a transparent background canvas, make the background of the
  \beamer-color |background canvas| empty, not the background of this
  color.

  \templatefontcolor{normal text in math text}\no\yes\no
  \colornote
  If the foreground of this color is set, normal text inside
  mathematical text (which is introduced using the |\text| command)
  will be typeset using this color. The background is currently
  ignored. 

  \fontnote
  This font is not used, currently. In particular, redefining this
  font will not have any effect. This may change in the future.

  \templatefontcolor{palette primary}\no\yes\no
  \colornote
  Layout themes (should) base the color of navigational elements and,
  possibly, also of other elements, on the four palette colors. The
  ``primary'' palette should be used for the most important
  navigational elements, which are usually the ones that change most
  often and hence require the most attention by the audience. The
  ``secondary'' and ``ternary'' are less important, the ``fourth'' one
  is least important.

  By default, the palette colors do not have a background and the
  foreground ranges from |structure.fg| to |black|.

  For the sidebar, there is an extra set of palette colors, see
  |palette sidebar primary|. 

  \templatefontcolor{palette secondary}\no\yes\no
  See |palette primary|.

  \templatefontcolor{palette ternary}\no\yes\no
  See |palette primary|.

  \templatefontcolor{palette fourth}\no\yes\no
  See |palette primary|.

  \templatefontcolor{palette sidebar primary}\no\yes\no
  Similar to |palette primary|, only outer themes (should) base the
  colors of elements in the sidebar of the four sidebar palette
  colors.

  \templatefontcolor{palette sidebar secondary}\no\yes\no
  See |palette sidebar primary|.

  \templatefontcolor{palette sidebar ternary}\no\yes\no
  See |palette sidebar primary|.

  \templatefontcolor{palette sidebar fourth}\no\yes\no
  See |palette sidebar primary|.
  

  \templatefontcolor{structure}\no\yes\yes
  This color/font is used for ``structural'' elements, that is, for
  all elements that are supposed to help the audience see the
  structure of your presentation. This color/font is used, by default,
  for the headings of blocks, for item buttons, for titles, and
  several other things. In most color themes, the colors for
  navigational elements in the headline or the footline are derived
  from the foreground color of |structure|. By changing the structure
  color you can easily change the ``basic color'' of your
  presentation, other than the color of normal text. See also the
  related color |local structure| and the related font
  |tiny structure|. 

  \templatefontcolor{tiny structure}\no\no\yes
  This special font is used for ``tiny'' structural text. Basically,
  this font should be used whenever a structural element uses a tiny
  font. The idea is that the tiny versions of the |structure| font
  often are not suitable. For example, it is often necessary to use a
  boldface version for them. Also, one might wish to have serif smallcaps
  structural text, but still retain normal sans-serif tiny structural
  text. 

  \templatefontcolor{titlelike}\no\yes\yes
  This color/font is a more specialized form of the |structure|
  color/font. It is the base for all elements that are ``like
  titles.'' This includes the frame title and subtitle as well as the
  document title and subtitle.
\end{itemize}



%% \subsection{Title Page}

%% \paragraph{Predefined Templates}\ 

%% \begin{command}{\beamertemplatelargetitlepage}
%%   Causes the title page to be typeset with a large font for the title.
%% \end{command}

%% \begin{command}{\beamertemplateboldtitlepage}
%%   Causes the title page to be typeset with a bold font for the title.
%% \end{command}



%% \paragraph{Template Installation Commands}\ 

%% \begin{command}{\usetitlepagetemplate\marg{title page template}}
%%   \example
%% \begin{verbatim}
%% \usetitlepagetemplate{
%%   \vbox{}
%%   \vfill
%%   \begin{centering}
%%     \Large\structure{\inserttitle}
%%     \vskip1em\par
%%     \normalsize\insertauthor\vskip1em\par
%%     {\scriptsize\insertinstitute\par}\par\vskip1em
%%     \insertdate\par\vskip1.5em
%%     \inserttitlegraphic
%%   \end{centering}
%%   \vfill
%% }
%% \end{verbatim}
%% \end{command}

%% If you wish to suppress the head- and footline in the title page, use
%% |\frame[plain]{\titlepage}|.




%% \paragraph{Inserts for these Templates}\ 

%% \begin{command}{\insertauthor}
%%   Inserts the author names into a template.
%% \end{command}

%% \begin{command}{\insertdate}
%%   Inserts the date into a template.
%% \end{command}

%% \begin{command}{\insertinstitute}
%%   Inserts the institute into a template.
%% \end{command}

%% \begin{command}{\inserttitle}
%%   Inserts a version of the document title into a template that is
%%   useful for the title page. 
%% \end{command}

%% \begin{command}{\insertsubtitle}
%%   Inserts a version of the document subtitle into a template that is
%%   useful for the title page. 
%% \end{command}

%% \begin{command}{\inserttitlegraphic}
%%   Inserts the title graphic into a template.
%% \end{command}



%% \subsection{Part Page}

%% \label{section-part-page-template}

%% \paragraph{Predefined Templates}\ 

%% \begin{command}{\beamertemplatelargepartpage}
%%   Causes the part pages to be typeset with a large font for the part name.
%% \end{command}

%% \begin{command}{\beamertemplateboldpartpage}
%%   Causes the part pages to be typeset with a bold font for the part name.
%% \end{command}


%% \paragraph{Template Installation Commands}\ 

%% \begin{command}{\usepartpagetemplate\marg{part page template}}
%%   \example
%% \begin{verbatim}
%% \usepartpagetemplate{
%%   \begin{centering}
%%     \Large\structure{\partname~\insertromanpartnumber}
%%     \vskip1em\par
%%     \insertpart\par
%%   \end{centering}
%%   }
%% \end{verbatim}
%% \end{command}


%% \paragraph{Inserts for these Templates}\ 

%% \begin{command}{\insertpart}
%%   Inserts the current part name.
%% \end{command}

%% \begin{command}{\insertpartnumber}
%%   Inserts the current part number as an Arabic number into a template.
%% \end{command}

%% \begin{command}{\insertpartromannumber}
%%   Inserts the current part number as a Roman number into a template.
%% \end{command}



%% \subsection{Frames}

%% \label{section-frame-template}

%% \paragraph{Template Installation Commands}\ 

%% \begin{command}{\useframetemplate\marg{begin of frame}\marg{end of
%%       frame}}
%%   \beamernote
%%   This command is currently \emph{not} available in the presentation
%%   modes.

%%   \articlenote
%%   The \meta{begin of frame} text is inserted at the beginning of each
%%   frame, when it is inserted into the article. The text \meta{end of
%%   frame} is inserted at the end. You can use this template to put,
%%   say, lines around a frame or to put the whole frame into  a
%%   minipage. By default, nothing is inserted.
  
%%   \example
%% \begin{verbatim}
%% \useframetemplate{\par\medskip\hrule\smallskip}{\par\smallskip\hrule\medskip}
%% \end{verbatim}
%% \end{command}





%% \subsection{Background}

%% \label{section-backgrounds}

%% \paragraph{Predefined Templates}\ 

%% \begin{command}{\beamertemplatesolidbackgroundcolor\marg{color}}
%%   Installs the given color as the background color for every frame.
  
%%   \example |\beamertemplatesolidbackgroundcolor{white!90!red}|
%% \end{command}

%% \begin{command}{\beamertemplateshadingbackground%
%%     \marg{color expression page bottom}\marg{color expression page top}}
%%   Installs a vertically shaded background such that the
%%   specified bottom color changes smoothly to the specified top
%%   color. \emph{Use with care: Background shadings are often
%%     distracting!} However, a very light shading with warm colors can 
%%   make a presentation more lively.
%%   \example
%% \begin{verbatim}
%% \beamertemplateshadingbackground{red!10}{blue!10}
%% %% Bottom is light red, top is light blue
%% \end{verbatim}
%% \end{command}


%% \begin{command}{\beamertemplategridbackground\oarg{spacing}}
%%   Installs a light grid as background with lines spaced apart by
%%   \meta{spacing}. Default is half a centimeter.

%%   \example |\beamertemplategridbackground[0.2cm]|
%% \end{command}


%% \paragraph{Template Installation Commands}\ 

%% \begin{command}{\usebackgroundtemplate\marg{background template}}
%%   Installs a new background template. Set the background color of
%%   |normal text| after you have called this macro to the average color
%%   of the background you have installed.
%%   \example
%% \begin{verbatim}
%% \usebackgroundtemplate{%
%%   \color{red}%
%%   \vrule  height\paperheight width\paperwidth%
%% }
%% \setbeamercolor{normal text}{bg=red}
%% \end{verbatim}
%% \end{command}







%% \subsection{Table of Contents}

%% \label{section-toc-templates}

%% %% \paragraph{Predefined Templates}\ 

%% %% \begin{command}{\beamertemplateplaintoc}
%% %%   Installs a simple table of contents template with indented subsections. 
%% %% \end{command}

%% %% \begin{command}{\beamertemplateballtoc}
%% %%   Installs a table of contents template in which small balls are shown
%% %%   before each section and subsection.
%% %% \end{command}

%% %% \begin{command}{\beamertemplatenumberedsectiontoc}
%% %%   Installs a table of contents template in which the sections are
%% %%   numbered. 
%% %% \end{command}

%% %% \begin{command}{\beamertemplatenumberedcirclesectiontoc}
%% %%   Installs a table of contents template in which the sections are
%% %%   numbered and the numbers are drawn on a small circle. 
%% %% \end{command}

%% %% \begin{command}{\beamertemplatenumberedballsectiontoc}
%% %%   Installs a table of contents template in which the sections are
%% %%   numbered and the numbers are drawn on a small ball. 
%% %% \end{command}

%% %% \begin{command}{\beamertemplatenumberedsubsectiontoc}
%% %%   Installs a table of contents template in which the subsections are
%% %%   numbered. 
%% %% \end{command}



%% \paragraph{Template Installation Commands}\ 

%% \begin{command}{\usetemplatetocsection\oarg{mix-in specification}%
%%     \marg{template}\opt{\marg{grayed template}}}
%%   Installs a \meta{template} for rendering sections in the table of
%%   contents. If the \meta{mix-in specification} is present, the
%%   \meta{grayed template} may not be present and the grayed sections
%%   names are obtained by mixing in the  \meta{mix-in specification}. 
%%   If \meta{mix-in specification} is not present,  \meta{grayed
%%     template} must be present and is used to render grayed section
%%   names. 
%%   \example
%% \begin{verbatim}
%% \usetemplatetocsection
%% {\color{structure}\inserttocsection}
%% {\color{structure!50}\inserttocsection}

%% \usetemplatetocsection[50!averagebackgroundcolor]
%% {\color{structure}\inserttocsection}
%% \end{verbatim}
%% \end{command}

%% \begin{command}{\usetemplatetocsubsection\oarg{mix-in specification}%
%%     \marg{template}\opt{\marg{grayed template}}}
%%   See |\usetemplatetocsection|.
%%   \example
%% \begin{verbatim}
%% \usetemplatetocsubsection
%% {\leavevmode\leftskip=1.5em\color{black}\inserttocsubsection\par}
%% {\leavevmode\leftskip=1.5em\color{black!50!white}\inserttocsubsection\par}

%% \usetemplatetocsection[50!averagebackgroundcolor]
%% {\leavevmode\leftskip=1.5em\color{black}\inserttocsubsection\par}
%% \end{verbatim}
%% \end{command}



%% \paragraph{Inserts for these Templates}\ 

%% \begin{command}{\inserttocsection}
%%   Inserts the table of contents version of the current section name
%%   into a template.
%% \end{command}

%% \begin{command}{\inserttocsectionnumber}
%%   Inserts the number of the current section (in the table of contents)
%%   into a template. 
%% \end{command}

%% \begin{command}{\inserttocsubsection}
%%   Inserts the table of contents version of the current subsection name
%%   into a template. 
%% \end{command}

%% \begin{command}{\inserttocsubsectionnumber}
%%   Inserts the number of the current subsection (in the table of
%%   contents) into a template. 
%% \end{command}





\subsection{Bibliography}

\label{section-bib-templates}

\paragraph{Predefined Templates}\

\begin{command}{\beamertemplatetextbibitems}
  Shows the citation text in front of references in a
  bibliography instead of a small symbol.
\end{command} 

\begin{command}{\beamertemplatearrowbibitems}
  Changes the symbol before references in a bibliography to
  a small arrow.
\end{command}

\begin{command}{\beamertemplatebookbibitems}
  Changes the symbol before references in a bibliography to
  a small book icon.
\end{command}

\begin{command}{\beamertemplatearticlebibitems}
  Changes the symbol before references in a bibliography to
  a small article icon. (Default)
\end{command}



\paragraph{Template Installation Commands}\ 

\begin{command}{\usebibitemtemplate\marg{citation template}}
  Installs a template for the citation text before the entry. (The 
  ``label'' of the item.)
  \example |\usebibitemtemplate{\color{structure}\insertbiblabel}|
\end{command}


\begin{command}{\usebibliographyblocktemplate%
    \marg{template 1}\marg{template 2}%
    \marg{template 3}\marg{template 4}}
  The text \meta{template~1} is inserted before the first block of the
  entry (the first block is all text before the first occurrence of a 
  |\newblock| command). The text \meta{template~2} is inserted before
  the second block (the text between the first and second occurrence
  of |\newblock|). Likewise for \meta{template~3} and \meta{template~4}. 

  The templates are inserted \emph{before} the blocks and you do not
  have access to the blocks themselves via insert commands. In the
  following example, the first |\par| commands ensure that the
  author, the title, and the journal are put on different lines. The
  color commands cause the author (first block) to be typeset using
  the theme color, the second block (title of the paper) to be typeset
  in black, and all other lines to be typeset in a washed-out version
  of the theme color. 
  \example
\begin{verbatim}
  \usebibliographyblocktemplate
  {\color{structure}}
  {\par\color{black}}
  {\par\color{structure!75}}
  {\par\color{structure!75}}
\end{verbatim}
\end{command}


\paragraph{Inserts for these Templates}\ 

\begin{command}{\insertbiblabel}
  Inserts the current citation label into a template.
\end{command}



%% \subsection{Frame Titles}

%% \label{section-continuation}

%% \paragraph{Predefined Templates}\

%% \begin{command}{\beamertemplateboldcenterframetitle}
%%   Typesets the frame title using a bold face and centers it.
%% \end{command}

%% \begin{command}{\beamertemplatelargeframetitle}
%%   Typesets the frame title using a large face and flushes it left.
%% \end{command}

%% \begin{command}{\beamertemplatecontinuationroman}
%%   Causes the text in the frame title informing that the
%%   current frame has been broken up into several pages to be Roman
%%   numbers. Thus if you have a frame with title ``Foo'' that is broken
%%   up into two pages, the first page will have the title ``Foo~I'' and the
%%   second will have ``Foo~II''. 
%% \end{command}

%% \begin{command}{\beamertemplatecontinuationtext}
%%   Causes the text in the frame title informing that the
%%   current frame has been broken up into several pages to be the text
%%   |\insertcontinuationtext| on all pages but the first. This text
%%   inserted by this insert is ``(cont.)'' by default. Thus if you have
%%   a frame with title ``Foo'' that is broken up into two pages, the
%%   first page will have the title ``Foo'' and the 
%%   second will have ``Foo~(cont.)'' or ``Foo~(Forts.)'' if you have
%%   redefined |\insertcontinuationtext| to ``(Forts.)''.
%% \end{command}


%% \paragraph{Template Installation Commands}\ 

%% \begin{command}{\useframetitletemplate\marg{frame title template}}
%%   \example
%% \begin{verbatim}
%% \useframetitletemplate{%
%%   \begin{centering}
%%     \structure{\textbf{\insertframetitle}}
%%     \par
%%     \small\structure{\textbf{\insertframesubtitle}}}
%%     \par
%%   \end{centering}
%% }
%% \end{verbatim}

%%   \articlenote
%%   This command is also available in |article| mode. By default, a new
%%   paragraph is created. You may wish to install a template that will
%%   simply suppress the frame title.
%% \end{command}



%% \begin{command}{\usecontinuationtemplate\marg{template}}
%%   The \meta{template} will be added at the end of the text inserted by
%%   the |\insertframetitle| text if a frame has the |allowframebreaks| option
%%   set. 
%%   \example
%% \begin{verbatim}
%% \usecontinuationtemplate{ \ifnum\insertcontinuationcount>1(Forts.)\fi}
%% \end{verbatim}
%% \end{command}



%% \paragraph{Inserts for these Templates}\ 

%% \begin{command}{\insertframetitle}
%%   Inserts the current frame title into a template. If the current
%%   frame has the option |allowframebreaks| set, at the end of this insert the
%%   template that has been set using |\usecontinuationtemplate| will be
%%   appended.
%% \end{command}

%% \begin{command}{\insertframesubtitle}
%%   Inserts the current frame subtitle into a template.
%% \end{command}

%% \begin{command}{\insertcontinuationtext}
%%   Inserts the text ``(cont.)'' into a template. Redefine this insert
%%   if you use a different language. This insert is used by the template
%%   installed by the command |\beamertemplatecontinuationtext|.
%% \end{command}

%% \begin{command}{\insertcontinuationcount}
%%   Inserts which page of the current frame is currently presented. If
%%   the |allowframebreaks| option is \emph{not} set, this number is 0.
%% \end{command}

%% \begin{command}{\insertcontinuationcountroman}
%%   Inserts which page of the current frame is currently presented as a
%%   Roman number.
%% \end{command}



\subsection{Headlines and Footlines}

\label{section-head-templates}

\paragraph{Predefined Templates}\ 

\begin{command}{\beamertemplateheadempty}
  Makes the headline empty.
\end{command}

\begin{command}{\beamertemplatefootempty}
  Makes the footline empty.
\end{command}

\begin{command}{\beamertemplatefootpagenumber}
  Shows only the page number in the footline.
\end{command}



\paragraph{Template Installation Commands}\ 

\begin{command}{\usefoottemplate\marg{footline template}}
  The final height of the footline is calculated by invoking this
  template just before the beginning of the document and by setting
  the footline height to the height of the template.
  \example
\begin{verbatim}
\usefoottemplate{\hfil\tiny{\color{black!50}\insertpagenumber}}
\end{verbatim}
or
\begin{verbatim}
\usefoottemplate{%
  \vbox{%
    \tinycolouredline{structure!75}%
      {\color{white}\textbf{\insertshortauthor\hfill\insertshortinstitute}}%
    \tinycolouredline{structure}%
      {\color{white}\textbf{\insertshorttitle}\hfill}%
    }}
\end{verbatim}
\end{command}

\begin{command}{\addtofoottemplate\marg{before}\marg{after}}
  Prepends the text \meta{before} to the current foot template and
  appends the text \meta{after}. This command is useful for adding
  something to a foot template that is installed by some theme.
  \example
\begin{verbatim}
\addtoheadtemplate
  {\vbox\bgroup}
  {%
    \vskip-.5cm%
    \includegraphics[height=.5cm]{myimage.pdf}%
  \egroup}
\end{verbatim}
\end{command}


\begin{command}{\useheadtemplate\marg{headline template}}
  See |\usefoottemplate|.
  \example
\begin{verbatim}
\useheadtemplate{%
  \vbox{%
  \vskip3pt%
  \beamerline{\insertnavigation{\paperwidth}}%
  \vskip1.5pt%
  \insertvrule{0.4pt}{structure!50}}%
}
\end{verbatim}
\end{command}

\begin{command}{\addtoheadtemplate\marg{before}\marg{after}}
  See |\addtofoottemplate|.
\end{command}


\paragraph{Inserts for these Templates}\ 

\begin{command}{\insertframenumber}
  Inserts the number of the current frame (not slide) into a template.
\end{command}

\begin{command}{\inserttotalframenumber}
  Inserts the total number of the frames (not slides) into a
  template. The number is only correct on the second run of \TeX\ on
  your document.
\end{command}

\begin{command}{\insertlogo}
  Inserts the logo(s) into a template.
\end{command}

\begin{command}{\insertnavigation\marg{width}}
  Inserts a horizontal navigation bar of the given \meta{width} into a
  template. The bar lists the sections and below them mini frames for
  each frame in that section.
\end{command}

\begin{command}{\insertpagenumber}
  Inserts the current page number into a template.
\end{command}

\begin{command}{\insertsection}
  Inserts the current section into a template.
\end{command}

\begin{command}{\insertsectionnavigation\marg{width}}
  Inserts a vertical navigation bar containing all sections, with the
  current section hilighted.
\end{command}

\begin{command}{\insertsectionnavigationhorizontal\marg{width}%
    \marg{left insert}\marg{right insert}}
  Inserts a horizontal navigation bar containing all sections, with
  the current section hilighted. The \meta{left insert} will be
  inserted to the left of the sections, the \marg{right insert} to the
  right. By inserting a triple fill (a
  |filll|) you can flush the bar to the left or right.
  \example
\begin{verbatim}
\insertsectionnavigationhorizontal{.5\textwidth}{\hskip0pt plus1filll}{}
\end{verbatim}
\end{command}

\begin{command}{\insertshortauthor\oarg{options}}
  Inserts the short version of the author into a template. The text
  will be printed in one long line, line breaks introduced using the
  |\\| command are suppressed.  The
  following \meta{options} may be given:
  \begin{itemize}
  \item
    \declare{|width=|\meta{width}}
    causes the text to be put into a multi-line minipage of the given
    size. Line breaks are still suppressed by default.
  \item
    \declare{|center|}
    centers the text inside the minipage created using the |width|
    option, rather than having it left aligned.
  \item
    \declare{|respectlinebreaks|}
    causes line breaks introduced by the |\\| command to be honored.    
  \end{itemize}

  \example |\insertauthor[width={3cm},center,respectlinebreaks]|
\end{command}

\begin{command}{\insertshortdate\oarg{options}}
  Inserts the short version of the date into a template. The same
  options as for |\insertshortauthor| may be given. 
\end{command}

\begin{command}{\insertshortinstitute\oarg{options}}
  Inserts the short version of the institute into a template. The same
  options as for |\insertshortauthor| may be given. 
\end{command}

\begin{command}{\insertshortpart\oarg{options}}
  Inserts the short version of the part name into a template. The same
  options as for |\insertshortauthor| may be given. 
\end{command}

\begin{command}{\insertshorttitle\oarg{options}}
  Inserts the short version of the document title into a template. Same
  options as for |\insertshortauthor| may be given. 
\end{command}

\begin{command}{\insertshortsubtitle\oarg{options}}
  Inserts the short version of the document subtitle. Same
  options as for |\insertshortauthor| may be given. 
\end{command}

\begin{command}{\insertsubsection}
  Inserts the current subsection into a template.
\end{command}

\begin{command}{\insertsubsectionnavigation\marg{width}}
  Inserts a vertical navigation bar containing all subsections of the
  current section, with the current subsection hilighted.
\end{command}

\begin{command}{\insertsubsectionnavigationhorizontal\marg{width}%
    \marg{left insert}\marg{right insert}}
  See |\insertsectionnavigationhorizontal|.
\end{command}


\begin{command}{\insertverticalnavigation\marg{width}}
  Inserts a vertical navigation bar of the given \meta{width} into a
  template. The bar shows a little table of contents. The individual
  lines are typeset using the templates
  |\usesectionsidetemplate| and |\usesubsectionsidetemplate|.
\end{command}

\begin{command}{\insertframestartpage}
  Inserts the page number of the first page of the current frame.
\end{command}

\begin{command}{\insertframeendpage}
  Inserts the page number of the last page of the current frame.
\end{command}

\begin{command}{\insertsubsectionstartpage}
  Inserts the page number of the first page of the current subsection.
\end{command}

\begin{command}{\insertsubsectionendpage}
  Inserts the page number of the last page of the current subsection.
\end{command}

\begin{command}{\insertsectionstartpage}
  Inserts the page number of the first page of the current section.
\end{command}

\begin{command}{\insertsectionendpage}
  Inserts the page number of the last page of the current section.
\end{command}

\begin{command}{\insertpartstartpage}
  Inserts the page number of the first page of the current part.
\end{command}

\begin{command}{\insertpartendpage}
  Inserts the page number of the last page of the current part.
\end{command}

\begin{command}{\insertpresentationstartpage}
  Inserts the page number of the first page of the presentation.
\end{command}

\begin{command}{\insertpresentationendpage}
  Inserts the page number of the last page of the presentation
  (excluding the appendix).
\end{command}


\begin{command}{\insertappendixstartpage}
  Inserts the page number of the first page of the appendix. If there
  is no appendix, this number is the last page of the document.
\end{command}

\begin{command}{\insertappendixendpage}
  Inserts the page number of the last page of the appendix. If there
  is no appendix, this number is the last page of the document.
\end{command}

\begin{command}{\insertdocumentstartpage}
  Inserts 1.
\end{command}

\begin{command}{\insertdocumentendpage}
  Inserts the page number of the last page of the document (including
  the appendix).
\end{command}




\subsection{Sidebars}

\label{section-sidebar-templates}

In the following, only the commands for the left sidebars are
listed. Each of these commands also exists for the right sidebar,
with ``left'' replaced by ``right'' everywhere.


\begin{command}{\useleftsidebartemplate\marg{horizontal size}\marg{template}}
  When the sidebar is typeset, the \meta{template} is invoked inside a
  |\vbox| of the height of the sidebar. Thus, the below example
  will produce a sidebar of half a centimeter width, in which the word
  ``top'' is printed just below the headline and ``bottom'' is printed
  just above the footline.
  \example
\begin{verbatim}
\useleftsidebartemplate{1cm}{
  top
  \vfill
  bottom
}
\end{verbatim}
\end{command}

\begin{command}{\useleftsidebarbackgroundtemplate\marg{template}}
  The template is shown behind whatever is shown in the left side
  bar. 
  \example
\begin{verbatim}
\useleftsidebarbackgroundtemplate
  {\color{red}\vrule height\paperheight width\beamer@leftsidebar}
\end{verbatim}
\end{command}


\begin{command}{\useleftsidebarcolortemplate\marg{color expression}}
  Uses the given color as background for the sidebar.
  \example
\begin{verbatim}
\useleftsidebarcolortemplate{\color{red}}
\useleftsidebarcolortemplate{\color[rgb]{1,0,0.5}}
\end{verbatim}
\end{command}

\begin{command}{\useleftsidebarverticalshadingtemplate\marg{bottom
      color expression}\marg{top color expression}}
  Installs a smooth vertical transition between the given colors as
  background for the sidebar.
  \example
\begin{verbatim}
\useleftsidebarverticalshadingtemplate{white}{red}
\end{verbatim}
\end{command}


\begin{command}{\useleftsidebarhorizontalshadingtemplate\marg{left end
      color expression}\marg{right end color expression}}
  Installs a smooth horizontal transition between the given colors as
  background for the sidebar.
  \example
\begin{verbatim}
\useleftsidebarhorizontalshadingtemplate{white}{red}
\end{verbatim}
\end{command}


\begin{command}{\usesectionsidetemplate\marg{current section
      template}\marg{other section template}}
  Both parameters should be |\hbox|es. The templates are used to
  typeset a section name inside a side navigation bar.
  \example
\begin{verbatim}
\usesectionsidetemplate
{\setbox\tempbox=\hbox{\color{black}\tiny{\kern3pt\insertsectionhead}}%
  \ht\tempbox=8pt%
  \dp\tempbox=2pt%
  \wd\tempbox=\beamer@sidebarwidth%
  \box\tempbox}
{\setbox\tempbox=\hbox{\color{structure!75}\tiny{\kern3pt\insertsectionhead}}%
  \ht\tempbox=8pt%
  \dp\tempbox=2pt%
  \wd\tempbox=\beamer@sidebarwidth%
  \box\tempbox}
\end{verbatim}
\end{command}



\begin{command}{\usesubsectionsidetemplate\marg{current subsection
      template}\marg{other subsection template}}
  See |\usesectionsidetemplate|.
  \example
\begin{verbatim}
\usesectionsidetemplate
{\setbox\tempbox=\hbox{\color{black}\tiny{\kern3pt\insertsectionhead}}%
  \ht\tempbox=8pt%
  \dp\tempbox=2pt%
  \wd\tempbox=\beamer@sidebarwidth%
  \box\tempbox}
{\setbox\tempbox=\hbox{\color{structure!75}\tiny{\kern3pt\insertsectionhead}}%
  \ht\tempbox=8pt%
  \dp\tempbox=2pt%
  \wd\tempbox=\beamer@sidebarwidth%
  \box\tempbox}
\end{verbatim}
\end{command}










\subsection{Buttons}
\label{section-navigation-buttons}

\paragraph{Predefined Templates}\ 

\begin{command}{\beamertemplateoutlinebuttons}
  Renders buttons as rectangles with rounded left and right
  border. Only the border (outline) is painted.
\end{command}

\begin{command}{\beamertemplatesolidbuttons}
  Renders buttons as filled rectangles with rounded left and right
  border.
\end{command}


\paragraph{Template Installation Commands}\ 

\begin{command}{\usebuttontemplate\marg{button template}}
  Installs a new button template. This template is invoked whenever a
  button should be rendered.
  \example
\begin{verbatim}
\usebuttontemplate{\color{structure}\insertbuttontext}
\end{verbatim}
\end{command}


\paragraph{Inserts}\ 

Inside the button template, the button text can be accessed via the
following command:

\begin{command}{\insertbuttontext}
  Inserts the text of the current button into a template. When called
  by  button creation commands, like |\beamerskipbutton|, the symbol
  will be part of this text.
\end{command}

The button creation commands automatically add the following three
inserts to the text to be rendered by |\insertbuttontext|:

\begin{command}{\insertgotosymbol}
  Inserts a small right-pointing arrow.
\end{command}

\begin{command}{\insertskipsymbol}
  Inserts a double right-pointing arrow.
\end{command}

\begin{command}{\insertreturnsymbol}
  Inserts a small left-pointing arrow.
\end{command}

You can redefine these commands to change these symbols.




\subsection{Navigation Bars}

\paragraph{Predefined Templates}\ 

\begin{command}{\beamertemplatecircleminiframe}
  Changes the symbols in a navigation bar used to represent
  a frame to a small circle.
\end{command}

\begin{command}{\beamertemplatecircleminiframeinverted}
  Changes the symbols in a navigation bar used to represent
  a frame to a small circle, but with the colors inverted. Use this if
  the navigation bar is shown on a dark background.
\end{command}

\begin{command}{\beamertemplatesphereminiframe}
  Changes the symbols in a navigation bar used to represent
  a frame to a small sphere.
\end{command}

\begin{command}{\beamertemplatesphereminiframeinverted}
  Changes the symbols in a navigation bar used to represent
  a frame to a small sphere, but with the colors inverted. Use this if
  the navigation bar is shown on a |structure| background.
\end{command}

\begin{command}{\beamertemplateboxminiframe}
  Changes the symbols in a navigation bar used to represent
  a frame to a small box.
\end{command}

\begin{command}{\beamertemplateticksminiframe}
  Changes the symbols in a navigation bar used to represent
  a frame to a small vertical bar of varying length.
\end{command}


\paragraph{Template Installation Commands}\ 

\begin{command}{\usesectionheadtemplate\marg{current section
      template}\marg{other section template}}
  The templates are used to render the section names in a navigation
  bar. 
  \example
\begin{verbatim}
\usesectionheadtemplate
  {\hfill\color{white}\tiny\textbf{\insertsectionheadnumber.\ \
    \insertsectionhead}}
  {\hfill\color{white!50!black}\tiny\textbf{\insertsectionheadnumber.\ \ 
    \insertsectionhead}}
\end{verbatim}
\end{command}
  

\begin{command}{\usesubsectionheadtemplate\marg{current subsection
      template}\marg{other subsection template}}
  See |\usesectionheadtemplate|.
  \example
\begin{verbatim}
\usesubsectionheadtemplate{\color{white}%
  \tiny\textbf{\insertsectionheadnumber.\insertsubsectionheadnumber\ \
  \insertsubsectionhead}}%
  {\color{white!50!beamerstructure}%
  \tiny\textbf{\insertsectionheadnumber.\insertsubsectionheadnumber\ \ 
  \insertsubsectionhead}}
\end{verbatim}
\end{command}

\begin{command}{\useminislidetemplate%
    \marg{template current frame icon}%
    \marg{template current subsection frame icon}\\%
    \marg{template other frame icon}%
    \marg{horizontal offset}%
    \marg{vertical offset}}
  The templates are used to draw frame icons in navigation bars. The
  offsets describe the offset between icons.
  \example
\begin{verbatim}
\useminislidetemplate
  {
    \color{structure}%
    \hskip-0.4pt\vrule height\boxsize width1.2pt%
  }  
  {%
    \color{structure}%
    \vrule height\boxsize width0.4pt%
  }
  {%
    \color{structure!50}%
    \vrule height\boxsize width0.4pt%
  }
  {.1cm}
  {.05cm}
\end{verbatim}
\end{command}


\paragraph{Inserts}\


\begin{command}{\insertsectionhead}
  Inserts the text of the section that is to be typeset in a
  navigation bar.
\end{command}

\begin{command}{\insertsubsectionhead}
  Inserts the text of the subsection that is to be typeset in a
  navigation bar. 
\end{command}

\begin{command}{\insertsectionheadnumber}
  Inserts the number of the section that is to be typeset in a
  navigation bar. 
\end{command}

\begin{command}{\insertsubsectionheadnumber}
  Inserts the number of the subsection that is to be typeset in a
  navigation bar. 
\end{command}

\begin{command}{\insertpartheadnumber}
  Inserts the number of the part of the current section of subsection
  that is to be typeset in a navigation bar. 
\end{command}





\subsection{Navigation Symbols}
\label{section-navigation-symbols-template}

\paragraph{Predefined Templates}\ 

\begin{command}{\beamertemplatenavigationsymbolsempty}
  Suppresses all navigation symbols.
\end{command}

\begin{command}{\beamertemplatenavigationsymbolsframe}
  Shows only the frame symbol as navigation symbol.
\end{command}

\begin{command}{\beamertemplatenavigationsymbolsvertical}
  Organizes the navigation symbols vertically.
\end{command}

\begin{command}{\beamertemplatenavigationsymbolshorizontal}
  Organizes the navigation symbols horizontally.
\end{command}



\paragraph{Template Installation Commands}\ 

\begin{command}{\usenavigationsymbolstemplate\marg{symbols template}}
  Installs a new symbols template. This template is invoked by themes
  at the place where the navigation symbols should be shown.
  \example
\begin{verbatim}
\usenavigationsymbolstemplate{\vbox{%
  \hbox{\insertslidenavigationsymbol}
  \hbox{\insertframenavigationsymbol}
  \hbox{\insertsubsectionnavigationsymbol}
  \hbox{\insertsectionnavigationsymbol}
  \hbox{\insertdocnavigationsymbol}
  \hbox{\insertbackfindforwardnavigationsymbol}}}
\end{verbatim}
\end{command}


\paragraph{Inserts for these Templates}\ 

The following inserts are useful for the navigation symbols template:

\begin{command}{\insertslidenavigationsymbol}
  Inserts the slide navigation symbol, see
  Section~\ref{section-navigation-symbols}.
\end{command}

\begin{command}{\insertframenavigationsymbol}
  Inserts the frame navigation symbol, see
  Section~\ref{section-navigation-symbols}.
\end{command}

\begin{command}{\insertsubsectionnavigationsymbol}
  Inserts the subsection navigation symbol, see
  Section~\ref{section-navigation-symbols}.
\end{command}

\begin{command}{\insertsectionnavigationsymbol}
  Inserts the section navigation symbol, see
  Section~\ref{section-navigation-symbols}.
\end{command}

\begin{command}{\insertdocnavigationsymbol}
  Inserts the presentation navigation symbol and (if necessary) the
  appendix navigation symbol, see
  Section~\ref{section-navigation-symbols}.
\end{command}

\begin{command}{\insertbackfindforwardnavigationsymbol}
  Inserts a back, a find, and a forward navigation symbol, see
  Section~\ref{section-navigation-symbols}.
\end{command}





\subsection{Footnotes}

\label{section-templates-footnotes}

\paragraph{Template Installation Commands}\

\begin{command}{\usefootnotetemplate\marg{footnote template}}
  \example
\begin{verbatim}
\usefootnotetemplate{
  \parindent 1em
  \noindent
  \hbox to 1.8em{\hfil\insertfootnotemark}\insertfootnotetext}
\end{verbatim}
\end{command}


\paragraph{Inserts for these Templates}\

\begin{command}{\insertfootnotemark}
  Inserts the current footnote mark (like a raised number) into a
  template. 
\end{command}

\begin{command}{\insertfootnotetext}
  Inserts the current footnote text into a template. 
\end{command}





\subsection{Captions}
\label{section-template-caption}

\paragraph{Predefined Templates}\

\begin{command}{\beamertemplatecaptionwithnumber}
  Changes the caption template such that the number of the
  table or figure is also shown.
\end{command}

\begin{command}{\beamertemplatecaptionownline}
  Changes the caption template such that the word ``Table''
  or ``Figure'' has its own line.
\end{command}



\paragraph{Template Installation Commands}\

\begin{command}{\usecaptiontemplate\marg{caption template}}
  \example
\begin{verbatim}
\usecaptiontemplate{
  \small
  \structure{\insertcaptionname~\insertcaptionnumber:}
  \insertcaption
}
\end{verbatim}
\end{command}



\paragraph{Inserts for these Templates}\
 
\begin{command}{\insertcaption}
  Inserts the text of the current caption into a template.
\end{command}

\begin{command}{\insertcaptionname}
  Inserts the name of the current caption into a template. This word
  is either ``Table'' or ``Figure'' or, if the |babel| package is
  used, some translation thereof.
\end{command}

\begin{command}{\insertcaptionnumber}
  Inserts the number of the current figure or table into a template.
\end{command}






%% \subsection{Lists (Itemizations, Enumerations, Descriptions)}

%% \label{section-template-enumerate}

%% \paragraph{Predefined Templates}\

%% \begin{command}{\beamertemplateballitem}
%%   Changes the symbols shown in an |itemize| and an |enumerate|
%%   environment to small plastic balls.
%% \end{command}

%% \begin{command}{\beamertemplatedotitem}
%%   Changes the symbols shown in an |itemize|
%%   environment to dots.
%% \end{command}

%% \begin{command}{\beamertemplatetriangleitem}
%%   Changes the symbols shown in an |itemize|
%%   environment to triangles.
%% \end{command}

%% \begin{command}{\beamertemplateenumeratealpha}
%%   Changes the labels of first-level enumerations to ``1.'', ``2.'',
%%   ``3.'', and so on, and to ``1.1'', ``1.2'', ``1.3'', and so on for
%%   the second level.
%% \end{command}



%\paragraph{Template Installation Commands}\

%\begin{command}{\useenumerateitemtemplate\marg{template}}
%  The \meta{template} is used to render the default item in the top
%  level of an enumeration. 
%  \example |\useenumerateitemtemplate{\insertenumlabel}|
%\end{command}


%\begin{command}{\useitemizeitemtemplate\marg{template}}
%  The \meta{template} is used to render the default item in the top
%  level of an itemize list.
%  \example |\useitemizeitemtemplate{\pgfuseimage{mybullet}}|
%\end{command}


%\begin{command}{\usesubitemizeitemtemplate\marg{template}}
%  The \meta{template} is used to render the default item in the
%  second level of an itemize list.
%  \example |\usesubitemizeitemtemplate{\pgfuseimage{mysubbullet}}|
%\end{command}

%\begin{command}{\usesubitemizeitemtemplate\marg{template}}
%  The \meta{template} is used to render the default item in the
%  third level of an itemize list.
%  \example |\usesubitemizeitemtemplate{\pgfuseimage{mysubbullet}}|
%\end{command}

%\begin{command}{\useitemizetemplate\marg{begin text}\marg{end text}}
%  The \meta{begin text} is inserted at the beginning of a top-level
%  itemize list, the \meta{end text} at its end.
%  \example |\useitemizetemplate{}{}|
%\end{command}

%\begin{command}{\usesubitemizetemplate\marg{begin text}\marg{end text}}
%  The \meta{begin text} is inserted at the beginning of a second-level
%  itemize list, the \meta{end text} at its end.
%  \example |\usesubitemizetemplate{\begin{small}}{\end{small}}|
%\end{command}

%\begin{command}{\usesubitemizetemplate\marg{begin text}\marg{end text}}
%  The \meta{begin text} is inserted at the beginning of a third-level
%  itemize list, the \meta{end text} at its end.
%  \example |\usesubitemizetemplate{\begin{footnotesize}}{\end{footnotesize}}|
%\end{command}


%\begin{command}{\useenumerateitemtemplate\marg{template}}
%  The \meta{template} is used to render the default item in the
%  top-level of an enumeration.  
%  \example
%  |\useenumerateitemtemplate{\insertenumlabel}|
%\end{command}

%\begin{command}{\useenumerateitemminitemplate\marg{template}}
%  The \meta{template} is used to render the items in an enumeration
%  where the optional argument \meta{mini template} is used (see
%  Section~\ref{section-enumerate}). 
%  \example
%  |\useenumerateitemminitemplate{\color{structure}\insertenumlabel}|
%\end{command}

%\begin{command}{\usesubenumerateitemtemplate\marg{template}}
%  The \meta{template} is used to render the default item in the second
%  level of an enumeration. 
%  \example
%  |\usesubenumerateitemtemplate{\insertenumlabel-\insertsubenumlabel}|
%\end{command}

%\begin{command}{\usesubenumerateitemtemplate\marg{template}}
%  The \meta{template} is used to render the default item in the third
%  level of an enumeration. 
%  \example
%\begin{verbatim}
%\usesubenumerateitemtemplate
%{\insertenumlabel-\insertsubenumlabel-\insertsubenumlabel}
%\end{verbatim}
%\end{command}


%\begin{command}{\useenumeratetemplate\marg{begin text}\marg{end text}}
%  The \meta{begin text} is inserted at the beginning of a top-level
%  enumeration, the \meta{end text} at its end.
%  \example |\useenumeratetemplate{}{}|
%\end{command}

%\begin{command}{\usesubenumeratetemplate\marg{begin text}\marg{end text}}
%  The \meta{begin text} is inserted at the beginning of a second-level
%  enumeration, the \meta{end text} at its end.
%  \example |\usesubenumeratetemplate{\begin{small}}{\end{small}}|
%\end{command}

%\begin{command}{\usesubenumeratetemplate\marg{begin text}\marg{end text}}
%  The \meta{begin text} is inserted at the beginning of a third-level
%  enumeration, the \meta{end text} at its end.
%  \example |\usesubenumeratetemplate{\begin{footnotesize}}{\end{footnotesize}}|
%\end{command}

%% \begin{command}{\usedescriptiontemplate\marg{description
%%       template}\marg{default width}}
%%   The \meta{default width} is used as width of the default item, if no
%%   other width is specified; the width |\labelsep| is
%%   automatically added to this parameter.
%%   \example
%%   |\usedescriptionitemtemplate{\color{structure}\insertdescriptionitem}{2cm}|
%% \end{command}


%% %\paragraph{Inserts for these Templates}\

%% \begin{command}{\insertdescriptionitem}
%%   Inserts the current item of a |description| environment into a
%%   template.
%% \end{command}

%\begin{command}{\insertenumlabel}
%  Normally, this command inserts the current number of the top-level
%  enumeration (as an Arabic number) into a template. However, in an
%  enumeration where the optional \meta{mini template} option is used,
%  this command inserts the current number rendered by this mini
%  template. For example, if the \meta{mini template} is |(i)| and this
%  command is used in the fourth item, |\insertenumlabel| would yield
%  |(iv)|. 
%\end{command}

%\begin{command}{\insertsubenumlabel}
%  Inserts the current number of the second-level enumeration (as an
%  Arabic number) into a template.
%\end{command}

%\begin{command}{\insertsubenumlabel}
%  Inserts the current number of the third-level enumeration (as an
%  Arabic number) into a template.
%\end{command}





\subsection{Hilighting Commands}


\paragraph{Template Installation Commands}\

\begin{command}{\usealerttemplate\marg{alert template
      begin}\marg{alert template end}}
  In an |\alert| command and in an |alertenv| environment, the text
  \meta{alert template begin} is inserted at the beginning, the text
  \meta{alert template end} at the end.
  
  \example |\usealerttemplate{\color{red}}{}|

  \articlenote
  This command is also available in |article| mode.
\end{command}

\begin{command}{\usestructuretemplate\marg{structure template
      begin}\marg{structure template end}}
  Same as for alerts.
  
  \example |\usestructuretemplate{\color{blue}}{}|

  \articlenote
  This command is also available in |article| mode.
\end{command}




\subsection{Block Environments}

\paragraph{Predefined Templates}\

\begin{command}{\beamertemplateboldblocks}
  Block titles are printed in bold.
\end{command}

\begin{command}{\beamertemplatelargeblocks}
  Block titles are printed slightly larger.
\end{command}

\begin{command}{\beamertemplateroundedblocks}
  Changes the block templates such that they are printed on a
  rectangular area with rounded corners.
\end{command}

\begin{command}{\beamertemplateshadowblocks}
  Changes the block templates such that they are printed on a
  rectangular area with rounded corners and a shadow.
\end{command}



\paragraph{Template Installation Commands}\

\begin{command}{\useblocktemplate\marg{block beginning
      template}\marg{block end template}}
  \example
\begin{verbatim}
\useblocktemplate
  {%
   \medskip%
    {\color{blockstructure}\textbf{\insertblockname}}%
    \par%
  }
  {\medskip}
\end{verbatim}

  \articlenote
  This command is also available in |article| mode.
\end{command}


\begin{command}{\usealertblocktemplate\marg{block beginning
      template}\marg{block end template}}
  \example
\begin{verbatim}
\usealertblocktemplate
  {%
    \medskip
    {\alert{\textbf{\insertblockname}}}%
  \par}
  {\medskip}
\end{verbatim}

  \articlenote
  This command is also available in |article| mode.
\end{command}


\begin{command}{\useexampleblocktemplate\marg{block beginning
      template}\marg{block end template}}
  \example
\begin{verbatim}
\useexampleblocktemplate
  {%
    \medskip
    \begingroup\color{darkgreen}{\textbf{\insertblockname}}
    \par}
  {%
     \endgroup
     \medskip
  }
\end{verbatim}

  \articlenote
  This command is also available in |article| mode.
\end{command}


\paragraph{Inserts for these Templates}\

\begin{command}{\insertblockname}
  Inserts the name of the current block into a template.
\end{command}



\subsection{Theorem Environments}

\label{section-theorems-templates}

\paragraph{Predefined Templates}\

\begin{command}{\beamertemplatetheoremssimple}
  Causes the theorem head and text to be directly passed to the
  |block| or |exampleblock| environment. All font specifications for
  theorems are ignored. 
\end{command}

\begin{command}{\beamertemplatetheoremsunnumbered}
  Causes theorems to be typeset as follows: The font specification for
  the body is honored, the font specification for the head is
  ignored. No theorem number is printed. This is the default.
\end{command}

\begin{command}{\beamertemplatetheoremsnumbered}
  Like |\beamertemplatetheoremsunnumbered|, except that the theorem
  number is printed for environments that are numbered.
\end{command}

\begin{command}{\beamertemplatetheoremsamslike}
  This causes theorems to be put in a |block| or |exampleblock|, but
  to be otherwise typeset as is normally done in |amsthm|. Thus the
  head font and body font depend on the setting for the theorem to be
  typeset and theorems are numbered. 
\end{command}



\paragraph{Template Installation Commands}\

\begin{command}{\usetheoremtemplate\marg{block beginning
      template}\marg{block end template}}
  \beamernote
  Whenever an environment declared using the command |\newtheorem| is
  to be typeset, the \meta{block beginning template} is inserted at
  the beginning and the \meta{block end template} at the end. If there
  is a overlay specification when an environment like |theorem| is
  used, this overlay specification will directly follow the
  \meta{block beginning template} upon invocation. This is even true
  if there was an optional argument to the |theorem| environment. This
  optional argument is available via the insert |\inserttheoremaddition|.

  Numerous inserts are available in this template, see below.  

  Before the template starts, the font is set to the body font
  prescribed by the environment to be typeset.
  
  \example The following typesets theorems like |amsthm|:
\begin{verbatim}
\usetheoremtemplate{\begin{\inserttheoremblockenv}
  {%
    \inserttheoremheadfont
    \inserttheoremname
    \inserttheoremnumber
    \ifx\inserttheoremaddition\@empty\else\ (\inserttheoremaddition)\fi%
    \inserttheorempunctuation
  }%
}{\end{\inserttheoremblockenv}}
\end{verbatim}

  \example In the following example, all font ``suggestions'' for the
  environment are suppressed or ignored; and the theorem number is
  suppressed.
\begin{verbatim}
\usetheoremtemplate{%
  \normalfont% ignore body font
  \begin{\inserttheoremblockenv}
  {%
    \inserttheoremname
    \ifx\inserttheoremaddition\@empty\else\ (\inserttheoremaddition)\fi%
  }%
}{\end{\inserttheoremblockenv}}
\end{verbatim}
  
  \articlenote
  This command is not available in |article| mode.
\end{command}


\paragraph{Inserts for these Templates}\

\begin{command}{\inserttheoremblockenv}
  This will normally expand to |block|, but if a theorem that has
  theorem style |example| is typeset, it will expand to
  |exampleblock|. Thus you can use this insert to decide which
  environment should be used when typesetting the theorem.
\end{command}

\begin{command}{\inserttheoremheadfont}
  This will expand to a font chainging command that switches to the
  font to be used in the head of the theorem. By not inserting it, you
  can ignore the head font.
\end{command}

\begin{command}{\inserttheoremname}
  This will expand to the name of the environment to be typeset (like
  ``Theorem'' or ``Corollary''). 
\end{command}


\begin{command}{\inserttheoremnumber}
  This will expand to the number of the current theorem preceeded by a
  space or to nothing, if the current theorem does not have a number.
\end{command}

\begin{command}{\inserttheoremaddition}
  This will expand to the optional argument given to the environment
  or will be empty, if there was no optional argument.
\end{command}

\begin{command}{\inserttheorempunctuation}
  This will expand to the punctuation character for the current
  environment. This is usually a period.
\end{command}





\subsection{Verse, Quotation and Quote Environments}


\paragraph{Template Installation Commands}\

\begin{command}{\usetemplateverse\marg{block beginning
      template}\marg{block end template}}
  In a |verse| environment, the \meta{block beginning template} is
  inserted before the verse, the \meta{block end template} after the
  verse. The margins are not setup in these templates; this is done in
  the |verse| environment and cannot be changed.
  \example |\usetemplateverse{\rmfamily\itshape}{}|
\end{command}


\begin{command}{\usetemplatequotation\marg{block beginning
      template}\marg{block end template}}
  Both in |quotation| and in |quote| environments, the \meta{block
    beginning template} is inserted before the quotation, the
  \meta{block end template} after the quotation. As for verses, the
  margins are not setup in these templates and cannot be changed.
  
  \example |\usetemplatequotation{\itshape}{}|
\end{command}




\subsection{Typesetting Notes}

\label{section-note-templates}

\paragraph{Predefined Templates}\

\begin{command}{\beamertemplatenoteplain}
  Causes all note pages to contain only the note text.
\end{command}

\begin{command}{\beamertemplatenotecompress}
  Causes the ``routing information'' at the top of a note to be
  smaller. 
\end{command}


\paragraph{Template Installation Commands}\

\begin{command}{\usetemplatenote\marg{note template}}
  Each note is typeset by inserting the \meta{note template}. The
  template should contain a mentioning of the insert |\insertnote|,
  which will contain the note text.
  
  \example |\usetemplatenote{\tiny\insertnote}|
\end{command}



\paragraph{Inserts for these Templates}\

\begin{command}{\insertnote}
  Inserts the text of the current note into the template.
\end{command}


\begin{command}{\insertslideintonotes\marg{magnification}}
  Inserts a ``mini picture'' of the last slide into the current
  note. The slide will be scaled by the given magnification.

  \example |\insertslideintonotes{0.25}|

  This will give a mini slide whose width and height are one fourth of
  the usual size.
\end{command}



%%% Local Variables: 
%%% mode: latex
%%% TeX-master: "beameruserguide"
%%% End: 
